
\documentclass{ctexart}
%%%%%%%%%%%%%%%%%%%%%%%%%%%%%%%%%%%%%%%%%%%%%%%%%%%%%%%%%%%%%%%%%%%%%%%%%%%%%%%%%%%%%%%%%%%%%%%%%%%%%%%%%%%%%%%%%%%%%%%%%%%%%%%%%%%%%%%%%%%%%%%%%%%%%%%%%%%%%%%%%%%%%%%%%%%%%%%%%%%%%%%%%%%%%%%%%%%%%%%%%%%%%%%%%%%%%%%%%%%%%%%%%%%%%%%%%%%%%%%%%%%%%%%%%%%%
\usepackage{amsmath}


\setcounter{MaxMatrixCols}{10}
%TCIDATA{OutputFilter=LATEX.DLL}
%TCIDATA{Version=5.00.0.2552}
%TCIDATA{<META NAME="SaveForMode" CONTENT="1">}
%TCIDATA{Created=Sunday, October 25, 2015 11:39:15}
%TCIDATA{LastRevised=Wednesday, May 18, 2016 21:36:08}
%TCIDATA{<META NAME="GraphicsSave" CONTENT="32">}
%TCIDATA{<META NAME="DocumentShell" CONTENT="Scientific Notebook\Blank Document">}
%TCIDATA{CSTFile=Math with theorems suppressed.cst}
%TCIDATA{PageSetup=72,72,72,72,0}
%TCIDATA{AllPages=
%F=36,\PARA{038<p type="texpara" tag="Body Text" >\hfill \thepage}
%}


\newtheorem{theorem}{Theorem}
\newtheorem{acknowledgement}[theorem]{Acknowledgement}
\newtheorem{algorithm}[theorem]{Algorithm}
\newtheorem{axiom}[theorem]{Axiom}
\newtheorem{case}[theorem]{Case}
\newtheorem{claim}[theorem]{Claim}
\newtheorem{conclusion}[theorem]{Conclusion}
\newtheorem{condition}[theorem]{Condition}
\newtheorem{conjecture}[theorem]{Conjecture}
\newtheorem{corollary}[theorem]{Corollary}
\newtheorem{criterion}[theorem]{Criterion}
\newtheorem{definition}[theorem]{Definition}
\newtheorem{example}[theorem]{Example}
\newtheorem{exercise}[theorem]{Exercise}
\newtheorem{lemma}[theorem]{Lemma}
\newtheorem{notation}[theorem]{Notation}
\newtheorem{problem}[theorem]{Problem}
\newtheorem{proposition}[theorem]{Proposition}
\newtheorem{remark}[theorem]{Remark}
\newtheorem{solution}[theorem]{Solution}
\newtheorem{summary}[theorem]{Summary}
\newenvironment{proof}[1][Proof]{\noindent\textbf{#1.} }{\ \rule{0.5em}{0.5em}}


\begin{document}



\bigskip\ 
泛函分析$\left(
1\right) $第一次习题课 题目%
与解答

1. 设n.l.s X上的一个次可加%
泛函$f:f(x+y)\leq f\left( x\right) +f\left( y\right) \forall
x,y\in X.$已知$f$在

$\left\{ x\in X|\left\Vert x\right\Vert =r\right\} $外非负,%
求证$f\left( x\right) \geq 0,\forall x\in X.$

解: $\forall x\in X,$if $x=0,f\left( 0\right) \leq f\left( 0\right)
+f\left( 0\right) \implies f\left( 0\right) \geq 0,$

if $x\neq 0,$there exists a positive integer n s.t. $nx\notin \left\{ x\in
X|\left\Vert x\right\Vert \leq r\right\} ,\implies f\left( nx\right) \geq 0,$

$f(x+y)\leq f\left( x\right) +f\left( y\right) \implies f\left( nx\right)
\leq nf\left( x\right) \implies f\left( x\right) \geq 0.$

2. 见课本第4页Corollary 1.4

3. 2的Corollary,show that for given $x,y\in X,$if $\forall f\in
X^{\ast },f\left( x\right) =f\left( y\right) ,$then $x=y.$

Solution: 用反证法.Suppose $x\neq y,$ then by 2
there exists $\tilde{f}\in X^{\ast },s.t.\left\Vert \tilde{f}\right\Vert =1$
and $\tilde{f}\left( x-y\right) =\left\Vert x-y\right\Vert \neq 0,$与%
条件中 $\forall f\in X^{\ast },f\left( x\right) =f\left(
y\right) $矛盾.

4.设$\NEG{R}^{2}$上的线性泛函$%
f\left( x\right) =\alpha \xi _{1}+\beta \xi _{2},\forall x=\left( \xi
_{1},\xi _{2}\right) \in \NEG{R}^{2}$,证明可将%
它保范延拓到$\NEG{R}^{3}$上.

证明: 利用Cauchy-Inequality 可求出$%
\left\vert f\left( x\right) \right\vert \leq \sqrt{\xi _{1}^{2}+\xi _{2}^{2}}%
\sqrt{\alpha ^{2}+\beta ^{2}},$若$\NEG{R}^{2}$取Euclid norm$%
\implies \left\Vert f\right\Vert \leq \sqrt{\alpha ^{2}+\beta ^{2}}.$再%
取$x=\left( \frac{\alpha }{\sqrt{\alpha ^{2}+\beta ^{2}}},\frac{\beta 
}{\sqrt{\alpha ^{2}+\beta ^{2}}}\right) ,$

$\implies \left\Vert f\right\Vert \geq \left\vert f\left( x\right)
\right\vert =\sqrt{\alpha ^{2}+\beta ^{2}}\implies \left\Vert f\right\Vert =%
\sqrt{\alpha ^{2}+\beta ^{2}},$

在$\NEG{R}^{3}$上定义$g:\NEG{R}^{3}->\NEG{R}$ $%
g\left( x\right) =\alpha \eta _{1}+\beta \eta _{2},\forall x=\left( \eta
_{1},\eta _{2},\eta _{3}\right) \in \NEG{R}^{3}.$可证$g$ 是%
$f$的延拓且由和上面同%
样的演绎$\left\Vert g\right\Vert =\sqrt{\alpha
^{2}+\beta ^{2}}.\implies g$ 是$f$ 的保范延%
拓.

(个人观点:2维空间是3维%
空间的一个子空间,貌%
似做了一个投影变换)

\bigskip

5.设$Y$为n.l.s$\qquad X的 一 个 子 
空 间 ,x_{0}\in X,d\left( x_{0},Y\right) >0.$试证%
明存在一个X上的有界%
线性泛函f,使得f$\left( Y\right)
=0,f\left( x_{0}\right) =d\left( x_{0},Y\right) ,$且$\left\Vert
f\right\Vert =1.$

\bigskip Solution: (refer to lecture note after the professor 介%
绍保范延拓的概念)

先在$x_{0}$和$Y$张成的子空%
间上定义一个有界线%
性泛函,

$g:t\left\{ x_{0}\right\} +Y\rightarrow td\left( x_{0},Y\right) .$由%
于$x_{0}\notin \bar{Y}$可证此定义合%
理,$g$ 的线性性质容易验%
证.

\bigskip $\forall x\in t\left\{ x_{0}\right\} +Y,$ there exists unqiue $t\in
R$ and $y\in Y,s.t.x=tx_{0}+y$

$\left\vert g\left( x\right) \right\vert =\left\vert t\right\vert d\left(
x_{0},Y\right) ,$if $t\neq 0,d\left( x_{0},Y\right) \leq \left\vert x_{0}+%
\frac{y}{t}\right\vert \implies \left\vert g\left( x\right) \right\vert \leq
\left\vert t\right\vert \left\vert x_{0}+\frac{y}{t}\right\vert $

$\leq \left\vert tx_{0}+y\right\vert =\left\vert x\right\vert \implies
\left\Vert g\right\Vert \leq 1,g$ is bounded. Further, by the definition of

$d\left( x_{0},Y\right) ,\exists \left\{ y_{n}\right\} \subset
Y,s.t.\left\Vert x_{0}-y_{n}\right\Vert =d\left( x_{0},y_{n}\right)
\rightarrow d\left( x_{0},Y\right) ,$

$d\left( x_{0},Y\right) $=$\left\vert g\left( x_{0}-y_{n}\right) \right\vert
\leq \left\Vert g\right\Vert \left\Vert x_{0}-y_{n}\right\Vert ,$let $%
n\rightarrow \infty $

$\implies d\left( x_{0},Y\right) \leq \left\Vert g\right\Vert d\left(
x_{0},Y\right) \implies \left\Vert g\right\Vert \geq 1.\implies \left\Vert
g\right\Vert =1.$

\bigskip (Alternative approach(the subscript is omitted in some places):

$\left\Vert f\right\Vert =\sup \frac{\left\vert f\left( m+tx_{0}\right)
\right\vert }{\left\Vert m+tx_{0}\right\Vert }=\sup \frac{\left\vert f\left(
m^{\prime }+x_{0}\right) \right\vert }{\left\Vert m^{\prime
}+x_{0}\right\Vert }=\frac{\left\vert f\left( x_{0}\right) \right\vert }{%
\inf \left\Vert m^{\prime }+x_{0}\right\Vert }=\frac{\left\vert f\left(
x_{0}\right) \right\vert }{dist\left( x,M\right) }.$)

Then by Hahn Banach Extension Thm, 存在一个X上%
的有界线性泛函f,使得f
is the extension of g$\implies f\left( x_{0}\right) =g\left( x_{0}\right)
=d\left( x_{0},Y\right) ,$

and since $g$ is linear continuous$\implies \left\Vert f\right\Vert
=\left\Vert g\right\Vert =1.$

6 设E为n.l.s X的一个子集,$y\in X,$%
记$X_{1}=spanE,$则$y\in \bar{X}_{1}\iff \forall $linear
continuous functional vanishes on X$_{1},$we have $f\left( y\right) =0.$

证明: $\implies $ obviously $\Longleftarrow $反证%
法,若$y\notin \bar{X}_{1}\implies d\left( y,X_{1}\right) >0,$%
由5的结论存在一个X上%
的有界线性泛函f,使得f$%
\left( Y\right) =0$ but $f\left( y\right) =d\left( x_{0},Y\right) \neq 0.$%
矛盾.

7设X是有界实数列全体,%
按普通线性运算构成%
一个线性空间,试证明%
存在f为X上的线性泛函%
使得对任意$x$=$\left( \alpha _{n}\right) \in
X. $

均有$\underset{n->\infty }{\lim }\inf \alpha _{n}\leq f\left(
x\right) \leq \underset{n->\infty }{\lim }\sup \alpha _{n}.$

(个人观点:此题为存在%
性证明,构造似乎很困%
难)

Solution:\bigskip 取$p\left( x\right) =\underset{n->\infty }{\lim }%
\sup \alpha _{n}$作为X上的次线性%
泛函,取X$_{1}=\left\{ 0\right\} ,$定义$%
f_{1}:f_{1}\left( 0\right) =0.$

$f_{1}$是X$_{1}$上的线性泛函,f$%
_{1}\left( 0\right) \leq p\left( 0\right) .$

由H-B延拓定理,存在一个X%
上的线性泛函使得f$\left(
x\right) \leq \underset{n->\infty }{\lim }\sup \alpha _{n}$

-f$\left( x\right) $=f$\left( -x\right) \leq \underset{n->\infty }{\lim }%
\sup \left( -\alpha _{n}\right) =-\underset{n->\infty }{\lim }\inf \alpha
_{n}\implies $

f$\left( x\right) \geq \underset{n->\infty }{\lim }\inf \alpha _{n}$

8.设X$_{1}$是实的赋范线性%
空间X中含有内点的凸%
集,$x_{0}\notin X_{1},$证明存在$f\in X^{\ast }$%
使得

$\sup \left\{ f\left( x\right) |x\in X_{1}\right\} \leq 1\leq f\left(
x_{0}\right) .$

证明,若0为X$_{1}$的一个内%
点,我们可以定义一个%
Minkowski泛函on X: $p\left( x\right) =\inf \left\{ \alpha >0|%
\frac{x}{\alpha }\in X_{1}\right\} ,($参见课本Lemma
1.2 on page 6,但此题不需开集%
的条件)

\bigskip 可以证明$p\left( \lambda x\right) =\lambda
p\left( x\right) ,\lambda >0,$and if $p\left( x\right) <1,$then we can find $%
\alpha <1,s.t.$

$\frac{x}{\alpha }\in X_{1},$由$X_{1}$的凸性$%
\implies \alpha \left( \frac{x}{\alpha }\right) +\left( 1-\alpha \right)
0\in X_{1}\implies x\in X_{1}.$

$\forall p\left( x\right) ,p\left( y\right) $,$\epsilon >0,$since $p\left( 
\frac{x}{p\left( x\right) +\epsilon }\right) =\frac{p\left( x\right) }{%
p\left( x\right) +\epsilon }<1\implies \frac{x}{p\left( x\right) +\epsilon }%
\in X_{1},$

similarly$\frac{y}{p\left( y\right) +\epsilon }\in X_{1},$由凸%
性$\implies t\frac{x}{p\left( x\right) +\epsilon }+\left( 1-t\right) 
\frac{y}{p\left( y\right) +\epsilon }=\frac{x+y}{p\left( x\right) +p\left(
y\right) +2\epsilon }\in X_{1}$

$($可以反解出$t$=$\frac{p\left( x\right)
+\epsilon }{p\left( x\right) +p\left( y\right) +2\epsilon }\in \left(
0,1\right) )\implies \frac{x+y}{p\left( x\right) +p\left( y\right)
+2\epsilon }\in X_{1}\implies p\left( \frac{x+y}{p\left( x\right) +p\left(
y\right) +2\epsilon }\right) <1$

(if $\frac{z}{1}\in X_{1},$by the definition of the gauge function, $p\left(
z\right) <1)$

$p\left( x+y\right) \leq p\left( x\right) +p\left( y\right) +2\epsilon .$%
由$\epsilon $任意性知$p\left( x+y\right) \leq
p\left( x\right) +p\left( y\right) $

Also we have $x_{0}\notin X_{1}\implies p\left( x_{0}\right) \geq 1,$

\bigskip 然后先在$x_{0}\NEG{R}$上定%
义一个线性泛函(事实%
上可定义f$\left( tx_{0}\right) =tp\left(
x_{0}\right) ,$证明$f$在此一维子%
空间上被p控制需要分t%
的正负讨论,参见Lemma 1.3 on page 6
of textbook.$)$

If $t>0,f\left( tx_{0}\right) =tp\left( x_{0}\right) =p\left( tx_{0}\right)
; $if $t\leq 0,f\left( tx_{0}\right) \leq 0\leq p\left( tx_{0}\right) .$

再用H-B THM 延拓到X上,此时%
仍有$f(x)\leq p\left( x\right) ,\forall x\in X$.

Since $0$ is the interior point of $X_{1},\implies \exists r>0,s.t.B\left(
0,r\right) \subset X_{1},\forall x\in X,\frac{r}{2\left\Vert x\right\Vert }%
x\in B\left( 0,r\right) \implies p\left( x\right) \leq \frac{2\left\Vert
x\right\Vert }{r}$

$\implies f\left( x\right) \leq p\left( x\right) \leq \frac{2\left\Vert
x\right\Vert }{r}\implies f$ is bounded.

\bigskip $\forall x\in X_{1},f\left( x\right) \leq p\left( x\right)
<1\implies \sup \left\{ f\left( x\right) |x\in X_{1}\right\} \leq 1,f\left(
x_{0}\right) =p\left( x_{0}\right) \geq 1$

$\implies \sup \left\{ f\left( x\right) |x\in X_{1}\right\} \leq 1\leq
f\left( x_{0}\right) .$

一般的,若$x^{\ast }$为$X_{1}$的内%
点,只需定义$p\left( x\right) =\inf \left\{
\alpha >0|\frac{x}{\alpha }+x^{\ast }\in X_{1}\right\} $仿照%
上面的推导同样可得%
出要证的结论.

9.设A是n.l.s.X中凸闭集,$x_{0}\notin A,$%
prove that 存在一个X上的有界%
线性泛函使得 $\sup \left\{ f\left(
x\right) |x\in A\right\} $严格小于$f\left(
x_{0}\right) $.

Solution:Let $\alpha =d\left( x_{0},A\right) >0$,$A_{1}=\left\{ x\in
X|\exists y\in A,s.t.\left\Vert x-y\right\Vert \leq \frac{\alpha }{2}%
\right\} $可以证明$A_{1}$是凸集%
.Indeed, $x_{1},x_{2}\in A_{1}\implies \exists y_{1},y_{2}\in A,\left\Vert
x_{i}-y_{i}\right\Vert \leq \frac{\alpha }{2},i=1,2$

$\left\Vert \left[ tx_{1}+\left( 1-t\right) x_{2}\right] -\left[
ty_{1}+\left( 1-t\right) y_{2}\right] \right\Vert =\left\Vert t\left(
x_{1}-y_{1}\right) +\left( 1-t\right) \left( x_{2}-y_{2}\right) \right\Vert $

$\leq t\left\Vert x_{1}-y_{1}\right\Vert +\left( 1-t\right) \left\Vert
x_{2}-y_{2}\right\Vert \leq t\frac{\alpha }{2}+\left( 1-t\right) \frac{%
\alpha }{2}\leq \frac{\alpha }{2}.$

Since $A$ is convex$\implies ty_{1}+\left( 1-t\right) y_{2}\in A\implies
tx_{1}+\left( 1-t\right) x_{2}\in A_{1}.$

$\implies A_{1}$ is convex. $\forall x\in A_{1},\exists y\in
A,s.t.\left\Vert x-y\right\Vert \leq \frac{\alpha }{2}$

$\left\Vert x-x_{0}\right\Vert =\left\Vert x-y+y-x_{0}\right\Vert \geq
\left\Vert y-x_{0}\right\Vert -\left\Vert x-y\right\Vert \geq \frac{\alpha }{%
2}$

$\bigskip \implies d\left( x_{0},A_{1}\right) \geq \frac{\alpha }{2}%
,x_{0}\notin A_{1}.$

By the defintion $A_{1}$ has interior points.

由上题结论,对$A_{1},$A中任%
一点均为A$_{1}$内点,为方%
便设0$\in A,$则0为A$_{1}$内点,则%
存在$f\in X^{\ast }$使得$\forall x\in A_{1},f\left(
x\right) \leq p\left( x\right) \leq 1,f\left( x_{0}\right) =p\left(
x_{0}\right) \geq 1.p\left( x\right) \leq 1$成立是因%
为$1$是$inf\left\{ \alpha _{1}>0,\alpha _{1}^{-1}x\in
A_{1}\right\} \leq 1$

为证$\bigskip \sup \left\{ f\left( x\right) |x\in A\right\}
<f\left( x_{0}\right) ,$只需说明$p\left(
x_{0}\right) >1.$

$\bigskip $因为我们设$0\in A,\alpha <d\left(
x_{0},0\right) =\left\Vert x_{0}\right\Vert \implies 1-\frac{\alpha }{%
4\left\Vert x_{0}\right\Vert }>0$

$\left( 1-\frac{\alpha }{4\left\Vert x_{0}\right\Vert }\right) x_{0}\notin
A_{1},$since $d\left( x_{0},A_{1}\right) \geq \frac{\alpha }{2}$

结合$A_{1}$为凸集,$p\left( x_{0}\right) $%
为某区间(右端点为+$\infty )$%
下确界$\implies p\left( x_{0}\right) \geq \left( 1-\frac{%
\alpha }{4\left\Vert x_{0}\right\Vert }\right) ^{-1}>1.$

(助教的解法中利用$A_{1}$%
是闭集的性质证$p\left(
x_{0}\right) >1$似乎不正确,因为$%
A_{1}$是闭集证不出来,

但如果$X$ is a Banach space,可以证%
明$A_{1}$是闭集,从而助教%
的解法行得通.)

10 Let X be a n.l.s. on field K. Show that $X^{\ast }$ has infinite
dimension if $X$ has infinite dimension.

Solution: Let $0\neq x_{1}\in X,$由题2知,$\exists f_{1}\in
X^{\ast },s.t.f_{1}\left( x_{1}\right) =\left\Vert x_{1}\right\Vert \neq 0,$

取$X_{1}=\left\{ \alpha x_{1},\alpha \in K\right\} ,$then $X_{1}$ is
closed linear space since $X_{1}$ has finite dimension.Since X has infinite
dimension$\implies \exists x_{2}\in X,s.t.x_{2}\notin X_{1}$

由题5知$\implies \exists f_{2}\in X^{\ast
},s.t.f_{2}\left( X_{1}\right) =0,f_{2}\left( x_{2}\right) =d\left(
x_{2},X_{1}\right) >0.$

\bigskip Suppose we proceed to $k$th step, and have $%
X_{i},x_{i},f_{i},i=1,2,..k,$

then we let $X_{k+1}=Span\left\{ x_{1},..x_{k}\right\} ,X_{k+1}$ is linear
closed$\implies \exists x_{k+1}\in X,s.t.x_{k+1}\notin X_{k}$

$\implies \exists f_{k+1}\in X^{\ast },s.t.f\left( X_{k}\right) =0,f\left(
x_{k+1}\right) =d\left( x_{k+1},X_{1}\right) >0.$

By this method we can proceed continuously and get $\left\{
f_{1},..f_{n},..\right\} $

For any finite $\left\{ f_{1},..f_{n}\right\} ,$suppose $\beta
_{1}f_{1}+..\beta _{n}f_{n}=0.$Acting this expression on $x_{1}$ gives $%
\beta _{1}f_{1}\left( x_{1}\right) =0,$since $f_{1}\left( x_{1}\right)
>0\implies \beta _{1}=0.$Similary we find all $\beta _{i}$ vanish

$\implies $ $\left\{ f_{1},..f_{n},..\right\} $ are linearly indepedent$%
\implies X^{\ast }$ has infinite dimension



\end{document}
