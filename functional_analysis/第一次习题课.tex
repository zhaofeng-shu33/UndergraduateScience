
\documentclass{article}
%%%%%%%%%%%%%%%%%%%%%%%%%%%%%%%%%%%%%%%%%%%%%%%%%%%%%%%%%%%%%%%%%%%%%%%%%%%%%%%%%%%%%%%%%%%%%%%%%%%%%%%%%%%%%%%%%%%%%%%%%%%%%%%%%%%%%%%%%%%%%%%%%%%%%%%%%%%%%%%%%%%%%%%%%%%%%%%%%%%%%%%%%%%%%%%%%%%%%%%%%%%%%%%%%%%%%%%%%%%%%%%%%%%%%%%%%%%%%%%%%%%%%%%%%%%%
\usepackage{amsmath}

\setcounter{MaxMatrixCols}{10}
%TCIDATA{OutputFilter=LATEX.DLL}
%TCIDATA{Version=5.00.0.2552}
%TCIDATA{<META NAME="SaveForMode" CONTENT="1">}
%TCIDATA{Created=Sunday, October 25, 2015 11:39:15}
%TCIDATA{LastRevised=Wednesday, May 18, 2016 21:36:08}
%TCIDATA{<META NAME="GraphicsSave" CONTENT="32">}
%TCIDATA{<META NAME="DocumentShell" CONTENT="Scientific Notebook\Blank Document">}
%TCIDATA{CSTFile=Math with theorems suppressed.cst}
%TCIDATA{PageSetup=72,72,72,72,0}
%TCIDATA{AllPages=
%F=36,\PARA{038<p type="texpara" tag="Body Text" >\hfill \thepage}
%}


\newtheorem{theorem}{Theorem}
\newtheorem{acknowledgement}[theorem]{Acknowledgement}
\newtheorem{algorithm}[theorem]{Algorithm}
\newtheorem{axiom}[theorem]{Axiom}
\newtheorem{case}[theorem]{Case}
\newtheorem{claim}[theorem]{Claim}
\newtheorem{conclusion}[theorem]{Conclusion}
\newtheorem{condition}[theorem]{Condition}
\newtheorem{conjecture}[theorem]{Conjecture}
\newtheorem{corollary}[theorem]{Corollary}
\newtheorem{criterion}[theorem]{Criterion}
\newtheorem{definition}[theorem]{Definition}
\newtheorem{example}[theorem]{Example}
\newtheorem{exercise}[theorem]{Exercise}
\newtheorem{lemma}[theorem]{Lemma}
\newtheorem{notation}[theorem]{Notation}
\newtheorem{problem}[theorem]{Problem}
\newtheorem{proposition}[theorem]{Proposition}
\newtheorem{remark}[theorem]{Remark}
\newtheorem{solution}[theorem]{Solution}
\newtheorem{summary}[theorem]{Summary}
\newenvironment{proof}[1][Proof]{\noindent\textbf{#1.} }{\ \rule{0.5em}{0.5em}}


\begin{document}


\bigskip\ 
\U{6cdb}\U{51fd}\U{5206}\U{6790}$\left(
1\right) $\U{7b2c}\U{4e00}\U{6b21}\U{4e60}\U{9898}\U{8bfe} \U{9898}\U{76ee}%
\U{4e0e}\U{89e3}\U{7b54}

1. \U{8bbe}n.l.s X\U{4e0a}\U{7684}\U{4e00}\U{4e2a}\U{6b21}\U{53ef}\U{52a0}%
\U{6cdb}\U{51fd}$f:f(x+y)\leq f\left( x\right) +f\left( y\right) \forall
x,y\in X.$\U{5df2}\U{77e5}$f$\U{5728}

$\left\{ x\in X|\left\Vert x\right\Vert =r\right\} $\U{5916}\U{975e}\U{8d1f},%
\U{6c42}\U{8bc1}$f\left( x\right) \geq 0,\forall x\in X.$

\U{89e3}: $\forall x\in X,$if $x=0,f\left( 0\right) \leq f\left( 0\right)
+f\left( 0\right) \implies f\left( 0\right) \geq 0,$

if $x\neq 0,$there exists a positive integer n s.t. $nx\notin \left\{ x\in
X|\left\Vert x\right\Vert \leq r\right\} ,\implies f\left( nx\right) \geq 0,$

$f(x+y)\leq f\left( x\right) +f\left( y\right) \implies f\left( nx\right)
\leq nf\left( x\right) \implies f\left( x\right) \geq 0.$

2. \U{89c1}\U{8bfe}\U{672c}\U{7b2c}4\U{9875}Corollary 1.4

3. 2\U{7684}Corollary,show that for given $x,y\in X,$if $\forall f\in
X^{\ast },f\left( x\right) =f\left( y\right) ,$then $x=y.$

Solution: \U{7528}\U{53cd}\U{8bc1}\U{6cd5}.Suppose $x\neq y,$ then by 2
there exists $\tilde{f}\in X^{\ast },s.t.\left\Vert \tilde{f}\right\Vert =1$
and $\tilde{f}\left( x-y\right) =\left\Vert x-y\right\Vert \neq 0,$\U{4e0e}%
\U{6761}\U{4ef6}\U{4e2d} $\forall f\in X^{\ast },f\left( x\right) =f\left(
y\right) $\U{77db}\U{76fe}.

4.\U{8bbe}$\NEG{R}^{2}$\U{4e0a}\U{7684}\U{7ebf}\U{6027}\U{6cdb}\U{51fd}$%
f\left( x\right) =\alpha \xi _{1}+\beta \xi _{2},\forall x=\left( \xi
_{1},\xi _{2}\right) \in \NEG{R}^{2}$,\U{8bc1}\U{660e}\U{53ef}\U{5c06}%
\U{5b83}\U{4fdd}\U{8303}\U{5ef6}\U{62d3}\U{5230}$\NEG{R}^{3}$\U{4e0a}.

\U{8bc1}\U{660e}: \U{5229}\U{7528}Cauchy-Inequality \U{53ef}\U{6c42}\U{51fa}$%
\left\vert f\left( x\right) \right\vert \leq \sqrt{\xi _{1}^{2}+\xi _{2}^{2}}%
\sqrt{\alpha ^{2}+\beta ^{2}},$\U{82e5}$\NEG{R}^{2}$\U{53d6}Euclid norm$%
\implies \left\Vert f\right\Vert \leq \sqrt{\alpha ^{2}+\beta ^{2}}.$\U{518d}%
\U{53d6}$x=\left( \frac{\alpha }{\sqrt{\alpha ^{2}+\beta ^{2}}},\frac{\beta 
}{\sqrt{\alpha ^{2}+\beta ^{2}}}\right) ,$

$\implies \left\Vert f\right\Vert \geq \left\vert f\left( x\right)
\right\vert =\sqrt{\alpha ^{2}+\beta ^{2}}\implies \left\Vert f\right\Vert =%
\sqrt{\alpha ^{2}+\beta ^{2}},$

\U{5728}$\NEG{R}^{3}$\U{4e0a}\U{5b9a}\U{4e49}$g:\NEG{R}^{3}->\NEG{R}$ $%
g\left( x\right) =\alpha \eta _{1}+\beta \eta _{2},\forall x=\left( \eta
_{1},\eta _{2},\eta _{3}\right) \in \NEG{R}^{3}.$\U{53ef}\U{8bc1}$g$ \U{662f}%
$f$\U{7684}\U{5ef6}\U{62d3}\U{4e14}\U{7531}\U{548c}\U{4e0a}\U{9762}\U{540c}%
\U{6837}\U{7684}\U{6f14}\U{7ece}$\left\Vert g\right\Vert =\sqrt{\alpha
^{2}+\beta ^{2}}.\implies g$ \U{662f}$f$ \U{7684}\U{4fdd}\U{8303}\U{5ef6}%
\U{62d3}.

(\U{4e2a}\U{4eba}\U{89c2}\U{70b9}:2\U{7ef4}\U{7a7a}\U{95f4}\U{662f}3\U{7ef4}%
\U{7a7a}\U{95f4}\U{7684}\U{4e00}\U{4e2a}\U{5b50}\U{7a7a}\U{95f4},\U{8c8c}%
\U{4f3c}\U{505a}\U{4e86}\U{4e00}\U{4e2a}\U{6295}\U{5f71}\U{53d8}\U{6362})

\bigskip

5.\U{8bbe}$Y$\U{4e3a}n.l.s$\qquad X\U{7684} \U{4e00} \U{4e2a} \U{5b50} 
\U{7a7a} \U{95f4} ,x_{0}\in X,d\left( x_{0},Y\right) >0.$\U{8bd5}\U{8bc1}%
\U{660e}\U{5b58}\U{5728}\U{4e00}\U{4e2a}X\U{4e0a}\U{7684}\U{6709}\U{754c}%
\U{7ebf}\U{6027}\U{6cdb}\U{51fd}f,\U{4f7f}\U{5f97}f$\left( Y\right)
=0,f\left( x_{0}\right) =d\left( x_{0},Y\right) ,$\U{4e14}$\left\Vert
f\right\Vert =1.$

\bigskip Solution: (refer to lecture note after the professor \U{4ecb}%
\U{7ecd}\U{4fdd}\U{8303}\U{5ef6}\U{62d3}\U{7684}\U{6982}\U{5ff5})

\U{5148}\U{5728}$x_{0}$\U{548c}$Y$\U{5f20}\U{6210}\U{7684}\U{5b50}\U{7a7a}%
\U{95f4}\U{4e0a}\U{5b9a}\U{4e49}\U{4e00}\U{4e2a}\U{6709}\U{754c}\U{7ebf}%
\U{6027}\U{6cdb}\U{51fd},

$g:t\left\{ x_{0}\right\} +Y\rightarrow td\left( x_{0},Y\right) .$\U{7531}%
\U{4e8e}$x_{0}\notin \bar{Y}$\U{53ef}\U{8bc1}\U{6b64}\U{5b9a}\U{4e49}\U{5408}%
\U{7406},$g$ \U{7684}\U{7ebf}\U{6027}\U{6027}\U{8d28}\U{5bb9}\U{6613}\U{9a8c}%
\U{8bc1}.

\bigskip $\forall x\in t\left\{ x_{0}\right\} +Y,$ there exists unqiue $t\in
R$ and $y\in Y,s.t.x=tx_{0}+y$

$\left\vert g\left( x\right) \right\vert =\left\vert t\right\vert d\left(
x_{0},Y\right) ,$if $t\neq 0,d\left( x_{0},Y\right) \leq \left\vert x_{0}+%
\frac{y}{t}\right\vert \implies \left\vert g\left( x\right) \right\vert \leq
\left\vert t\right\vert \left\vert x_{0}+\frac{y}{t}\right\vert $

$\leq \left\vert tx_{0}+y\right\vert =\left\vert x\right\vert \implies
\left\Vert g\right\Vert \leq 1,g$ is bounded. Further, by the definition of

$d\left( x_{0},Y\right) ,\exists \left\{ y_{n}\right\} \subset
Y,s.t.\left\Vert x_{0}-y_{n}\right\Vert =d\left( x_{0},y_{n}\right)
\rightarrow d\left( x_{0},Y\right) ,$

$d\left( x_{0},Y\right) $=$\left\vert g\left( x_{0}-y_{n}\right) \right\vert
\leq \left\Vert g\right\Vert \left\Vert x_{0}-y_{n}\right\Vert ,$let $%
n\rightarrow \infty $

$\implies d\left( x_{0},Y\right) \leq \left\Vert g\right\Vert d\left(
x_{0},Y\right) \implies \left\Vert g\right\Vert \geq 1.\implies \left\Vert
g\right\Vert =1.$

\bigskip (Alternative approach(the subscript is omitted in some places):

$\left\Vert f\right\Vert =\sup \frac{\left\vert f\left( m+tx_{0}\right)
\right\vert }{\left\Vert m+tx_{0}\right\Vert }=\sup \frac{\left\vert f\left(
m^{\prime }+x_{0}\right) \right\vert }{\left\Vert m^{\prime
}+x_{0}\right\Vert }=\frac{\left\vert f\left( x_{0}\right) \right\vert }{%
\inf \left\Vert m^{\prime }+x_{0}\right\Vert }=\frac{\left\vert f\left(
x_{0}\right) \right\vert }{dist\left( x,M\right) }.$)

Then by Hahn Banach Extension Thm, \U{5b58}\U{5728}\U{4e00}\U{4e2a}X\U{4e0a}%
\U{7684}\U{6709}\U{754c}\U{7ebf}\U{6027}\U{6cdb}\U{51fd}f,\U{4f7f}\U{5f97}f
is the extension of g$\implies f\left( x_{0}\right) =g\left( x_{0}\right)
=d\left( x_{0},Y\right) ,$

and since $g$ is linear continuous$\implies \left\Vert f\right\Vert
=\left\Vert g\right\Vert =1.$

6 \U{8bbe}E\U{4e3a}n.l.s X\U{7684}\U{4e00}\U{4e2a}\U{5b50}\U{96c6},$y\in X,$%
\U{8bb0}$X_{1}=spanE,$\U{5219}$y\in \bar{X}_{1}\iff \forall $linear
continuous functional vanishes on X$_{1},$we have $f\left( y\right) =0.$

\U{8bc1}\U{660e}: $\implies $ obviously $\Longleftarrow $\U{53cd}\U{8bc1}%
\U{6cd5},\U{82e5}$y\notin \bar{X}_{1}\implies d\left( y,X_{1}\right) >0,$%
\U{7531}5\U{7684}\U{7ed3}\U{8bba}\U{5b58}\U{5728}\U{4e00}\U{4e2a}X\U{4e0a}%
\U{7684}\U{6709}\U{754c}\U{7ebf}\U{6027}\U{6cdb}\U{51fd}f,\U{4f7f}\U{5f97}f$%
\left( Y\right) =0$ but $f\left( y\right) =d\left( x_{0},Y\right) \neq 0.$%
\U{77db}\U{76fe}.

7\U{8bbe}X\U{662f}\U{6709}\U{754c}\U{5b9e}\U{6570}\U{5217}\U{5168}\U{4f53},%
\U{6309}\U{666e}\U{901a}\U{7ebf}\U{6027}\U{8fd0}\U{7b97}\U{6784}\U{6210}%
\U{4e00}\U{4e2a}\U{7ebf}\U{6027}\U{7a7a}\U{95f4},\U{8bd5}\U{8bc1}\U{660e}%
\U{5b58}\U{5728}f\U{4e3a}X\U{4e0a}\U{7684}\U{7ebf}\U{6027}\U{6cdb}\U{51fd}%
\U{4f7f}\U{5f97}\U{5bf9}\U{4efb}\U{610f}$x$=$\left( \alpha _{n}\right) \in
X. $

\U{5747}\U{6709}$\underset{n->\infty }{\lim }\inf \alpha _{n}\leq f\left(
x\right) \leq \underset{n->\infty }{\lim }\sup \alpha _{n}.$

(\U{4e2a}\U{4eba}\U{89c2}\U{70b9}:\U{6b64}\U{9898}\U{4e3a}\U{5b58}\U{5728}%
\U{6027}\U{8bc1}\U{660e},\U{6784}\U{9020}\U{4f3c}\U{4e4e}\U{5f88}\U{56f0}%
\U{96be})

Solution:\bigskip \U{53d6}$p\left( x\right) =\underset{n->\infty }{\lim }%
\sup \alpha _{n}$\U{4f5c}\U{4e3a}X\U{4e0a}\U{7684}\U{6b21}\U{7ebf}\U{6027}%
\U{6cdb}\U{51fd},\U{53d6}X$_{1}=\left\{ 0\right\} ,$\U{5b9a}\U{4e49}$%
f_{1}:f_{1}\left( 0\right) =0.$

$f_{1}$\U{662f}X$_{1}$\U{4e0a}\U{7684}\U{7ebf}\U{6027}\U{6cdb}\U{51fd},f$%
_{1}\left( 0\right) \leq p\left( 0\right) .$

\U{7531}H-B\U{5ef6}\U{62d3}\U{5b9a}\U{7406},\U{5b58}\U{5728}\U{4e00}\U{4e2a}X%
\U{4e0a}\U{7684}\U{7ebf}\U{6027}\U{6cdb}\U{51fd}\U{4f7f}\U{5f97}f$\left(
x\right) \leq \underset{n->\infty }{\lim }\sup \alpha _{n}$

-f$\left( x\right) $=f$\left( -x\right) \leq \underset{n->\infty }{\lim }%
\sup \left( -\alpha _{n}\right) =-\underset{n->\infty }{\lim }\inf \alpha
_{n}\implies $

f$\left( x\right) \geq \underset{n->\infty }{\lim }\inf \alpha _{n}$

8.\U{8bbe}X$_{1}$\U{662f}\U{5b9e}\U{7684}\U{8d4b}\U{8303}\U{7ebf}\U{6027}%
\U{7a7a}\U{95f4}X\U{4e2d}\U{542b}\U{6709}\U{5185}\U{70b9}\U{7684}\U{51f8}%
\U{96c6},$x_{0}\notin X_{1},$\U{8bc1}\U{660e}\U{5b58}\U{5728}$f\in X^{\ast }$%
\U{4f7f}\U{5f97}

$\sup \left\{ f\left( x\right) |x\in X_{1}\right\} \leq 1\leq f\left(
x_{0}\right) .$

\U{8bc1}\U{660e},\U{82e5}0\U{4e3a}X$_{1}$\U{7684}\U{4e00}\U{4e2a}\U{5185}%
\U{70b9},\U{6211}\U{4eec}\U{53ef}\U{4ee5}\U{5b9a}\U{4e49}\U{4e00}\U{4e2a}%
Minkowski\U{6cdb}\U{51fd}on X: $p\left( x\right) =\inf \left\{ \alpha >0|%
\frac{x}{\alpha }\in X_{1}\right\} ,($\U{53c2}\U{89c1}\U{8bfe}\U{672c}Lemma
1.2 on page 6,\U{4f46}\U{6b64}\U{9898}\U{4e0d}\U{9700}\U{5f00}\U{96c6}%
\U{7684}\U{6761}\U{4ef6})

\bigskip \U{53ef}\U{4ee5}\U{8bc1}\U{660e}$p\left( \lambda x\right) =\lambda
p\left( x\right) ,\lambda >0,$and if $p\left( x\right) <1,$then we can find $%
\alpha <1,s.t.$

$\frac{x}{\alpha }\in X_{1},$\U{7531}$X_{1}$\U{7684}\U{51f8}\U{6027}$%
\implies \alpha \left( \frac{x}{\alpha }\right) +\left( 1-\alpha \right)
0\in X_{1}\implies x\in X_{1}.$

$\forall p\left( x\right) ,p\left( y\right) $,$\epsilon >0,$since $p\left( 
\frac{x}{p\left( x\right) +\epsilon }\right) =\frac{p\left( x\right) }{%
p\left( x\right) +\epsilon }<1\implies \frac{x}{p\left( x\right) +\epsilon }%
\in X_{1},$

similarly$\frac{y}{p\left( y\right) +\epsilon }\in X_{1},$\U{7531}\U{51f8}%
\U{6027}$\implies t\frac{x}{p\left( x\right) +\epsilon }+\left( 1-t\right) 
\frac{y}{p\left( y\right) +\epsilon }=\frac{x+y}{p\left( x\right) +p\left(
y\right) +2\epsilon }\in X_{1}$

$($\U{53ef}\U{4ee5}\U{53cd}\U{89e3}\U{51fa}$t$=$\frac{p\left( x\right)
+\epsilon }{p\left( x\right) +p\left( y\right) +2\epsilon }\in \left(
0,1\right) )\implies \frac{x+y}{p\left( x\right) +p\left( y\right)
+2\epsilon }\in X_{1}\implies p\left( \frac{x+y}{p\left( x\right) +p\left(
y\right) +2\epsilon }\right) <1$

(if $\frac{z}{1}\in X_{1},$by the definition of the gauge function, $p\left(
z\right) <1)$

$p\left( x+y\right) \leq p\left( x\right) +p\left( y\right) +2\epsilon .$%
\U{7531}$\epsilon $\U{4efb}\U{610f}\U{6027}\U{77e5}$p\left( x+y\right) \leq
p\left( x\right) +p\left( y\right) $

Also we have $x_{0}\notin X_{1}\implies p\left( x_{0}\right) \geq 1,$

\bigskip \U{7136}\U{540e}\U{5148}\U{5728}$x_{0}\NEG{R}$\U{4e0a}\U{5b9a}%
\U{4e49}\U{4e00}\U{4e2a}\U{7ebf}\U{6027}\U{6cdb}\U{51fd}(\U{4e8b}\U{5b9e}%
\U{4e0a}\U{53ef}\U{5b9a}\U{4e49}f$\left( tx_{0}\right) =tp\left(
x_{0}\right) ,$\U{8bc1}\U{660e}$f$\U{5728}\U{6b64}\U{4e00}\U{7ef4}\U{5b50}%
\U{7a7a}\U{95f4}\U{4e0a}\U{88ab}p\U{63a7}\U{5236}\U{9700}\U{8981}\U{5206}t%
\U{7684}\U{6b63}\U{8d1f}\U{8ba8}\U{8bba},\U{53c2}\U{89c1}Lemma 1.3 on page 6
of textbook.$)$

If $t>0,f\left( tx_{0}\right) =tp\left( x_{0}\right) =p\left( tx_{0}\right)
; $if $t\leq 0,f\left( tx_{0}\right) \leq 0\leq p\left( tx_{0}\right) .$

\U{518d}\U{7528}H-B THM \U{5ef6}\U{62d3}\U{5230}X\U{4e0a},\U{6b64}\U{65f6}%
\U{4ecd}\U{6709}$f(x)\leq p\left( x\right) ,\forall x\in X$.

Since $0$ is the interior point of $X_{1},\implies \exists r>0,s.t.B\left(
0,r\right) \subset X_{1},\forall x\in X,\frac{r}{2\left\Vert x\right\Vert }%
x\in B\left( 0,r\right) \implies p\left( x\right) \leq \frac{2\left\Vert
x\right\Vert }{r}$

$\implies f\left( x\right) \leq p\left( x\right) \leq \frac{2\left\Vert
x\right\Vert }{r}\implies f$ is bounded.

\bigskip $\forall x\in X_{1},f\left( x\right) \leq p\left( x\right)
<1\implies \sup \left\{ f\left( x\right) |x\in X_{1}\right\} \leq 1,f\left(
x_{0}\right) =p\left( x_{0}\right) \geq 1$

$\implies \sup \left\{ f\left( x\right) |x\in X_{1}\right\} \leq 1\leq
f\left( x_{0}\right) .$

\U{4e00}\U{822c}\U{7684},\U{82e5}$x^{\ast }$\U{4e3a}$X_{1}$\U{7684}\U{5185}%
\U{70b9},\U{53ea}\U{9700}\U{5b9a}\U{4e49}$p\left( x\right) =\inf \left\{
\alpha >0|\frac{x}{\alpha }+x^{\ast }\in X_{1}\right\} $\U{4eff}\U{7167}%
\U{4e0a}\U{9762}\U{7684}\U{63a8}\U{5bfc}\U{540c}\U{6837}\U{53ef}\U{5f97}%
\U{51fa}\U{8981}\U{8bc1}\U{7684}\U{7ed3}\U{8bba}.

9.\U{8bbe}A\U{662f}n.l.s.X\U{4e2d}\U{51f8}\U{95ed}\U{96c6},$x_{0}\notin A,$%
prove that \U{5b58}\U{5728}\U{4e00}\U{4e2a}X\U{4e0a}\U{7684}\U{6709}\U{754c}%
\U{7ebf}\U{6027}\U{6cdb}\U{51fd}\U{4f7f}\U{5f97} $\sup \left\{ f\left(
x\right) |x\in A\right\} $\U{4e25}\U{683c}\U{5c0f}\U{4e8e}$f\left(
x_{0}\right) $.

Solution:Let $\alpha =d\left( x_{0},A\right) >0$,$A_{1}=\left\{ x\in
X|\exists y\in A,s.t.\left\Vert x-y\right\Vert \leq \frac{\alpha }{2}%
\right\} $\U{53ef}\U{4ee5}\U{8bc1}\U{660e}$A_{1}$\U{662f}\U{51f8}\U{96c6}%
.Indeed, $x_{1},x_{2}\in A_{1}\implies \exists y_{1},y_{2}\in A,\left\Vert
x_{i}-y_{i}\right\Vert \leq \frac{\alpha }{2},i=1,2$

$\left\Vert \left[ tx_{1}+\left( 1-t\right) x_{2}\right] -\left[
ty_{1}+\left( 1-t\right) y_{2}\right] \right\Vert =\left\Vert t\left(
x_{1}-y_{1}\right) +\left( 1-t\right) \left( x_{2}-y_{2}\right) \right\Vert $

$\leq t\left\Vert x_{1}-y_{1}\right\Vert +\left( 1-t\right) \left\Vert
x_{2}-y_{2}\right\Vert \leq t\frac{\alpha }{2}+\left( 1-t\right) \frac{%
\alpha }{2}\leq \frac{\alpha }{2}.$

Since $A$ is convex$\implies ty_{1}+\left( 1-t\right) y_{2}\in A\implies
tx_{1}+\left( 1-t\right) x_{2}\in A_{1}.$

$\implies A_{1}$ is convex. $\forall x\in A_{1},\exists y\in
A,s.t.\left\Vert x-y\right\Vert \leq \frac{\alpha }{2}$

$\left\Vert x-x_{0}\right\Vert =\left\Vert x-y+y-x_{0}\right\Vert \geq
\left\Vert y-x_{0}\right\Vert -\left\Vert x-y\right\Vert \geq \frac{\alpha }{%
2}$

$\bigskip \implies d\left( x_{0},A_{1}\right) \geq \frac{\alpha }{2}%
,x_{0}\notin A_{1}.$

By the defintion $A_{1}$ has interior points.

\U{7531}\U{4e0a}\U{9898}\U{7ed3}\U{8bba},\U{5bf9}$A_{1},$A\U{4e2d}\U{4efb}%
\U{4e00}\U{70b9}\U{5747}\U{4e3a}A$_{1}$\U{5185}\U{70b9},\U{4e3a}\U{65b9}%
\U{4fbf}\U{8bbe}0$\in A,$\U{5219}0\U{4e3a}A$_{1}$\U{5185}\U{70b9},\U{5219}%
\U{5b58}\U{5728}$f\in X^{\ast }$\U{4f7f}\U{5f97}$\forall x\in A_{1},f\left(
x\right) \leq p\left( x\right) \leq 1,f\left( x_{0}\right) =p\left(
x_{0}\right) \geq 1.p\left( x\right) \leq 1$\U{6210}\U{7acb}\U{662f}\U{56e0}%
\U{4e3a}$1$\U{662f}$inf\left\{ \alpha _{1}>0,\alpha _{1}^{-1}x\in
A_{1}\right\} \leq 1$

\U{4e3a}\U{8bc1}$\bigskip \sup \left\{ f\left( x\right) |x\in A\right\}
<f\left( x_{0}\right) ,$\U{53ea}\U{9700}\U{8bf4}\U{660e}$p\left(
x_{0}\right) >1.$

$\bigskip $\U{56e0}\U{4e3a}\U{6211}\U{4eec}\U{8bbe}$0\in A,\alpha <d\left(
x_{0},0\right) =\left\Vert x_{0}\right\Vert \implies 1-\frac{\alpha }{%
4\left\Vert x_{0}\right\Vert }>0$

$\left( 1-\frac{\alpha }{4\left\Vert x_{0}\right\Vert }\right) x_{0}\notin
A_{1},$since $d\left( x_{0},A_{1}\right) \geq \frac{\alpha }{2}$

\U{7ed3}\U{5408}$A_{1}$\U{4e3a}\U{51f8}\U{96c6},$p\left( x_{0}\right) $%
\U{4e3a}\U{67d0}\U{533a}\U{95f4}(\U{53f3}\U{7aef}\U{70b9}\U{4e3a}+$\infty )$%
\U{4e0b}\U{786e}\U{754c}$\implies p\left( x_{0}\right) \geq \left( 1-\frac{%
\alpha }{4\left\Vert x_{0}\right\Vert }\right) ^{-1}>1.$

(\U{52a9}\U{6559}\U{7684}\U{89e3}\U{6cd5}\U{4e2d}\U{5229}\U{7528}$A_{1}$%
\U{662f}\U{95ed}\U{96c6}\U{7684}\U{6027}\U{8d28}\U{8bc1}$p\left(
x_{0}\right) >1$\U{4f3c}\U{4e4e}\U{4e0d}\U{6b63}\U{786e},\U{56e0}\U{4e3a}$%
A_{1}$\U{662f}\U{95ed}\U{96c6}\U{8bc1}\U{4e0d}\U{51fa}\U{6765},

\U{4f46}\U{5982}\U{679c}$X$ is a Banach space,\U{53ef}\U{4ee5}\U{8bc1}%
\U{660e}$A_{1}$\U{662f}\U{95ed}\U{96c6},\U{4ece}\U{800c}\U{52a9}\U{6559}%
\U{7684}\U{89e3}\U{6cd5}\U{884c}\U{5f97}\U{901a}.)

10 Let X be a n.l.s. on field K. Show that $X^{\ast }$ has infinite
dimension if $X$ has infinite dimension.

Solution: Let $0\neq x_{1}\in X,$\U{7531}\U{9898}2\U{77e5},$\exists f_{1}\in
X^{\ast },s.t.f_{1}\left( x_{1}\right) =\left\Vert x_{1}\right\Vert \neq 0,$

\U{53d6}$X_{1}=\left\{ \alpha x_{1},\alpha \in K\right\} ,$then $X_{1}$ is
closed linear space since $X_{1}$ has finite dimension.Since X has infinite
dimension$\implies \exists x_{2}\in X,s.t.x_{2}\notin X_{1}$

\U{7531}\U{9898}5\U{77e5}$\implies \exists f_{2}\in X^{\ast
},s.t.f_{2}\left( X_{1}\right) =0,f_{2}\left( x_{2}\right) =d\left(
x_{2},X_{1}\right) >0.$

\bigskip Suppose we proceed to $k$th step, and have $%
X_{i},x_{i},f_{i},i=1,2,..k,$

then we let $X_{k+1}=Span\left\{ x_{1},..x_{k}\right\} ,X_{k+1}$ is linear
closed$\implies \exists x_{k+1}\in X,s.t.x_{k+1}\notin X_{k}$

$\implies \exists f_{k+1}\in X^{\ast },s.t.f\left( X_{k}\right) =0,f\left(
x_{k+1}\right) =d\left( x_{k+1},X_{1}\right) >0.$

By this method we can proceed continuously and get $\left\{
f_{1},..f_{n},..\right\} $

For any finite $\left\{ f_{1},..f_{n}\right\} ,$suppose $\beta
_{1}f_{1}+..\beta _{n}f_{n}=0.$Acting this expression on $x_{1}$ gives $%
\beta _{1}f_{1}\left( x_{1}\right) =0,$since $f_{1}\left( x_{1}\right)
>0\implies \beta _{1}=0.$Similary we find all $\beta _{i}$ vanish

$\implies $ $\left\{ f_{1},..f_{n},..\right\} $ are linearly indepedent$%
\implies X^{\ast }$ has infinite dimension


\end{document}
