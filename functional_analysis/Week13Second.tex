
\documentclass{article}
\usepackage{amsmath}

%%%%%%%%%%%%%%%%%%%%%%%%%%%%%%%%%%%%%%%%%%%%%%%%%%%%%%%%%%%%%%%%%%%%%%%%%%%%%%%%%%%%%%%%%%%%%%%%%%%%%%%%%%%%%%%%%%%%%%%%%%%%%%%%%%%%%%%%%%%%%%%%%%%%%%%%%%%%%%%%%%%%%%%%%%%%%%%%%%%%%%%%%%%%%%%%%%%%%%%%%%%%%%%%%%%%%%%%%%%%%%%%%%%%
%TCIDATA{OutputFilter=LATEX.DLL}
%TCIDATA{Version=5.00.0.2552}
%TCIDATA{<META NAME="SaveForMode" CONTENT="1">}
%TCIDATA{Created=Sunday, December 13, 2015 21:42:54}
%TCIDATA{LastRevised=Monday, December 14, 2015 00:13:50}
%TCIDATA{<META NAME="GraphicsSave" CONTENT="32">}
%TCIDATA{<META NAME="DocumentShell" CONTENT="Scientific Notebook\Blank Document">}
%TCIDATA{CSTFile=Math with theorems suppressed.cst}
%TCIDATA{PageSetup=72,72,72,72,0}
%TCIDATA{AllPages=
%F=36,\PARA{038<p type="texpara" tag="Body Text" >\hfill \thepage}
%}


\newtheorem{theorem}{Theorem}
\newtheorem{acknowledgement}[theorem]{Acknowledgement}
\newtheorem{algorithm}[theorem]{Algorithm}
\newtheorem{axiom}[theorem]{Axiom}
\newtheorem{case}[theorem]{Case}
\newtheorem{claim}[theorem]{Claim}
\newtheorem{conclusion}[theorem]{Conclusion}
\newtheorem{condition}[theorem]{Condition}
\newtheorem{conjecture}[theorem]{Conjecture}
\newtheorem{corollary}[theorem]{Corollary}
\newtheorem{criterion}[theorem]{Criterion}
\newtheorem{definition}[theorem]{Definition}
\newtheorem{example}[theorem]{Example}
\newtheorem{exercise}[theorem]{Exercise}
\newtheorem{lemma}[theorem]{Lemma}
\newtheorem{notation}[theorem]{Notation}
\newtheorem{problem}[theorem]{Problem}
\newtheorem{proposition}[theorem]{Proposition}
\newtheorem{remark}[theorem]{Remark}
\newtheorem{solution}[theorem]{Solution}
\newtheorem{summary}[theorem]{Summary}
\newenvironment{proof}[1][Proof]{\noindent\textbf{#1.} }{\ \rule{0.5em}{0.5em}}
\input{tcilatex}

\begin{document}


$\bigskip $

$\bigskip 5.11$

$F^{\prime }\left( a\right) $ should be defined as an element of $%
H,s.t.F\left( a+h\right) =F\left( a\right) +\left( F^{\prime }\left(
a\right) ,h\right) +o\left( \left\Vert h\right\Vert \right) .$

$\bigskip $And we further assume that $u\in K.$

Then $\left( 1-t\right) u+tv\in K,\forall 0<t<1,$since $K$ is convex

$\implies F\left( \left( 1-t\right) u+tv\right) \geq F\left( u\right) $

$\implies F\left( u+t\left( v-u\right) \right) \geq F\left( u\right) $

$\implies F\left( u\right) +t(F^{\prime }\left( u\right) ,v-u)+\left\Vert
v-u\right\Vert o\left( t\right) \geq F\left( u\right) $

$\left( F^{\prime }\left( u\right) ,v-u\right) +\left\Vert v-u\right\Vert
o\left( 1\right) \geq 0$

Let $t\rightarrow 0,\left( ii\right) $ follows.

$\left( ii\right) \implies \left( i\right) F\left( u+t\left( v-u\right)
\right) =F\left( tv+\left( 1-t\right) u\right) ,\forall 0<t<1$

Since $F$ is convex$\qquad F\left( u+t\left( v-u\right) \right) \leq
tF\left( v\right) +\left( 1-t\right) F\left( u\right) $

$\implies F\left( u+t\left( v-u\right) \right) -F\left( u\right) \leq
t\left( F\left( v\right) -F\left( u\right) \right) $

from $\left( 1\right) \frac{F\left( u+t\left( v-u\right) \right) -F\left(
u\right) }{t}>0,$as $t\rightarrow 0\implies F\left( v\right) \geq F\left(
u\right) ,\forall v\in K.$

5.12

We can choose $\left\{ u_{n}\right\} \subset M,\left\Vert u_{n}\right\Vert
=1,s.t.\left( f,u_{n}\right) \rightarrow m.$

Since $M$ is closed and $\left\{ u_{n}\right\} $ is bounded, we can find a
subsequence $\left\{ u_{n_{k}}\right\} $ of $\left\{ u_{n}\right\} ,s.t.$

$u_{n_{k}}$ converges to some $u\in M,$ also $\left\Vert
u_{n_{k}}\right\Vert \rightarrow \left\Vert u\right\Vert \implies \left\Vert
u\right\Vert =1$

$\implies m=\underset{k\rightarrow \infty }{\lim }\left( f,u_{n_{k}}\right)
=\left( f,u\right) .$ $m$ can be achieved by some $u$ satisfying $\left\Vert
u\right\Vert =1$ and $u\in M.$

$\bigskip $

$\forall v\in M,\left\Vert f-v\right\Vert ^{2}=\left\Vert f\right\Vert
^{2}-2\left( f,\frac{v}{\left\Vert v\right\Vert }\right) \left\Vert
v\right\Vert +\left\Vert v\right\Vert ^{2},$

Suppose the direction $\frac{v}{\left\Vert v\right\Vert }$ is fixed and the
length $\left\Vert v\right\Vert $ is changable.

To minimize $\left\Vert f-v\right\Vert ^{2}$ requires $\left\Vert
v\right\Vert =\left( f,\frac{v}{\left\Vert v\right\Vert }\right) ,$ and 

$\left\Vert f-v\right\Vert ^{2}=\left\Vert f\right\Vert ^{2}-\left( f,\frac{v%
}{\left\Vert v\right\Vert }\right) ^{2}$ in such case.

(We know that $\left\vert m\right\vert \geq \left\vert \left( f,\frac{v}{%
\left\Vert v\right\Vert }\right) \right\vert $ by definition Notice that we
can change the size of $u$ in $\inf \left( f,u\right) ,$ which implies that  
$\left\vert m\right\vert =\underset{\underset{\left\Vert u\right\Vert =1}{%
u\in M}}{\sup }\left( f,u\right) .)$

$\implies \left\Vert f-mu\right\Vert ^{2}=\left\Vert f\right\Vert
^{2}-m^{2}\leq \left\Vert f\right\Vert ^{2}-\left( f,\frac{v}{\left\Vert
v\right\Vert }\right) ^{2}$

$\implies mu=d\left( f,M\right) .$ By Thm5.2 $mu$ is unique, and the length
of projection vector is fixed as $m$. Therefore, $m$ is also uniquely
determined.

2. $\left( 1\right) M=E=span\left\{ \varphi _{1},\varphi _{2},\varphi
_{3}\right\} .$ Using Gram-Schmit Orthogonal Method we can assume that $%
\left( \varphi _{i},\varphi _{j}\right) =\delta _{ij}$

If $\left\Vert \mu _{1}\varphi _{1}+\mu _{2}\varphi _{2}+\mu _{3}\varphi
_{3}\right\Vert =1$

$\implies \mu _{1}^{2}+\mu _{2}^{2}+\mu _{3}^{2}=1\implies $

$\left( f,\mu _{1}\varphi _{1}+\mu _{2}\varphi _{2}+\mu _{3}\varphi
_{3}\right) =\mu _{1}\left( f,\varphi _{1}\right) +\mu _{2}\left( f,\varphi
_{2}\right) +\mu _{3}\left( f,\varphi _{3}\right) \geq -\underset{1\leq
i\leq 3}{\max }\left\{ \left\vert \left( f,\varphi _{i}\right) \right\vert
\right\} $

$\bigskip $And the equality can be taken. $\implies m=-\underset{1\leq i\leq
3}{\max }\left\{ \left\vert \left( f,\varphi _{i}\right) \right\vert
\right\} .$

$\bigskip \left( 2\right) M=E^{\bot }.$We still assume that $\left( \varphi
_{i},\varphi _{j}\right) =\delta _{ij},$

$f^{\prime }=f-\left( f,\varphi _{1}\right) \varphi _{1}-\left( f,\varphi
_{2}\right) \varphi _{2}-\left( f,\varphi _{3}\right) \varphi _{3}\in M,$

Also $\left\vert \left( f,u\right) \right\vert =\left\vert \left( f^{\prime
},u\right) \right\vert \leq \left\Vert f^{\prime }\right\Vert $ and the
equality can be taken at

$u=\frac{-f^{\prime }}{\left\Vert f^{\prime }\right\Vert }\implies \left(
f,u\right) =\left( f^{\prime },u\right) \geq -\left\Vert f^{\prime
}\right\Vert .\implies $

$m=-\left\Vert f-\left( f,\varphi _{1}\right) \varphi _{1}-\left( f,\varphi
_{2}\right) \varphi _{2}-\left( f,\varphi _{3}\right) \varphi
_{3}\right\Vert $

$\left( 3\right) $ We only need to get an orthogonal basis from $\left\{
\varphi _{i},i=1,2,3\right\} $

The norm of $L^{2}\left( 0,1\right) $ is defined by $\left( g,h\right)
=\int_{0}^{1}g\left( t\right) h\left( t\right) dt\implies $

$\phi _{1}\left( t\right) =t,\phi _{2}\left( t\right) =t-\frac{4}{3}%
t^{2},\phi _{3}\left( t\right) =t^{3}+\frac{2}{5}t-\frac{4}{3}%
t^{2},s.t.\left( \phi _{i},\phi _{j}\right) =0,as$ $i\neq j$

After normalization we can apply the conclusion in $\left( 2\right) $ to
find $m.$

\end{document}
