
\documentclass{article}
\usepackage{amssymb}

%%%%%%%%%%%%%%%%%%%%%%%%%%%%%%%%%%%%%%%%%%%%%%%%%%%%%%%%%%%%%%%%%%%%%%%%%%%%%%%%%%%%%%%%%%%%%%%%%%%%%%%%%%%%%%%%%%%%%%%%%%%%%%%%%%%%%%%%%%%%%%%%%%%%%%%%%%%%%%%%%%%%%%%%%%%%%%%%%%%%%%%%%%%%%%%%%%%%%%%%%%%%%%%%%%%%%%%%%%%%%%%%%%%%
\usepackage{amsmath}

\setcounter{MaxMatrixCols}{10}
\def\TEXTsymbol#1{\mbox{$#1$}}%
\def\NEG#1{\leavevmode\hbox{\rlap{\thinspace/}{$#1$}}}%
\def\QATOPD#1#2#3#4{{#3 \atopwithdelims#1#2 #4}}%
\def\QTP#1{}
\def\func#1{\mathop{\rm #1}}%
%TCIDATA{Version=5.00.0.2552}
%TCIDATA{<META NAME="SaveForMode" CONTENT="1">}
%TCIDATA{Created=Wednesday, November 04, 2015 08:47:01}
%TCIDATA{LastRevised=Saturday, November 07, 2015 15:41:31}
%TCIDATA{<META NAME="GraphicsSave" CONTENT="32">}
%TCIDATA{<META NAME="DocumentShell" CONTENT="Standard LaTeX\Blank - Standard LaTeX Article">}
%TCIDATA{CSTFile=40 LaTeX article.cst}

\newtheorem{theorem}{Theorem}
\newtheorem{acknowledgement}[theorem]{Acknowledgement}
\newtheorem{algorithm}[theorem]{Algorithm}
\newtheorem{axiom}[theorem]{Axiom}
\newtheorem{case}[theorem]{Case}
\newtheorem{claim}[theorem]{Claim}
\newtheorem{conclusion}[theorem]{Conclusion}
\newtheorem{condition}[theorem]{Condition}
\newtheorem{conjecture}[theorem]{Conjecture}
\newtheorem{corollary}[theorem]{Corollary}
\newtheorem{criterion}[theorem]{Criterion}
\newtheorem{definition}[theorem]{Definition}
\newtheorem{example}[theorem]{Example}
\newtheorem{exercise}[theorem]{Exercise}
\newtheorem{lemma}[theorem]{Lemma}
\newtheorem{notation}[theorem]{Notation}
\newtheorem{problem}[theorem]{Problem}
\newtheorem{proposition}[theorem]{Proposition}
\newtheorem{remark}[theorem]{Remark}
\newtheorem{solution}[theorem]{Solution}
\newtheorem{summary}[theorem]{Summary}
\newenvironment{proof}[1][Proof]{\noindent\textbf{#1.} }{\ \rule{0.5em}{0.5em}}


\begin{document}


\bigskip 赵丰\qquad 2013012178\qquad 泛函第8%
周作业

2.8

We first show that $T$ is a closed operator and it is equivalent to show
that if $x_{n}\rightarrow x$ in $E,$

$Tx_{n}\rightarrow y$ in $E^{\ast },$then $Tx=y.$ $\forall z\in E,$by the
condition we have $\left\langle T(x_{n}-z),x_{n}-z\right\rangle \geq 0,$by

the linearity of $T,\implies \left\langle Tx_{n},x_{n}\right\rangle
-\left\langle Tx_{n},z\right\rangle -\left\langle Tz,x_{n}-z\right\rangle
\geq 0$ $\left( \ast \right) .$

$\left\langle Tx_{n},z\right\rangle \rightarrow \left\langle
y,z\right\rangle ,$since $Tx_{n}\rightarrow y;$

$\left\langle Tz,x_{n}-z\right\rangle \rightarrow \left\langle
Tz,x-z\right\rangle ,$since$\qquad x_{n}\rightarrow x;$

$\left\vert \left\langle Tx_{n},x_{n}\right\rangle -\left\langle
y,x\right\rangle \right\vert \leq \left\vert \left\langle
Tx_{n},x_{n}\right\rangle -\left\langle y,x_{n}\right\rangle \right\vert
+\left\vert \left\langle y,x_{n}\right\rangle -\left\langle y,x\right\rangle
\right\vert \rightarrow 0;\implies \left\langle Tx_{n},x_{n}\right\rangle
\rightarrow \left\langle y,x\right\rangle .$

Taking the limit $n\rightarrow \infty $ in $\left( \ast \right) \implies
\left\langle y,x-z\right\rangle $ $\geq \left\langle Tz,x-z\right\rangle
\forall z\in E.$Let $z^{\prime }=z-x,$

$\implies \left\langle y,z^{\prime }\right\rangle \geq \left\langle T\left(
x-z^{\prime }\right) ,z^{\prime }\right\rangle \implies \left\langle
Tx-y,z^{\prime }\right\rangle \leq \left\langle Tz^{\prime },z^{\prime
}\right\rangle ,$Substituting $tz^{\prime }$ for $z^{\prime }$ gives$\left(
t\in R\right) $:

$\left\langle Tx-y,z^{\prime }\right\rangle t\leq \left\langle Tz^{\prime
},z^{\prime }\right\rangle t^{2},$holds $\forall t\in R.\implies
\left\langle Tx-y,z^{\prime }\right\rangle =0\forall z^{\prime }\in E.$

Therefore $T$ is a closed operator. Since E is a Banach space and E* is
also. Then we can

apply the closed graph Thm to conclude that T is linear bounded operator
from E to E*.

2.9

Like 2.8,we first show that $T$ is a closed operator. If $x_{n}\rightarrow x$
in $E,$

$Tx_{n}\rightarrow z$ in $E^{\ast },$And $\forall y\in E,$by the condition
we have $\left\langle Tx_{n},y-x\right\rangle =\left\langle T\left(
y-x\right) ,x_{n}\right\rangle .$

Taking the limit $n\rightarrow \infty $ gives $\left\langle
z,y-x\right\rangle =\left\langle T\left( y-x\right) ,x\right\rangle $

On the other hand $\left\langle T\left( y-x\right) ,x\right\rangle
=\left\langle Ty,x\right\rangle -\left\langle Tx,x\right\rangle
=\left\langle Tx,y\right\rangle -\left\langle Tx,x\right\rangle
=\left\langle Tx,y-x\right\rangle $

$\implies \left\langle z,y-x\right\rangle =\left\langle Tx,y-x\right\rangle
,\forall y\in E\implies z=Tx$

\bigskip Like 2.8,from the closed graph Thm, the conclusion follows.

2.10 $\left( a\right) $Suppose $T\left( M\right) $ is closed, for each
sequence $e_{k}+g_{k}\in M+N\left( T\right) ,$which converges to $e,$below
we show that $e\in M+N\left( T\right) .$ Applying $T$ to this sequece gives
a sequence $T\left( e_{k}\right) $ in $T\left( M\right) ,$which converges to 
$T\left( e\right) $ by the continuity of $T$. Since $T\left( M\right) $ is
closed, $T\left( e\right) \in T\left( M\right) .$Then we can find $\tilde{e}%
\in M,s.t.T\left( \tilde{e}\right) =T\left( e\right) ,$since $T$ is
surjective. It follows that $T\left( e-\tilde{e}\right) =0\implies e-\tilde{e%
}\in N\left( T\right) \implies e\in M+N\left( T\right) .$

Conversely, we show that $T\left( \left( M+N\left( T\right) \right)
^{c}\right) =\left( T\left( M+N\left( T\right) \right) \right) ^{c}.$

$\forall y\in T\left( \left( M+N\left( T\right) \right) ^{c}\right) ,$there
exists $x\in E/\left( M+N\left( T\right) \right) ,s.t.T\left( x\right) =y.$
If $T\left( x\right) \in T\left( M+N\left( T\right) \right) ,$then

$T\left( x\right) =T\left( x^{\prime }\right) ,x^{\prime }\in M+N\left(
T\right) ,x-x^{\prime }\in N\left( T\right) \implies x\in M+N\left( T\right)
.$A contradiction.

$\implies y=T\left( x\right) \notin T\left( M+N\left( T\right) \right)
\implies y\in \left( T\left( M+N\left( T\right) \right) \right) ^{c}\implies
T\left( \left( M+N\left( T\right) \right) ^{c}\right) \subset \left( T\left(
M+N\left( T\right) \right) \right) ^{c}.$

And for any subset of E, $T\left( A^{c}\right) \supset \left( T\left(
A\right) \right) ^{c}$. Indeed, $\forall y\in \left( T\left( A\right)
\right) ^{c},$

there exists $x\in E,s.t.T\left( x\right) =y\notin T\left( A\right)
.\implies x\notin A\implies x\in A^{c}\implies y\in T\left( A^{c}\right)
\implies T\left( A^{c}\right) \supset \left( T\left( A\right) \right) ^{c}.$

As a result, $T\left( \left( M+N\left( T\right) \right) ^{c}\right) =\left(
T\left( M+N\left( T\right) \right) \right) ^{c}$ is shown.

Since $T$ is surjective from $E$ to $F$, where both spaces are Banach
space,by open mapping Thm,

$T$ maps every open set onto open set. If $M+N\left( T\right) $ is closed
,then its complement $\left( M+N\left( T\right) \right) ^{c}$ is open$%
\implies $

$\left( T\left( M+N\left( T\right) \right) \right) ^{c}$ is open$\implies $ $%
T\left( M+N\left( T\right) \right) =T\left( M\right) $ is closed.

In conclusion $T\left( M\right) $ is closed$\iff M+N\left( T\right) $ is
closed.

$\left( b\right) $ $N\left( T\right) $ is a linear subspace of $E$. $N\left(
T\right) $ is also closed since its dimension is finite. Then by Proposition
11.4 on page 350 of textbook we can show that

their sum-space $M+N\left( T\right) $ is also closed. Then by the conclusion
in $\left( a\right) $ it follows that $T\left( M\right) $ is closed.

2.12

Since dimN$\left( T\right) $\TEXTsymbol{<}$\infty ,$ it has a complement L
in E, that is ,$\forall x\in E,x$ has unique decomposition $x=x^{\prime
}+x_{e},s.t.T\left( x\right) =T\left( x_{e}\right) .$If we restrict T on L,
Then T is injective from L to F. Indeed, suppose that $T\left( x_{1}\right)
=T\left( x_{2}\right) ,$then $x_{1}-x_{2}\in N\left( T\right) ,$we choose $%
x_{1},x_{2}\in L\implies x_{1}-x_{2}\in L\cap N\left( T\right) =\left\{
0\right\} \implies x_{1}=x_{2}.$

L is a closed subspace of E, also a Banach space. Then we get a bijective
mapping from L to T$\left( L\right) =T\left( R\right) ,$ which by the
condition is closed, also a Banach space$.$Then by the inverse operator Thm,

there exists a constant $c,s.t.\forall x\in L,y\in R\left( T\right) $ we
have:

$\left\Vert x\right\Vert \leq c\left\Vert Tx\right\Vert .$

We define a new norm on E as $\left\Vert x\right\Vert _{1}=\left\Vert
Tx\right\Vert _{F}+\left\vert x\right\vert ,$to show the new norm is
equivalent to the original norm. We should show that $\left( E,\left\Vert
{}\right\Vert _{1}\right) $ is a Banach space.Suppose $\left\Vert
x_{n}-x_{m}\right\Vert _{1}\rightarrow 0\implies \left\Vert
Tx_{n}-Tx_{m}\right\Vert _{F}\rightarrow 0$

Since F is Banach space,there exists $y\in F,s.t.\left\Vert
Tx_{n}-y\right\Vert _{F}\rightarrow 0.$

$y\in R\left( T\right) ,$ since $R\left( T\right) $ is closed$\implies $%
there exists $x\in E,s.t.T\left( x\right) =y.$To show $\left\Vert
x_{n}-x\right\Vert _{1}\rightarrow 0,$we only need to show $\left\vert
x_{n}-x\right\vert \rightarrow 0.$Decomposing $x_{n}=x_{n}^{\prime
}+x_{e}^{\left( n\right) },x=x^{\prime }+x_{e},$where $x_{n}^{\prime },x\in
N\left( T\right) ,x_{e}^{\left( n\right) },x_{e}\in L.$ Since $\left\Vert
x_{e}^{\left( n\right) }-x_{e}\right\Vert \leq c\left\Vert Tx_{e}^{\left(
n\right) }-Tx_{e}\right\Vert =\left\Vert Tx_{n}-y\right\Vert \rightarrow
0\implies x_{e}^{\left( n\right) }\rightarrow x_{e}.$ Further we have  $%
\left\vert x_{n}-x_{m}\right\vert \rightarrow 0$

Seperating $x_{n}^{\prime },x^{\prime }$ out gives $\left\vert x_{n}^{\prime
}-x_{m}^{\prime }\right\vert \leq \allowbreak \left\vert
x_{n}-x_{m}\right\vert +\left\vert x_{e}^{\left( n\right) }-x_{e}^{\left(
m\right) }\right\vert \rightarrow 0.$Notice that dimN$\left( T\right)
<\infty ,\implies N\left( T\right) $ is complete$\implies $there exists $%
x^{\ast }\in N\left( T\right) ,s.t.\left\vert x_{n}^{\prime }-x^{\ast
}\right\vert \rightarrow 0\implies x_{n}\rightarrow x^{\ast }+x_{e}$ in $%
\left\vert {}\right\vert .$But we know $x_{n}\rightarrow x^{\prime }+x_{e}$
in

$\left\vert {}\right\vert \implies x^{\ast }+x_{e}=x^{\prime }+x_{e}\implies
x^{\prime }=x^{\ast }.$That is $x_{n}\rightarrow x+x_{e}$ =x in $\left\vert
{}\right\vert .$As a result, $\left( E,\left\Vert {}\right\Vert _{1}\right) $
is a Banach space. $\left\Vert x\right\Vert _{1}\leq \left\Vert T\right\Vert
\left\Vert x\right\Vert _{E}+M\left\Vert x\right\Vert _{E}=\left( \left\Vert
T\right\Vert +M\right) \left\Vert x\right\Vert _{E},$by Cor2.8$\implies $%
there exists a constant c,s.t.

$\left\Vert x\right\Vert _{E}\leq c\left\Vert x\right\Vert _{1}=c\left(
\left\Vert Tx\right\Vert _{F}+\left\vert x\right\vert \right) ,\forall x\in
E.$

2.11

\bigskip In $F=\ell ^{1}$ we can find a basis $\left\{
f_{1},f_{2},..\right\} $ with $\left\Vert f_{i}\right\Vert =1$

Since T is surjective, by open mapping thm, for some constant c, we can find 
$e_{i}$ s.t. $T\left( e_{i}\right) =f_{i},$ and $\left\Vert e_{i}\right\Vert
\leq c\forall i.$ 

S=Span$\left\{ e_{1},..e_{n}..\right\} $ is a subspace of E. And for any z$%
\in E,T\left( z\right) $ can be represented uniquely as 

T$\left( z\right) =\underset{i=1}{\overset{\infty }{\sum }}z_{i}f_{i}.$

Now $\forall x=\underset{i=1}{\overset{\infty }{\sum }}x_{i}f_{i}\in
F,\left\Vert x\right\Vert =\underset{i=1}{\overset{\infty }{\sum }}%
\left\vert x_{i}\right\vert <\infty ,$

\bigskip Consider s$_{n}=\underset{i=1}{\overset{n}{\sum }}x_{i}e_{i}\in
E,\left\Vert s_{n+p}-\NEG{s}_{n}\right\Vert =\left\Vert \underset{i=n+1}{%
\overset{n+p}{\sum }}x_{i}e_{i}\right\Vert \leq c\underset{i=n+1}{\overset{%
n+p}{\sum }}\left\vert x_{i}\right\vert ,$

since $\underset{i=1}{\overset{\infty }{\sum }}\left\vert x_{i}\right\vert
<\infty ,$by Cauchy's Criterion for real number, follows $\underset{i=n+1}{%
\overset{n+p}{\sum }}\left\vert x_{i}\right\vert \rightarrow 0\implies
\left\Vert s_{n+p}-\NEG{s}_{n}\right\Vert \rightarrow 0,$

since E is a Banach space$\implies \left\{ s_{n}\right\} $ converges in E,
and the limit is denoted as $\underset{i=1}{\overset{\infty }{\sum }}%
x_{i}e_{i}.$

We then define $Sx=\underset{i=1}{\overset{\infty }{\sum }}x_{i}e_{i}\in E,$%
such definition is well-defined as shown above,

and S is also a linear mapping from F to E. 

\bigskip To show T$\left( \underset{i=1}{\overset{\infty }{\sum }}%
x_{i}e_{i}\right) =\left( \underset{i=1}{\overset{\infty }{\sum }}%
x_{i}f_{i}\right) ,$

in the equality, T$\left( \underset{i=1}{\overset{n}{\sum }}%
x_{i}e_{i}\right) =\left( \underset{i=1}{\overset{n}{\sum }}%
x_{i}f_{i}\right) ,$we let n$\rightarrow \infty ,$by the continuity of T
follows 

T$\left( \underset{i=1}{\overset{\infty }{\sum }}x_{i}e_{i}\right) =\left( 
\underset{i=1}{\overset{\infty }{\sum }}x_{i}f_{i}\right) $

Therefore S satisfies $T\circ S=I_{F}.$ To show the continuity of $S,$%
suppose $\underset{i=1}{\overset{\infty }{\sum }}\left\vert x_{i}\right\vert
\leq 1,\left\Vert Sx\right\Vert \leq \underset{i=1}{\overset{\infty }{\sum }}%
\left\vert x_{i}\right\vert \left\Vert e_{i}\right\Vert \leq c$

\bigskip $\left\Vert S\right\Vert \leq c$ and the proof is complete$%
\boxtimes .$

2.16 Define a mapping T from E/G$\cap L$ to E/L by T$\left( x+G\cap L\right)
=x+L.$ 

\bigskip if x-y$\in G\cap L,x-y\in L$ therefore this mapping is
well-defined. It is obvious that T is surjective and linear.

By the discussion on P368 of the textbook, we can show that $dist\left(
\cdot ,L\right) $ is a norm on E/L and

$dist\left( \cdot ,G\cap L\right) $ is a norm on E/G$\cap $L. Since $%
\left\Vert T\left( x\right) \right\Vert _{E/L}=dist\left( x,L\right) \leq
dist\left( x,G\cap L\right) =\left\Vert x\right\Vert _{E/G\cap L},\implies $%
T is bdd.

The condition says that $\left\Vert x\right\Vert _{E/G\cap L}\leq
C\left\Vert T\left( x\right) \right\Vert _{E/L}\implies T$ is injective.
Therefore T$^{-1}$ exists and is a linear bdd

operator. If we restrict T on G/G$\cap L,G$ is a closed subspace of E, also
a Banach space, then we get

a mapping from G/G$\cap L$ to $\left( G+L\right) /L,$ which is bijective and
linear bdd. $\implies G+L/L$ is closed in E/L.

Consider $\pi :E\rightarrow E/L$ which takes the equivalent class. By the
discussion on P368,$\pi $ is a linear bdd surjective operator.$\implies
G+L=\pi ^{-1}\left( \left( G+L\right) /L\right) $ is closed. We can also
show this directly. Suppose $\left\{ x_{n}\right\} \in G+L\rightarrow x,$%
then $dist\left( x_{n}-x,L\right) \leq \left\Vert x_{n}-x\right\Vert
\rightarrow 0,\implies x_{n}+L\rightarrow x+L$ in E/L.

Since G+L/L is closed and $x_{n}+L\in \left( G+L\right) /L$, $x+L\in \left(
G+L\right) /L,\implies x\in G+L.$ And G+L is closed.        

2.17 $\left( a\right) $ $\overline{D\left( A\right) }=E$ means that every
continuous function can be approximated by differential function,which is
guaranteed by Weistrass' approximation Thm, since every continuous function
can be approximated by polynomials.

$\left( b\right) $ Suppose $f_{n}\in C^{1}\left( \left[ 0,1\right] \right)
\rightarrow f,f_{n}^{\prime }\in C\left( \left[ 0,1\right] \right)
\rightarrow g,$ then by the definition of max norm on C$\left( \left[ 0,1%
\right] \right) $

$f_{n}^{\prime }\rightarrow g$ uniformly,  $f_{n}\left( x\right)
=f_{n}\left( 0\right) +\int_{0}^{x}f_{n}^{\prime }\left( t\right) dt\implies
\left\vert f_{n}\left( x\right) -f\left( 0\right) -\int_{0}^{x}g\left(
t\right) dt\right\vert \leq \left\vert f_{n}\left( 0\right) -f\left(
0\right) \right\vert +\int_{0}^{x}\left\vert f_{n}^{\prime }\left( t\right)
-g\left( t\right) \right\vert dt$

$\leq \left\vert f_{n}\left( 0\right) -f\left( 0\right) \right\vert
+\left\Vert f_{n}^{\prime }-g\right\Vert \rightarrow 0\implies \left\Vert
f_{n}-f\left( 0\right) -\int_{0}^{x}g\left( t\right) dt\right\Vert
\rightarrow 0,f_{n}\rightarrow f\left( 0\right) +\int_{0}^{x}g\left(
t\right) dt\in C^{1}\left( \left[ 0,1\right] \right) .$

By the uniqueness of the limit, $f=f\left( 0\right) +\int_{0}^{x}g\left(
t\right) dt$ and $f^{\prime }=g\implies $the operator A is closed.

$\left( c\right) $Suppose $f_{n}\in C^{2}\left( \left[ 0,1\right] \right)
\rightarrow f,f_{n}^{\prime }\in C^{1}\left( \left[ 0,1\right] \right)
\rightarrow g.$Similar to $\left( b\right) $ we can show that $f^{\prime
}=g,g$ does 

not necessarily belong to $C^{1}\left( \left[ 0,1\right] \right) \implies f$
does not necessarily belong to $C^{2}\left( \left[ 0,1\right] \right) .$

The counter example: $f_{n}^{\prime }\left( x\right) =\sqrt{\left( x-\frac{1%
}{2}\right) ^{2}+\frac{1}{n}}\rightarrow g=\left\vert x-\frac{1}{2}%
\right\vert \notin C^{1}\left( \left[ 0,1\right] \right) ,$We can get $f_{n}$
by direct integral, and $f_{n}\in C^{2}\left( \left[ 0,1\right] \right) $ $%
\rightarrow \int \left\vert x-\frac{1}{2}\right\vert dx\notin C^{2}\left( %
\left[ 0,1\right] \right) .$

2.27 F=R$\left( T\right) \oplus X,$Let G=$\left( E,X\right) ,$equipped with
the norm $\left\Vert \left( x,y\right) \right\Vert =\left\Vert x\right\Vert
_{E}+\left\Vert y\right\Vert _{F}.$

For$\left( x_{n},y_{n}\right) \in G,$satisfying $\left\Vert \left(
x_{n+p},y_{n+p}\right) -\left( x_{n},y_{n}\right) \right\Vert \rightarrow 0$,%
$\left\Vert x_{n+p}-x_{n}\right\Vert _{E}\rightarrow 0,\left\Vert
y_{n+p}-y_{n}\right\Vert _{E}\rightarrow 0;$

Since E is a Banach space, there exists $x\in E,s.t.\left\Vert
x_{n}-x\right\Vert _{E}\rightarrow 0;$

Since X is a finite subspace of F and F is a Banach space, X is complete and
there exists $y\in X,s.t.$

$\left\Vert y_{n}-y\right\Vert _{F}\rightarrow 0\implies \left\Vert \left(
x_{n},y_{n}\right) -\left( x,y\right) \right\Vert \rightarrow 0$, and $%
\left( x,y\right) \in G.\implies G$ is a Banach space.

Define a mapping from G to F as $T\left( x,y\right) =Tx+y,$ Since F=R$\left(
T\right) \oplus X,$ T is bijective.

$\left\Vert T\left( x,y\right) \right\Vert =\left\Vert Tx+y\right\Vert
_{F}\leq \left\Vert y\right\Vert +\left\Vert Tx\right\Vert \leq \left\Vert
y\right\Vert +\left\Vert T\right\Vert \left\Vert x\right\Vert _{E}\leq \max
\left\{ 1,\left\Vert T\right\Vert \right\} \left\Vert \left( x,y\right)
\right\Vert \implies $

T is bdd. By Inverse mapping Thm, T$^{-1}$ is bdd. Then T maps the closed
set $\left( E,\left\{ 0\right\} \right) $ to a closed set

in E, and the closed set is $R\left( T\right) .$


\end{document}
