
\documentclass{article}
\usepackage{amsmath}

%%%%%%%%%%%%%%%%%%%%%%%%%%%%%%%%%%%%%%%%%%%%%%%%%%%%%%%%%%%%%%%%%%%%%%%%%%%%%%%%%%%%%%%%%%%%%%%%%%%%%%%%%%%%%%%%%%%%%%%%%%%%%%%%%%%%%%%%%%%%%%%%%%%%%%%%%%%%%%%%%%%%%%%%%%%%%%%%%%%%%%%%%%%%%%%%%%%%%%%%%%%%%%%%%%%%%%%%%%%%%%%%%%%%
%TCIDATA{OutputFilter=LATEX.DLL}
%TCIDATA{Version=5.00.0.2552}
%TCIDATA{<META NAME="SaveForMode" CONTENT="1">}
%TCIDATA{Created=Tuesday, December 29, 2015 19:04:50}
%TCIDATA{LastRevised=Tuesday, December 29, 2015 22:03:22}
%TCIDATA{<META NAME="GraphicsSave" CONTENT="32">}
%TCIDATA{<META NAME="DocumentShell" CONTENT="Scientific Notebook\Blank Document">}
%TCIDATA{CSTFile=Math with theorems suppressed.cst}
%TCIDATA{PageSetup=72,72,72,72,0}
%TCIDATA{AllPages=
%F=36,\PARA{038<p type="texpara" tag="Body Text" >\hfill \thepage}
%}


\newtheorem{theorem}{Theorem}
\newtheorem{acknowledgement}[theorem]{Acknowledgement}
\newtheorem{algorithm}[theorem]{Algorithm}
\newtheorem{axiom}[theorem]{Axiom}
\newtheorem{case}[theorem]{Case}
\newtheorem{claim}[theorem]{Claim}
\newtheorem{conclusion}[theorem]{Conclusion}
\newtheorem{condition}[theorem]{Condition}
\newtheorem{conjecture}[theorem]{Conjecture}
\newtheorem{corollary}[theorem]{Corollary}
\newtheorem{criterion}[theorem]{Criterion}
\newtheorem{definition}[theorem]{Definition}
\newtheorem{example}[theorem]{Example}
\newtheorem{exercise}[theorem]{Exercise}
\newtheorem{lemma}[theorem]{Lemma}
\newtheorem{notation}[theorem]{Notation}
\newtheorem{problem}[theorem]{Problem}
\newtheorem{proposition}[theorem]{Proposition}
\newtheorem{remark}[theorem]{Remark}
\newtheorem{solution}[theorem]{Solution}
\newtheorem{summary}[theorem]{Summary}
\newenvironment{proof}[1][Proof]{\noindent\textbf{#1.} }{\ \rule{0.5em}{0.5em}}
\input{tcilatex}

\begin{document}


6.8 1 Consider $T:E\rightarrow R\left( T\right) ,$where $E$ and $R\left(
T\right) $ are both Banach spaces since $R\left( T\right) $ is a closed
subspace of Banach space $F.$

By open mapping theorem, there exists a constant $c>0,s.t$

$B\left( 0,c\right) \subset T\left( B_{E}\right) \implies \overline{B\left(
0,c\right) }\subset \overline{T\left( B_{E}\right) }.$

Since $T$ is a compact operator, $\overline{T\left( B_{E}\right) }$ is
compact, the same is true for its closed subset.$\implies \overline{B\left(
0,c\right) }$ is compact$\implies \overline{B_{R\left( T\right) }\text{ }}$
is compact$\implies R\left( T\right) $ is finite dimensional by Riesz
Theorem. 

$T$ is a finite rank operator.

2 Consider $\tilde{T}:E\backslash N\left( T\right) \rightarrow R\left(
T\right) ,E\backslash N\left( T\right) $ is the quotient space$\left( \func{%
mod}N\left( T\right) \right) .$

$\pi :E\rightarrow E\backslash N\left( T\right) $ is the canonical map,
which is surjective. It is known that 

$E\backslash N\left( T\right) $ is a Banach space, then by open mapping
theorem

$B_{E\backslash N\left( T\right) }\left( 0,c\right) \subset B_{E},$ by
definition $\tilde{T}B_{E\backslash N\left( T\right) }\left( 0,c\right)
\subset TB_{E}\implies \overline{\tilde{T}B_{E\backslash N\left( T\right)
}\left( 0,c\right) }$

is compact$\implies \tilde{T}$ is compact. $\tilde{T}$ is injective.$%
\implies \tilde{T}$ \ is bijective.

$\implies $Then $\tilde{T}$ allows the inverse operator $\tilde{T}$ $^{-1}$%
,which is bounded.

Then $I_{E\backslash N\left( T\right) }=\tilde{T}$ $^{-1}\circ \tilde{T}$ is
a compact operator by the composite property of compact operator.$\implies 
\overline{B_{E\backslash N\left( T\right) }}$ is compact$\implies
E\backslash N\left( T\right) $ is finite dimensional.

$\implies \dim E=\dim N\left( T\right) \times \dim E\backslash N\left(
T\right) <\infty .$

We can also prove this proposition by decomposing $E=N\left( T\right) +L,s.t.
$

$N\left( T\right) \cap L=\left\{ 0\right\} .$This is possible since $N\left(
T\right) $ is of finite dimension.

Then we restrict $T$ onto $L:T_{|L}:L\rightarrow R\left( T\right) ,$by the
definition of $N\left( T\right) .$

This is a bijective mapping. Similar to above deduction, we can show $\dim
L<\infty \implies \dim E<\infty $

6.9$\left( 1\right) \implies \left( 2\right) :E$ can be decomposed as $%
E=N\left( T\right) +L,N\left( T\right) \cap L=\left\{ 0\right\} $

then we define $P:E\rightarrow N\left( T\right) ,$as $P\left( x\right) =y,$%
where $x=y+z,y\in N\left( T\right) $ and 

$z\in L.P$ is bounded from former chapters. $\dim R\left( P\right) =\dim
N\left( T\right) <\infty $

$\implies P$ is a finite rank projection operator. 

Below we can assume $F$ is closed, otherwise we can replace $F$ with $%
R\left( T\right) .$

To deduce the conclusion, we argue by contradiction$\implies $

$\exists u_{n}\in T,\left\Vert u_{n}\right\Vert =1,s.t.\left\Vert
Tu_{n}\right\Vert _{F}+\left\Vert Pu_{n}\right\Vert _{E}<\frac{1}{n}.$

By the open mapping Thm $cB_{F}\subset TB_{E},\implies \exists v_{n}\in
B_{E}\left( 0,\frac{1}{nc}\right) ,s.t.$

$Tv_{n}=Tu_{n}\implies v_{n}-u_{n}\in N\left( T\right) ,$and $\underset{%
n\rightarrow \infty }{\lim }\left\Vert v_{n}-u_{n}\right\Vert =1.$

On the other hand $\left\Vert v_{n}-u_{n}\right\Vert =\left\Vert
Pv_{n}-Pu_{n}\right\Vert \leq \frac{1}{n}+\left\Vert Pv_{n}\right\Vert $

$\leq \frac{1}{n}+C\left\Vert v_{n}\right\Vert \leq \frac{1}{n}+\frac{C}{nc}%
\rightarrow 0.$A contradiction$\implies \exists C,s.t.$

$\left\Vert u\right\Vert _{E}\leq C\left( \left\Vert Tu\right\Vert
_{F}+\left\Vert Pu\right\Vert _{E}\right) ,\forall u\in E.$

$\left( 2\right) \implies \left( 3\right) $ Since $P$ is a finite rank
projection operator, define $G:=P\left( E\right) ,$which is finite
dimensional therefore a Banach space,

and let $Q=P.$Then the conclusion follows.

$\left( 3\right) \implies \left( 1\right) \left\Vert u\right\Vert _{E}\leq
C\left\Vert Qu\right\Vert _{G},\forall u\in N\left( T\right) .$

$\left\Vert Qu\right\Vert _{G}\geq \frac{\left\Vert u\right\Vert _{E}}{C}%
\implies \frac{\overline{B_{N\left( T\right) }}}{C}\subset \overline{B}_{Q},$%
since $Q\in K\left( E,G\right) $

$\implies \overline{B_{N\left( T\right) }}$ is compact therefore $N\left(
T\right) $ is finite dimensional.

Below we assume $T$ is injective, otherwise consider $E=N\left( T\right)
\oplus L,$

and restrict $T$ on $L$ and try to show $L$ is closed.

\bigskip $\left\Vert u\right\Vert _{E}\leq C\left( \left\Vert Tu\right\Vert
_{F}+\left\Vert Qu\right\Vert _{G}\right) ,$Suppose $T\left( x_{n}\right)
\rightarrow y$

\left\Vert x_{n}\right\Vert _{E}$\leq C\left( \left\Vert Tx_{n}\right\Vert
_{F}+\left\Vert Qx_{n}\right\Vert _{G}\right) ,Q\frac{x_{n}}{\left\Vert
x_{n}\right\Vert }$allows a convergent subsequence 

$\implies Q\frac{x_{n}}{\left\Vert x_{n}\right\Vert }$is a Cauchy sequence

Suppose $\left\Vert x_{n}\right\Vert $ is unbounded, further assuming $%
\left\Vert x_{n}\right\Vert \rightarrow \infty ,$without loss of generality.

$\left\Vert \frac{x_{n}}{\left\Vert x_{n}\right\Vert }-\frac{y_{n}}{%
\left\Vert y_{n}\right\Vert }\right\Vert $ $\leq C\left( \frac{\left\Vert
Tx_{n}\right\Vert _{F}}{\left\Vert x_{n}\right\Vert }+\frac{\left\Vert
Tx_{m}\right\Vert _{F}}{\left\Vert x_{m}\right\Vert }+\left\Vert Q\left( 
\frac{x_{n}}{\left\Vert x_{n}\right\Vert }\right) -Q\left( \frac{x_{m}}{%
\left\Vert x_{m}\right\Vert }\right) \right\Vert \right) $

$\rightarrow 0,as$ $n,m\rightarrow \infty , $since $Tx_{n}$ is
bounded.

$\implies \frac{x_{n}}{\left\Vert x_{n}\right\Vert }$ is a Cauchy sequence. 

$\frac{x_{n}}{\left\Vert x_{n}\right\Vert }\rightarrow z,T\frac{x_{n}}{%
\left\Vert x_{n}\right\Vert }\rightarrow Tz\implies Tz=0,\implies z=0,$%
contradict with $\left\Vert z\right\Vert =1.$

$\implies \left\Vert x_{n}\right\Vert $ is bounded. Therefore we can extract
a subsequence s.t.

$\bigskip x_{n_{k}}\rightarrow x\in E\implies Tx=y\in R\left( T\right) $

$\implies R\left( T\right) $ is closed.

$\left( 2\right) $As above, without loss of generality, we can assume $T$ is
bijective.

Then $T$ allows the inverse operator $T^{-1},$

To show $\overline{B_{N\left( T+S\right) }}$ is compact, take a sequence $%
\left\{ x_{n}\right\} $ from it

Then $\left\Vert x_{n}\right\Vert \leq 1$ and $\left( T+S\right) \left(
x_{n}\right) =0,x_{n}=-T^{-1}\circ Sx_{n},$

next to show $\left\{ x_{n}\right\} $ allows a convergent subsequence. Since 
$S$ is a compact operator, $\left\{ Sx_{n}\right\} $ is bounded$\implies $

$\left\{ Sx_{n}\right\} $ allows a convergent subsequence $\left\{
Sx_{n_{k}}\right\} ,$

\bigskip $\implies x_{n_{k}}$ is convergent from $x_{n}=-T^{-1}\circ Sx_{n}.$

$\implies \overline{B_{N\left( T+S\right) }}$ is compact, and therefore $%
N\left( T+S\right) $ is finite dimensional.

$\left( T+S\right) x_{n}\rightarrow z,$

Since $R\left( T\right) $ is closed, 

\end{document}
