
\documentclass{article}
\usepackage{amssymb}

%%%%%%%%%%%%%%%%%%%%%%%%%%%%%%%%%%%%%%%%%%%%%%%%%%%%%%%%%%%%%%%%%%%%%%%%%%%%%%%%%%%%%%%%%%%%%%%%%%%%%%%%%%%%%%%%%%%%%%%%%%%%%%%%%%%%%%%%%%%%%%%%%%%%%%%%%%%%%%%%%%%%%%%%%%%%%%%%%%%%%%%%%%%%%%%%%%%%%%%%%%%%%%%%%%%%%%%%%%%%%%%%%%%%
\usepackage{amsmath}

\setcounter{MaxMatrixCols}{10}
%TCIDATA{OutputFilter=LATEX.DLL}
%TCIDATA{Version=5.00.0.2552}
%TCIDATA{<META NAME="SaveForMode" CONTENT="1">}
%TCIDATA{Created=Friday, November 27, 2015 16:29:57}
%TCIDATA{LastRevised=Saturday, November 28, 2015 11:31:31}
%TCIDATA{<META NAME="GraphicsSave" CONTENT="32">}
%TCIDATA{<META NAME="DocumentShell" CONTENT="Standard LaTeX\Blank - Standard LaTeX Article">}
%TCIDATA{CSTFile=40 LaTeX article.cst}

\newtheorem{theorem}{Theorem}
\newtheorem{acknowledgement}[theorem]{Acknowledgement}
\newtheorem{algorithm}[theorem]{Algorithm}
\newtheorem{axiom}[theorem]{Axiom}
\newtheorem{case}[theorem]{Case}
\newtheorem{claim}[theorem]{Claim}
\newtheorem{conclusion}[theorem]{Conclusion}
\newtheorem{condition}[theorem]{Condition}
\newtheorem{conjecture}[theorem]{Conjecture}
\newtheorem{corollary}[theorem]{Corollary}
\newtheorem{criterion}[theorem]{Criterion}
\newtheorem{definition}[theorem]{Definition}
\newtheorem{example}[theorem]{Example}
\newtheorem{exercise}[theorem]{Exercise}
\newtheorem{lemma}[theorem]{Lemma}
\newtheorem{notation}[theorem]{Notation}
\newtheorem{problem}[theorem]{Problem}
\newtheorem{proposition}[theorem]{Proposition}
\newtheorem{remark}[theorem]{Remark}
\newtheorem{solution}[theorem]{Solution}
\newtheorem{summary}[theorem]{Summary}
\newenvironment{proof}[1][Proof]{\noindent\textbf{#1.} }{\ \rule{0.5em}{0.5em}}


\begin{document}


\bigskip \bigskip \U{8d75}\U{4e30}\qquad 2013012178\qquad \qquad Functional
Analysis 11\U{5468} Coursework

3.1$\forall f\in E^{\prime },f\left( A\right) $ is compact since $A$ is
compact. Indeed, for an open covering $\left\{ S_{i}\right\} $ of $f\left(
A\right) .\left\{ f^{-1}\left( S_{i}\right) \right\} $ is an open covering
of $A,$ since $f$ is continuous and $f^{-1}\left( S_{i}\right) $ is $\sigma
-open.$By the compactness of $A,A$ allows a finite covering$A\subset $ $%
\underset{i=1}{\overset{m}{\cup }}f^{-1}\left( S_{i}\right) $

$\implies f\left( A\right) \subset $ $\underset{i=1}{\overset{m}{\cup }}%
S_{i}.$

Every compact set in R is bounded. Then we can invoke the corallory of UBT

$\implies A$ is bounded in $E.$

3.2 $\sigma _{n}\rightharpoonup x\iff f\left( \sigma _{n}\right) \rightarrow
f\left( x\right) ,\forall f\in E^{\prime }\iff \frac{f\left( x_{1}\right)
+..+f\left( x_{n}\right) }{n}\rightarrow f\left( x\right) \forall f\in
E^{\prime }$

By the Stolz's formula in R,$\Longleftarrow f\left( x_{n}\right) \rightarrow
f\left( x\right) \forall f\in E^{\prime }\iff x_{n}\rightharpoonup x.$

The direct approach: suppose $x_{n}\rightharpoonup x,$that is, for any basis
of nbhd $U=\left\{ y|\left\vert f_{i}\left( y-x\right) \right\vert <\epsilon
,for\text{ }i\in I,finite\right\} $ of $x,$there exists N, s.t.$x_{n}\in
U.for$ $n\geq N,\implies \left\vert f_{i}\left( x_{n}-x\right) \right\vert <%
\frac{\epsilon }{2}\left( \text{we change the nbhd to }<\frac{\epsilon }{2}%
\right) ,$now $\sigma _{n}=\frac{x_{1}+..+x_{N}}{n}+\frac{x_{N+1}+..x_{n}}{n}%
,$

$\left\vert f_{i}\left( \sigma _{n}-x\right) \right\vert \leq $ $\left\vert
f_{i}\left( \frac{x_{1}+..+x_{N}}{n}\right) \right\vert +\frac{1}{n}\underset%
{j=N+1}{\overset{n}{\sum }}\left\vert f_{i}\left( x_{j}-x\right) \right\vert
\leq $ $\left\vert f_{i}\left( \frac{x_{1}+..+x_{N}}{n}\right) \right\vert +%
\frac{\epsilon }{2}\frac{n-N}{n},$

By the continuity of $f_{i},$fix N, we can find $N^{\prime },s.t.\left\vert
f_{i}\left( \frac{x_{1}+..+x_{N}}{n}\right) \right\vert <\frac{\epsilon }{2}%
, $

$for$ $i\in I,finite\implies \left\vert f_{i}\left( \sigma _{n}-x\right)
\right\vert <\frac{\epsilon }{2}+\frac{\epsilon }{2}\frac{n-N}{n}<\epsilon
,\implies \sigma _{n}\in U,for$ $n>N^{\prime }$

$\implies \sigma _{n}\rightharpoonup x.$

3.6 For any basis of nbhd $U_{\epsilon }\left( x\right) $ of E, $%
u^{-1}\left( U_{\epsilon }\left( x\right) \right) ,v^{-1}\left( U_{\epsilon
}\left( x\right) \right) $ are open since $u$ and $v$ are continuous. If $O$
is an open set of $E,$ to prove $\left( u+v\right) ^{-1}\left( O\right) $ is
open, we can select $x_{0}\in $ $\left( u+v\right) ^{-1}\left( O\right) ,$%
then we have $u\left( x_{0}\right) +v\left( x_{0}\right) \in O,$then we can
find a basis of neighborhood $U_{\epsilon }\left( u\left( x_{0}\right)
+v\left( x_{0}\right) \right) \subset O.$

We can write $U$ explicitly:$U=\left\{ y|\left\vert f_{i}\left( y-u\left(
x_{0}\right) -v\left( x_{0}\right) \right) \right\vert <\epsilon ,for\text{ }%
i\in I,finite\right\} ,$

To show $\left( u+v\right) ^{-1}\left( U\right) =\left\{ y|\left\vert
f_{i}\left( \left( u+v\right) \left( x\right) -u\left( x_{0}\right) -v\left(
x_{0}\right) \right) \right\vert <\epsilon \right\} \ \left( u+v\right)
^{-1}\left( O\right) $is an open nbhd of $x_{0},$

Notice that $u^{-1}\left( U_{\epsilon /2}\left( x_{0}\right) \right)
=\left\{ x|\left\vert f_{i}\left( u\left( x\right) -u\left( x_{0}\right)
\right) \right\vert <\frac{\epsilon }{2}\right\} $ is open,that is, $x_{0}$
is an interior point of $u^{-1}\left( U_{\epsilon /2}\left( x_{0}\right)
\right) $

$\exists $nbhd$X_{1}\left( x_{0}\right) \subset u^{-1}\left( U_{\epsilon
/2}\left( x_{0}\right) \right) .$Similarly, $\exists $nbhd$X_{1}\left(
x_{0}\right) \subset v^{-1}\left( U_{\epsilon /2}\left( x_{0}\right) \right) 
$

$X_{1}\left( x_{0}\right) \cap X_{1}\left( x_{0}\right) \subset u^{-1}\left(
U_{\epsilon /2}\left( x_{0}\right) \right) \cap v^{-1}\left( U_{\epsilon
/2}\left( x_{0}\right) \right) \subset \left( u+v\right) ^{-1}\left(
U\right) .$

By the definition of open set in $X,X_{1}\left( x_{0}\right) \cap
X_{1}\left( x_{0}\right) $ is open in $X$ hence an open nbhd of $x_{0}.$%
Therefore for any $x\in \left( u+v\right) ^{-1}\left( O\right) $ in $X,$we
find an open set $U^{\prime }\left( x\right) $ s.t.

$x\subset U^{\prime }\left( x\right) \subset \left( u+v\right) ^{-1}\left(
O\right) \implies \left( u+v\right) ^{-1}\left( O\right) $ is open and $%
x\longmapsto u\left( x\right) +v\left( x\right) $ is

continuous from X into E equipped with $\sigma \left( E,E^{\ast }\right) .$

3.3 $\overline{A}$ is closed and convex. By Thm 3.7, it is closed in the
weak topology. $\overline{A}^{\sigma \left( E,E^{\ast }\right) }$ is closed
in the weak topology, to show it is convex. suppose $x,y\in \overline{A}%
^{\sigma \left( E,E^{\ast }\right) },\exists x_{n}\rightharpoonup
x,y_{n}\rightharpoonup y,\implies f\left( x_{n}\right) \rightarrow f\left(
x\right) ,f\left( y_{n}\right) \rightarrow f\left( y\right) ,\forall f\in
E^{\prime }\implies $

$f\left( tx_{n}+\left( 1-t\right) y_{n}\right) \rightarrow f\left( tx+\left(
1-t\right) y\right) $ by the continuity of $f$

$\implies tx_{n}+\left( 1-t\right) y_{n}\rightharpoonup tx+\left( 1-t\right)
y\implies tx+\left( 1-t\right) y\in \overline{A}^{\sigma \left( E,E^{\ast
}\right) }.$

By Thm 3.7, it is closed in the strong topology. Then from $A\subset 
\overline{A}^{\sigma \left( E,E^{\ast }\right) },$ we take the closure in
the strong sense $\implies \overline{A}\subset \overline{A}^{\sigma \left(
E,E^{\ast }\right) };$from $A\subset \overline{A},$ we take the closure in
the weak sense $\implies \overline{A}^{\sigma \left( E,E^{\ast }\right)
}\subset \overline{A}.\implies A=\overline{A}^{\sigma \left( E,E^{\ast
}\right) }$

3.4$\left( 1\right) $

\bigskip $x_{n}\rightharpoonup x\implies x\in \overline{conv\left( \underset{%
i=n}{\overset{\infty }{\cup }}\left\{ x_{i}\right\} \right) }^{\sigma \left(
E,E^{\ast }\right) }\forall n,$from 3.3 we know that $\overline{conv\left( 
\underset{i=n}{\overset{\infty }{\cup }}\left\{ x_{i}\right\} \right) }%
^{\sigma \left( E,E^{\ast }\right) }=\overline{conv\left( \underset{i=n}{%
\overset{\infty }{\cup }}\left\{ x_{i}\right\} \right) },$therefore we can
find $y_{n}\in conv\left( \underset{i=n}{\overset{\infty }{\cup }}\left\{
x_{i}\right\} \right) ,s.t.\left\Vert y_{n}-x\right\Vert \leq \frac{1}{n}.$

$\implies y_{n}\rightarrow x$ strongly.

$\left( 2\right) $ First we can show that $\underset{n=1}{\overset{\infty }{%
\cup }}conv\left( \underset{i=1}{\overset{n}{\cup }}\left\{ x_{i}\right\}
\right) =conv\left( \underset{i=1}{\overset{\infty }{\cup }}\left\{
x_{i}\right\} \right) .$

Then we take the weak closure $\overline{\underset{n=1}{\overset{\infty }{%
\cup }}conv\left( \underset{i=1}{\overset{n}{\cup }}\left\{ x_{i}\right\}
\right) }^{\sigma \left( E,E^{\ast }\right) }=\overline{conv\left( \underset{%
i=n}{\overset{\infty }{\cup }}\left\{ x_{i}\right\} \right) }^{\sigma \left(
E,E^{\ast }\right) }$

$\implies x\in $ $\overline{conv\left( \underset{i=n}{\overset{\infty }{\cup 
}}\left\{ x_{i}\right\} \right) }=$ $\overline{\underset{n=1}{\overset{%
\infty }{\cup }}conv\left( \underset{i=1}{\overset{n}{\cup }}\left\{
x_{i}\right\} \right) },$therefore we can find $z_{n_{k}}\in conv\left( 
\underset{i=1}{\overset{n_{k}}{\cup }}\left\{ x_{i}\right\} \right)
,s.t.n_{k+1}>n_{k},$and $\left\Vert x-z_{n_{k}}\right\Vert \leq \frac{1}{k}%
\implies z_{n_{k}}\rightarrow x$ strongly.

Then we let $z_{i}=z_{n_{k}}\in conv\left( \underset{i=1}{\overset{i}{\cup }}%
\left\{ x_{i}\right\} \right) ,for$ $n_{k}\leq i<n_{k+1},$ $%
z_{1},..z_{n_{1}-1}$ can be chosen arbitrarily$\implies z_{n}\rightarrow x$
strongly.

3.5 Assume $x_{n}\nrightarrow x$ strongly.Then we can find $x_{n_{k}}\in
K,s.t.\left\Vert x_{n_{k}}-x\right\Vert >\epsilon _{0},$

By H-B Thm, $\forall x\notin \overline{B\left( x,\epsilon _{0}\right) }$
there exists $\varphi _{x}\in E^{\prime },s.t.\varphi _{x}\left( y\right)
<\alpha _{x}<\varphi _{x}\left( x\right) ,\forall y\in \overline{B\left(
x,\epsilon _{0}\right) }.$

$\underset{x\in K,x\notin \overline{B\left( x,\epsilon _{0}\right) }}{\cup }%
\left\{ y|\alpha _{x}<\varphi _{x}\left( y\right) \right\} \cup \overline{%
B\left( x,\epsilon _{0}\right) }$ is an open covering of K$\implies $

there exists a finite sub-covering $\underset{i=1}{\overset{m}{\cup }}%
\left\{ y|\alpha _{x_{\left( i\right) }}<\varphi _{x_{\left( i\right)
}}\left( y\right) \right\} \cup \overline{B\left( x,\epsilon _{0}\right) }%
\supset K.$

$\left\{ x_{n_{k}}\right\} \subset \underset{i=1}{\overset{m}{\cup }}\left\{
y|\alpha _{x_{\left( i\right) }}<\varphi _{x_{\left( i\right) }}\left(
y\right) \right\} ,$therefore there is an infinite term in at least one set.

Suppose $\left\{ x_{n_{k_{j}}}\right\} \subset \left\{ y|\alpha _{x_{\left(
1\right) }}<\varphi _{x_{\left( 1\right) }}\left( y\right) \right\} ,\varphi
_{x_{\left( 1\right) }}\left( x_{n_{k_{j}}}\right) $ $\rightarrow \varphi
_{x_{\left( 1\right) }}\left( x\right) ,$since $x_{n}\rightarrow x$ weakly.

$\implies \alpha _{x_{\left( 1\right) }}\leq \varphi _{x_{\left( 1\right)
}}\left( x\right) ,$which is a contradiction with $\varphi _{x_{\left(
1\right) }}\left( y\right) <\alpha _{x_{\left( 1\right) }}\forall y\in 
\overline{B\left( x,\epsilon _{0}\right) }.$

\end{document}
