
\documentclass{article}
\usepackage{amssymb}
\usepackage{amsmath}

%%%%%%%%%%%%%%%%%%%%%%%%%%%%%%%%%%%%%%%%%%%%%%%%%%%%%%%%%%%%%%%%%%%%%%%%%%%%%%%%%%%%%%%%%%%%%%%%%%%%%%%%%%%%%%%%%%%%%%%%%%%%%%%%%%%%%%%%%%%%%%%%%%%%%%%%%%%%%%%%%%%%%%%%%%%%%%%%%%%%%%%%%%%%%%%%%%%%%%%%%%%%%%
\def\TEXTsymbol#1{\mbox{$#1$}}%
\def\NEG#1{\leavevmode\hbox{\rlap{\thinspace/}{$#1$}}}%
\def\QATOPD#1#2#3#4{{#3 \atopwithdelims#1#2 #4}}%
\def\QTP#1{}
\def\func#1{\mathop{\rm #1}}%
%TCIDATA{Version=5.00.0.2552}
%TCIDATA{<META NAME="SaveForMode" CONTENT="1">}
%TCIDATA{Created=Tuesday, September 22, 2015 07:33:08}
%TCIDATA{LastRevised=Tuesday, September 22, 2015 17:29:24}
%TCIDATA{<META NAME="GraphicsSave" CONTENT="32">}
%TCIDATA{<META NAME="DocumentShell" CONTENT="Scientific Notebook\Blank Document">}
%TCIDATA{CSTFile=Math with theorems suppressed.cst}
%TCIDATA{PageSetup=72,72,72,72,0}
%TCIDATA{AllPages=
%F=36,\PARA{038<p type="texpara" tag="Body Text" >\hfill \thepage}
%}


\newtheorem{theorem}{Theorem}
\newtheorem{acknowledgement}[theorem]{Acknowledgement}
\newtheorem{algorithm}[theorem]{Algorithm}
\newtheorem{axiom}[theorem]{Axiom}
\newtheorem{case}[theorem]{Case}
\newtheorem{claim}[theorem]{Claim}
\newtheorem{conclusion}[theorem]{Conclusion}
\newtheorem{condition}[theorem]{Condition}
\newtheorem{conjecture}[theorem]{Conjecture}
\newtheorem{corollary}[theorem]{Corollary}
\newtheorem{criterion}[theorem]{Criterion}
\newtheorem{definition}[theorem]{Definition}
\newtheorem{example}[theorem]{Example}
\newtheorem{exercise}[theorem]{Exercise}
\newtheorem{lemma}[theorem]{Lemma}
\newtheorem{notation}[theorem]{Notation}
\newtheorem{problem}[theorem]{Problem}
\newtheorem{proposition}[theorem]{Proposition}
\newtheorem{remark}[theorem]{Remark}
\newtheorem{solution}[theorem]{Solution}
\newtheorem{summary}[theorem]{Summary}
\newenvironment{proof}[1][Proof]{\noindent\textbf{#1.} }{\ \rule{0.5em}{0.5em}}


\begin{document}


%\FRA

$\left( 1\right) $for every $x_{1},x_{2}\in L^{2}\left[ a,b\right] $ and $%
s\in \left[ a,b\right] ,\mu \in R.$

$T\left( x_{1}+\mu x_{2}\right) \left( s\right) =\int_{a}^{b}K\left(
s,t\right) \left( x_{1}\left( t\right) +\mu x_{2}\left( t\right) \right)
dt=\int_{a}^{b}K\left( s,t\right) x_{1}\left( t\right) dt+\mu
\int_{a}^{b}K\left( s,t\right) x_{2}\left( t\right) dt=T\left( x_{1}\right)
\left( s\right) +\mu T\left( x_{2}\right) \left( s\right) $

$\rightarrow T\left( x_{1}+x_{2}\right) =T\left( x_{1}\right) +\mu T\left(
x_{2}\right) .\rightarrow T$ is linear.

The norm in $L^{2}\left[ a,b\right] $ is defined as $\left\Vert x\right\Vert
=\sqrt{\int_{a}^{b}x^{2}\left( t\right) dt}.$

By Cauchy-Inequality,

$\left\vert T\left( x\right) \left( s\right) \right\vert =\left\vert
\int_{a}^{b}K\left( s,t\right) x\left( t\right) dt\right\vert \leq \sqrt{%
\int_{a}^{b}\int_{a}^{b}K^{2}\left( s,t\right) dtds}\left\Vert x\right\Vert $

$\rightarrow \left\Vert T\left( x\right) \right\Vert \leq \sqrt{%
\int_{a}^{b}\int_{a}^{b}K^{2}\left( s,t\right) dtds}\left\Vert x\right\Vert ,
$

$\rightarrow T$ is bounded.

$\left( 2\right) $ By the definition of the norm of functional, $\left\Vert
T\right\Vert =\underset{\underset{\left\Vert x\right\Vert \neq 0}{x\in L^{2}%
\left[ a,b\right] }}{\sup }\frac{\left\Vert T\left( x\right) \right\Vert }{%
\left\Vert x\right\Vert }\leq \sqrt{\int_{a}^{b}\int_{a}^{b}K^{2}\left(
s,t\right) dtds}.$

%\FRA

Solution:$\left\Vert T\right\Vert =\underset{x\in R^{n}}{\underset{%
\left\Vert x\right\Vert _{\ell ^{1}}=1}{\sup }}\left\Vert T\left( x\right)
\right\Vert _{\ell ^{1}}=\underset{x_{i}\in R}{\underset{\underset{i=1}{%
\overset{n}{\sum }}\left\vert x_{i}\right\vert =1}{\sup }}\left\vert 
\underset{i=1}{\overset{n}{\sum }}x_{i}T\left( e_{i}\right) \right\vert ,$%
where $e_{i}=\left( 0,...\underset{ith\text{ pos}}{\underbrace{1}}%
,...,0\right) .$

$\rightarrow \left\Vert T\right\Vert =\underset{1\leq i\leq n}{\max }%
\left\vert T\left( e_{i}\right) \right\vert .$

%\FRA

Proof: Let $x=\left( x_{1},...,x_{n},...\right) ,y=\left(
y_{1},...,y_{n},...\right) ,\mu \in R.\qquad $

$\qquad \qquad \qquad P_{n}\left( x+\mu y\right) =\left( x_{1}+\mu
y_{1},...,x_{n}+\mu y_{n},0,0,...\right) $

$=\left( x_{1},...,x_{n},0,0,...\right) +\mu \left(
y_{1},...,y_{n},0,0,...\right) =P_{n}\left( x\right) +\mu P_{n}\left(
y\right) .$

For $\left\Vert x\right\Vert _{\ell ^{2}}=1,$that is $\underset{i=1}{\overset%
{\infty }{\sum }}x_{i}^{2}=1,\left\Vert P_{n}\left( x\right) \right\Vert
_{\ell ^{2}}=\sqrt{\underset{i=1}{\overset{n}{\sum }}x_{i}^{2}}\leq 1,$

$\rightarrow P_{n}$ is linear bounded operator.

$\left\Vert P_{n}\right\Vert =\underset{\underset{i=1}{\overset{\infty }{%
\sum }}x_{i}^{2}=1}{\sup }\sqrt{\underset{i=1}{\overset{n}{\sum }}x_{i}^{2}}%
=1,$

Below we prove that $P_{n}$ is not convergent by norm.

By negation, $P_{n}\rightarrow P$ by norm then $\left\Vert P\left( x\right)
-P_{n}\left( x\right) \right\Vert _{\ell ^{2}}\rightarrow 0$ for any given $%
x\in \ell ^{2}.$

We can show that $P\left( x\right) =x,$otherwise assuming $P\left( x\right)
_{k}\neq x_{k},\left\Vert P\left( x\right) -P_{n}\left( x\right) \right\Vert
_{\ell ^{2}}\geq $ $\left\vert P\left( x\right) _{k}-x_{k}\right\vert
\nrightarrow 0.$

However,$\left\Vert P-P_{n}\right\Vert =\underset{\underset{i=1}{\overset{%
\infty }{\sum }}x_{i}^{2}=1}{\sup }\sqrt{\underset{i=n}{\overset{\infty }{%
\sum }}x_{i}^{2}}=1.$A contradiction with $\left\Vert P-P_{n}\right\Vert
\rightarrow 0.$

%\FRA

Proof: $\left\Vert A_{n}B_{n}-AB\right\Vert \leq \left\Vert A_{n}\right\Vert
\left\Vert B_{n}-B\right\Vert +\left\Vert B\right\Vert \left\Vert
A_{n}-A\right\Vert ,$

It follows from $\left\Vert A_{n}-A\right\Vert \rightarrow 0$ that there
exists $M$ such that $\left\Vert A_{n}\right\Vert \leq M.$

Hence $\left\Vert A_{n}B_{n}-AB\right\Vert \rightarrow 0.$

%\FRA

$N\left( f\right) =\left\{ x\in X:f\left( x\right) =0\right\} $ is a closed
set in $\left( X,\left\Vert \cdot \right\Vert \right) \iff N^{c}\left(
f\right) =\left\{ x\in X:f\left( x\right) \neq 0\right\} $ is an open set in 
$\left( X,\left\Vert \cdot \right\Vert \right) \iff \forall x\in N^{c}\left(
f\right) ,$ there exists a $\delta >0,s.t.y\in N^{c}\left( f\right) $
provided that $\left\Vert y-x\right\Vert \leq \delta .$

Proof: Necessity, assume f is continuous and $x\in N^{c}\left( f\right) .$%
Assume $f\left( x\right) >0$ without loss of generality.

Since f is continuous, there exists a $\delta >0,s.t.\left\vert f\left(
x\right) -f\left( y\right) \right\vert <f\left( x\right) $ provided that $%
\left\Vert y-x\right\Vert \leq \delta .$

$\rightarrow f\left( y\right) >0\rightarrow y\in N^{c}\left( f\right) $ $%
\rightarrow N\left( f\right) $ is closed from the above equivalence.

Sufficiency: we only need to show f is continuous at zero element of $X.$
Since $N\left( f\right) $ is closed,

$N_{z}\left( f\right) =\left\{ x\in X:f\left( x\right) =z\right\} $ is also
closed (either null set or the translated result of  $N\left( f\right) $
from the linearity of $f$). From the above proof we can deduce that both $%
\left\{ x\in X:f\left( x\right) >z\right\} $ and $\left\{ x\in X:f\left(
x\right) <z\right\} $ are open. Below is the brief proof of this
sub-conclusion based on the linear property of f. Assume $f\left( x\right)
>z,$then $x\in $ $N_{z}^{c}\left( f\right) ,$ and there exists a $\delta
>0,s.t.y\in N_{z}^{c}\left( f\right) $ provided that $\left\Vert
y-x\right\Vert \leq \delta .$If there exist $y_{1}$ and $y_{2},$both
belonging to $N_{z}^{c}\left( f\right) $ but $f\left( y_{1}\right) >z$ and $%
f\left( y_{2}\right) <z,$then

there exists a real number $t=\frac{z-f\left( y_{2}\right) }{f\left(
y_{1}\right) -f\left( y_{2}\right) }\in \left( 0,1\right) ,$such that $%
f\left( ty_{1}+\left( 1-t\right) y_{2}\right) =z.$

But $\left\Vert ty_{1}+\left( 1-t\right) y_{2}-x\right\Vert \leq t\left\Vert
y_{1}-x\right\Vert +\left( 1-t\right) \left\Vert y_{2}-x\right\Vert \leq
\delta ,$which means that $ty_{1}+\left( 1-t\right) y_{2}\in $ $%
N_{z}^{c}\left( f\right) ,$

A contradiction. Since  $f\left( x\right) >z,$ it follows that $f\left(
y\right) >z$ provided that $\left\Vert y-x\right\Vert \leq \delta .$ That
is, $\left\{ x\in X:f\left( x\right) >z\right\} $ is open. 

Then $\forall \epsilon >0,$ $\left\{ x\in X:-\epsilon <f\left( x\right)
<\epsilon \right\} $ is open and zero element of $X$ belongs to this set, it
follows that  f is continuous at zero element of $X.$ 

%\FRA

$\left( 1\right) \left\Vert x\right\Vert _{X}\geq 0$ and $\left\Vert
x\right\Vert _{X}=0\iff \left\vert x\left( t\right) \right\vert \equiv 0$
from the continuity of x.

$\left\Vert \mu x\right\Vert _{X}=\underset{a\leq t\leq b}{\max }\left\vert
\mu x\left( t\right) \right\vert +\underset{a\leq t\leq b}{\max }\left\vert
\mu x^{\prime }\left( t\right) \right\vert =\left\vert \mu \right\vert
\left( \underset{a\leq t\leq b}{\max }\left\vert x\left( t\right)
\right\vert +\underset{a\leq t\leq b}{\max }\left\vert x^{\prime }\left(
t\right) \right\vert \right) =\left\vert \mu \right\vert \left\Vert
x\right\Vert _{X}.$

$\left\Vert x+y\right\Vert _{X}\leq \underset{a\leq t\leq b}{\max }%
\left\vert x\left( t\right) +y\left( t\right) \right\vert +\underset{a\leq
t\leq b}{\max }\left\vert x^{\prime }\left( t\right) +y^{\prime }\left(
t\right) \right\vert \leq \underset{a\leq t\leq b}{\max }\left\vert x\left(
t\right) \right\vert +\underset{a\leq t\leq b}{\max }\left\vert y\left(
t\right) \right\vert +\underset{a\leq t\leq b}{\max }\left\vert x^{\prime
}\left( t\right) \right\vert +\underset{a\leq t\leq b}{\max }\left\vert
y^{\prime }\left( t\right) \right\vert $

=$\left\Vert x\right\Vert _{X}+\left\Vert y\right\Vert _{X}.$ As a result, $%
\left\Vert \cdot \right\Vert _{X}$ is well defined norm on $C^{1}\left[ a,b%
\right] .$

$\left( 2\right) f\left( x+\mu y\right) =\left( x+\mu y\right) ^{\prime
}\left( t_{0}\right) =x^{\prime }\left( t_{0}\right) +\mu y^{\prime }\left(
t_{0}\right) =f\left( x\right) +\mu f\left( y\right) .$

$\forall \left\Vert x\right\Vert _{X}=1,\underset{a\leq t\leq b}{\max }%
\left\vert x^{\prime }\left( t\right) \right\vert \leq 1\rightarrow
\left\vert f\left( x\right) \right\vert \leq 1\rightarrow f$ is continuous
since $f$ is bounded.

$\left( 3\right) $Redefine the norm on  $C\left[ a,b\right] $ as $\left\Vert
x\right\Vert =\underset{a\leq t\leq b}{\max }\left\vert x\left( t\right)
\right\vert .\left\{ \left\vert f\left( x\right) \right\vert ,\left\Vert
x\right\Vert =1\right\} $ is not bounded, for example,

$\frac{d}{dx}\sin \left( n\frac{x-\frac{a+b}{2}}{\left( b-a\right) /2}%
\right) _{|x=\frac{a+b}{2}}=\frac{2n}{b-a}\rightarrow \infty ,$while $%
\left\vert \sin \left( n\frac{x-\frac{a+b}{2}}{\left( b-a\right) /2}\right)
\right\vert \leq 1.$As a result, $f$ is not continuous.

%\FRA

$f_{1}$ is not linear. Without loss of generality, assume $\left[ a,b\right]
=\left[ 0,2\pi \right] ,f_{1}\left( \sin x\right) =1,$

$f_{1}\left( \sqrt{3}\cos x\right) =\sqrt{3},f_{1}\left( \sin x+\sqrt{3}\cos
x\right) =2\neq f_{1}\left( \sin x\right) +f_{1}\left( \sqrt{3}\cos x\right)
.$

Assume $x_{i}\rightarrow x,$ that is $\underset{a\leq t\leq b}{\max }%
\left\vert x_{i}\left( t\right) -x\left( t\right) \right\vert \rightarrow 0.$%
Then $\left\vert f\left( x_{i}\right) -f\left( x\right) \right\vert $

$=\left\vert \underset{a\leq t\leq b}{\max }\left( x_{i}\left( t\right)
-x\left( t\right) \right) \right\vert \leq $ $\underset{a\leq t\leq b}{\max }%
\left\vert x_{i}\left( t\right) -x\left( t\right) \right\vert \rightarrow 0.$

Therefore, $f_{1}$ is a continuous functional.

\end{document}
