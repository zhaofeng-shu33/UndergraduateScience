
\documentclass{article}
\usepackage{amssymb}
\usepackage{amsmath}

%%%%%%%%%%%%%%%%%%%%%%%%%%%%%%%%%%%%%%%%%%%%%%%%%%%%%%%%%%%%%%%%%%%%%%%%%%%%%%%%%%%%%%%%%%%%%%%%%%%%%%%%%%%%%%%%%%%%%%%%%%%%%%%%%%%%%%%%%%%%%%%%%%%%%%%%%%%%%%%%%%%%%%%%%%%%%%%%%%%%%%%%%%%%%%%%%%%%%%%%%%%%%%
\def\TEXTsymbol#1{\mbox{$#1$}}%
\def\NEG#1{\leavevmode\hbox{\rlap{\thinspace/}{$#1$}}}%
\def\QATOPD#1#2#3#4{{#3 \atopwithdelims#1#2 #4}}%
\def\QTP#1{}
\def\func#1{\mathop{\rm #1}}%
%TCIDATA{Version=5.00.0.2552}
%TCIDATA{<META NAME="SaveForMode" CONTENT="1">}
%TCIDATA{Created=Wednesday, October 21, 2015 22:33:14}
%TCIDATA{LastRevised=Wednesday, October 21, 2015 22:40:35}
%TCIDATA{<META NAME="GraphicsSave" CONTENT="32">}
%TCIDATA{<META NAME="DocumentShell" CONTENT="Scientific Notebook\Blank Document">}
%TCIDATA{CSTFile=Math with theorems suppressed.cst}
%TCIDATA{PageSetup=72,72,72,72,0}
%TCIDATA{AllPages=
%F=36,\PARA{038<p type="texpara" tag="Body Text" >\hfill \thepage}
%}


\newtheorem{theorem}{Theorem}
\newtheorem{acknowledgement}[theorem]{Acknowledgement}
\newtheorem{algorithm}[theorem]{Algorithm}
\newtheorem{axiom}[theorem]{Axiom}
\newtheorem{case}[theorem]{Case}
\newtheorem{claim}[theorem]{Claim}
\newtheorem{conclusion}[theorem]{Conclusion}
\newtheorem{condition}[theorem]{Condition}
\newtheorem{conjecture}[theorem]{Conjecture}
\newtheorem{corollary}[theorem]{Corollary}
\newtheorem{criterion}[theorem]{Criterion}
\newtheorem{definition}[theorem]{Definition}
\newtheorem{example}[theorem]{Example}
\newtheorem{exercise}[theorem]{Exercise}
\newtheorem{lemma}[theorem]{Lemma}
\newtheorem{notation}[theorem]{Notation}
\newtheorem{problem}[theorem]{Problem}
\newtheorem{proposition}[theorem]{Proposition}
\newtheorem{remark}[theorem]{Remark}
\newtheorem{solution}[theorem]{Solution}
\newtheorem{summary}[theorem]{Summary}
\newenvironment{proof}[1][Proof]{\noindent\textbf{#1.} }{\ \rule{0.5em}{0.5em}}


\begin{document}


酉空间Cauthy Inequality Proof:

$<u_{1},u_{2}>=\overline{<u_{2},u_{1}>},$

$<tu_{1},u_{2}>=t<u_{1},u_{2}>,t\in \NEG{C}.$

$\left\Vert u_{1}+tu_{2}\right\Vert ^{2}\geq 0\implies t\bar{t}\left\Vert
u_{2}\right\Vert ^{2}+\bar{t}<u_{1},u_{2}>+t<u_{2},u_{1}>+\left\Vert
u_{1}\right\Vert ^{2}\geq 0$

Let $t=-\frac{<u_{1},u_{2}>}{\left\Vert u_{2}\right\Vert ^{2}}\in \NEG{C},$%
这是从二次函数取极%
小值的点的推广,但复%
线性空间的情况不可%
用判别式法,否则只能%
得到

$\left\vert \func{Re}<u_{1},u_{2}>\right\vert ^{2}\leq \left\Vert
u_{1}\right\Vert \left\Vert u_{2}\right\Vert .$

Then it follows that $\frac{\left\vert <u_{2},u_{1}>\right\vert ^{2}}{%
\left\Vert u_{2}\right\Vert ^{2}}-2\frac{\left\vert <u_{2},u_{1}>\right\vert
^{2}}{\left\Vert u_{2}\right\Vert ^{2}}+\left\Vert u_{1}\right\Vert ^{2}\geq
0$

$\implies \left\vert <u_{2},u_{1}>\right\vert ^{2}\leq \left\Vert
u_{1}\right\Vert \left\Vert u_{2}\right\Vert .\boxtimes $

\end{document}
