
\documentclass{article}
%%%%%%%%%%%%%%%%%%%%%%%%%%%%%%%%%%%%%%%%%%%%%%%%%%%%%%%%%%%%%%%%%%%%%%%%%%%%%%%%%%%%%%%%%%%%%%%%%%%%%%%%%%%%%%%%%%%%%%%%%%%%%%%%%%%%%%%%%%%%%%%%%%%%%%%%%%%%%%%%%%%%%%%%%%%%%%%%%%%%%%%%%%%%%%%%%%%%%%%%%%%%%%%%%%%%%%%%%%%%%%%%%%%%%%%%%%%%%%%%%%%%%%%%%%%%
\usepackage{amssymb}
\usepackage{amsmath}

\setcounter{MaxMatrixCols}{10}
%TCIDATA{OutputFilter=LATEX.DLL}
%TCIDATA{Version=5.00.0.2552}
%TCIDATA{<META NAME="SaveForMode" CONTENT="1">}
%TCIDATA{Created=Saturday, October 03, 2015 21:18:14}
%TCIDATA{LastRevised=Wednesday, October 07, 2015 21:33:08}
%TCIDATA{<META NAME="GraphicsSave" CONTENT="32">}
%TCIDATA{<META NAME="DocumentShell" CONTENT="Scientific Notebook\Blank Document">}
%TCIDATA{CSTFile=Math with theorems suppressed.cst}
%TCIDATA{PageSetup=72,72,72,72,0}
%TCIDATA{AllPages=
%F=36,\PARA{038<p type="texpara" tag="Body Text" >\hfill \thepage}
%}


\newtheorem{theorem}{Theorem}
\newtheorem{acknowledgement}[theorem]{Acknowledgement}
\newtheorem{algorithm}[theorem]{Algorithm}
\newtheorem{axiom}[theorem]{Axiom}
\newtheorem{case}[theorem]{Case}
\newtheorem{claim}[theorem]{Claim}
\newtheorem{conclusion}[theorem]{Conclusion}
\newtheorem{condition}[theorem]{Condition}
\newtheorem{conjecture}[theorem]{Conjecture}
\newtheorem{corollary}[theorem]{Corollary}
\newtheorem{criterion}[theorem]{Criterion}
\newtheorem{definition}[theorem]{Definition}
\newtheorem{example}[theorem]{Example}
\newtheorem{exercise}[theorem]{Exercise}
\newtheorem{lemma}[theorem]{Lemma}
\newtheorem{notation}[theorem]{Notation}
\newtheorem{problem}[theorem]{Problem}
\newtheorem{proposition}[theorem]{Proposition}
\newtheorem{remark}[theorem]{Remark}
\newtheorem{solution}[theorem]{Solution}
\newtheorem{summary}[theorem]{Summary}
\newenvironment{proof}[1][Proof]{\noindent\textbf{#1.} }{\ \rule{0.5em}{0.5em}}
\input{tcilatex}

\begin{document}


\bigskip \U{8d75}\U{4e30}\qquad 2013012178\qquad \qquad Functional Analysis
Second Coursework

\FRAME{ftbpF}{4.5in}{0.9in}{0pt}{}{}{Figure}{\special{language "Scientific
Word";type "GRAPHIC";maintain-aspect-ratio TRUE;display "USEDEF";valid_file
"T";width 4.5in;height 0.9in;depth 0pt;original-width
9.6548in;original-height 2.0557in;cropleft "0";croptop "1";cropright
"1";cropbottom "0";tempfilename 'NVNF0K03.wmf';tempfile-properties "XPR";}}

1. The equality should be understood in the meaning of isomorphism.

For each $f\in X^{\prime }\times Y^{\prime },f=\left( f_{1},f_{2}\right) ,$
where $f_{1}\in X^{\prime }$ and $f_{2}\in Y^{\prime }.$

We define a mapping $F:X^{\prime }\times Y^{\prime }$ to $\left( X\times
Y\right) ^{\prime }$: $U\left( f\right) \left( x,y\right) =f_{1}\left(
x\right) +f_{2}\left( y\right) .$

\bigskip Then We can verify that $U\left( f\right) $ is a continuous linear
functional on $X\times Y.$

Indeed, $\left\Vert U\left( f\right) \right\Vert =$ $\underset{\left\Vert
\left( x,y\right) \right\Vert \leq 1}{\sup }\left\vert U\left( f\right)
\left( x,y\right) \right\vert =\underset{\left\Vert x\right\Vert +\left\Vert
y\right\Vert \leq 1}{\sup }\left\vert f_{1}\left( x\right) +f_{2}\left(
y\right) \right\vert \leq \left\Vert f_{1}\right\Vert +\left\Vert
f_{2}\right\Vert <\infty $

$U\left( f\right) \in \left( X\times Y\right) ^{\prime }$ and the mapping $F$
is well-defined.

To show $F$ is an isometry, we need show that $\underset{\left\Vert
x\right\Vert +\left\Vert y\right\Vert \leq 1}{\sup }\left\vert f_{1}\left(
x\right) +f_{2}\left( y\right) \right\vert $ =$\left\Vert \left(
f_{1},f_{2}\right) \right\Vert $

By the definition of norm on $X^{\prime }\times Y^{\prime }$, it is
equivalent to show$\underset{\left\Vert x\right\Vert +\left\Vert
y\right\Vert \leq 1}{\sup }\left\vert f_{1}\left( x\right) +f_{2}\left(
y\right) \right\vert =\max \left\{ \left\Vert f_{1}\right\Vert ,\left\Vert
f_{2}\right\Vert \right\} .$

$\underset{\left\Vert x\right\Vert +\left\Vert y\right\Vert \leq 1}{\sup }%
\left\vert f_{1}\left( x\right) +f_{2}\left( y\right) \right\vert \geq 
\underset{\left\Vert x\right\Vert \leq 1}{\sup }\left\vert f_{1}\left(
x\right) +f_{2}\left( 0\right) \right\vert =\left\Vert f_{1}\right\Vert ,$

Similarly,$\underset{\left\Vert x\right\Vert +\left\Vert y\right\Vert \leq 1}%
{\sup }\left\vert f_{1}\left( x\right) +f_{2}\left( y\right) \right\vert
\geq \left\Vert f_{2}\right\Vert .$

$\implies \underset{\left\Vert x\right\Vert +\left\Vert y\right\Vert \leq 1}{%
\sup }\left\vert f_{1}\left( x\right) +f_{2}\left( y\right) \right\vert \geq
\max \left\{ \left\Vert f_{1}\right\Vert ,\left\Vert f_{2}\right\Vert
\right\} .$

Assume $\underset{\left\Vert x\right\Vert +\left\Vert y\right\Vert \leq 1}{%
\sup }\left\vert f_{1}\left( x\right) +f_{2}\left( y\right) \right\vert
>\max \left\{ \left\Vert f_{1}\right\Vert ,\left\Vert f_{2}\right\Vert
\right\} $

Then we can find $x_{0}$ and $y_{0}$ satisfying $\left\Vert x_{0}\right\Vert
+\left\Vert y_{0}\right\Vert \leq 1,$ s.t. $\left\vert f_{1}\left(
x_{0}\right) +f_{2}\left( y_{0}\right) \right\vert >\max \left\{ \left\Vert
f_{1}\right\Vert ,\left\Vert f_{2}\right\Vert \right\} $

$\left\vert f_{1}\left( x_{0}\right) +f_{2}\left( y_{0}\right) \right\vert
\leq \left\vert f_{1}\left( x_{0}\right) \right\vert +\left\vert f_{2}\left(
y_{0}\right) \right\vert =\left\Vert x_{0}\right\Vert \left\vert f_{1}\left( 
\frac{x_{0}}{\left\Vert x_{0}\right\Vert }\right) \right\vert +\left\Vert
y_{0}\right\Vert \left\vert f_{2}\left( \frac{y_{0}}{\left\Vert
y_{0}\right\Vert }\right) \right\vert \leq \left\Vert x_{0}\right\Vert
\left\Vert f_{1}\right\Vert +\left\Vert y_{0}\right\Vert \left\Vert
f_{2}\right\Vert $

$\leq \left\Vert x_{0}\right\Vert \max \left\{ \left\Vert f_{1}\right\Vert
,\left\Vert f_{2}\right\Vert \right\} +\left\Vert y_{0}\right\Vert \max
\left\{ \left\Vert f_{1}\right\Vert ,\left\Vert f_{2}\right\Vert \right\}
\leq \left( \left\Vert x_{0}\right\Vert +\left\Vert y_{0}\right\Vert \right)
\max \left\{ \left\Vert f_{1}\right\Vert ,\left\Vert f_{2}\right\Vert
\right\} \leq \max \left\{ \left\Vert f_{1}\right\Vert ,\left\Vert
f_{2}\right\Vert \right\} $

$\implies \max \left\{ \left\Vert f_{1}\right\Vert ,\left\Vert
f_{2}\right\Vert \right\} <\max \left\{ \left\Vert f_{1}\right\Vert
,\left\Vert f_{2}\right\Vert \right\} ,$A contradiction!

Then the equality $\underset{\left\Vert x\right\Vert +\left\Vert
y\right\Vert \leq 1}{\sup }\left\vert f_{1}\left( x\right) +f_{2}\left(
y\right) \right\vert =\max \left\{ \left\Vert f_{1}\right\Vert ,\left\Vert
f_{2}\right\Vert \right\} $ holds and $F$ is an isometry.

Next we show $F$ is onto.

If $g\in \left( X\times Y\right) ^{\prime },$then $g$ is a continuous linear
functional on $X\times Y.$

$f_{1}\left( x\right) :=g(x,0),$then $f_{1}\in X^{\prime };$

$f_{2}\left( y\right) :=g(0,y),$then $f_{2}\in Y^{\prime };$

Let $f=\left( f_{1},f_{2}\right) .$

$Uf\left( x,y\right) =f_{1}\left( x\right) +f_{2}\left( y\right) =g\left(
x,0\right) +g\left( 0,y\right) =g(x,y).$

$\implies Uf=g.$

As a result,$F$ is an isometrically isomorphic map. And $X^{\prime }\times
Y^{\prime }\overset{is-is}{=}\left( X\times Y\right) ^{\prime }.$

\bigskip \FRAME{ftbpF}{4.4996in}{1.9977in}{0pt}{}{}{Figure}{\special%
{language "Scientific Word";type "GRAPHIC";display "USEDEF";valid_file
"T";width 4.4996in;height 1.9977in;depth 0pt;original-width
7.8214in;original-height 4.4313in;cropleft "0";croptop "1";cropright
"1";cropbottom "0";tempfilename 'NVO3UV00.wmf';tempfile-properties "XPR";}}

\bigskip Problem 2: We define a mapping $F$ from $\left( \Pi
_{n}X_{n}^{\prime }\right) _{q}$ to $\left( \left( \Pi _{n}X_{n}\right)
_{p}\right) ^{\prime }$

$\forall f\in \left( \Pi _{n}X_{n}^{\prime }\right) _{q},f=\left(
f_{n}\right) ,f_{n}\in X_{n}^{\prime }$ and $\left\Vert f\right\Vert
_{q}=\left( \overset{\infty }{\underset{n=1}{\sum }}\left\Vert
f_{n}\right\Vert ^{q}\right) ^{1/q}<\infty $

$\left\langle F\left( f\right) ,x\right\rangle =\overset{\infty }{\underset{%
n=1}{\sum }}\left\langle f_{n},x_{n}\right\rangle \U{ff0c} $where $x=\left(
x_{n}\right) ,x_{n}\in X_{n},\left\Vert x\right\Vert _{p}=\left( \overset{%
\infty }{\underset{n=1}{\sum }}\left\Vert x_{n}\right\Vert ^{p}\right)
^{1/p}<\infty .$

By H\"{o}lder's inequality, $\overset{\infty }{\underset{n=1}{\sum }}%
\left\vert \left\langle f_{n},x_{n}\right\rangle \right\vert \leq $ $\overset%
{\infty }{\underset{n=1}{\sum }}\left\Vert f_{n}\right\Vert \left\Vert
x_{n}\right\Vert \leq \left\Vert f\right\Vert _{q}\left\Vert x\right\Vert
_{p}<\infty .$

And obviously $F\left( f\right) $ is a linear functional on $\left( \Pi
_{n}X_{n}\right) _{p}.$

To show that $F\left( f\right) $ is bounded, $\left\vert \left\langle
F\left( f\right) ,x\right\rangle \right\vert \leq $ $\overset{\infty }{%
\underset{n=1}{\sum }}\left\vert \left\langle f_{n},x_{n}\right\rangle
\right\vert \leq \left\Vert f\right\Vert _{q}\left\Vert x\right\Vert
_{p}=\left\Vert f\right\Vert _{q},$for any $\left\Vert x\right\Vert _{p}=1.$

$\implies \left\Vert F\left( f\right) \right\Vert \leq \left\Vert
f\right\Vert _{q}$ and $F\left( f\right) $ is a continuous linear functional
on $\left( \Pi _{n}X_{n}\right) _{p}.$

$\implies F\left( f\right) \in \left( \left( \Pi _{n}X_{n}\right)
_{p}\right) ^{\prime }$ and the mapping is well defined.

Next we show that $F\left( f\right) $ preserves the norm and we only need to
show $\left\Vert F\left( f\right) \right\Vert \geq \left\Vert f\right\Vert
_{q}.$

\bigskip For given n and any $\epsilon >0,$we can find $x_{i}\in X_{i}$ $%
s.t.\left\Vert x_{i}\right\Vert =1$ and $\left\vert \left\langle
f_{i},x_{i}\right\rangle \right\vert >\left( 1-\epsilon \right) \left\Vert
f_{i}\right\Vert ,$

Let $x=\left( sgn\left( \left\langle f_{1},x_{1}\right\rangle \right)
\left\Vert f_{1}\right\Vert ^{q-1}x_{1},...,sgn\left( \left\langle
f_{n},x_{n}\right\rangle \right) \left\Vert f_{n}\right\Vert
^{q-1}x_{n},0,0...\right) .$

$\left\vert \left\langle F\left( f\right) ,x\right\rangle \right\vert \leq
\left\Vert F\left( f\right) \right\Vert \left\Vert x\right\Vert _{p}\implies 
$

$\overset{n}{\underset{i=1}{\sum }}\left\vert \left\langle
f_{i},x_{i}\right\rangle \right\vert \left\Vert f_{i}\right\Vert ^{q-1}\leq
\left\Vert F\left( f\right) \right\Vert \left( \overset{n}{\underset{i=1}{%
\sum }}\left\Vert f_{i}\right\Vert ^{p\left( q-1\right) }\right)
^{1/p}\implies $

\bigskip $\overset{n}{\underset{i=1}{\sum }}\left( 1-\epsilon \right)
\left\Vert f_{i}\right\Vert ^{q}\leq \left\Vert F\left( f\right) \right\Vert
\left( \overset{n}{\underset{i=1}{\sum }}\left\Vert f_{i}\right\Vert
^{p\left( q-1\right) }\right) ^{1/p}\implies $

$\overset{n}{\underset{i=1}{\sum }}\left( 1-\epsilon \right) \left\Vert
f_{i}\right\Vert ^{q}\leq \left\Vert F\left( f\right) \right\Vert \left( 
\overset{n}{\underset{i=1}{\sum }}\left\Vert f_{i}\right\Vert ^{q}\right)
^{1/p}\implies $

$\left( 1-\epsilon \right) \left( \overset{n}{\underset{i=1}{\sum }}%
\left\Vert f_{i}\right\Vert ^{q}\right) ^{1-\frac{1}{p}}\leq \left\Vert
F\left( f\right) \right\Vert \implies $

$\left( 1-\epsilon \right) \left( \overset{n}{\underset{i=1}{\sum }}%
\left\Vert f_{i}\right\Vert ^{q}\right) ^{1/q}\leq \left\Vert F\left(
f\right) \right\Vert $ holdes for $\forall \epsilon >0$ and n$\in \NEG%
{N}\implies $

$\left( \overset{\infty }{\underset{i=1}{\sum }}\left\Vert f_{i}\right\Vert
^{q}\right) ^{1/q}\leq \left\Vert F\left( f\right) \right\Vert ,$that is $%
\left\Vert F\left( f\right) \right\Vert \geq \left\Vert f\right\Vert _{q}.$

Next we show that $F$ is onto.

For each $g\in \left( \left( \Pi _{n}X_{n}\right) _{p}\right) ^{\prime },$%
First we show that $g\left( x\right) =\underset{n->\infty }{\lim }\overset{n}%
{\underset{i=1}{\sum }}g\left( x_{i}\right) ,$where $x=\left( x_{i}\right)
,\left\Vert x\right\Vert _{p}\leq \infty .$

This is due to the fact that $x=\underset{n->\infty }{\lim }\overset{n}{%
\underset{i=1}{\sum }}x_{i}$ and the continuity and linearity of $g.$

Define $\left\langle f_{i},x_{i}\right\rangle =g\left( x_{i}\right) ,$ and $%
f_{i}$ is obviously a continuous linear functional on $X_{i}.$

Then define $f=\left( f_{i}\right) ,$ $\left\langle F\left( f\right)
,x\right\rangle =\overset{\infty }{\underset{i=1}{\sum }}\left\langle
f_{i},x_{i}\right\rangle =g\left( x\right) ,$and $F\left( f\right) =g.$

Then we only need to verify $\left\Vert f\right\Vert _{q}<\infty .$ By the
same process in the proof of $\left\Vert F\left( f\right) \right\Vert \geq
\left\Vert f\right\Vert _{q},$

this inequality also holds for such case. Then $\left\Vert f\right\Vert
_{q}\leq \left\Vert F\left( f\right) \right\Vert =\left\Vert g\right\Vert
<\infty .$

\bigskip Then $f$ is indeed the pre-image of $g.$

$\implies F$ is an isometrically isomorphic map. And $\left( \Pi
_{n}X_{n}^{\prime }\right) _{q}$ $\overset{is-is}{=}\left( \left( \Pi
_{n}X_{n}\right) _{p}\right) ^{\prime }.$

The proof is complete.$\boxtimes $

H.Brezis \U{6559}\U{6750}\U{ff0c} 22\U{9875}\U{ff0c} \U{505a}1.9\U{9898}%
\U{3002}

\FRAME{ftbpF}{4.8404in}{2.0877in}{0pt}{}{}{Figure}{\special{language
"Scientific Word";type "GRAPHIC";display "USEDEF";valid_file "T";width
4.8404in;height 2.0877in;depth 0pt;original-width 10.933in;original-height
5.0298in;cropleft "0";croptop "1";cropright "1";cropbottom "0";tempfilename
'NVOK8000.wmf';tempfile-properties "XPR";}}

\bigskip 1. Since the dimension is finite, the normed space is complete.

To show $C_{n}$ is compact, we construct a mapping from a simplex of $\NEG%
{R}_{n}$ to $C_{n}:$

$S_{n}=\{\left( t_{1},t_{2},...,t_{n}\right) \in \NEG{R}^{n}|\overset{n}{%
\underset{i=1}{\sum }}t_{i}=1\}->C_{n}$

$f\left( t_{1},t_{2},...,t_{n}\right) =$ $\overset{n}{\underset{i=1}{\sum }}%
t_{i}x_{i}$ onto $C_{n}.$

Then $f$ is a linear continuous mapping.

Since $S_{n}$ is compact, it follows that $C_{n}$ is compact,too. 

$\left\{ x_{n}\right\} \subset \underset{i=1}{\overset{\infty }{\cup }}%
C_{i}\subset C$ since the convex set $\underset{i=1}{\overset{\infty }{\cup }%
}C_{i}$ is a subset of the convex set $C.$

Since $\left\{ x_{n}\right\} $ is dense in $C,$ it follows that $\underset{%
i=1}{\overset{\infty }{\cup }}C_{i}$ is dense in C.

$\FRAME{itbpF}{4.5in}{0.9in}{0in}{}{}{Figure}{\special{language "Scientific
Word";type "GRAPHIC";maintain-aspect-ratio TRUE;display "USEDEF";valid_file
"T";width 4.5in;height 0.9in;depth 0in;original-width
7.9321in;original-height 1.2496in;cropleft "0";croptop "1";cropright
"1";cropbottom "0";tempfilename 'NVOLAN01.wmf';tempfile-properties "XPR";}}$

$\left( 2\right) $Since $0\notin C$ and $C_{n}$ is a subset of C, 0$\notin
C_{n}.$

Using the geometric form of Hahn-Banach Thm, we can find a $f\in E^{\ast }$
that seperates $C_{n}$ and $\left\{ 0\right\} .$

We can adjust the function $f$ by dividing some constant(i.e. -1) s.t. $%
\left\langle f,x\right\rangle \geq \left\langle f,0\right\rangle =0$ for
every $x\in C_{n}$ and $\left\Vert f\right\Vert =1\left( \text{dividing the
norm itself}\right) .$Therefore $f$ is the required $f_{n}.$

$\left( 3\right) $From $\left( 2\right) $, for each n, we can find a
correspondign $f_{n},s.t.$

$\left\Vert f_{n}\right\Vert =1$ and $\left\langle f_{n},x\right\rangle \geq
0\qquad \forall x\in C_{n}\qquad \left( \ast \right) $

Notice that $C_{n}$ is an increasing set sequence,the property of $f_{n}$
holds for any $x\in C_{i},i=1,2,...n$. 

Thus we can revise $\left( \ast \right) $ as

$\left\Vert f_{n}\right\Vert =1$ and $\left\langle f_{n},x\right\rangle \geq
0\qquad \forall x\in C_{i},i\leq n\qquad \left( \ast 2\right) $

Since $E^{\ast \text{ }}$is complete, from the bounded set $\left\{
f_{n}\right\} \left( \text{where }f_{n}\text{ satisfies 2.}\right) $ we can
find a convergent subset $\left\{ f_{n_{k}}\right\} $ s.t. 

$f_{n_{k}}->f\in E^{\ast }$ as $k->\infty .$

We can first fix the set $C_{i}$ and let $n_{k}->\infty $ in $\left( \ast
2\right) ,$

by continuity we conclude that 

$\left\Vert f\right\Vert =1$ and $\left\langle f,x\right\rangle \geq 0$ $%
\forall x\in C_{i}.$

And we conclude from the arbitrary choice of i that:

$\left\Vert f\right\Vert =1$ and $\left\langle f,x\right\rangle \geq 0$ $%
\forall x\in \underset{i=1}{\overset{\infty }{\cup }}C_{i}.$

Since $\underset{i=1}{\overset{\infty }{\cup }}C_{i}$ is dense in C, also
from the continuity of f$\implies $

$\left\langle f,x\right\rangle \geq 0$ $\forall x\in C.$

$\bigskip \left( 4\right) $ from $\left( 3\right) $ it is shown that if one
of the convex set reduces to a point $\left( e.g.\left\{ 0\right\} \right) ,$
then the conclusion holds. We can follow the example in the proof of
geometric form of Hahn-Banach Thm and form a Mikowiski subtraction of two
sets:

Define $D:=A-B.$

Since $A\cap B=\varnothing ,0\notin D.$Then we can find a function $\in
E^{\ast },s.t.$

$f\left( z\right) \geq 0$ for any $z\in D.$Then

$f\left( x-y\right) \geq 0$ holds for any $x\in A$ and $y\in B.$

Let $\alpha =\underset{y\in B}{\inf }f\left( y\right) .$Since $f$ is linear, 
$f\left( x\right) \leq f\left( y\right) $ for any $x\in A$ and $y\in
B\implies $

$f\left( x\right) \leq \alpha \leq f\left( y\right) $ for any $x\in A$ and $%
y\in B$.

Then $H=\left\{ x\in E;f\left( x\right) =\alpha \right\} $ is required
hyperplane that seperates A and B.

\end{document}
