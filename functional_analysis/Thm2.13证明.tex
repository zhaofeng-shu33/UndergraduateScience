
\documentclass{article}
\usepackage{amsmath}

%%%%%%%%%%%%%%%%%%%%%%%%%%%%%%%%%%%%%%%%%%%%%%%%%%%%%%%%%%%%%%%%%%%%%%%%%%%%%%%%%%%%%%%%%%%%%%%%%%%%%%%%%%%%%%%%%%%%%%%%%%%%%%%%%%%%%%%%%%%%%%%%%%%%%%%%%%%%%%%%%%%%%%%%%%%%%%%%%%%%%%%%%%%%%%%%%%%%%%%%%%%%%%%%%%%%%%%%%%%%%%%%%%%%
%TCIDATA{OutputFilter=LATEX.DLL}
%TCIDATA{Version=5.00.0.2552}
%TCIDATA{<META NAME="SaveForMode" CONTENT="1">}
%TCIDATA{Created=Sunday, November 15, 2015 11:57:49}
%TCIDATA{LastRevised=Sunday, November 15, 2015 11:58:12}
%TCIDATA{<META NAME="GraphicsSave" CONTENT="32">}
%TCIDATA{<META NAME="DocumentShell" CONTENT="Scientific Notebook\Blank Document">}
%TCIDATA{CSTFile=Math with theorems suppressed.cst}
%TCIDATA{PageSetup=72,72,72,72,0}
%TCIDATA{AllPages=
%F=36,\PARA{038<p type="texpara" tag="Body Text" >\hfill \thepage}
%}


\newtheorem{theorem}{Theorem}
\newtheorem{acknowledgement}[theorem]{Acknowledgement}
\newtheorem{algorithm}[theorem]{Algorithm}
\newtheorem{axiom}[theorem]{Axiom}
\newtheorem{case}[theorem]{Case}
\newtheorem{claim}[theorem]{Claim}
\newtheorem{conclusion}[theorem]{Conclusion}
\newtheorem{condition}[theorem]{Condition}
\newtheorem{conjecture}[theorem]{Conjecture}
\newtheorem{corollary}[theorem]{Corollary}
\newtheorem{criterion}[theorem]{Criterion}
\newtheorem{definition}[theorem]{Definition}
\newtheorem{example}[theorem]{Example}
\newtheorem{exercise}[theorem]{Exercise}
\newtheorem{lemma}[theorem]{Lemma}
\newtheorem{notation}[theorem]{Notation}
\newtheorem{problem}[theorem]{Problem}
\newtheorem{proposition}[theorem]{Proposition}
\newtheorem{remark}[theorem]{Remark}
\newtheorem{solution}[theorem]{Solution}
\newtheorem{summary}[theorem]{Summary}
\newenvironment{proof}[1][Proof]{\noindent\textbf{#1.} }{\ \rule{0.5em}{0.5em}}
\input{tcilatex}

\begin{document}


\bigskip Thm2.12 The topological complement of G$\subset E,$is unique. That
is, if it has a complement F,s.t. G$\oplus F=E,$then F is unique.

(i) T admits a left inverse.

(ii) R(T ) = T (E) is closed and admits a complement in F.

$\left( i\right) \implies \left( ii\right) $ Suppose there exists $S\in
L\left( F,E\right) ,s.t.S\circ T=I_{E}.$

$\bigskip \forall x_{n}\in R\left( T\right) $ and $x_{n}\rightarrow x,$there
exists $y_{n}\in E,s.t.T\left( y_{n}\right) =x_{n}\implies S\circ T\left(
y_{n}\right) =y_{n}$

$\implies S\left( x_{n}\right) =y_{n},$by the continuity of $S\implies
y_{n}=S\left( x_{n}\right) \rightarrow S\left( x\right) $

by the continuity of $T\implies x_{n}=T\left( y_{n}\right) \rightarrow
T\circ S\left( x\right) \in R\left( T\right) .$

We also show that $x=T\circ S\left( x\right) $ for $x\in R\left( T\right) .$

By the continuity of $S,N\left( S\right) $ is closed. Next we show that

$F=N\left( S\right) \oplus R\left( T\right) .$

If $x\in N\left( S\right) \cap R\left( T\right) ,x=T\left( y\right) $ and $%
S\circ T\left( y\right) =S\left( x\right) =0\implies y=0\implies x=0$

$\implies N\left( S\right) \cap R\left( T\right) =\emptyset .$

$\forall x\in F,T\circ S\left( x\right) \in R\left( T\right) $ and $x-T\circ
S\left( x\right) \in N\left( S\right) $ and $F=N\left( S\right) \oplus
R\left( T\right) $ is proved.

$\left( ii\right) \implies \left( i\right) $ Suppose $F$ admits the
decomposition $F=R\left( T\right) \oplus L.$We can define a projection from
F to its subspace R$\left( T\right) $

$\forall x\in F,Px\in R\left( T\right) ,$since $T$ is injective, there
exists unique $y\in E,$

$s.t.T\left( y\right) =Px,$then we define $S\left( x\right) =y,$and this
definition is well-defined.

Further, $\forall y\in E,S\left( T\left( y\right) \right) =y\implies S\circ
T=I_{E}.$

To complete the proof, we need to show that $S$ is linear continuous. 

The linearity is trival. $\forall x\in F,x=x^{\prime }+l^{\prime },$and $%
Sx=Sx^{\prime }$ by definition of $S.$

$\implies \forall $ $x\in F,\left\Vert Sx\right\Vert _{E}\leq \underset{%
x=x^{\prime }+l^{\prime },x^{\prime }\in R\left( T\right) }{\sup }\left\Vert
Sx^{\prime }\right\Vert \leq \underset{x^{\prime }\in R\left( T\right) }{%
\sup }\left\Vert Sx^{\prime }\right\Vert $

$=\allowbreak \underset{x^{\prime }=T\left( y\right) ,y\in T}{\sup }%
\left\Vert S\circ T\left( y\right) \right\Vert \leq \underset{x^{\prime
}=T\left( y\right) ,y\in T}{\sup }\left\Vert y\right\Vert ,$we know that $T$
is bijective from $T$ to $R\left( T\right) $ and $T^{-1}$ is continuous in
this sense $\implies \left\Vert Sx\right\Vert _{E}\leq \left\Vert
T^{-1}\right\Vert \left\Vert x^{\prime }\right\Vert \leq \left\Vert
T^{-1}\right\Vert \left\Vert x\right\Vert _{F}$

$\implies S$ is continuous. 

\end{document}
