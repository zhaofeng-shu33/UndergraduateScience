
\documentclass{article}
\usepackage{amssymb}
\usepackage{amsmath}

%%%%%%%%%%%%%%%%%%%%%%%%%%%%%%%%%%%%%%%%%%%%%%%%%%%%%%%%%%%%%%%%%%%%%%%%%%%%%%%%%%%%%%%%%%%%%%%%%%%%%%%%%%%%%%%%%%%%%%%%%%%%%%%%%%%%%%%%%%%%%%%%%%%%%%%%%%%%%%%%%%%%%%%%%%%%%%%%%%%%%%%%%%%%%%%%%%%%%%%%%%%%%%
\def\TEXTsymbol#1{\mbox{$#1$}}%
\def\NEG#1{\leavevmode\hbox{\rlap{\thinspace/}{$#1$}}}%
\def\QATOPD#1#2#3#4{{#3 \atopwithdelims#1#2 #4}}%
\def\QTP#1{}
\def\func#1{\mathop{\rm #1}}%
%TCIDATA{Version=5.00.0.2552}
%TCIDATA{<META NAME="SaveForMode" CONTENT="1">}
%TCIDATA{Created=Sunday, October 18, 2015 12:30:35}
%TCIDATA{LastRevised=Sunday, October 18, 2015 20:56:46}
%TCIDATA{<META NAME="GraphicsSave" CONTENT="32">}
%TCIDATA{<META NAME="DocumentShell" CONTENT="Scientific Notebook\Blank Document">}
%TCIDATA{CSTFile=Math with theorems suppressed.cst}
%TCIDATA{PageSetup=72,72,72,72,0}
%TCIDATA{AllPages=
%F=36,\PARA{038<p type="texpara" tag="Body Text" >\hfill \thepage}
%}


\newtheorem{theorem}{Theorem}
\newtheorem{acknowledgement}[theorem]{Acknowledgement}
\newtheorem{algorithm}[theorem]{Algorithm}
\newtheorem{axiom}[theorem]{Axiom}
\newtheorem{case}[theorem]{Case}
\newtheorem{claim}[theorem]{Claim}
\newtheorem{conclusion}[theorem]{Conclusion}
\newtheorem{condition}[theorem]{Condition}
\newtheorem{conjecture}[theorem]{Conjecture}
\newtheorem{corollary}[theorem]{Corollary}
\newtheorem{criterion}[theorem]{Criterion}
\newtheorem{definition}[theorem]{Definition}
\newtheorem{example}[theorem]{Example}
\newtheorem{exercise}[theorem]{Exercise}
\newtheorem{lemma}[theorem]{Lemma}
\newtheorem{notation}[theorem]{Notation}
\newtheorem{problem}[theorem]{Problem}
\newtheorem{proposition}[theorem]{Proposition}
\newtheorem{remark}[theorem]{Remark}
\newtheorem{solution}[theorem]{Solution}
\newtheorem{summary}[theorem]{Summary}
\newenvironment{proof}[1][Proof]{\noindent\textbf{#1.} }{\ \rule{0.5em}{0.5em}}


\begin{document}


$\bigskip $泛函第五周作业 赵%
丰2013012178

$\bigskip 1.$Since $f$ maps $x_{1}\NEG{R}$ to the whole real number set, for
each $x\in E$ we can find an $\alpha $ s.t. $f\left( x\right) =\alpha
f\left( x_{1}\right) =f\left( \alpha x_{1}\right) ,$ then $f\left( x-\alpha
x_{1}\right) =0\implies x-\alpha x_{1}\in N_{f}\implies \forall x\in E,$we
can decompose $x$ as $x=\alpha x_{1}+y,$where $y\in N_{f}$ and $\alpha \in 
\NEG{R}.$

We can show the uniqueness of the decomposition.

Consider $x=\alpha _{1}x_{1}+y_{1}=\alpha _{2}x_{1}+y_{2}$

Acting $f$ on both sides yields $\alpha _{1}f\left( x_{1}\right) =\alpha
_{2}f\left( x_{1}\right) .$

Since $f\left( x_{1}\right) \neq 0,\alpha _{1}=\alpha _{2},$and $y_{1}=y_{2}$
follows. 

2. we construct a linear functional on $x_{0}\NEG{R},$ a subspace of $E$

$g\left( tx_{0}\right) =t\alpha _{0}\left\Vert x_{0}\right\Vert ,$ and it is
easy to verify that $\left\Vert g\right\Vert =\alpha _{0}.$

For a sub-linear $p\left( x\right) =\alpha _{0}\left\Vert x\right\Vert $
defined on $E,g\left( tx_{0}\right) \leq p\left( tx_{0}\right) $ and by
Hahn-Banach Thm, we can extend $g$

to the whole space $E.$We denote the extended function as $f,$ then $f\left(
x\right) \leq p\left( x\right) \implies $

$\left\vert f\left( x\right) \right\vert \leq \alpha _{0}\left\Vert
x\right\Vert \implies \left\Vert f\right\Vert \leq \alpha _{0},$but as an
extention $\left\Vert f\right\Vert \geq \left\Vert g\right\Vert =\alpha
_{0}\implies $

$\left\Vert f\right\Vert =\alpha _{0}.$And $\left\langle
f,x_{0}\right\rangle =\left\langle f,x_{0}\right\rangle =\alpha
_{0}\left\Vert x_{0}\right\Vert =\left\Vert f\right\Vert \left\Vert
x_{0}\right\Vert $

Textbook P21 Problem 7

$\left( 1\right) \forall x,y\in \bar{C}$ ,there exists two sequnce $\left\{
x_{n}\right\} ,\left\{ y_{n}\right\} \subset C$ $s.t.x_{n}\rightarrow
x,y_{n}\rightarrow y.$

Then $tx+\left( 1-t\right) y=\underset{n\rightarrow \infty }{\lim }%
tx_{n}+\left( 1-t\right) y_{n},\forall 0<t<1.$

Since C is convex, $tx_{n}+\left( 1-t\right) y_{n}\in C,$then by the closed
property of $\bar{C},tx+\left( 1-t\right) y\in \bar{C}.$

$\therefore \bar{C}$ is convex.

$\forall x,y\in IntC,$We can find a $r>0,s.t.B\left( x,r\right) \cup B\left(
y,r\right) \subset C.$

Then by the convex property of C, $t\left( x+h\right) +\left( 1-t\right)
\left( y+h\right) \in C,\forall 0<t<1$ and $h\in B\left( 0,r\right) .$

Then $\forall z=tx+\left( 1-t\right) y,$ $B\left( z,r\right) \subset C,$%
since $z+h=t\left( x+h\right) +\left( 1-t\right) \left( y+h\right) \in
C,\forall h\in B\left( 0,r\right) .$

$\implies z\in IntC$ and $IntC$ is convex.

$\left( 2\right) $Since $y\in IntC,$there exists $r>0,s.t.B\left( y,r\right)
\subset C.$Then $\forall 0<t<1$ and $h\in B\left( 0,r\right) ,$

$tx+\left( 1-t\right) \left( y+h\right) \in C$

For each $z=tx+\left( 1-t\right) y,B\left( z,r\left( 1-t\right) \right)
\subset C,$

since $z+\left( 1-t\right) h=tx+\left( 1-t\right) \left( y+h\right) \in
C,\forall h\in B\left( 0,r\right) .$

$\implies z\in IntC.$

$\left( 3\right) $We only need to show that $\bar{C}\subset $ $\overline{IntC%
}$

Since  $\overline{IntC}$ is not empty, there exists $x_{0}\in $ $\overline{%
IntC},$ and $\forall x\in C,$ by$\left( 2\right) $ $tx+\left( 1-t\right)
x_{0}\in $ $IntC.$

Let $t\rightarrow 1\implies x\in $ $\overline{IntC}.\implies C\subset $ $%
\overline{IntC}.$Taking the closure on both sides yield $\bar{C}\subset $ $%
\overline{IntC}\boxtimes $ 

\end{document}
