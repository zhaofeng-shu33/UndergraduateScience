
\documentclass{article}
%%%%%%%%%%%%%%%%%%%%%%%%%%%%%%%%%%%%%%%%%%%%%%%%%%%%%%%%%%%%%%%%%%%%%%%%%%%%%%%%%%%%%%%%%%%%%%%%%%%%%%%%%%%%%%%%%%%%%%%%%%%%%%%%%%%%%%%%%%%%%%%%%%%%%%%%%%%%%%%%%%%%%%%%%%%%%%%%%%%%%%%%%%%%%%%%%%%%%%%%%%%%%%%%%%%%%%%%%%%%%%%%%%%%%%%%%%%%%%%%%%%%%%%%%%%%
\usepackage{amssymb}
\usepackage{amsmath}

\setcounter{MaxMatrixCols}{10}
%TCIDATA{OutputFilter=LATEX.DLL}
%TCIDATA{Version=5.00.0.2552}
%TCIDATA{<META NAME="SaveForMode" CONTENT="1">}
%TCIDATA{Created=Friday, October 23, 2015 18:55:44}
%TCIDATA{LastRevised=Sunday, November 22, 2015 22:19:59}
%TCIDATA{<META NAME="GraphicsSave" CONTENT="32">}
%TCIDATA{<META NAME="DocumentShell" CONTENT="Scientific Notebook\Blank Document">}
%TCIDATA{CSTFile=Math with theorems suppressed.cst}
%TCIDATA{PageSetup=72,72,72,72,0}
%TCIDATA{AllPages=
%F=36,\PARA{038<p type="texpara" tag="Body Text" >\hfill \thepage}
%}


\newtheorem{theorem}{Theorem}
\newtheorem{acknowledgement}[theorem]{Acknowledgement}
\newtheorem{algorithm}[theorem]{Algorithm}
\newtheorem{axiom}[theorem]{Axiom}
\newtheorem{case}[theorem]{Case}
\newtheorem{claim}[theorem]{Claim}
\newtheorem{conclusion}[theorem]{Conclusion}
\newtheorem{condition}[theorem]{Condition}
\newtheorem{conjecture}[theorem]{Conjecture}
\newtheorem{corollary}[theorem]{Corollary}
\newtheorem{criterion}[theorem]{Criterion}
\newtheorem{definition}[theorem]{Definition}
\newtheorem{example}[theorem]{Example}
\newtheorem{exercise}[theorem]{Exercise}
\newtheorem{lemma}[theorem]{Lemma}
\newtheorem{notation}[theorem]{Notation}
\newtheorem{problem}[theorem]{Problem}
\newtheorem{proposition}[theorem]{Proposition}
\newtheorem{remark}[theorem]{Remark}
\newtheorem{solution}[theorem]{Solution}
\newtheorem{summary}[theorem]{Summary}
\newenvironment{proof}[1][Proof]{\noindent\textbf{#1.} }{\ \rule{0.5em}{0.5em}}
\input{tcilatex}

\begin{document}


赵丰\qquad 2013012178\qquad 泛函第六%
周作业

1.14 $\left( a\right) $ The linearity and closed property of $X$ and $Y$ are
easy to check. To show $\overline{X+Y}=E=\ell ^{1}$, note that for every
sequence $x$=$\left\{ x_{\left( n\right) }\right\} \in E$, we can construct $%
y^{\left( n\right) }=\left(
2x_{2},x_{2},4x_{4},x_{4},...2^{n}x_{2n},x_{2n},0,0...\right) \in Y$

$t^{\left( n\right) }=\left(
x_{1}-2x_{2},0,x_{3}-4x_{4},...x_{2n-1}-2^{n}x_{2n},0,0...\right) \in
X,\forall n\in N.$Further $y^{\left( n\right) }+t^{\left( n\right) }=\left(
x_{1},x_{2},...x_{2n},0,0\right) $

-\TEXTsymbol{>}$x$ as $n->\infty .\implies $ $\overline{X+Y}=E.$

2 Otherwise we have $c=x+y,$where $x\in X,y\in Y\implies y=\left( 1,\frac{1}{%
2},1,\frac{1}{4},1,\frac{1}{8},...\right) ,y\notin \ell ^{1}.$Contradiction!

3. by 2, $Y\cap Z=\emptyset .$If there exists a linear bdd functional s.t. $%
f\left( z\right) \leq \alpha \leq f\left( y\right) ,\forall z\in Z,y\in Y.$

Then from $z=x-c$ follows $f\left( x-y\right) \leq f\left( c\right) ,$by $1,$
$f\left( e\right) \leq f\left( c\right) \forall e\in E\implies f$ vanishes
identically on E.

Therefore it is impossible to seperate $Z$ and Y by a closed hyperplane.

4. The above conclusion holds for $E=\ell ^{p},1<p<\infty $ and for $%
E=c_{0}=\left\{ x=\left\{ x_{n}\right\} |x_{n}->0\right\} .$

1.15

1. By the continuity of $f\in E,C^{\ast \ast }$ is closed.

For each $x\in C$ and $f\in C^{\ast },\left\langle f,x\right\rangle \leq
1\implies x\in C^{\ast \ast }\implies C\subset C^{\ast \ast }\implies \bar{C}%
\subset C^{\ast \ast }.$

Suppose $x\in C^{\ast \ast },$and $x\notin \bar{C},$then by convex set
separation Thm,there exists $g\in E^{\ast },s.t.g\left( c\right) <1,\forall
c\in \bar{C},$but $g\left( x\right) >1,$since we can scale $g.$

$\implies g\in C^{\ast }\implies \left\langle g,x\right\rangle \leq 1,$a
contradiction.

2. If $C$ is a linear space, then $f\in C^{\ast }\implies \left\langle
f,x\right\rangle \equiv 0$ on C, otherwise $\left\langle
f,tx_{0}\right\rangle ->+\infty ,$as $\left\vert t\right\vert ->\infty ,$for 
$tx_{0}\in C\implies f\in C^{\bot }.$ Obviously,$C^{\bot }\implies C^{\ast
}. $

$\implies C^{\ast }=C^{\bot }.$

1.16

\bigskip for $x\in N^{\bot },x=\left\{ x_{n}\right\} $ with $\underset{n=1}{%
\overset{\infty }{\sum }}\left\vert x_{n}\right\vert <\infty ,\underset{n=1}{%
\overset{\infty }{\sum }}x_{n}y_{n}=0,$for any $\left\{ y_{n}\right\} \in
c_{0},$ that is $\underset{n->\infty }{\lim }y_{n}=0.$

We choose $y_{n}=\frac{x_{n}}{2^{n}}\implies x_{n}=0,$hence $N^{\bot
}=\left\{ 0\right\} .$

$N^{\bot \bot }=E^{\ast }\neq N$

1.17 $\left( 1\right) $ $M^{\bot }$ is the kernel of $I_{M^{\bot }},$where $%
I_{K}$ is the indicator function of K.

$\left( 2\right) $By Ex 3, $dist\left( x,M\right) =\underset{\underset{%
\left\Vert g\right\Vert \leq 1}{g\in M^{\bot }}}{\max }<g,x>=\underset{g\in
M^{\bot }}{\max }\frac{<g,x>}{\left\Vert g\right\Vert }.$Suppose $%
\left\langle f,x\right\rangle >0,$and $\left\Vert f\right\Vert =1,$then for
any $g\in M^{\bot }$\bigskip ,

$\left\Vert g\right\Vert =1,\left\langle f-g,x\right\rangle \neq 0,$since $%
f-g\in M^{\bot }$ and $x\notin M.$ If $\left\langle f-g,x\right\rangle <0,$%
since $M^{\bot }$ is convex, we can

find 0\TEXTsymbol{<}t\TEXTsymbol{<}1,s.t. \TEXTsymbol{<}$tf+\left(
1-t\right) \left( f-g\right) ,x>=0,$A contradiction! $\therefore
\left\langle f-g,x\right\rangle >0$ and $\left\langle f,x\right\rangle
>\left\langle g,x\right\rangle \implies $

$dist\left( x,M\right) =\underset{\underset{\left\Vert g\right\Vert \leq 1}{%
g\in M^{\bot }}}{\max }<g,x>=\frac{\left\vert <f,x>\right\vert }{\left\Vert
f\right\Vert }.$

A direct method: we may as well assume that $\left\Vert f\right\Vert =1,$to
show that $\left\vert <f,x>\right\vert =dist\left( x,M\right) ,\forall
x\notin M.$

For one $x_{0}\notin M,$ we restrict $f$ on S=span$\left\{ x_{0},M\right\} .$%
For each $y\in S,y=tx_{0}+m,$with $m\in M.$

$f\left( y\right) =tf\left( x_{0}\right) ,0\neq \left\vert \left\langle
f,x_{0}\right\rangle \right\vert =\left\vert \left\langle
f,x_{0}-m\right\rangle \right\vert \leq \left\Vert f\right\Vert \left\vert
x_{0}-m\right\vert \leq dist\left( x_{0},M\right) .$

On the other hand, 1=$\left\Vert f\right\Vert =\underset{n\rightarrow \infty 
}{\lim }\frac{\left\langle f,x_{0}-m_{n}\right\rangle }{\left\Vert
x_{0}-m_{n}\right\Vert }=\underset{n\rightarrow \infty }{\lim }\frac{%
\left\langle f,x_{0}\right\rangle }{\left\Vert x_{0}-m_{n}\right\Vert }\leq 
\frac{\left\langle f,x_{0}\right\rangle }{dist\left( x_{0},M\right) }%
\implies dist\left( x_{0},M\right) \leq \left\langle f,x_{0}\right\rangle $

Then by the conclusion in the first problem of Coursework 4,\FRAME{ftbpF}{%
4.5in}{0.9in}{0in}{}{}{Figure}{\special{language "Scientific Word";type
"GRAPHIC";maintain-aspect-ratio TRUE;display "USEDEF";valid_file "T";width
4.5in;height 0.9in;depth 0in;original-width 9.5432in;original-height
1.8057in;cropleft "0";croptop "1";cropright "1";cropbottom "0";tempfilename
'NWRPJ306.wmf';tempfile-properties "XPR";}}

S=E, and $\left\vert <f,x>\right\vert =dist\left( x,M\right) $ is proved
directly.

the norm on $E$ is defined as $\left\Vert u\right\Vert =\underset{0\leq
t\leq 1}{\max }\left\vert u\left( t\right) \right\vert ,$ $\left\vert
\left\langle f,u\right\rangle \right\vert \leq \left\Vert u\right\Vert
\implies \left\Vert f\right\Vert \leq 1.$

for $u_{n}=\QATOPD\{ . {nx,0\leq x\leq \frac{1}{n}}{1,\frac{1}{n}<x\leq
1},\left\Vert u_{n}\right\Vert =1,\left\langle f,u_{n}\right\rangle =1-\frac{%
1}{2n}->1\implies \left\Vert f\right\Vert =\underset{n\dot{-}\infty }{\lim }%
\left\langle f,u_{n}\right\rangle =1.$

Then by 2. dist$\left( u,M\right) =\left\vert \int_{0}^{1}u\left( t\right)
dt\right\vert .$

If there exists $v\in M,$that is $\int_{0}^{1}v\left( t\right)
dt=0,s.t.\left\Vert u-v\right\Vert =\left\vert \int_{0}^{1}u\left( t\right)
dt\right\vert \neq 0\iff \underset{0\leq t\leq 1}{\max }\left\vert u\left(
t\right) -v\left( t\right) \right\vert =\left\vert \int_{0}^{1}u\left(
t\right) dt\right\vert ,$

then $\left\vert u\left( t\right) -v\left( t\right) \right\vert \leq
\left\vert \int_{0}^{1}u\left( t\right) dt\right\vert $

$u\left( t\right) -v\left( t\right) \leq \left\vert \int_{0}^{1}u\left(
t\right) dt\right\vert $ and $v\left( t\right) -u\left( t\right) \leq
\left\vert \int_{0}^{1}u\left( t\right) dt\right\vert .$

Integrating the above two inequalities from 0 to 1 gives

$\int_{0}^{1}v\left( t\right) dt\geq \left\vert \int_{0}^{1}u\left( t\right)
dt\right\vert -\int_{0}^{1}u\left( t\right) dt$ and $\int_{0}^{1}v\left(
t\right) dt\leq \left\vert \int_{0}^{1}u\left( t\right) dt\right\vert
+\int_{0}^{1}u\left( t\right) dt.$

If $\int_{0}^{1}u\left( t\right) dt>0,\int_{0}^{1}v\left( t\right) dt\geq 0,$
and the equality holds only if $u\left( t\right) -v\left( t\right)
=\int_{0}^{1}u\left( s\right) ds,$then we let t=0,

$\int_{0}^{1}u\left( t\right) dt=0,$a contradiction. Similar result follows
from $\int_{0}^{1}u\left( t\right) dt<0.$

In conclusion, $\underset{v\in M}{\inf }\left\Vert u-v\right\Vert $ is never
achieved for any $u\in E\backslash M.$

\bigskip

\bigskip

\bigskip

\end{document}
