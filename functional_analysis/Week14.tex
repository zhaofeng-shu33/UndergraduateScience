
\documentclass{article}
\usepackage{amssymb}
\usepackage{amsmath}

%%%%%%%%%%%%%%%%%%%%%%%%%%%%%%%%%%%%%%%%%%%%%%%%%%%%%%%%%%%%%%%%%%%%%%%%%%%%%%%%%%%%%%%%%%%%%%%%%%%%%%%%%%%%%%%%%%%%%%%%%%%%%%%%%%%%%%%%%%%%%%%%%%%%%%%%%%%%%%%%%%%%%%%%%%%%%%%%%%%%%%%%%%%%%%%%%%%%%%%%%%%%%%
%TCIDATA{OutputFilter=LATEX.DLL}
%TCIDATA{Version=5.00.0.2552}
%TCIDATA{<META NAME="SaveForMode" CONTENT="1">}
%TCIDATA{Created=Friday, December 18, 2015 18:24:56}
%TCIDATA{LastRevised=Friday, December 18, 2015 23:54:37}
%TCIDATA{<META NAME="GraphicsSave" CONTENT="32">}
%TCIDATA{<META NAME="DocumentShell" CONTENT="Scientific Notebook\Blank Document">}
%TCIDATA{CSTFile=Math with theorems suppressed.cst}
%TCIDATA{PageSetup=72,72,72,72,0}
%TCIDATA{AllPages=
%F=36,\PARA{038<p type="texpara" tag="Body Text" >\hfill \thepage}
%}


\newtheorem{theorem}{Theorem}
\newtheorem{acknowledgement}[theorem]{Acknowledgement}
\newtheorem{algorithm}[theorem]{Algorithm}
\newtheorem{axiom}[theorem]{Axiom}
\newtheorem{case}[theorem]{Case}
\newtheorem{claim}[theorem]{Claim}
\newtheorem{conclusion}[theorem]{Conclusion}
\newtheorem{condition}[theorem]{Condition}
\newtheorem{conjecture}[theorem]{Conjecture}
\newtheorem{corollary}[theorem]{Corollary}
\newtheorem{criterion}[theorem]{Criterion}
\newtheorem{definition}[theorem]{Definition}
\newtheorem{example}[theorem]{Example}
\newtheorem{exercise}[theorem]{Exercise}
\newtheorem{lemma}[theorem]{Lemma}
\newtheorem{notation}[theorem]{Notation}
\newtheorem{problem}[theorem]{Problem}
\newtheorem{proposition}[theorem]{Proposition}
\newtheorem{remark}[theorem]{Remark}
\newtheorem{solution}[theorem]{Solution}
\newtheorem{summary}[theorem]{Summary}
\newenvironment{proof}[1][Proof]{\noindent\textbf{#1.} }{\ \rule{0.5em}{0.5em}}


\begin{document}


\bigskip \bigskip \bigskip 赵丰\qquad 2013012178\qquad \qquad
Functional Analysis 14周 Coursework

5.22 1 Since $C$ is closed convex in the strong topology, also in the weak
topology$\implies u\in C$ and $Tu\in C$ by the definition of operator $T.$

$\left\Vert Tu-Tu_{n}\right\Vert \leq \left\Vert u-u_{n}\right\Vert ,$ since 
$u_{n}\overset{w}{\rightarrow }u,Tu_{n}\overset{w}{\rightarrow }Tu$ by the
definition of weak topology. $\implies u_{n}-Tu_{n}\overset{w}{\rightarrow }%
u-Tu.$ We also have $u_{n}-Tu_{n}\overset{s}{\rightarrow }f.$ also weakly.
Then by the uniqueness of limit under weak topology follows $u-Tu=f.$

2. We define $T_{1/n}\left( u\right) =\left( 1-\epsilon \right) Tu+a\epsilon
,$ where $a\in C.$ The existence of $a$ is to gurantee that $T_{1/n}$ is
from $C$ to $C.$ since $C$ is convex. then for any $v_{1},v_{2}.$

we have $\left\Vert T_{1/n}\left( v_{1}\right) -T_{1/n}\left( v_{2}\right)
\right\Vert \leq \left( 1-\epsilon \right) \left\Vert v_{1}-v_{2}\right\Vert
.$Then by Banach fixed point theorem we can find $u_{n},s.t.T_{1/n}\left(
u_{n}\right) =u_{n}.$ Notice that $C$ is bounded. As a result, the sequence $%
\left\{ u_{n}\right\} $ is bounded. Then we can find a subsequence

$\left\{ u_{n_{k}}\right\} $ $s.t.u_{n_{k}}\overset{w}{\rightarrow }u\in C.$ 
$u_{n_{k}}-Tu_{n_{k}}=-\epsilon Tu_{k}+a\epsilon \rightarrow 0$ strongly.
Then by 1,

we have $Tu=u.$That is,$T$ has a fixed point.

5.23

$\left\vert T\sigma -\underset{i=1}{\overset{n}{\sum }}\alpha
_{i}Tu_{i}\right\vert ^{2}=\left\vert \underset{i=1}{\overset{n}{\sum }}%
\alpha _{i}\left( T\sigma -Tu_{i}\right) \right\vert ^{2}=\underset{i,j=1}{%
\overset{n}{\sum }}\alpha _{i}\alpha _{j}\left\langle T\sigma
-Tu_{i},T\sigma -Tu_{j}\right\rangle $

$=\frac{1}{2}\underset{i,j=1}{\overset{n}{\sum }}\alpha _{i}\alpha
_{j}[\left\Vert T\sigma -Tu_{i}\right\Vert ^{2}+\left\Vert T\sigma
-Tu_{j}\right\Vert ^{2}-\left\Vert Tu_{i}-Tu_{j}\right\Vert ^{2}]$ by cosine
Thm and $\alpha _{i}\geq 0.$

Thus we only need to show that 

$\underset{i,j=1}{\overset{n}{\sum }}\alpha _{i}\alpha _{j}\left( \left\Vert
T\sigma -Tu_{i}\right\Vert ^{2}+\left\Vert T\sigma -Tu_{j}\right\Vert
^{2}\right) \leq \underset{i,j=1}{\overset{n}{\sum }}\alpha _{i}\alpha
_{j}\left\Vert u_{i}-u_{j}\right\Vert ^{2}$

$\bigskip $Since $T$ is a contraction,

$\underset{i,j=1}{\overset{n}{\sum }}\alpha _{i}\alpha _{j}\left( \left\Vert
T\sigma -Tu_{i}\right\Vert ^{2}+\left\Vert T\sigma -Tu_{j}\right\Vert
^{2}\right) \leq \underset{i,j=1}{\overset{n}{\sum }}\alpha _{i}\alpha
_{j}\left( \left\Vert \sigma -u_{i}\right\Vert ^{2}+\left\Vert \sigma
-u_{j}\right\Vert ^{2}\right) $

$=\underset{i,j=1}{\overset{n}{\sum }}\alpha _{i}\alpha _{j}\left(
2\left\Vert \sigma \right\Vert ^{2}+\left\Vert u_{i}\right\Vert
^{2}+\left\Vert u_{j}\right\Vert ^{2}-2\left\langle \sigma
,u_{i}\right\rangle -2\left\langle \sigma ,u_{j}\right\rangle \right) $

$=-2\left\Vert \sigma \right\Vert ^{2}+\underset{i,j=1}{\overset{n}{\sum }}%
\alpha _{i}\alpha _{j}\left( \left\Vert u_{i}\right\Vert ^{2}+\left\Vert
u_{i}\right\Vert ^{2}\right) $

$=\underset{i,j=1}{\overset{n}{\sum }}\alpha _{i}\alpha _{j}\left(
\left\Vert u_{i}\right\Vert ^{2}+\left\Vert u_{i}\right\Vert
^{2}-2\left\langle u_{1},u_{2}\right\rangle \right) ,$since the term is zero
if $i=j$

$\underset{i,j=1}{\overset{n}{\sum }}\alpha _{i}\alpha _{j}\left\Vert
u_{i}-u_{j}\right\Vert ^{2}.\boxtimes $

The equality is reached when T is an isometry.

5.25 1. For each $u_{n},$there exists $v_{n}\in K,s.t.dist\left(
u_{n},K\right) =\left\Vert u_{n}-v_{n}\right\Vert ,$ since $K$ is close
convex. Then $dist\left( u_{1},K\right) \leq \left\Vert
u_{1}-v_{2}\right\Vert \leq \left\Vert u_{2}-v_{2}\right\Vert =dist\left(
u_{2},K\right) $

$\leq \left\Vert u_{2}-v_{3}\right\Vert \leq \left\Vert
u_{3}-v_{3}\right\Vert =\left\Vert u_{2}-v_{2}\right\Vert ..$

Thus the sequence $dist\left( u_{n},K\right) $ is nonincreasing.

2. From 1 $v_{n}=P_{K}u_{n},$ From Ex5.4

$\bigskip \left\Vert v_{m}-v_{n}\right\Vert \leq \left\Vert
u_{n}-v_{m}\right\Vert -\left\Vert u_{n}-v_{n}\right\Vert ,$if $m\leq n$

$\leq \left\Vert u_{m}-v_{m}\right\Vert -\left\Vert u_{n}-v_{n}\right\Vert ,$
if $m>n$ we can rewrite 

$\left\Vert v_{m}-v_{n}\right\Vert \leq \left\Vert u_{m}-v_{n}\right\Vert
-\left\Vert u_{m}-v_{m}\right\Vert \leq \left\Vert u_{n}-v_{n}\right\Vert
-\left\Vert u_{m}-v_{m}\right\Vert $

$\implies \left\Vert v_{m}-v_{n}\right\Vert \leq \left\vert dist\left(
u_{m},K\right) -dist\left( u_{n},K\right) \right\vert .$From 1, $dist\left(
u_{m},K\right) $ converges.

$\implies \left\{ v_{n}\right\} $ is a Cauchy sequence and therefore $%
\exists v,s.t.v_{n}\rightarrow v\in K,$since $K$ is closed.

3. It is obvious that $\left\{ u_{n}\right\} $ is bounded and therefore
admits a weakly convergent subsequence $\left\{ u_{n_{k}}\right\} $. By the
property of projection and $\bar{u}\in K,$

$\left\langle \bar{u}-v_{n_{k}},u_{n_{k}}-v_{n_{k}}\right\rangle \leq 0.$
Taking the limit $k\rightarrow \infty ,$we get

$\left\langle \bar{u}-l,\bar{u}-l\right\rangle \leq 0\implies \bar{u}=l$

If $u_{n}\overset{w}{\nrightarrow }l,$then we can find $f\in H$ and a
subsequence $\left\{ u_{n_{t}}\right\} ,s.t$

$\left\vert \left\langle f,u_{n_{t}}-l\right\rangle \right\vert >\epsilon
_{0}\forall t,$for $\bigskip \left\{ u_{n_{t}}\right\} ,\left( P\right) $
holds and it admits a subsequence which converges weakly to $l.$But this
contradicts with $\left\vert \left\langle f,u_{n_{t}}-l\right\rangle
\right\vert >\epsilon _{0}.$

4. Since $\underset{n\rightarrow \infty }{\lim }\left\Vert
u_{n}-v\right\Vert $ exists $\forall v\in K,\left\langle
u_{n},w-v\right\rangle =\frac{\left\Vert u_{n}+w\right\Vert ^{2}-\left\Vert
u_{n}+v\right\Vert ^{2}-\left\Vert w\right\Vert ^{2}-\left\Vert v\right\Vert
^{2}}{2}$ 

converges as $n\rightarrow \infty .$ Thus the function $\varphi \left(
x\right) :H=\cup _{\lambda >0}\left( K-K\right) \rightarrow R$

definded by $\varphi \left( x\right) =\underset{n\rightarrow \infty }{\lim }%
\left\langle u_{n},x\right\rangle $ exists. since $x=\lambda ^{\prime
}\left( w-v\right) ,$

for some $w,v\in K.$ $\varphi $ is linear and bounded, since $\left\{
u_{n}\right\} $ is bounded. $\implies \varphi \in H^{\prime }.$Then by Riesz
Representation Thm, $\exists u\in H,s.t.\varphi \left( x\right)
=\left\langle u,x\right\rangle $

$\implies u_{n}\overset{w}{\rightarrow }u.$From $\left\langle
v-v_{n},u_{n}-v_{n}\right\rangle \leq 0\forall v\in K\implies $

$\left\langle v-l,u_{n}-v_{n}\right\rangle +\left\langle
l-v_{n},u_{n}-v_{n}\right\rangle \rightarrow \left\langle
v-l,u-l\right\rangle ,$ since $v_{n}\overset{s}{\rightarrow }v$

$\implies $\bigskip $\left\langle v-l,u-l\right\rangle \leq 0.$By the
uniqueness of the projection follows that $l=P_{K}u.$

5. By translation and dilation we can assume that $K$ contains a closed unit
sphere $B_{H}.$ Since $0\in B_{H}\subset K\implies \left\Vert
u_{n}\right\Vert $ decreases $\implies \exists \alpha ,s.t.$

$\left\Vert u_{n}\right\Vert \rightarrow \alpha .$ If $\alpha <1.$ Then for
sufficient large n, $u_{n}\in K\implies v_{n}=u_{n}$

from 2 we have shown that $u_{n}$ converges strongly to some limit.

If $\alpha \geq 1,$ Then we can show $P_{K}u_{n}=\frac{u_{n}}{\left\Vert
u_{n}\right\Vert }\in K.$ Indeed,

$\left\langle v-\frac{u_{n}}{\left\Vert u_{n}\right\Vert },u_{n}-\frac{u_{n}%
}{\left\Vert u_{n}\right\Vert }\right\rangle =\left( 1-\frac{1}{\left\Vert
u_{n}\right\Vert }\right) \left( \left\langle v,u_{n}\right\rangle
-\left\Vert u_{n}\right\Vert \right) ,$ $\left\Vert u_{n}\right\Vert \geq
\left\vert \left\langle v,u_{n}\right\rangle \right\vert \implies $

$\left\langle v-\frac{u_{n}}{\left\Vert u_{n}\right\Vert },u_{n}-\frac{u_{n}%
}{\left\Vert u_{n}\right\Vert }\right\rangle \leq 0\forall v\in K.$

Then $\frac{u_{n}}{\left\Vert u_{n}\right\Vert }$ converges strongly.
Combined with the convergence of $\left\Vert u_{n}\right\Vert $ follows the
strong convergence of $u_{n}.$

$\rightarrow $

\end{document}
