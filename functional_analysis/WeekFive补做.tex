
\documentclass{article}
%%%%%%%%%%%%%%%%%%%%%%%%%%%%%%%%%%%%%%%%%%%%%%%%%%%%%%%%%%%%%%%%%%%%%%%%%%%%%%%%%%%%%%%%%%%%%%%%%%%%%%%%%%%%%%%%%%%%%%%%%%%%%%%%%%%%%%%%%%%%%%%%%%%%%%%%%%%%%%%%%%%%%%%%%%%%%%%%%%%%%%%%%%%%%%%%%%%%%%%%%%%%%%%%%%%%%%%%%%%%%%%%%%%%%%%%%%%%%%%%%%%%%%%%%%%%
\usepackage{amssymb}
\usepackage{amsmath}

\setcounter{MaxMatrixCols}{10}
%TCIDATA{OutputFilter=LATEX.DLL}
%TCIDATA{Version=5.00.0.2552}
%TCIDATA{<META NAME="SaveForMode" CONTENT="1">}
%TCIDATA{Created=Wednesday, October 21, 2015 21:56:03}
%TCIDATA{LastRevised=Saturday, November 21, 2015 16:31:43}
%TCIDATA{<META NAME="GraphicsSave" CONTENT="32">}
%TCIDATA{<META NAME="DocumentShell" CONTENT="Scientific Notebook\Blank Document">}
%TCIDATA{CSTFile=Math with theorems suppressed.cst}
%TCIDATA{PageSetup=72,72,72,72,0}
%TCIDATA{AllPages=
%F=36,\PARA{038<p type="texpara" tag="Body Text" >\hfill \thepage}
%}


\newtheorem{theorem}{Theorem}
\newtheorem{acknowledgement}[theorem]{Acknowledgement}
\newtheorem{algorithm}[theorem]{Algorithm}
\newtheorem{axiom}[theorem]{Axiom}
\newtheorem{case}[theorem]{Case}
\newtheorem{claim}[theorem]{Claim}
\newtheorem{conclusion}[theorem]{Conclusion}
\newtheorem{condition}[theorem]{Condition}
\newtheorem{conjecture}[theorem]{Conjecture}
\newtheorem{corollary}[theorem]{Corollary}
\newtheorem{criterion}[theorem]{Criterion}
\newtheorem{definition}[theorem]{Definition}
\newtheorem{example}[theorem]{Example}
\newtheorem{exercise}[theorem]{Exercise}
\newtheorem{lemma}[theorem]{Lemma}
\newtheorem{notation}[theorem]{Notation}
\newtheorem{problem}[theorem]{Problem}
\newtheorem{proposition}[theorem]{Proposition}
\newtheorem{remark}[theorem]{Remark}
\newtheorem{solution}[theorem]{Solution}
\newtheorem{summary}[theorem]{Summary}
\newenvironment{proof}[1][Proof]{\noindent\textbf{#1.} }{\ \rule{0.5em}{0.5em}}


\begin{document}


\bigskip Functional Analysis \ Week Five Second

Problem 1.8

$\left( a\right) $Since $C$ is bounded above by $M_{1}$, if $p\left(
x\right) =0,\forall \epsilon >0,\exists 0<\alpha <\epsilon ,s.t.\frac{x}{%
\alpha }\in C.$

$M_{1}\geq \left\Vert \frac{x}{\alpha }\right\Vert \geq \frac{\left\Vert
x\right\Vert }{\epsilon },$hence $\left\Vert x\right\Vert =0,$otherwise we
let $\epsilon ->0$ in the inequality and get a contradiction.

Since $C$ is symmtric, $p\left( x\right) =p\left( -x\right) $ by the
definition of $p.$Then from $p\left( \lambda x\right) =\lambda p\left(
x\right) $,for positive $\lambda ,$it holds as well for $\lambda \in \NEG{R}.
$

Combined with sub-linear property of $p\left( \cdot \right) ,$we conclude
that $p\left( \cdot \right) $ is a norm on $E.$Below we show that there
exists two positive constants $c_{1},c_{2},s.t.c_{1}\left\Vert x\right\Vert
\leq p\left( x\right) \leq c_{2}\left\Vert x\right\Vert ,c_{2}$ can be taken
as $M$ found in lemma 2. We again use the bounded property of $C$, $\exists
B\left( 0,\delta \right) \subset C\implies p\left( z\right) >m_{2},$for any $%
z\in \partial B\left( 0,\delta \right) ,$otherwise, we have points belonging
to $C$ and $\frac{x}{\alpha }\in C,$with arbitrary small $\alpha ,$%
contradicting with bdd property of $C.$By rescaling $p\left( \frac{x}{%
\left\Vert x\right\Vert }\right) >m_{2}^{\prime }.$Thus we can take $%
c_{1}=m_{2}^{\prime },$and $c_{1}\left\Vert x\right\Vert \leq p\left(
x\right) $ follows. In conclusion the norm defined by $p\left( \cdot \right) 
$ and $\left\Vert \cdot \right\Vert $ are equivalent.

$\left( b\right) $for $u_{1},u_{2}\in C,\int_{0}^{1}\left\vert u_{i}\left(
t\right) \right\vert ^{2}dt<1,i=1,2,\forall s\in \left( 0,1\right) ,$

$\int_{0}^{1}\left\vert su_{1}\left( t\right) +\left( 1-s\right) u_{2}\left(
t\right) \right\vert ^{2}dt=s^{2}\int_{0}^{1}\left\vert u_{1}\left( t\right)
\right\vert ^{2}dt+\left( 1-s\right) ^{2}\int_{0}^{1}\left\vert u_{2}\left(
t\right) \right\vert ^{2}dt+s\left( 1-s\right) \left(
\int_{0}^{1}u_{1}\left( t\right) u_{2}^{\ast }\left( t\right) +u_{2}\left(
t\right) u_{1}^{\ast }\left( t\right) dt\right) $

$\leq s^{2}\int_{0}^{1}\left\vert u_{1}\left( t\right) \right\vert
^{2}dt+\left( 1-s\right) ^{2}\int_{0}^{1}\left\vert u_{2}\left( t\right)
\right\vert ^{2}dt+s\left( 1-s\right) \left( \int_{0}^{1}u_{1}\left(
t\right) u_{2}^{\ast }\left( t\right) +u_{2}\left( t\right) u_{1}^{\ast
}\left( t\right) dt\right) \qquad \left( \text{Cauchy Inequality}\right) $

$\leq s^{2}\int_{0}^{1}\left\vert u_{1}\left( t\right) \right\vert
^{2}dt+\left( 1-s\right) ^{2}\int_{0}^{1}\left\vert u_{2}\left( t\right)
\right\vert ^{2}dt+2s\left( 1-s\right) \int_{0}^{1}\left\vert u_{1}\left(
t\right) \right\vert ^{2}dt\int_{0}^{1}\left\vert u_{2}\left( t\right)
\right\vert ^{2}dt$

$\leq s^{2}+\left( 1-s\right) ^{2}+2s\left( 1-s\allowbreak \right)
=\allowbreak 1$

$\implies su_{1}+\left( 1-s\right) u_{2}\in C$ and $C$ is convex.

$C$ is obviously symmetric and $0\in C$

$C$ is unbouded.

$p\left( u\right) =\inf \left\{ \alpha >0,\alpha \in \NEG{R},\frac{u}{\alpha 
}\in C\right\} $

$\frac{u}{\alpha }\in C\iff \sqrt{\int_{0}^{1}\left\vert u\left( t\right)
\right\vert ^{2}dt}\leq \alpha \implies p\left( u\right) =\sqrt{%
\int_{0}^{1}\left\vert u\left( t\right) \right\vert ^{2}dt},.$

$p\left( \cdot \right) $ defines the $L_{2}$-norm on $E$.

$p\left( \cdot \right) $ is not equivalent to $\left\Vert \cdot \right\Vert
, $since we can find a function sequence $\left\{ v_{n}\right\} $ with
constant $p$-norm but its

$\left\Vert \cdot \right\Vert $-norm is unbounded$\implies c_{1}\left\Vert
v_{n}\right\Vert \leq p\left( v_{n}\right) $ with positive $c_{1}$ holds for
every n is impossible.

Problem 1.10

$\bigskip \left( a\right) \left( A\right) \implies \left( B\right) :$

If there exists some $f\in E^{\ast },s.t.\left\langle f,x_{i}\right\rangle
=\alpha _{i},\forall i\in I.\left\vert \underset{i\in J}{\sum }\beta
_{i}\alpha _{i}\right\vert =\left\vert \underset{i\in J}{\sum }\beta
_{i}\left\langle f,x_{i}\right\rangle \right\vert =\left\vert \left\langle f,%
\underset{i\in J}{\sum }\beta _{i}x_{i}\right\rangle \right\vert $

$\leq \left\Vert f\right\Vert \left\Vert \underset{i\in J}{\sum }\beta
_{i}x_{i}\right\Vert .$

$\left( B\right) \implies \left( A\right) :$First we define a functional on
the subspace $S$ of $E$ spanned by $\left\{ x_{i},i\in I\right\} .$

$g\left( x\right) =\underset{i\in J}{\sum }\beta _{i}\alpha _{i},$provided $%
x=\underset{i\in J}{\sum }\beta _{i}x_{i}$ for some finite subset $J$ of $I.$

Such definition is well-defined. Indeed for every $x\in S,$ $x$ can be
expressed by finite elements from $\left\{ x_{i},i\in I\right\} .$ Secondly,
if $x=\underset{i\in J_{1}}{\sum }\beta _{i}^{\left( 1\right) }x_{i}^{\left(
1\right) }=\underset{i\in J_{2}}{\sum }\beta _{i}^{\left( 2\right)
}x_{i}^{\left( 2\right) },$and $g_{1}\left( x\right) =\underset{i\in J}{\sum 
}\beta _{i}^{\left( 1\right) }\alpha _{i}^{\left( 1\right) },g_{2}\left(
x\right) =\underset{i\in J}{\sum }\beta _{i}^{\left( 2\right) }\alpha
_{i}^{\left( 2\right) }$ as a result, we only need to show that $g_{1}\left(
x\right) =g_{2}\left( x\right) ,$which is guaranteed by $\left( B\right) ,$%
since we have

$\left\vert g_{1}\left( x\right) -g_{2}\left( x\right) \right\vert
=\left\vert \underset{i\in J_{1}}{\sum }\beta _{i}^{\left( 1\right) }\alpha
_{i}^{\left( 1\right) }-\underset{i\in J_{2}}{\sum }\beta _{i}^{\left(
2\right) }\alpha _{i}^{\left( 2\right) }\right\vert =\left\vert \underset{%
i\in J_{1}\cup J_{2}}{\sum }\beta _{i}\alpha _{i}\right\vert \leq
M\left\vert \underset{i\in J_{1}\cup J_{2}}{\sum }\beta _{i}x_{i}\right\vert
=\left\vert \underset{i\in J_{1}}{\sum }\beta _{i}^{\left( 1\right)
}x_{i}^{\left( 1\right) }-\underset{i\in J_{2}}{\sum }\beta _{i}^{\left(
2\right) }x_{i}^{\left( 2\right) }\right\vert $

$=0,$that is $g_{1}\left( x\right) =g_{2}\left( x\right) .$

$g$ is obviously linear and by $\left( B\right) $ $\left\vert g\left(
x\right) \right\vert \leq M\left\Vert x\right\Vert ,g$ is bounded$\implies g$
is a linear functional on $S.$

Then using Corallary of Hahn-Banach Thm, we can extend $g$ to the whole
space and get a linear functional $f$ on E and $f_{|S}=g,$thus $f$ satisfies 
$\left( A\right) .$

Problem $1.11$

$\left( A\right) \implies \left( B\right) :\left\vert \underset{i=1}{\overset%
{n}{\sum }}\beta _{i}\alpha _{i}\right\vert \leq \left\vert \underset{i=1}{%
\overset{n}{\sum }}\beta _{i}\left\langle f_{i},x_{\epsilon }\right\rangle
\right\vert \leq \left\vert \left\langle \underset{i=1}{\overset{n}{\sum }}%
\beta _{i}f_{i},x_{\epsilon }\right\rangle \right\vert \leq \left\Vert 
\underset{i=1}{\overset{n}{\sum }}\beta _{i}f_{i}\right\Vert \left\Vert
x_{\epsilon }\right\Vert \leq \left\Vert \underset{i=1}{\overset{n}{\sum }}%
\beta _{i}f_{i}\right\Vert \left( M+\epsilon \right) .$

Let $\epsilon ->0\implies \left\vert \underset{i=1}{\overset{n}{\sum }}\beta
_{i}\alpha _{i}\right\vert \leq M\left\Vert \underset{i=1}{\overset{n}{\sum }%
}\beta _{i}f_{i}\right\Vert .$

$\left( B\right) \implies \left( A\right) :$

\bigskip Define $\varphi \left( x\right) :=\left( f_{1}\left( x\right)
,...,f_{n}\left( x\right) \right) ,x\in E,\varphi \left( x\right) \in \NEG%
{R}^{n},\vec{\gamma}=\left( \alpha _{1},...,\alpha _{n}\right) .$

$\left( A\right) \iff \varphi ^{-1}\left( \vec{\gamma}\right) \cap \overline{%
B\left( 0,M\right) }\neq \varnothing ,$now assume it is not true,that is $%
\varphi ^{-1}\left( \gamma \right) \cap \overline{B\left( 0,M\right) }%
=\varnothing \implies $

$\left\{ \vec{\gamma}\right\} \cap \varphi \left( \overline{B\left(
0,M\right) }\right) =\varnothing .$Since $\varphi $ is linear continuous
mapping, it maps $\overline{B\left( 0,M\right) }$ onto a convex closed set

in $\NEG{R}^{n},$by Convex Set Segregation Law, there exists a hyperplane in 
$\NEG{R}^{n}$ strictly separating $\left\{ \vec{\gamma}\right\} $ and $%
\varphi \left( \overline{B\left( 0,M\right) }\right) .$That means there
exists $\left( \beta _{1},...\beta _{n}\right) =:\vec{\beta},s.t.\vec{\beta}%
\cdot \varphi \left( x\right) <c<\vec{\beta}\cdot \vec{\gamma},\forall x\in 
\overline{B\left( 0,M\right) }\implies $

$M\underset{i=1}{\overset{n}{\sum }}\beta _{i}f_{i}\left( x\right) <c$%
\TEXTsymbol{<}$\underset{i=1}{\overset{n}{\sum }}\beta _{i}\alpha
_{i},\forall x\in \overline{B\left( 0,1\right) },$by the definition of $%
\left\Vert \underset{i=1}{\overset{n}{\sum }}\beta _{i}f_{i}\right\Vert ,$%
taking the upper bound on the lefthand of the inequality gives $M\left\Vert 
\underset{i=1}{\overset{n}{\sum }}\beta _{i}f_{i}\right\Vert \leq c<\underset%
{i=1}{\overset{n}{\sum }}\beta _{i}\alpha _{i},$a contradiction and the
proof is complete.

(The standard answer hints that we should first show the result for
indepedent $\left\{ f_{i}\right\} $ and then take the maximal indepedent
sets of $\left\{ f_{i}\right\} .$The Segregation Law used is based on finite
dimensional space, established in Ex 1.9; But I wonder whether it is
possible to conclude from the strict separation law of Second Geometric Form
of Hahn Banach Thm. Is there any problem about claiming that $\varphi \left( 
\overline{B\left( 0,M\right) }\right) $ is closed?)

Problem 1.12

$\left( A\right) \implies \left( B\right) :$ $\left\langle \underset{i=1}{%
\overset{n}{\sum }}\beta _{i}f_{i},x\right\rangle =\underset{i=1}{\overset{n}%
{\sum }}\beta _{i}\left\langle f_{i},x\right\rangle =\underset{i=1}{\overset{%
n}{\sum }}\beta _{i}\alpha _{i}\implies \underset{i=1}{\overset{n}{\sum }}%
\beta _{i}\alpha _{i}=0$

$\left( B\right) \implies \left( A\right) :$Notice that in this problem
there is no norm defined in E.

The proof should rely on induction. for n=1,if f1 不恒为0,
we can find $x,s.t.$

$f_{1}\left( x\right) =\alpha _{1}.$ Suppose the conclusion is true for
n=k-1, that is we have found $x,s.t.f_{i}\left( x\right) =\alpha
_{i},i=1,..k-1.$If $\left( f_{1},..f_{k-1}\right) $ are linearly indepedent$,
$

If we assume $\left( \alpha _{1},..\alpha _{k}\right) \notin R\left( \varphi
\right) ,$where $\varphi \left( x\right) =\left( f_{1}\left( x\right)
,..f_{k}\left( x\right) \right) ,$using the conclusion from Ex 1.9, we can
find $\beta _{1},..\beta _{k}$ 不全为零,s.t. $\beta
_{1}f_{1}+..+\beta _{k}f_{k}\left( x\right) $ $\leq \beta _{1}\alpha
_{1}+..+\beta _{k}\alpha _{k},\forall x\in E\implies \beta
_{1}f_{1}+..+\beta _{k}f_{k}=0.$

\bigskip If $f_{k}$ is linear indepedent with $f_{1},..f_{k-1},$then $\beta
_{k}=0\implies $ $\beta _{1}f_{1}+..+\beta _{k-1}f_{k-1}=0,$

对不全为零的$\left( \beta _{1},..\beta
_{k-1}\right) $成立,这与$\left(
f_{1},..f_{k-1}\right) $ are linearly indepedent矛盾.

Then $f_{k}$ is linear depedent with $f_{1},..f_{k-1},\implies \exists
\left( \beta _{1},..\beta _{k-1}\right) \in R^{k-1},s.t.f_{k}=\beta
_{1}f_{1}+..+\beta _{k-1}f_{k-1}$

由条件B,$\alpha _{k}=\beta _{1}\alpha _{1}+..+\beta
_{k-1}\alpha _{k-1},$

$f_{k}\left( x\right) =\beta _{1}f_{1}\left( x\right) +..+\beta
_{k-1}f_{k-1}\left( x\right) =\beta _{1}\alpha _{1}+..+\beta _{k-1}\alpha
_{k-1}=\alpha _{k}.$

因此对$\left( f_{1},..f_{k-1}\right) $ are linearly
indepedent的情形n=k也成立.

一般的,可以从$\left(
f_{1},..f_{k-1}\right) $中取出极大无%
关组,这时构造的$\varphi $的%
维数就要相应地降为%
极大无关组的维数.由%
归纳假设可知对任意%
正整数n$\left( B\right) \implies \left( A\right) $成%
立.

\end{document}
