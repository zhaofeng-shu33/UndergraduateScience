
\documentclass{article}
%%%%%%%%%%%%%%%%%%%%%%%%%%%%%%%%%%%%%%%%%%%%%%%%%%%%%%%%%%%%%%%%%%%%%%%%%%%%%%%%%%%%%%%%%%%%%%%%%%%%%%%%%%%%%%%%%%%%%%%%%%%%%%%%%%%%%%%%%%%%%%%%%%%%%%%%%%%%%%%%%%%%%%%%%%%%%%%%%%%%%%%%%%%%%%%%%%%%%%%%%%%%%%%%%%%%%%%%%%%%%%%%%%%%%%%%%%%%%%%%%%%%%%%%%%%%
\usepackage{amssymb}
\usepackage{amsmath}

\setcounter{MaxMatrixCols}{10}
\def\TEXTsymbol#1{\mbox{$#1$}}%
\def\NEG#1{\leavevmode\hbox{\rlap{\thinspace/}{$#1$}}}%
\def\QATOPD#1#2#3#4{{#3 \atopwithdelims#1#2 #4}}%
\def\QTP#1{}
\def\func#1{\mathop{\rm #1}}%
%TCIDATA{Version=5.00.0.2552}
%TCIDATA{<META NAME="SaveForMode" CONTENT="1">}
%TCIDATA{Created=Tuesday, September 15, 2015 17:15:15}
%TCIDATA{LastRevised=Saturday, October 03, 2015 21:18:11}
%TCIDATA{<META NAME="GraphicsSave" CONTENT="32">}
%TCIDATA{<META NAME="DocumentShell" CONTENT="Scientific Notebook\Blank with Theorem Tags">}
%TCIDATA{CSTFile=Math.cst}
%TCIDATA{PageSetup=72,72,72,72,0}
%TCIDATA{AllPages=
%F=36,\PARA{038<p type="texpara" tag="Body Text" >\hfill \thepage}
%}


\newtheorem{theorem}{Theorem}
\newtheorem{acknowledgement}[theorem]{Acknowledgement}
\newtheorem{algorithm}[theorem]{Algorithm}
\newtheorem{axiom}[theorem]{Axiom}
\newtheorem{case}[theorem]{Case}
\newtheorem{claim}[theorem]{Claim}
\newtheorem{conclusion}[theorem]{Conclusion}
\newtheorem{condition}[theorem]{Condition}
\newtheorem{conjecture}[theorem]{Conjecture}
\newtheorem{corollary}[theorem]{Corollary}
\newtheorem{criterion}[theorem]{Criterion}
\newtheorem{definition}[theorem]{Definition}
\newtheorem{example}[theorem]{Example}
\newtheorem{exercise}[theorem]{Exercise}
\newtheorem{lemma}[theorem]{Lemma}
\newtheorem{notation}[theorem]{Notation}
\newtheorem{problem}[theorem]{Problem}
\newtheorem{proposition}[theorem]{Proposition}
\newtheorem{remark}[theorem]{Remark}
\newtheorem{solution}[theorem]{Solution}
\newtheorem{summary}[theorem]{Summary}
\newenvironment{proof}[1][Proof]{\noindent\textbf{#1.} }{\ \rule{0.5em}{0.5em}}


\begin{document}


%\FRA

\bigskip Proof: Denote $F^{\prime }\left( x\right) =\left\{ f\in E^{\ast
};||f||\leq ||x||\text{ and }<f,x>=||x||^{2}\right\} .$ If $f\in F^{\prime
}\left( x\right) ,$then $||f||\geq |f\left( \frac{x}{||x||}\right) |=\frac{%
|f\left( x\right) |}{||x||}=\frac{<f,x>}{||x||}=||x||,$therefore $%
||f||=||x|| $ and $F^{\prime }(x)=F(x).$

From the corollary of Hahn-Banach Theorem, $F(x)$ is nonempty. If $\left\{
f_{n}\right\} \subset F(x)$ and $\left\{ f_{n}\right\} $ converges,

to $f,$from $||f_{n}||=||x||,$ it follows $||f||=||x||,$ since $%
|||f||-||f_{n}|||\leq ||f-f_{n}||\rightarrow 0$ . Therefore $F(x)$ is closed.

For $f_{1}$, $f_{2}$ $\in F(x),$it follows $||tf_{1}+(1-t)f_{2}||\leq
t||f_{1}||+(1-t)||f_{2}||\leq t||x||+(1-t)||x||=||x||$ for $0<t<1$ and

\TEXTsymbol{<}$%
tf_{1}+(1-t)f_{2},x>=t<f_{1},x>+(1-t)<f_{2},x>=t||x||^{2}+(1-t)||x||^{2}=||x||^{2}, 
$ it follows that

$tf_{1}+(1-t)f_{2}\in F(x).$ and $F(x)$ is convex.

\begin{problem}
\begin{proof}
\begin{problem}
2 prove that if E$^{\ast }$ is strictly convex, then F(x) contains a single
point.

\begin{proof}
For $f_{1}$, $f_{2}$ $\in F(x)$ and $f_{1}\neq f_{2},$ $%
||tf_{1}+(1-t)f_{2}||<||x||,$contradicting $tf_{1}+(1-t)f_{2}\in F(x).$ It
follows that F(x) contains a single point.
\end{proof}
\end{problem}
\end{proof}

%\FRA

\begin{proof}
Denote $F^{\prime }(x)=\left\{ f\in E^{\ast };\frac{1}{2}||y||^{2}-\frac{1}{2%
}||x||^{2}\geq <f,y-x>\text{ }\forall y\in E\right\} $ For any $f\in
F(x),y\in E,$

\TEXTsymbol{\vert}\TEXTsymbol{<}f,$\frac{y}{||y||}>|\leq ||x||\qquad
\rightarrow <f,y>\leq ||x||||y||\leq \frac{1}{2}(||x||^{2}+||y||^{2})%
\rightarrow \frac{1}{2}||y||^{2}-\frac{1}{2}||x||^{2}\geq <f,y>-||x||^{2}=$
\end{proof}
\end{problem}

\QTP{Body Math}
$\qquad \qquad \qquad \qquad \qquad <f,y>-<f,x>=<f,y-x>.$ That is, $f\in
F^{\prime }(x).$

\QTP{Body Math}
$\qquad \qquad \qquad \qquad \qquad Conversely,$For any $f\in F^{\prime
}(x), $take $||u||=1,$and $y=tu,$where $u\in E$ (such u exists,e.g.$\frac{x}{%
||x||} $)$,$ and t

\begin{proof}
is arbitary real number. Then we get: $\frac{1}{2}t^{2}-<f,u>t+<f,x>-\frac{1%
}{2}||x||^{2}\geq 0,$ for such inequality to hold, the determinant is
nonpositive,i.e. $<f,u>^{2}-2(<f,x>-\frac{1}{2}||x||^{2})\leq 0.$

Let $u=\frac{x}{||x||},$we can get $(<f,x>-||x||^{2})^{2}\leq 0,$ therefore $%
<f,x>=||x||^{2}.$

Futhermore, since $<f,u>^{2}\leq 2(<f,x>-\frac{1}{2}||x||^{2})$ for any 
\TEXTsymbol{\vert}\TEXTsymbol{\vert}u\TEXTsymbol{\vert}\TEXTsymbol{\vert}=1.
By the definition of $||f||$ and the property of linearity of $f,||f||=%
\underset{\underset{u\in E}{||u||=1}}{\sup }|<f,E>|\leq
2<f,x>-||x||^{2}=||x||^{2}.$ By the conclusion in problem (1), it follows
that $f\in F(x).$ Hence $F(x)=F^{\prime }(x),$the proof it complete.$%
\boxtimes $
\end{proof}

%\FRA

(a) $||f||_{E^{\ast }}=\underset{\underset{x\in E}{||x||\leq 1}}{\sup }%
|f(x)|=\underset{\underset{x\in E}{||x||\leq 1}}{\sup }|\underset{i=1}{%
\overset{n}{\sum }}x_{i}<f,e_{i}>|=\underset{\underset{x\in E}{\underset{i=1}%
{\overset{n}{\sum |}}x_{i}|\leq 1}}{\sup }|\underset{i=1}{\overset{n}{\sum }}%
x_{i}f_{i}|=\underset{1\leq i\leq n}{\max }|f_{i}|,$

(b) By definition %\FRA


For abitrary $x\in E,$ we get two equations of $F(x)$ (denoted as $f$ below)

$\underset{1\leq i\leq n}{\max }|<f,e_{i}>|=\underset{i=1}{\overset{n}{\sum }%
}|x_{i}|\qquad \qquad \qquad \qquad \qquad \qquad \qquad (1)$

$\underset{i=1}{\overset{n}{\sum }}x_{i}<f,e_{i}>=(\underset{i=1}{\overset{n}%
{\sum }}|x_{i}|)^{2}\qquad \qquad \qquad \qquad \qquad \ (2)$

from (1) and (2) we get:

$\underset{i=1}{\overset{n}{\sum }}\frac{x_{i}}{\underset{i=1}{\overset{n}{%
\sum }}|x_{i}|}<f,e_{i}>=\underset{1\leq i\leq n}{\max }|<f,e_{i}>|$, which
is possible only if all $<f,e_{i}>$ equal $<f,e_{1}>$ for example.

Thus the above equation reduces to

$\underset{i=1}{\overset{n}{\sum }}x_{i}<f,e_{1}>=<f,e_{1}>^{2},$since $%
<f,e_{1}>$ is nonzero for all nonzero $\underset{i=1}{\overset{n}{\sum }}%
|x_{i}|,$ it follows that

$<f,e_{i}>=\underset{i=1}{\overset{n}{\sum }}x_{i},$ and F maps x to a
function f on E, such that $f(y)=\underset{i=1}{\overset{n}{\sum }}x_{i}%
\underset{i=1}{\overset{n}{\sum }}y_{i},$where $y=\underset{i=1}{\overset{n}{%
\sum }}y_{i}e_{i}$

%\FRA

(a)$||f||_{E^{\ast }}=\underset{i=1}{\overset{n}{\sum }}|f_{i}|,$

(b)F maps x to a function f on E, such that $f(y)=\frac{\underset{i=1}{%
\overset{n}{\sum }}x_{i}}{n^{2}}\underset{i=1}{\overset{n}{\sum }}y_{i},$%
where $y=\underset{i=1}{\overset{n}{\sum }}y_{i}e_{i}$

%\FRA

Using H\"{o}lder's Inequality we can show (a) $||f||_{E^{\ast }}=\left( 
\underset{i=1}{\overset{n}{\sum }}|f_{i}|^{q}\right) ^{\frac{1}{q}},$ where
q satisfies $\frac{1}{p}+\frac{1}{q}=1.$

(b)F maps x to a function f on E, such that $f(y)=\overset{n}{\sum }%
sgn(x_{i})|x_{i}|^{p-1}y_{i},$where $y=\underset{i=1}{\overset{n}{\sum }}%
y_{i}e_{i}$

%\FRA

%\FRA
1 $\left( a\right) $ $f$ is linear and $\left\Vert f\right\Vert =\underset{%
\left\Vert u\right\Vert \leq 1}{\sup }\left\vert f\left( u\right)
\right\vert =\underset{\left\Vert u\right\Vert \leq 1}{\sup }\left\vert
\int_{0}^{1}u\left( t\right) dt\right\vert \leq 1.$f is bounded$\implies f$
is continous linear functional on $E,$thus belonging to $E^{\ast }.$

$\left\Vert f\right\Vert _{E^{\ast }}=1.$

$\left( b\right) \int_{0}^{1}u\left( t\right) dt=1$ for $\underset{t\in
\lbrack 0,1]}{\max }\left\vert u\left( t\right) \right\vert =1,u\left(
0\right) =0.$

Since $u\left( t\right) $ is continuous at 0, there exists $\delta \in
\left( 0,1\right) ,s.t.\left\vert u\left( t\right) \right\vert <\frac{1}{2},$%
in $\left( 0,\delta \right) .$

Then $\int_{0}^{1}u\left( t\right) dt=\int_{0}^{\delta }u\left( t\right)
dt+\int_{\delta }^{1}u\left( t\right) dt\leq \frac{1}{2}\delta +\left(
1-\delta \right) <1.$A contradiction!

\end{document}
