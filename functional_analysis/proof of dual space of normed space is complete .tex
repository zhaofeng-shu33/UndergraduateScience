
\documentclass{article}
\usepackage{amssymb}
\usepackage{amsmath}

%%%%%%%%%%%%%%%%%%%%%%%%%%%%%%%%%%%%%%%%%%%%%%%%%%%%%%%%%%%%%%%%%%%%%%%%%%%%%%%%%%%%%%%%%%%%%%%%%%%%%%%%%%%%%%%%%%%%%%%%%%%%%%%%%%%%%%%%%%%%%%%%%%%%%%%%%%%%%%%%%%%%%%%%%%%%%%%%%%%%%%%%%%%%%%%%%%%%%%%%%%%%%%
%TCIDATA{OutputFilter=LATEX.DLL}
%TCIDATA{Version=5.00.0.2552}
%TCIDATA{<META NAME="SaveForMode" CONTENT="1">}
%TCIDATA{Created=Tuesday, September 15, 2015 15:55:04}
%TCIDATA{LastRevised=Tuesday, September 15, 2015 16:39:24}
%TCIDATA{<META NAME="GraphicsSave" CONTENT="32">}
%TCIDATA{<META NAME="DocumentShell" CONTENT="Scientific Notebook\Blank with Theorem Tags">}
%TCIDATA{CSTFile=Math.cst}
%TCIDATA{PageSetup=72,72,72,72,0}
%TCIDATA{AllPages=
%F=36,\PARA{038<p type="texpara" tag="Body Text" >\hfill \thepage}
%}


\newtheorem{theorem}{Theorem}
\newtheorem{acknowledgement}[theorem]{Acknowledgement}
\newtheorem{algorithm}[theorem]{Algorithm}
\newtheorem{axiom}[theorem]{Axiom}
\newtheorem{case}[theorem]{Case}
\newtheorem{claim}[theorem]{Claim}
\newtheorem{conclusion}[theorem]{Conclusion}
\newtheorem{condition}[theorem]{Condition}
\newtheorem{conjecture}[theorem]{Conjecture}
\newtheorem{corollary}[theorem]{Corollary}
\newtheorem{criterion}[theorem]{Criterion}
\newtheorem{definition}[theorem]{Definition}
\newtheorem{example}[theorem]{Example}
\newtheorem{exercise}[theorem]{Exercise}
\newtheorem{lemma}[theorem]{Lemma}
\newtheorem{notation}[theorem]{Notation}
\newtheorem{problem}[theorem]{Problem}
\newtheorem{proposition}[theorem]{Proposition}
\newtheorem{remark}[theorem]{Remark}
\newtheorem{solution}[theorem]{Solution}
\newtheorem{summary}[theorem]{Summary}
\newenvironment{proof}[1][Proof]{\noindent\textbf{#1.} }{\ \rule{0.5em}{0.5em}}
\input{tcilatex}

\begin{document}


\begin{problem}
Let E be a n.v.s. and E$^{\ast }$ be its dual space, that is E$^{\ast
}=\{f:E->\NEG{R}$ and f is linear and continuous\} with the norm \TEXTsymbol{%
\vert}\TEXTsymbol{\vert}f\TEXTsymbol{\vert}\TEXTsymbol{\vert}$_{E^{\ast }}=%
\underset{\underset{x\in E}{||x||\leq 1}}{\sup }|f(x)|.$Show that E$^{\ast }$
is a Banach space.

\begin{proof}
For a Cauchy sequence \{f$_{n}\}$ of E$^{\ast }$ and for every $\epsilon ,$
there exists N$_{\epsilon }$ such that for every i, j\TEXTsymbol{>}N$%
\epsilon ,||f_{i}-f_{j}||<\epsilon .$It follows that \TEXTsymbol{\vert}f$%
_{i}(x)-f_{j}(x)|<\epsilon $ for each \TEXTsymbol{\vert}\TEXTsymbol{\vert}x%
\TEXTsymbol{\vert}\TEXTsymbol{\vert}$\leq 1$ and x$\in E.\qquad \qquad
\qquad \qquad \qquad \qquad (1)$

For all such x, by Cauthy Criterion for real number, \{f$_{n}(x)\}$
converges in R. Denote this limit as f(x), which is defined on all x$\in E$
and \TEXTsymbol{\vert}\TEXTsymbol{\vert}x\TEXTsymbol{\vert}\TEXTsymbol{\vert}%
$\leq 1.$ And for each x'$\in E,$define f(x')=\TEXTsymbol{\vert}\TEXTsymbol{%
\vert}x'\TEXTsymbol{\vert}\TEXTsymbol{\vert}$f(\frac{x^{\prime }}{%
||x^{\prime }||})$,thus we get a limit function on the whole E.
\end{proof}
\end{problem}

\ \ Firstly, we will show f lies in E$^{\ast },$which means that f is a
linear continuous function. By definition of f, it is sufficient to show
such property for all \TEXTsymbol{\vert}\TEXTsymbol{\vert}x\TEXTsymbol{\vert}%
\TEXTsymbol{\vert}$\leq 1.$Since f(x)=$\underset{n->\infty }{\lim }f_{n}(x),$
the linearlity of f is obtained from the corresponding property of limit
operation. To show the continuity of f, the uniform convergent of f$_{n}$ is
needed, which is guranteed by the convergence of \TEXTsymbol{\vert}$%
\left\vert f_{i}-f_{j}\right\vert |$ (the Cauthy Criterion for uniform
convergence). Then from the continuity of f$_{n}$ it follows that f is
continuous. 

\ \ Secondly, we show that $||f_{n}-f||$ converges to zero, which is trivial
from the Cauthy Criterion for uniform convergence. Therefore $\{f_{n}\}$ has
limits in E$^{\ast }$ and the proof is complete.$\boxtimes $

\end{document}
