
\documentclass{article}
\usepackage{amssymb}

%%%%%%%%%%%%%%%%%%%%%%%%%%%%%%%%%%%%%%%%%%%%%%%%%%%%%%%%%%%%%%%%%%%%%%%%%%%%%%%%%%%%%%%%%%%%%%%%%%%%%%%%%%%%%%%%%%%%%%%%%%%%%%%%%%%%%%%%%%%%%%%%%%%%%%%%%%%%%%%%%%%%%%%%%%%%%%%%%%%%%%%%%%%%%%%%%%%%%%%%%%%%%%%%%%%%%%%%%%%%%%%%%%%%
\usepackage{amsmath}

\setcounter{MaxMatrixCols}{10}
%TCIDATA{OutputFilter=LATEX.DLL}
%TCIDATA{Version=5.00.0.2552}
%TCIDATA{<META NAME="SaveForMode" CONTENT="1">}
%TCIDATA{Created=Tuesday, October 27, 2015 23:06:56}
%TCIDATA{LastRevised=Sunday, November 01, 2015 20:02:02}
%TCIDATA{<META NAME="GraphicsSave" CONTENT="32">}
%TCIDATA{<META NAME="DocumentShell" CONTENT="Scientific Notebook\Blank Document">}
%TCIDATA{CSTFile=Math with theorems suppressed.cst}
%TCIDATA{PageSetup=72,72,72,72,0}
%TCIDATA{AllPages=
%F=36,\PARA{038<p type="texpara" tag="Body Text" >\hfill \thepage}
%}


\newtheorem{theorem}{Theorem}
\newtheorem{acknowledgement}[theorem]{Acknowledgement}
\newtheorem{algorithm}[theorem]{Algorithm}
\newtheorem{axiom}[theorem]{Axiom}
\newtheorem{case}[theorem]{Case}
\newtheorem{claim}[theorem]{Claim}
\newtheorem{conclusion}[theorem]{Conclusion}
\newtheorem{condition}[theorem]{Condition}
\newtheorem{conjecture}[theorem]{Conjecture}
\newtheorem{corollary}[theorem]{Corollary}
\newtheorem{criterion}[theorem]{Criterion}
\newtheorem{definition}[theorem]{Definition}
\newtheorem{example}[theorem]{Example}
\newtheorem{exercise}[theorem]{Exercise}
\newtheorem{lemma}[theorem]{Lemma}
\newtheorem{notation}[theorem]{Notation}
\newtheorem{problem}[theorem]{Problem}
\newtheorem{proposition}[theorem]{Proposition}
\newtheorem{remark}[theorem]{Remark}
\newtheorem{solution}[theorem]{Solution}
\newtheorem{summary}[theorem]{Summary}
\newenvironment{proof}[1][Proof]{\noindent\textbf{#1.} }{\ \rule{0.5em}{0.5em}}
\input{tcilatex}

\begin{document}


\bigskip \bigskip \U{8d75}\U{4e30}\qquad 2013012178\qquad \qquad Functional
Analysis Seventh Week

2.2$\left( a\right) $ $p\left( 0\right) =p\left( 0+0\right) \leq p\left(
0\right) +p\left( 0\right) $ by $\left( i\right) \implies p\left( 0\right)
\geq 0$

On the other hand,$p\left( 0\right) =p\left( x+\left( -x\right) \right) \leq
p\left( x\right) +p\left( -x\right) ,$where $x\in E\implies p\left( 0\right)
=\underset{x\in E}{\min }p\left( x\right) .$

If $p\left( 0\right) >0,$ it is impossible to find a sequence$\left\{
x_{n}\right\} $ in E, s.t.p$\left( x_{n}\right) ->0,$since p$\left(
x_{n}\right) \geq p\left( 0\right) >0\implies $

p$\left( 0\right) =0.$

Following the hint, $\forall \epsilon >0,$we consider $F_{n}=\left\{ \lambda
\in R;\left\vert p\left( \lambda x_{k}\right) \right\vert \leq \epsilon
,\forall k\geq n\right\} ;$

From $\left( iii\right) $ we have $\underset{n=1}{\overset{\infty }{\cup }}%
F_{n}=R.$Then by Baire Category Thm, there exists at least one $F_{n}$ for
some n$,$which has interior points.

\bigskip Now we argue by contradiction that $p\left( \alpha _{n}x_{n}\right)
\nrightarrow 0,$then there exists $\epsilon _{0}$ s.t. $p\left( \alpha
_{n}x_{n}\right) \geq \epsilon _{0}$ holds for a subsequence of $\left\{
\alpha _{n}x_{n}\right\} .$ We can further choose a subsequence s.t. $\alpha
_{n}\rightarrow \alpha ,$since $\left\{ \alpha _{n}\right\} $ is a bounded
sequence in $R.$To simplify the notation we still use the subscript n to
denote this subsequence.

For $\frac{1}{2}\epsilon _{0},$ from the above deduction we can find a 

$F_{n_{0}}$ =$\left\{ \lambda \in R;\left\vert p\left( \lambda x_{k}\right)
\right\vert \leq \frac{\epsilon _{0}}{2},\forall k\geq n_{0}\right\} $which
has interior points.

Suppose $\lambda _{0}$ is an interior point of $F_{n_{0}},$ 

we can find $N_{1},s.t.\left( \lambda _{0}+\alpha _{n}-\alpha \right) \in
F_{n_{0}},$for $n>N_{1},$since  $\alpha _{n}\rightarrow \alpha $

Also from $\left( ii\right) ,$ there exists $N_{2},s.t.$ $p\left( \alpha
x_{n}\right) <\frac{\epsilon _{0}}{2},$for $n>N_{2}.\implies $

$p\left( \alpha _{n}x_{n}\right) \leq p\left( \left( \lambda _{0}+\alpha
_{n}-\alpha \right) x_{n}\right) +p\left( \left( \alpha -\lambda _{0}\right)
x_{n}\right) <\frac{\epsilon _{0}}{2}+\frac{\epsilon _{0}}{2}=\epsilon
_{0},\forall n>\max \left\{ N_{1},N_{2},n_{0}\right\} .$A contradiction with 
$p\left( \alpha _{n}x_{n}\right) \geq \epsilon _{0},\forall n\in N.$

\bigskip And if $\left\{ x_{n}\right\} $ is a sequence in E s.t. $p\left(
x_{n}-x\right) \rightarrow 0,$for some $x\in E,$ and $\left\{ \alpha
_{n}\right\} \subset R,\alpha _{n}\rightarrow \alpha .$

Then $p\left( \alpha _{n}x_{n}\right) \leq p\left( \alpha _{n}\left(
x_{n}-x\right) \right) +p\left( \alpha _{n}x\right) ,$ from the above
deduction we know that  $p\left( \alpha _{n}\left( x_{n}-x\right) \right)
\rightarrow 0.$

$p\left( \alpha _{n}x\right) \leq p\left( \left( \alpha _{n}-\alpha \right)
x\right) +p\left( \alpha x\right) ,$since $p\left( \lambda x\right) $ is
continuous about $\lambda $ and $p\left( 0x\right) =0\implies p\left( \left(
\alpha _{n}-\alpha \right) x\right) \rightarrow 0$

$\implies \underset{n->\infty }{\lim }\sup p\left( \alpha _{n}x_{n}\right)
\leq p\left( \alpha x\right) .$

$p\left( \alpha x\right) \leq p\left( \alpha _{n}x\right) +p\left( \left(
\alpha _{n}-\alpha \right) x\right) \leq p\left( \alpha _{n}x_{n}\right)
+p\left( \alpha _{n}\left( x-x_{n}\right) \right) +p\left( \left( \alpha
_{n}-\alpha \right) x\right) ,\implies p\left( \alpha x\right) \leq \underset%
{n->\infty }{\lim }\inf p\left( \alpha _{n}x\right) $

$\implies \underset{n->\infty }{\lim }p\left( \alpha _{n}x_{n}\right)
=p\left( \alpha x\right) $

2.3 By Cor2.3,$\left\Vert T_{n}\right\Vert \leq c,\forall n\in N,$ and $T\in
L\left( E,F\right) .$

If $x_{n}\rightarrow x,\left\vert T_{n}x_{n}-Tx\right\vert =\left\vert
T_{n}x_{n}-T_{n}x+T_{n}x-Tx\right\vert \leq \left\vert
T_{n}x_{n}-T_{n}x\right\vert +\left\vert T_{n}x-Tx\right\vert .$

By condition we know $\left\vert T_{n}x-Tx\right\vert \rightarrow 0,$ and $%
\left\vert T_{n}x_{n}-T_{n}x\right\vert \leq \left\Vert T_{n}\right\Vert
\left\vert x_{n}-x\right\vert \leq c\left\vert x_{n}-x\right\vert
\rightarrow 0$

$\implies \left\vert T_{n}x_{n}-Tx\right\vert \rightarrow 0,$that is $%
T_{n}x_{n}\rightarrow 0.$

\bigskip 2.4 For each fixed $x\in E$ and $\left\Vert x\right\Vert \leq 1,$%
from $\left( i\right) $ follows $f_{x}=a\left( x,\cdot \right) $ is linear
functional on $F\implies \left( f_{x}\right) _{x\in E}$ is a family of
continuous linear functional from $E$ to $R.$From $\left( ii\right) $
follows $\underset{x\in E,\left\Vert x\right\Vert \leq 1}{\sup }\left\vert
f_{x}\left( y\right) \right\vert =$ $\underset{x\in E,\left\Vert
x\right\Vert \leq 1}{\sup }\left\vert a\left( x,y\right) \right\vert \leq
\left\Vert a\left( \cdot ,y\right) \right\Vert .$ Then by UBT, there exists
a constant $c,s.t.\left\Vert f_{x}\right\Vert \leq c,$for any $\left\Vert
x\right\Vert \leq 1$

$\implies \left\vert a\left( x,y\right) \right\vert =\left\vert f_{x}\left(
y\right) \right\vert \leq \left\Vert f_{x}\right\Vert \leq c,$for $%
\left\Vert y\right\Vert \leq 1.$

Now for arbitrary $x\in E,y\in F,$ substituting respectively$\frac{x}{%
\left\Vert x\right\Vert },\frac{y}{\left\Vert y\right\Vert }$ for $x,y$ gives

$\left\vert a\left( \frac{x}{\left\Vert x\right\Vert },\frac{y}{\left\Vert
y\right\Vert }\right) \right\vert \leq c.$By the linearily of the bilinear
functional a$\implies $

$\left\vert a\left( x,y\right) \right\vert \leq c\left\Vert x\right\Vert
\left\Vert y\right\Vert \boxtimes $

2.5 Following the hint ,we introduce $g_{n}\left( x\right) =\frac{%
f_{n}\left( x\right) }{1+\epsilon _{n}\left\Vert f_{n}\right\Vert },$then $%
\left\langle g_{n},x\right\rangle =\frac{\left\langle f_{n},x\right\rangle }{%
1+\epsilon _{n}\left\Vert f_{n}\right\Vert }\leq \frac{\epsilon
_{n}\left\Vert f_{n}\right\Vert }{1+\epsilon _{n}\left\Vert f_{n}\right\Vert 
}+\frac{C\left( x\right) }{1+\epsilon _{n}\left\Vert f_{n}\right\Vert }$

$\leq 1+C\left( x\right) ,\implies g_{n}$ is bounded pointwise for $%
\left\Vert x\right\Vert <r,$ and by the linearily of $g_{n}\implies g_{n}$
is bounded $\forall x\in E.$By UBT, there exists a constant $c$,s.t.$%
\left\vert \left\langle g_{n},x\right\rangle \right\vert \leq c\left\vert
x\right\vert ,\forall x\in E\forall n\in N.$

$\implies \left\vert \left\langle f_{n},x\right\rangle \right\vert \leq
c\left\vert x\right\vert \left( 1+\epsilon _{n}\left\Vert f_{n}\right\Vert
\right) \implies \left\Vert f_{n}\right\Vert \leq c\left( 1+\epsilon
_{n}\left\Vert f_{n}\right\Vert \right) \implies \left\Vert f_{n}\right\Vert
\leq \frac{c}{1-c\epsilon _{n}},\implies \underset{n->\infty }{\lim }%
\left\Vert f_{n}\right\Vert \leq c,$since $\underset{n->\infty }{\lim }%
\epsilon _{n}=0.\implies \left\{ \left\Vert f_{n}\right\Vert \right\} $ is
bounded in $R\implies \left( f_{n}\right) $ is bounded.

Three added problem:

1.$\varphi \in C\left[ 0,1\right] \U{ff0c} T\varphi \left( t\right)
=\int_{0}^{t}\varphi \left( \xi \right) d\xi ,T\varphi \left( 0\right)
=0,T\varphi ^{\prime }\left( t\right) =\varphi \left( t\right) \in C\left[
0,1\right] \implies T:C\left[ 0,1\right] \implies C^{1,0}\left[ 0,1\right] $

$T$ is obviously linear and $\left\vert T\varphi \right\vert \leq
\int_{0}^{t}\left\vert \varphi \left( \xi \right) \right\vert d\xi \leq 
\underset{0\leq \xi \leq 1}{\max }\left\vert \varphi \left( \xi \right)
\right\vert =\left\Vert \varphi \right\Vert \implies T$ is bounded and $%
\left\Vert T\right\Vert \leq 1.$

For each $C^{1,0}$ function $\phi $ on $\left[ 0,1\right] ,$we can find a
continuous function $\phi ^{\prime }$ on $\left[ 0,1\right] $  $T\phi
^{\prime }=\phi \implies T$ is surjective. If $\int_{0}^{t}\varphi
_{1}\left( \xi \right) d\xi =\int_{0}^{t}\varphi _{2}\left( \xi \right) d\xi
,$taking the derivative gives $\varphi _{1}\left( t\right) =\varphi
_{2}\left( t\right) ,\forall t\in \left[ 0,1\right] \implies $

$T$ is injective. Hence $T^{-1}$ exists. And $T^{-1}\phi =\phi ^{\prime }.$%
Let $\phi _{n}=\sin nx,\left( n=2,3...\right) \left\Vert \phi
_{n}\right\Vert =1$ but $T^{-1}\phi _{n}=n\cos nx\implies \left\Vert
T^{-1}\phi _{n}\right\Vert =n\rightarrow \infty \implies T^{-1}$ is
unbounded.

2. $T$ is an linear continuous bijective functional from $X$ to $Y.$ By open
mapping Thm, there exists $\delta >0,$

$s.t.T\left( B\left( 0,1\right) \right) \supset B\left( 0,\delta \right) ,$
by linearily $T\left( B\left( 0,a\right) \right) =aT\left( B\left(
0,1\right) \right) \supset aB\left( 0,\delta \right) =B\left( 0,a\delta
\right) ,\forall a>0.$

3. It is easy to verify $T$ is linear, and $\left\Vert Tx\right\Vert
=\left\vert \underset{n=1}{\overset{\infty }{\sum }}a_{n}\xi _{n}\right\vert
\leq \underset{n\in N}{\sup }\left\vert a_{n}\right\vert \underset{n=1}{%
\overset{\infty }{\sum }}\left\vert \xi _{n}\right\vert =\left\Vert
x\right\Vert \underset{n\in N}{\sup }\left\vert a_{n}\right\vert $

$\implies T$ is bounded. $T$ is obviously injective.If $\underset{n\in N}{%
\inf }\left\vert a_{n}\right\vert >0\qquad $for $\left\{ \eta _{n}\right\}
\in E,T\left\{ \frac{\eta _{n}}{a_{n}}\right\} =\left\{ \eta _{n}\right\} ,$

where $\underset{n=1}{\overset{\infty }{\sum }}\left\vert \frac{\eta _{n}}{%
a_{n}}\right\vert \leq \frac{1}{\underset{n\in N}{\inf }\left\vert
a_{n}\right\vert }$ $\underset{n=1}{\overset{\infty }{\sum }}\left\vert \eta
_{n}\right\vert <\infty ,$that is $\left\{ \frac{\eta _{n}}{a_{n}}\right\}
\in E$ and $T$ is surjective.

Conversely, if $T$ is surjective, then $\left\{ \frac{\eta _{n}}{a_{n}}%
\right\} \in E$ for any $\left\{ \eta _{n}\right\} \in E.$We define a linear
functional from $E$ to R as $f_{\alpha }\left( \left\{ \eta _{n}\right\}
\right) =$ $\underset{n=1}{\overset{\infty }{\sum }}\epsilon _{n}\frac{\eta
_{n}}{a_{n}},$where the index set $I=\left\{ \alpha |\alpha =\left( \epsilon
_{1},\epsilon _{2},..\epsilon _{n},..\right) ,\epsilon _{i}=1\text{ or }%
-1\right\} $

We can verify that the functional family is bounded pointwisely$\overset{UBT}%
{\rightarrow }$ there exists a constant $c>0$,$s.t.$

$\left\vert f_{I}\left( \left\{ \eta _{n}\right\} \right) \right\vert \leq
c\left\Vert \left\{ \eta _{n}\right\} \right\Vert ,\forall \left\{ \eta
_{n}\right\} \in E,\alpha \in I\implies \left\vert \underset{n=1}{\overset{%
\infty }{\sum }}\epsilon _{n}\frac{\eta _{n}}{a_{n}}\right\vert \leq c$ $%
\underset{n=1}{\overset{\infty }{\sum }}\left\vert \eta _{n}\right\vert ,$%
take $\epsilon _{n}=\left( 0,0,..\underset{nth}{\underbrace{1}},0,0..\right) 
$

gives $\left\vert \frac{\eta _{n}}{a_{n}}\right\vert \leq c$ $\underset{k=1}{%
\overset{\infty }{\sum }}\left\vert \eta _{k}\right\vert ,$ if we choose $%
\left\{ \eta _{n}\right\} s.t.\eta _{n}\neq 0\forall n\in N\implies
\left\vert a_{n}\right\vert \geq \frac{c\underset{k=1}{\overset{\infty }{%
\sum }}\left\vert \eta _{k}\right\vert }{\left\vert \eta _{n}\right\vert }%
\geq c\implies $

$\underset{n\in N}{\inf }\left\vert a_{n}\right\vert \geq c>0.$As a result,$T
$ is surjective$\iff $ $\underset{n\in N}{\inf }\left\vert a_{n}\right\vert
>0.$By OMT and the condition that $T$ is injective, it follows that 

$T$ is surjective$\iff T^{-1}$ exists and bounded. Therefore $T$ is
surjective$\iff $ $\underset{n\in N}{\inf }\left\vert a_{n}\right\vert
>0\boxtimes $

\end{document}
