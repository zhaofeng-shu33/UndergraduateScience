
\documentclass{ctexart}
%%%%%%%%%%%%%%%%%%%%%%%%%%%%%%%%%%%%%%%%%%%%%%%%%%%%%%%%%%%%%%%%%%%%%%%%%%%%%%%%%%%%%%%%%%%%%%%%%%%%%%%%%%%%%%%%%%%%%%%%%%%%%%%%%%%%%%%%%%%%%%%%%%%%%%%%%%%%%%%%%%%%%%%%%%%%%%%%%%%%%%%%%%%%%%%%%%%%%%%%%%%%%%%%%%%%%%%%%%%%%%%%%%%%%%%%%%%%%%%%%%%%%%%%%%%%
\usepackage{amssymb}
\usepackage{amsmath}

\setcounter{MaxMatrixCols}{10}
\def\TEXTsymbol#1{\mbox{$#1$}}%
\def\NEG#1{\leavevmode\hbox{\rlap{\thinspace/}{$#1$}}}%
\def\QATOPD#1#2#3#4{{#3 \atopwithdelims#1#2 #4}}%
\def\QTP#1{}
\def\func#1{\mathop{\rm #1}}%
%TCIDATA{Version=5.00.0.2552}
%TCIDATA{<META NAME="SaveForMode" CONTENT="1">}
%TCIDATA{Created=Friday, November 06, 2015 19:20:26}
%TCIDATA{LastRevised=Wednesday, May 18, 2016 21:44:43}
%TCIDATA{<META NAME="GraphicsSave" CONTENT="32">}
%TCIDATA{<META NAME="DocumentShell" CONTENT="Scientific Notebook\Blank Document">}
%TCIDATA{CSTFile=Math with theorems suppressed.cst}
%TCIDATA{PageSetup=72,72,72,72,0}
%TCIDATA{AllPages=
%F=36,\PARA{038<p type="texpara" tag="Body Text" >\hfill \thepage}
%}


\newtheorem{theorem}{Theorem}
\newtheorem{acknowledgement}[theorem]{Acknowledgement}
\newtheorem{algorithm}[theorem]{Algorithm}
\newtheorem{axiom}[theorem]{Axiom}
\newtheorem{case}[theorem]{Case}
\newtheorem{claim}[theorem]{Claim}
\newtheorem{conclusion}[theorem]{Conclusion}
\newtheorem{condition}[theorem]{Condition}
\newtheorem{conjecture}[theorem]{Conjecture}
\newtheorem{corollary}[theorem]{Corollary}
\newtheorem{criterion}[theorem]{Criterion}
\newtheorem{definition}[theorem]{Definition}
\newtheorem{example}[theorem]{Example}
\newtheorem{exercise}[theorem]{Exercise}
\newtheorem{lemma}[theorem]{Lemma}
\newtheorem{notation}[theorem]{Notation}
\newtheorem{problem}[theorem]{Problem}
\newtheorem{proposition}[theorem]{Proposition}
\newtheorem{remark}[theorem]{Remark}
\newtheorem{solution}[theorem]{Solution}
\newtheorem{summary}[theorem]{Summary}
\newenvironment{proof}[1][Proof]{\noindent\textbf{#1.} }{\ \rule{0.5em}{0.5em}}

\begin{document}




T is a closed operator$\iff $G$\left( T\right) $ is closed in $X\times Y\iff
\forall \left\{ x_{n}\right\} \subset X,x\in X,y\in Y,$satisfying $\left(
x_{n},Tx_{n}\right) \rightarrow \left( x,y\right) ,$then we have $x\in
D\left( T\right) ,y=Tx,i.e.\left( x,y\right) \in G\left( T\right) .$

\bigskip

1. Let $X,Y$ be two n.l.s,T:X$\rightarrow Y$ is a closed linear operator,
and $T^{-1}$ exists,show that $T^{-1}$ is closed.

Pf: $\forall \left\{ y_{n}\right\} \subset R\left( T\right) ,y\in R\left(
T\right) ,x\in x,$satisfying $y_{n}\rightarrow y,T^{-1}y_{n}\rightarrow x,$
to prove that $y\in D\left( T^{-1}\right) ,T^{-1}y=x,$

Since T is closed, T$\left( T^{-1}y_{n}\right) =y_{n}\rightarrow
y,T^{-1}y_{n}\rightarrow x\implies y=Tx,i.e.T^{-1}y=x\boxtimes $

2. $,$Let T: X$\rightarrow Y$ is a linear operator, dimX=dimY\TEXTsymbol{<}$%
\infty ,$show that R$\left( T\right) =Y\iff T^{-1}$ exists(T is injective).

Pf: $\left( \Longrightarrow \right) $ only need to show that T is injective.
Tx=0$\Longrightarrow x=0.$Take a basis $\left\{ b_{1},..b_{n}\right\} $ of
Y, $\exists \left\{ e_{1},..e_{n}\right\} ,s.t.b_{i}=Te_{i},$let $%
k_{1}e_{1}+..+k_{n}e_{n}=0,T\left( \sum k_{i}e_{i}\right) =0\implies \Sigma
k_{i}b_{i}=0\implies k_{i}=0.$

$\implies \left\{ e_{1},..e_{n}\right\} $ is linearly indepedent and since
dimX=n,

they are indeed a basis of X.

Let $x=\Sigma \alpha _{i}e_{i},T\left( x\right) =0\implies \alpha
_{i}=0\implies x=0\implies T$ is injective $\implies T$ is bijective $%
\implies T^{-1}$ exists.

$\left( \Longleftarrow \right) $ T: X$\rightarrow Y$ first show that dimR$%
\left( T\right) \leq \dim X.$Take a basis of X:$\left\{
a_{1},..a_{n}\right\} ,R\left( T\right) =span\left\{ Ta_{1},..Ta_{n}\right\}
\leq n=\dim X.$

Similarly, apply the above conclusion to T$^{-1}\implies R\left( T\right)
\rightarrow X,\implies \dim X\leq \dim R\left( T\right) \implies \dim X=\dim
R\left( T\right) $

$\implies R\left( T\right) =Y,$because they are both finite dimensional
space with same dimension.

3. X,Y are n.l.s T X$\rightarrow Y$ is a bounded linear operator. If $%
\exists b>0,s.t.\forall x\in X,$ we have $\left\Vert Tx\right\Vert \geq
b\left\Vert x\right\Vert $

then $T^{-1}$ $\left( R\left( T\right) \rightarrow X\right) $exists and
bounded.

Pf: T is injective,that is Tx=0$\implies x=0,$

\bigskip $\left\Vert Tx\right\Vert \geq b\left\Vert x\right\Vert \implies
\left\Vert T^{-1}y\right\Vert \leq \frac{1}{b}\left\Vert y\right\Vert
\implies T^{-1}$ is bounded.

4.$\left( a\right) $ T R$^{2}\rightarrow R,\left( \xi _{1},\xi _{2}\right)
\rightarrow \xi _{2},$show that T is an open mapping.

Pf: $\forall $open set $U\subset R^{2},$then we show that T$\left( U\right) $
is open in R.

$\forall x\in T\left( U\right) ,\exists \left( x,y\right) \in U\implies
B\left( \left( x,y\right) ,\delta \right) \subset U\implies \left( x-\delta
,x+\delta \right) \subset T\left( U\right) $

$\implies $T$\left( U\right) $ is open in R.

$\left( b\right) $ S:$R^{2}\rightarrow R^{2},\left( \xi _{1},\xi _{2}\right)
\rightarrow \left( \xi _{1},0\right) ,$ then show that S is not an open
mapping.

A two-dimensional open ball is mapped to a segment, which is not open in R$%
^{2}.$

$\left( c\right) $ Open mapping T does not necessary map a closed set onto a
closed set.

Solution:$\left( a\right) $ gives a definition of T, which maps a closed set 
$\left\{ \left( x,y\right) |xy=1\right\} $ onto R\TEXTsymbol{\backslash}$%
\left\{ 0\right\} ,$ which is open.

5. X is a Banach space, L and M are two closed subspaces of X, and X=L$%
\oplus M,$show that the point sequence $\left\{ x_{n}\right\} \rightarrow
x\iff \exists x_{l}^{\left( n\right) }\rightarrow x_{l}$ and $x_{m}^{\left(
n\right) }\rightarrow x_{m},s.t.x_{n}=x_{l}^{\left( n\right) }+x_{m}^{\left(
n\right) },x=x_{l}+x_{m}.$

$\left( \implies ,\text{refer to Thm 2.10 on page 37 of textbook}\right) $%
Define a new norm $\left\Vert \cdot \right\Vert _{1}$ on X,:$\left\Vert
x\right\Vert _{1}=\left\Vert l\right\Vert +\left\Vert m\right\Vert ,x=l+m$
uniquely,then $\left\Vert x\right\Vert \leq \left\Vert x\right\Vert _{1}.$
We define an identical mapping from $\left( X,\left\Vert {}\right\Vert
\right) $ $\rightarrow \left( X,\left\Vert {}\right\Vert _{1}\right) ,$We
should verify that $\left( X,\left\Vert {}\right\Vert _{1}\right) $ is a
Banach space. Take $\left\{ x_{n}\right\} \subset \left( X,\left\Vert
{}\right\Vert _{1}\right) $ is a Cauchy sequence, $\left\Vert
x_{n}-x_{p}\right\Vert _{1}\rightarrow 0,\left( n,p\rightarrow \infty
\right) .$

$\implies \left\Vert x_{l}^{\left( n\right) }-x_{l}^{\left( p\right)
}\right\Vert +\left\Vert x_{m}^{\left( n\right) }-x_{m}^{\left( p\right)
}\right\Vert \rightarrow 0\implies \left\{ x_{m}^{\left( n\right) }\right\} $
and $\left\{ x_{l}^{\left( n\right) }\right\} $ are both Cauchy sequence in
L and M. $\implies \exists x_{m}\leftarrow x_{m}^{\left( n\right)
},x_{l}\leftarrow x_{l}^{\left( n\right) },\implies x_{n}\rightarrow
x_{m}+x_{l}$ in $\left\Vert \cdot \right\Vert _{1}.$

\bigskip By inverse mapping Thm,$\exists c>0$ $s.t.\left\Vert x\right\Vert
_{1}\leq c\left\Vert x\right\Vert .$

$\left( \Longleftarrow \right) $Obviously.

6. $X,X_{1},X_{2}$ are Banach spaces,$T_{1}:X\rightarrow
X_{1},T_{2}:X\rightarrow X_{2}$ are closed l.op. and D$\left( T_{1}\right)
\subset D\left( T_{2}\right) ,$show that

there exists c\TEXTsymbol{>}0,s.t. $\left\Vert T_{2}x\right\Vert \leq
c\left( \left\Vert T_{1}x\right\Vert +\left\Vert x\right\Vert \right)
,\forall x\in X.$

Pf: Define a new norm $\left\Vert \cdot \right\Vert _{1}$on X$:\left\Vert
x\right\Vert _{1}=\left\Vert T_{1}x\right\Vert +\left\Vert x\right\Vert .$

Clearly,we have $\left\Vert x\right\Vert _{1}\leq \left\Vert x\right\Vert .$

Suppose $\left\Vert x_{n}-x_{p}\right\Vert _{1}\rightarrow 0,\left(
n,p\rightarrow \infty \right) ,\implies \left\Vert x_{n}-x_{p}\right\Vert
\rightarrow 0.$

$X$ is a Banach space $\implies \exists x\in X,s.t.x_{n}\rightarrow x.$

Also we have $\left\Vert T_{1}x_{n}-T_{1}x_{p}\right\Vert \rightarrow
0,X_{1} $ is a Banach space$\implies T_{1}x_{n}\rightarrow y.$

$T_{1}$ is closed $\implies y=T_{1}x\implies x_{n}\rightarrow x$ in $%
\left\Vert \cdot \right\Vert _{1}.$

$\implies \left( X,\left\Vert \cdot \right\Vert _{1}\right) $ is a Banach
space. Similar to 5, we can show the equivalence of $\left\Vert \cdot
\right\Vert $ and $\left\Vert \cdot \right\Vert _{1},$that is, $\exists
c_{1}>0,s.t.\left\Vert x\right\Vert \leq c_{1}\left\Vert x\right\Vert _{1}$

$T_{2}:X\rightarrow X_{2},$ by closed graph Thm

$\implies T_{2}$ is continuous$\left\Vert T_{2}x\right\Vert \leq c^{\prime
}\left\Vert x\right\Vert \leq c_{1}c^{\prime }\left\Vert x\right\Vert _{1}$

$\implies \left\Vert T_{2}x\right\Vert \leq c\left\Vert T_{1}x\right\Vert
+\left\Vert x\right\Vert ,$where $c=c_{1}c^{\prime }.$

(上面是我的做法,助教%
解法大意为在X$\times X_{1}$上%
定义norm $\left\Vert \left( x,x_{1}\right) \right\Vert
=\left\Vert x\right\Vert +\left\Vert x_{1}\right\Vert $

然后证明X$\times X_{1}$的子空%
间G$\left( T_{1}\right) $ is a Banach space,做一个%
把G$\left( T_{1}\right) $映到X$_{2}$的映%
射

$\left( x,T_{1}x\right) \rightarrow T_{2}x,$证明此映%
射 is closed,再用closed graph Thm即得要%
证的结论)

7. Let X be a Banach space, P: X$\rightarrow X$ is a linear operator and
satisfies P$^{2}=P.$

Show that P is bounded if both N$\left( P\right) $ and R$\left( P\right) $
are closed.

$\forall x_{n}\in X,x_{n}\rightarrow x,Px_{n}\rightarrow y,\implies
Px_{n}-x_{n}\rightarrow y-x.$

Notice that $P\left( Px_{n}-x_{n}\right) =P^{2}x_{n}-Px_{n}=0$

N$\left( P\right) $ is closed$\implies y-x\in N\left( P\right) \implies
P\left( y\right) =P\left( x\right) $

Also R$\left( P\right) $ are closed and $Px_{n}\rightarrow y\implies \exists
x_{1}\in X,s.t.P\left( x_{1}\right) =y$

$\implies P^{2}x_{1}=Px\implies y=Px_{1}=P^{2}x_{1}=Px\implies P$ is closed.

Also R$\left( P\right) $ is a Banach space,by closed graph Thm$\implies $P
is bounded

8.Let X,Y be normed vector space, T: D$\left( T\right) \subset X\rightarrow
Y $ be a bounded linear operator.$\left( 1\right) $ if D$\left( T\right) $
is closed,show that T is closed.

$\left( 2\right) $ if T is closed, and Y is complete, show that D$\left(
T\right) $ is closed.

Solution$\left( 1\right) $Suppose X$_{n}\in D\left( T\right) \rightarrow
x,Tx_{n}\rightarrow y.$

D$\left( T\right) $ is closed$\implies x\in D\left( T\right) ,T$ is bounded $%
\implies Tx_{n}\rightarrow Tx=y$

$\implies $T is closed

$\left( 2\right) $ Suppose x$_{n}\in D\left( T\right) \rightarrow
x,\left\Vert Tx_{n}-Tx_{p}\right\Vert \leq c\left\Vert
x_{n}-x_{p}\right\Vert \rightarrow 0\left( n,p\rightarrow \infty \right) $

$\left\{ Tx_{n}\right\} $ is a Cauchy sequence in Y$\implies \exists y\in
Y,s.t.Tx_{n}\rightarrow y$

T is closed$\implies x\in D\left( T\right) \implies $D$\left( T\right) $ is
closed.

9 Let X,Y be two linear spaces, T: D$\left( T\right) \subset X\rightarrow Y$
is a linear operator.

Then T can be extended linearly to \~{T} s.t. G$\left( \tilde{T}\right) =%
\overline{G\left( T\right) }$ iff there are no elements like

$\left( 0,y\right) _{y\neq 0}$ in $\overline{G\left( T\right) }.$

$\left( \implies \right) \left( 0,y\right) \in \overline{G\left( T\right) }%
=G\left( \tilde{T}\right) ,\tilde{T}\left( 0\right) =y\implies y=0$

$\left( \Longleftarrow \right) \forall \left( x,y\right) \in \overline{%
G\left( T\right) },$we define $\tilde{T}\left( x\right) =y.$

To verify such definition is reasonable, suppose we have $\left(
x,y_{1}\right) ,\left( x,y_{2}\right) \in \overline{G\left( T\right) }$

$\implies \exists \left\{ x_{n}^{\left( i\right) }\right\} \in D\left(
T\right) ,s.t.x_{n}^{\left( i\right) }\rightarrow x,T\left( x_{n}^{\left(
i\right) }\right) \rightarrow y_{i},i=1,2\implies x_{n}^{\left( 1\right)
}-x_{n}^{\left( 2\right) }\rightarrow 0,$

$T\left( x_{n}^{\left( 1\right) }-x_{n}^{\left( 2\right) }\right) =T\left(
x_{n}^{\left( 1\right) }\right) -T\left( x_{n}^{\left( 2\right) }\right)
\rightarrow y_{1}-y_{2},$and $x_{n}^{\left( 1\right) }-x_{n}^{\left(
2\right) }\in D\left( T\right) $ $\implies \left( 0,y_{1}-y_{2}\right) \in 
\overline{G\left( T\right) }\implies y_{1}=y_{2}$ by the condition.

By the definition of \~{T},G$\left( \tilde{T}\right) =\overline{G\left(
T\right) }.$

And as $x\in D\left( T\right) ,\left( x,T\left( x\right) \right) \in 
\overline{G\left( T\right) },$by the uniqueness of image$\implies \tilde{T}%
\left( x\right) =T\left( x\right) $

$\implies \tilde{T}$ is an extension of T.

Suppose $\left( x_{i},\tilde{T}\left( x_{i}\right) \right) \in \overline{%
G\left( T\right) },i=1,2;\exists \left\{ x_{n}^{\left( i\right) }\right\}
\in D\left( T\right) ,s.t.x_{n}^{\left( i\right) }\rightarrow x_{i},T\left(
x_{n}^{\left( i\right) }\right) \rightarrow \tilde{T}\left( x_{i}\right)
,i=1,2$

$\implies x_{n}^{\left( 1\right) }+x_{n}^{\left( 2\right) }\rightarrow
x_{1}+x_{2},$

$T\left( x_{n}^{\left( 1\right) }+x_{n}^{\left( 2\right) }\right) =T\left(
x_{n}^{\left( 1\right) }\right) +T\left( x_{n}^{\left( 2\right) }\right)
\rightarrow \tilde{T}\left( x_{1}\right) +\tilde{T}\left( x_{2}\right) ,$and 
$x_{n}^{\left( 1\right) }+x_{n}^{\left( 2\right) }\in D\left( T\right) $

$\implies \left( x_{1}+x_{2},\tilde{T}\left( x_{1}\right) +\tilde{T}\left(
x_{2}\right) \right) \in \overline{G\left( T\right) }$

$\implies \tilde{T}\left( x_{1}+x_{2}\right) =\tilde{T}\left( x_{1}\right) +%
\tilde{T}\left( x_{2}\right) $

$\implies \tilde{T}$ is an linear extension of T.

10 Let $\left( X,\left\Vert \cdot \right\Vert _{1}\right) ,\left(
X,\left\Vert \cdot \right\Vert _{2}\right) $ be two Banach spaces, $%
\left\Vert x_{n}\right\Vert _{1}\rightarrow 0\implies \left\Vert
x_{n}\right\Vert _{2}\rightarrow 0.$

Show that $\left\Vert \cdot \right\Vert _{1}$ is equivalent to $\left\Vert
\cdot \right\Vert _{2}.$

The condition says that the identify mapping from $\left( X,\left\Vert \cdot
\right\Vert _{1}\right) $ to $\left( X,\left\Vert \cdot \right\Vert
_{2}\right) $ is continuous at 0. By linearity, it is continuous everywhere
and it is also bounded:

$\left\Vert x\right\Vert _{2}\leq c\left\Vert x\right\Vert _{1},\left(
X,\left\Vert \cdot \right\Vert _{1}\right) ,\left( X,\left\Vert \cdot
\right\Vert _{2}\right) $ be two Banach spaces,by Inverse Mapping Thm

$\implies \left\Vert \cdot \right\Vert _{1}$ is equivalent to $\left\Vert
\cdot \right\Vert _{2}.$


\end{document}
