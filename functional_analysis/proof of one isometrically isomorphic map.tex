
\documentclass{article}
\usepackage{amssymb}
\usepackage{amsmath}

%%%%%%%%%%%%%%%%%%%%%%%%%%%%%%%%%%%%%%%%%%%%%%%%%%%%%%%%%%%%%%%%%%%%%%%%%%%%%%%%%%%%%%%%%%%%%%%%%%%%%%%%%%%%%%%%%%%%%%%%%%%%%%%%%%%%%%%%%%%%%%%%%%%%%%%%%%%%%%%%%%%%%%%%%%%%%%%%%%%%%%%%%%%%%%%%%%%%%%%%%%%%%%
%TCIDATA{OutputFilter=LATEX.DLL}
%TCIDATA{Version=5.00.0.2552}
%TCIDATA{<META NAME="SaveForMode" CONTENT="1">}
%TCIDATA{Created=Monday, September 28, 2015 11:28:18}
%TCIDATA{LastRevised=Monday, September 28, 2015 13:30:19}
%TCIDATA{<META NAME="GraphicsSave" CONTENT="32">}
%TCIDATA{<META NAME="DocumentShell" CONTENT="Standard LaTeX\Blank - Standard LaTeX Article">}
%TCIDATA{CSTFile=40 LaTeX article.cst}

\newtheorem{theorem}{Theorem}
\newtheorem{acknowledgement}[theorem]{Acknowledgement}
\newtheorem{algorithm}[theorem]{Algorithm}
\newtheorem{axiom}[theorem]{Axiom}
\newtheorem{case}[theorem]{Case}
\newtheorem{claim}[theorem]{Claim}
\newtheorem{conclusion}[theorem]{Conclusion}
\newtheorem{condition}[theorem]{Condition}
\newtheorem{conjecture}[theorem]{Conjecture}
\newtheorem{corollary}[theorem]{Corollary}
\newtheorem{criterion}[theorem]{Criterion}
\newtheorem{definition}[theorem]{Definition}
\newtheorem{example}[theorem]{Example}
\newtheorem{exercise}[theorem]{Exercise}
\newtheorem{lemma}[theorem]{Lemma}
\newtheorem{notation}[theorem]{Notation}
\newtheorem{problem}[theorem]{Problem}
\newtheorem{proposition}[theorem]{Proposition}
\newtheorem{remark}[theorem]{Remark}
\newtheorem{solution}[theorem]{Solution}
\newtheorem{summary}[theorem]{Summary}
\newenvironment{proof}[1][Proof]{\noindent\textbf{#1.} }{\ \rule{0.5em}{0.5em}}
\input{tcilatex}

\begin{document}


Show that $\left( l^{1}\right) ^{\prime }\overset{is-is}{=}l^{\infty }.$

Proof: $\forall t\in l^{\infty },$define $F_{t}\left( s\right) =\underset{i=1%
}{\overset{\infty }{\sum }}t_{i}s_{i},$for $s=\left\{ s_{i}\right\}
_{i=1}^{\infty }\in l^{1}.$

Since $t=\left\{ t_{i}\right\} _{i=1}^{\infty }$ is bounded, the definition
is proper and obviously $F_{t}$ is linear.

Furthermore, $\left\vert F_{t}\left( s\right) \right\vert \leq \left\vert
t\right\vert _{\infty }\left\vert s\right\vert _{1}\rightarrow \left\Vert
F_{t}\right\Vert \leq \left\vert t\right\vert _{\infty }.$

On the other hand, let $s_{n}=\underset{i=1}{\overset{n}{\sum }}a_{i}e_{i},$%
where $e_{i}=\{\delta _{ij}\}_{j=1}^{\infty }.$

$\left\vert F_{t}\left( s_{n}\right) \right\vert \leq \left\Vert
F_{t}\right\Vert \left\vert s_{n}\right\vert _{1}\rightarrow \left\vert 
\underset{i=1}{\overset{n}{\sum }}a_{i}F\left( e_{i}\right) \right\vert \leq
\left\Vert F_{t}\right\Vert \underset{i=1}{\overset{n}{\sum }}\left\vert
a_{i}\right\vert ,$

Let $a_{i}=\left\{ 
\begin{array}{c}
1,\text{if }\left\vert t_{i}\right\vert =\underset{1\leq k\leq n}{\max }%
\left\vert t_{k}\right\vert  \\ 
0,\text{otherwise}%
\end{array}%
\right. $ for $i=1,2...n.$Combined with $F\left( e_{i}\right) =\underset{j=1}%
{\overset{\infty }{\sum }}t_{j}\delta _{ij}=t_{i},$

it follows that $\underset{1\leq k\leq n}{\max }\left\vert t_{k}\right\vert
\leq \left\Vert F_{t}\right\Vert .$

Let $n\rightarrow \infty ,\implies \left\vert t\right\vert _{\infty }\leq
\left\Vert F_{t}\right\Vert .\implies \left\Vert F_{t}\right\Vert
=\left\vert t\right\vert _{\infty },F_{t}$ is linear bounded operator on $%
l^{1}\implies F_{t}\in \left( l^{1}\right) ^{\prime }$ .

Then we can define a functional $U$ from $l^{\infty }$ to $\left(
l^{1}\right) ^{\prime },s.t.U\left( t\right) =F_{t}.$

To show the functional is onto, we take an arbitrary function $F$ from $%
\left( l^{1}\right) ^{\prime }.$

For each $s=\left\{ s_{i}\right\} _{i=1}^{\infty }\in l^{1},$first we show
that 

$s=\underset{n->\infty }{\lim }\underset{j=1}{\overset{n}{\sum }}%
s_{j}e_{j}.\left\vert s-\underset{j=1}{\overset{n}{\sum }}%
s_{j}e_{j}\right\vert _{1}=\underset{j=n+1}{\overset{\infty }{\sum }}%
\left\vert s_{j}\right\vert \rightarrow 0,$as $n\rightarrow \infty ,$since $%
\underset{j=1}{\overset{\infty }{\sum }}\left\vert s_{j}\right\vert
=\left\vert s\right\vert _{1}$ converges.

then $F\left( s\right) =F\left( \underset{n->\infty }{\lim }\underset{j=1}{%
\overset{n}{\sum }}s_{j}e_{j}\right) =\underset{n->\infty }{\lim }F\left( 
\underset{j=1}{\overset{n}{\sum }}s_{j}e_{j}\right) $, since $F$ is
continuous.

the above equality can be further rearranged to $\underset{n->\infty }{\lim }%
\underset{j=1}{\overset{n}{\sum }}s_{j}F\left( e_{j}\right) =\underset{j=1}{%
\overset{\infty }{\sum }}s_{j}F\left( e_{j}\right) $

Denote $t=\left\{ t_{i}\right\} _{i=1}^{\infty }=\left\{ F\left(
e_{j}\right) \right\} _{j=1}^{\infty },$ we get $F\left( s\right) =\underset{%
j=1}{\overset{\infty }{\sum }}s_{j}t_{j},$which has the same form as $F_{t}$
provided that

$\left\{ F\left( e_{j}\right) \right\} _{j=1}^{\infty }\in l^{\infty }.$

The condition follows from the bounded property of $F.$

In the equality $F\left( s\right) =\underset{j=1}{\overset{\infty }{\sum }}%
s_{j}t_{j},$we repeat the same manipulation as above$\left( \text{take
special }s\right) ,$we can get

$\underset{1\leq k\leq n}{\max }\left\vert t_{k}\right\vert \leq \left\Vert
F\right\Vert .$Let $n\rightarrow \infty ,$ $t=$ $\underset{k\geq 1}{\sup }%
\left\vert t_{k}\right\vert $ is indeed bounded.

Thus $U\left( t\right) =F,$and the functional is onto.

$\left( l^{1}\right) ^{\prime }\overset{is-is}{=}l^{\infty }.$

The proof is complete.$\boxtimes $

\end{document}
