
\documentclass{article}
\usepackage{amssymb}
\usepackage{amsmath}

%%%%%%%%%%%%%%%%%%%%%%%%%%%%%%%%%%%%%%%%%%%%%%%%%%%%%%%%%%%%%%%%%%%%%%%%%%%%%%%%%%%%%%%%%%%%%%%%%%%%%%%%%%%%%%%%%%%%%%%%%%%%%%%%%%%%%%%%%%%%%%%%%%%%%%%%%%%%%%%%%%%%%%%%%%%%%%%%%%%%%%%%%%%%%%%%%%%%%%%%%%%%%%
%TCIDATA{OutputFilter=LATEX.DLL}
%TCIDATA{Version=5.00.0.2552}
%TCIDATA{<META NAME="SaveForMode" CONTENT="1">}
%TCIDATA{Created=Tuesday, September 15, 2015 09:39:58}
%TCIDATA{LastRevised=Tuesday, September 15, 2015 15:47:46}
%TCIDATA{<META NAME="GraphicsSave" CONTENT="32">}
%TCIDATA{<META NAME="DocumentShell" CONTENT="Standard LaTeX\Blank - Standard LaTeX Article">}
%TCIDATA{CSTFile=40 LaTeX article.cst}

\newtheorem{theorem}{Theorem}
\newtheorem{acknowledgement}[theorem]{Acknowledgement}
\newtheorem{algorithm}[theorem]{Algorithm}
\newtheorem{axiom}[theorem]{Axiom}
\newtheorem{case}[theorem]{Case}
\newtheorem{claim}[theorem]{Claim}
\newtheorem{conclusion}[theorem]{Conclusion}
\newtheorem{condition}[theorem]{Condition}
\newtheorem{conjecture}[theorem]{Conjecture}
\newtheorem{corollary}[theorem]{Corollary}
\newtheorem{criterion}[theorem]{Criterion}
\newtheorem{definition}[theorem]{Definition}
\newtheorem{example}[theorem]{Example}
\newtheorem{exercise}[theorem]{Exercise}
\newtheorem{lemma}[theorem]{Lemma}
\newtheorem{notation}[theorem]{Notation}
\newtheorem{problem}[theorem]{Problem}
\newtheorem{proposition}[theorem]{Proposition}
\newtheorem{remark}[theorem]{Remark}
\newtheorem{solution}[theorem]{Solution}
\newtheorem{summary}[theorem]{Summary}
\newenvironment{proof}[1][Proof]{\noindent\textbf{#1.} }{\ \rule{0.5em}{0.5em}}


\begin{document}


\FRAME{ftbpF}{5.7692in}{1.7694in}{0pt}{}{}{Figure}{\special{language
"Scientific Word";type "GRAPHIC";maintain-aspect-ratio TRUE;display
"USEDEF";valid_file "T";width 5.7692in;height 1.7694in;depth
0pt;original-width 9.5986in;original-height 2.9179in;cropleft "0";croptop
"1";cropright "1";cropbottom "0";tempfilename
'NUPKBM01.wmf';tempfile-properties "XPR";}}Proof: for finite-dimensional
vector space the conclusion is trivial from the step by step spanning of
linear independent group. For infinite-dimensional space, Let S=\{a linear
indepedent group of E\}, every element of S can be both finite or infinite
subset of E. Then we can use Zorn's lemma to show that S has a maximum
element with the partial relationship "set inclusion". 

Firstly, we show that every "chain" of S has an upperbound. For a chian C'
of S, suppose C'=\{e1,e2...\},Let up=$\underset{i=1}{\overset{\infty }{\cup }%
}e_{i}.$Since C' is a totally ordered set, for each finite elements $%
g_{p1},g_{p2},...g_{pt}$ of up, we can find a e$_{i}$ such that $%
g_{p1},g_{p2}...g_{pt\text{ }}$all belong to e$_{i},$hence they are linearly
independent. The choice is arbitrary and therefore up is a linear
independent group of E, which belongs to S. It is obviously that up is the
upperbound of C'. Then the proof of "every chian has an upperbound is
complete.

Secondly, by Zorn's lemma, there exists a maximal $M$ of S, that is, for
each element $A$ of S, it is impossible that $A\nsupseteqq M.$ Then for each
element $e$ of E, if the negation of the conclusion holds, then we add $e$
to $M,$which is also a linear indepedent group by definition, but containing
M thus contrasting the maximal property of $M.$ In this way the proof of the
conclusion is complete.$\boxdot $ 

\end{document}
