
\documentclass{article}
\usepackage{amsmath}

%%%%%%%%%%%%%%%%%%%%%%%%%%%%%%%%%%%%%%%%%%%%%%%%%%%%%%%%%%%%%%%%%%%%%%%%%%%%%%%%%%%%%%%%%%%%%%%%%%%%%%%%%%%%%%%%%%%%%%%%%%%%%%%%%%%%%%%%%%%%%%%%%%%%%%%%%%%%%%%%%%%%%%%%%%%%%%%%%%%%%%%%%%%%%%%%%%%%%%%%%%%%%%%%%%%%%%%%%%%%%%%%%%%%
%TCIDATA{OutputFilter=LATEX.DLL}
%TCIDATA{Version=5.00.0.2552}
%TCIDATA{<META NAME="SaveForMode" CONTENT="1">}
%TCIDATA{Created=Sunday, December 06, 2015 21:58:16}
%TCIDATA{LastRevised=Sunday, December 06, 2015 21:59:05}
%TCIDATA{<META NAME="GraphicsSave" CONTENT="32">}
%TCIDATA{<META NAME="DocumentShell" CONTENT="Scientific Notebook\Blank Document">}
%TCIDATA{CSTFile=Math with theorems suppressed.cst}
%TCIDATA{PageSetup=72,72,72,72,0}
%TCIDATA{AllPages=
%F=36,\PARA{038<p type="texpara" tag="Body Text" >\hfill \thepage}
%}


\newtheorem{theorem}{Theorem}
\newtheorem{acknowledgement}[theorem]{Acknowledgement}
\newtheorem{algorithm}[theorem]{Algorithm}
\newtheorem{axiom}[theorem]{Axiom}
\newtheorem{case}[theorem]{Case}
\newtheorem{claim}[theorem]{Claim}
\newtheorem{conclusion}[theorem]{Conclusion}
\newtheorem{condition}[theorem]{Condition}
\newtheorem{conjecture}[theorem]{Conjecture}
\newtheorem{corollary}[theorem]{Corollary}
\newtheorem{criterion}[theorem]{Criterion}
\newtheorem{definition}[theorem]{Definition}
\newtheorem{example}[theorem]{Example}
\newtheorem{exercise}[theorem]{Exercise}
\newtheorem{lemma}[theorem]{Lemma}
\newtheorem{notation}[theorem]{Notation}
\newtheorem{problem}[theorem]{Problem}
\newtheorem{proposition}[theorem]{Proposition}
\newtheorem{remark}[theorem]{Remark}
\newtheorem{solution}[theorem]{Solution}
\newtheorem{summary}[theorem]{Summary}
\newenvironment{proof}[1][Proof]{\noindent\textbf{#1.} }{\ \rule{0.5em}{0.5em}}


\begin{document}


Uniform Convex 定义等价性的证%
明,即$\left\Vert x\right\Vert \leq 1,\left\Vert y\right\Vert
\leq 1$的条件可以弱化为

$\left\Vert x\right\Vert =1,\left\Vert y\right\Vert =1.$

事实上,$\forall \epsilon >0,\left\Vert x-y\right\Vert
>\epsilon ,$with $\left\Vert x\right\Vert \leq 1,\left\Vert y\right\Vert
\leq 1,$to show there exists

$\delta _{\epsilon }>0,s.t.\left\Vert x+y\right\Vert <2-\delta _{\epsilon }.$
Consider $\frac{x}{\left\Vert x\right\Vert }-\frac{y}{\left\Vert
y\right\Vert }$

$=\frac{\left( x-y\right) \left\Vert y\right\Vert +y\left( \left\Vert
y\right\Vert -\left\Vert x\right\Vert \right) }{\left\Vert x\right\Vert
\left\Vert y\right\Vert }\implies $ $\left\Vert \frac{x}{\left\Vert
x\right\Vert }-\frac{y}{\left\Vert y\right\Vert }\right\Vert \geq \frac{%
\left\Vert y-x\right\Vert -\left( \left\vert \left\Vert y\right\Vert
-\left\Vert x\right\Vert \right\vert \right) }{\left\Vert x\right\Vert },$by
triangular inequality of the norm. If $\left\vert \left\Vert y\right\Vert
-\left\Vert x\right\Vert \right\vert <\frac{\epsilon }{2},$then $\left\Vert 
\frac{x}{\left\Vert x\right\Vert }-\frac{y}{\left\Vert y\right\Vert }%
\right\Vert \geq \frac{\epsilon }{2},$ by the known conclusion for $%
\left\Vert \frac{x}{\left\Vert x\right\Vert }\right\Vert =1,\left\Vert \frac{%
y}{\left\Vert y\right\Vert }\right\Vert =1\implies \left\Vert \frac{x}{%
\left\Vert x\right\Vert }+\frac{y}{\left\Vert y\right\Vert }\right\Vert
<2-\delta _{\epsilon }^{\prime },$for this case. $\left\Vert \frac{x}{%
\left\Vert x\right\Vert }+\frac{y}{\left\Vert y\right\Vert }\right\Vert \geq
\left\Vert \frac{\left( x+y\right) \left\Vert y\right\Vert +y\left\Vert
x\right\Vert -y\left\Vert y\right\Vert }{\left\Vert x\right\Vert \left\Vert
y\right\Vert }\right\Vert \geq \frac{\left\Vert x+y\right\Vert \left\Vert
y\right\Vert -\left\Vert y\right\Vert \left\vert \left\Vert x\right\Vert
-\left\Vert y\right\Vert \right\vert }{\left\Vert x\right\Vert \left\Vert
y\right\Vert }$

$=\frac{\left\Vert x+y\right\Vert -\left\vert \left\Vert x\right\Vert
-\left\Vert y\right\Vert \right\vert }{\left\Vert x\right\Vert }\geq
\left\Vert x+y\right\Vert -\left\vert \left\Vert x\right\Vert -\left\Vert
y\right\Vert \right\vert ,$further we choose

$\left\vert \left\Vert x\right\Vert -\left\Vert y\right\Vert \right\vert
<\min \left\{ \frac{\delta _{\epsilon }^{\prime }}{2},\frac{\epsilon }{2}%
\right\} \implies $ $\left\Vert \frac{x}{\left\Vert x\right\Vert }+\frac{y}{%
\left\Vert y\right\Vert }\right\Vert \geq \left\Vert x+y\right\Vert -\frac{%
\delta _{\epsilon }^{\prime }}{2}\implies \left\Vert x+y\right\Vert <2-\frac{%
\delta _{\epsilon }^{\prime }}{2}.$

If $\left\vert \left\Vert y\right\Vert -\left\Vert x\right\Vert \right\vert
\geq \min \left\{ \frac{\delta _{\epsilon }^{\prime }}{2},\frac{\epsilon }{2}%
\right\} ,$ We can directly choose $\delta _{\epsilon }^{\prime \prime
}<2-\left\Vert x+y\right\Vert .$

Indeed, $\left\Vert x+y\right\Vert \leq \left\Vert x\right\Vert +\left\Vert
y\right\Vert $

$=2\max \left\{ \left\Vert x\right\Vert ,\left\Vert y\right\Vert \right\}
-\left\vert \left\Vert x\right\Vert -\left\Vert y\right\Vert \right\vert
\leq 2-\min \left\{ \frac{\delta _{\epsilon }^{\prime }}{2},\frac{\epsilon }{%
2}\right\} $

$\implies \min \left\{ \frac{\delta _{\epsilon }^{\prime }}{2},\frac{%
\epsilon }{2}\right\} \leq 2-\left\Vert x+y\right\Vert ,$ Thus we can choose 
$\delta _{\epsilon }^{\prime \prime }<\min \left\{ \frac{\delta _{\epsilon
}^{\prime }}{2},\frac{\epsilon }{2}\right\} ,$

$\implies \left\Vert x+y\right\Vert <2-\delta _{\epsilon }^{\prime \prime },$
Combining the two cases together, $\left\Vert x+y\right\Vert <2-\min \left\{
\delta _{\epsilon }^{\prime \prime },\frac{\delta _{\epsilon }^{\prime }}{2}%
\right\} ,\forall \left\Vert x\right\Vert \leq 1,\left\Vert y\right\Vert
\leq 1,$with $\left\Vert x-y\right\Vert >\epsilon .$

Suppose E is a reflexive Banach space and M is a closed linear space of E.
Show that M is also a reflexive space.

We only need to show that the canonical map $J:M\rightarrow M^{\prime \prime
}$ is surjective.

Given $x^{\prime \prime }\in M^{\prime \prime },x^{\prime \prime }:M^{\prime
}\rightarrow R,$we define  $\tilde{x}^{\prime \prime }$ by  $\tilde{x}%
^{\prime \prime }\left( x^{\prime }\right) :=x^{\prime \prime }\left(
x_{|M}^{\prime }\right) ,\forall x^{\prime }\in E,$

where $x_{|M}^{\prime }$ is the restriction of $x^{\prime }:E\rightarrow R$
on $M.$

Then we get $\tilde{x}^{\prime \prime }:E^{\prime }\rightarrow R.$

$\left\vert \tilde{x}^{\prime \prime }\left( x^{\prime }\right) \right\vert
\leq \left\Vert x^{\prime \prime }\right\Vert \left\vert x^{\prime
}\right\vert <\infty ,$ $\implies \tilde{x}^{\prime \prime }\in E^{\prime
\prime }$

Since E is reflexive, we can find $x\in E,s.t.$

$\tilde{x}^{\prime \prime }\left( x^{\prime }\right) =x^{\prime }\left(
x\right) \forall x^{\prime }\in E^{\prime }.$

To show $x\in M,$ suppose $x\notin M,$ then since $M$ is closed linear
space, we can find $f\in E^{\prime },s.t.f_{|M}=0,f\left( x\right)
>0,\implies $

$\tilde{x}^{\prime \prime }\left( f\right) =f\left( x\right) >0,$ but from
the definition of $\tilde{x}^{\prime \prime },$ $\tilde{x}^{\prime \prime
}\left( f\right) =0.$ A contradiction

$\implies x\in M.$

$\forall f\in M^{\prime },J\left( x\right) \left( f\right) =f\left( x\right)
,$ by H-B Thm, $f$=$g_{|M},g\in E^{\prime }\implies $

$J\left( x\right) \left( g_{|M}\right) =g_{|M}\left( x\right) =g\left(
x\right) ,$ 

Also we have $x^{\prime \prime }\left( g_{|M}\right) =\tilde{x}^{\prime
\prime }\left( g\right) =g\left( x\right) \implies x^{\prime \prime }\left(
f\right) =J\left( x\right) \left( f\right) \forall f\in M^{\prime }\implies
x^{\prime \prime }=J\left( x\right) .$

\end{document}
