
\documentclass{article}
\usepackage{amsmath}

%%%%%%%%%%%%%%%%%%%%%%%%%%%%%%%%%%%%%%%%%%%%%%%%%%%%%%%%%%%%%%%%%%%%%%%%%%%%%%%%%%%%%%%%%%%%%%%%%%%%%%%%%%%%%%%%%%%%%%%%%%%%%%%%%%%%%%%%%%%%%%%%%%%%%%%%%%%%%%%%%%%%%%%%%%%%%%%%%%%%%%%%%%%%%%%%%%%%%%%%%%%%%%%%%%%%%%%%%%%%%%%%%%%%
%TCIDATA{OutputFilter=LATEX.DLL}
%TCIDATA{Version=5.00.0.2552}
%TCIDATA{<META NAME="SaveForMode" CONTENT="1">}
%TCIDATA{Created=Saturday, November 14, 2015 20:50:38}
%TCIDATA{LastRevised=Sunday, November 15, 2015 22:24:45}
%TCIDATA{<META NAME="GraphicsSave" CONTENT="32">}
%TCIDATA{<META NAME="DocumentShell" CONTENT="Standard LaTeX\Blank - Standard LaTeX Article">}
%TCIDATA{CSTFile=40 LaTeX article.cst}

\newtheorem{theorem}{Theorem}
\newtheorem{acknowledgement}[theorem]{Acknowledgement}
\newtheorem{algorithm}[theorem]{Algorithm}
\newtheorem{axiom}[theorem]{Axiom}
\newtheorem{case}[theorem]{Case}
\newtheorem{claim}[theorem]{Claim}
\newtheorem{conclusion}[theorem]{Conclusion}
\newtheorem{condition}[theorem]{Condition}
\newtheorem{conjecture}[theorem]{Conjecture}
\newtheorem{corollary}[theorem]{Corollary}
\newtheorem{criterion}[theorem]{Criterion}
\newtheorem{definition}[theorem]{Definition}
\newtheorem{example}[theorem]{Example}
\newtheorem{exercise}[theorem]{Exercise}
\newtheorem{lemma}[theorem]{Lemma}
\newtheorem{notation}[theorem]{Notation}
\newtheorem{problem}[theorem]{Problem}
\newtheorem{proposition}[theorem]{Proposition}
\newtheorem{remark}[theorem]{Remark}
\newtheorem{solution}[theorem]{Solution}
\newtheorem{summary}[theorem]{Summary}
\newenvironment{proof}[1][Proof]{\noindent\textbf{#1.} }{\ \rule{0.5em}{0.5em}}


\begin{document}


\bigskip 赵丰\qquad 2013012178\qquad 泛函第9%
周作业

2.13 Let E and F be two Banach spaces. Prove that the set

$\Omega $ = \{T ∈ L(E, F); T admits a left inverse\}

is open in L(E, F).

\bigskip We first prove that $O=$ \{T ∈ L(E, F); T is bijective\} is
open. $\forall T\in O,\left\Vert x\right\Vert =1,$

$\left\Vert T\right\Vert =\underset{\left\Vert x\right\Vert =1}{inf}%
\left\Vert T\left( x\right) \right\Vert _{F}=\delta >0,$take $\epsilon =\min
\left\{ \frac{\delta }{2},\frac{1}{2}\left\Vert T^{-1}\right\Vert
^{-1}\right\} ,$and we can show that $U\left( T,\delta \right) \subset O.$

Indeed, $U\left( T,\delta \right) =$ \{T+g \TEXTsymbol{\vert}g ∈ L(E,
F) and $\left\Vert g\right\Vert <\epsilon $\},

$\left\Vert \left\langle T+g,x\right\rangle \right\Vert _{F}\geq \left\Vert
Tx\right\Vert -\left\Vert gx\right\Vert \geq \delta -\frac{\delta }{2}%
>0,\forall \left\Vert x\right\Vert =1$

$\implies \left( T+g\right) ^{-1}\left( 0\right) =\left\{ 0\right\} $ and by
linearily 

$T+g$ is injective. $\forall y\in F,$by the surjectivity of $T,$ there
exists $x\in E,s.t.$

$T\left( x_{0}\right) =y\implies \left\Vert \left( T+g\right) \left(
x_{0}\right) -y\right\Vert =\left\Vert g\left( x_{0}\right) \right\Vert
<\epsilon ,$

for $g\left( x_{0}\right) ,$ we can find $x_{1}\in E,s.t.T\left(
x_{1}\right) =-g\left( x_{0}\right) ,$

$\left\Vert g\left( x_{1}\right) \right\Vert \leq \left\Vert g\right\Vert
\left\Vert x_{1}\right\Vert \leq \left\Vert g\right\Vert \left\Vert
T^{-1}\right\Vert \left\Vert g\left( x_{0}\right) \right\Vert \leq \frac{1}{2%
}\epsilon ,$

$\left\Vert \left( T+g\right) \left( x_{0}+x_{1}\right) -y\right\Vert
=\left\Vert g\left( x_{1}\right) \right\Vert ,$

Similarly, we can use induction to proceed, and there exists $x_{n}\in E,$

$s.t.T\left( x_{n}\right) =-g\left( x_{n-1}\right) ,\left\Vert g\left(
x_{n}\right) \right\Vert \leq \left\Vert g\right\Vert \left\Vert
T^{-1}\right\Vert \left\Vert g\left( x_{n-1}\right) \right\Vert \leq \frac{1%
}{2^{n}}\epsilon .$

\bigskip $\left\Vert \left( T+g\right) \left( \underset{i=0}{\overset{n}{%
\sum }}x_{i}\right) -y\right\Vert =\left\Vert g\left( x_{n}\right)
\right\Vert $

And we also have $\left\Vert x_{n}\right\Vert \leq \left\Vert
T^{-1}\right\Vert \left\Vert g\left( x_{n-1}\right) \right\Vert \leq \frac{1%
}{2}\left\Vert x_{n-1}\right\Vert \implies \underset{n=0}{\overset{\infty }{%
\sum }}x_{n}$ converges 

in E since E is a Banach space, by the continuity of $\left( T+g\right) $

$\left\Vert \left( T+g\right) \left( \underset{n=0}{\overset{\infty }{\sum }}%
x_{n}\right) -y\right\Vert =\left\Vert \underset{n=0}{\overset{\infty }{\sum 
}}\left( T+g\right) \left( x_{n}\right) -y\right\Vert $

=$\underset{n\rightarrow \infty }{\lim }\left\Vert \underset{i=0}{\overset{n}%
{\sum }}\left( T+g\right) \left( x_{i}\right) -y\right\Vert $=$\underset{%
n\rightarrow \infty }{\lim }\left\Vert g\left( x_{n}\right) \right\Vert =0.$

This equality holds since $x_{n}\rightarrow 0\implies g\left( x_{n}\right)
\rightarrow 0.\implies y$ has preimage $\ \qquad \left( T+g\right) \left( 
\underset{n=0}{\overset{\infty }{\sum }}x_{n}\right) .$ 

$\implies O$ is open in L(E,F). 

$\forall T\in \Omega ,$since T has left inverse, we can easily show that T
is injective, then 

by Thm 2.13 $F$ allows the decomposition $F=R\left( T\right) \oplus L,$and
we can define a projection $P$ from F to $R\left( T\right) .$Then $P\circ T$
is bijective from E to R$\left( T\right) .$And since

$O^{\prime }=$ \{f ∈ L(E, R$\left( T\right) $); f is bijective\} is
open, $\ \ \exists \epsilon >0,s.t.\forall g:E\rightarrow R\left( T\right)
,\left\Vert g\right\Vert <\epsilon ,P\circ T+g$ is bijective.

Then $\forall g:E\rightarrow F,\left\Vert g\right\Vert <\frac{\epsilon }{%
\left\Vert P\right\Vert },$ $P\circ g$ satisfies $\left\Vert P\circ
g\right\Vert <\left\Vert P\right\Vert \left\Vert g\right\Vert <\epsilon ,$

and $P\circ g\in O^{\prime }\implies P\circ T+P\circ g$ is bijective

$\implies \exists S_{g},s.t.S_{g}\circ \left( P\circ T+P\circ g\right)
=I_{E}\implies S_{g}\circ P\circ \left( T+g\right) =I_{E}\implies $

$T+g$ has left inverse $\forall g:E\rightarrow F,\left\Vert g\right\Vert <%
\frac{\epsilon }{\left\Vert P\right\Vert }\implies \Omega $ is open.

2.19 $\left( 1\right) $A* E-\TEXTsymbol{>}E*, defined by $\left\langle
A^{\ast }v,u\right\rangle =\left\langle Au,v\right\rangle $ $\forall u\in
D\left( A\right) ,$ 

and we can extended $A^{\ast }v$ to E since D$\left( A\right) $ is dense in
E. 

$\forall u\in N\left( A\right) ,Au=0$ in E. then for any $v\in A,$by $\left(
1\right) $ we have

$\left\langle A\left( u+v\right) ,u+v\right\rangle \geq -C\left\Vert A\left(
u+v\right) \right\Vert ^{2}\implies $

$\left\langle Av,v\right\rangle +C\left\Vert Av\right\Vert ^{2}+\left\langle
Av,u\right\rangle \geq 0.$ we can replace $v$ by $av$,where $a$ is arbitrary
real number$\implies \left[ \left( Av,v\right) +C\left\Vert Av\right\Vert
^{2}\right] a^{2}+\left\langle Av,u\right\rangle a\geq 0,\forall a\in R.$

Since $\left( Av,v\right) +C\left\Vert Av\right\Vert ^{2}\geq 0,\implies
\left\langle Av,u\right\rangle =0\forall v\in E.\implies \left\langle
A^{\ast }u,v\right\rangle =0\forall v\in E$

$\implies A^{\ast }u=0$ in E.

$\left( 2\right) $Since G$\left( A\right) $ is closed, it is Banach space in
E$\times E^{\ast }$ with the norm $\left\Vert \cdot \right\Vert
_{E}+\left\Vert \cdot \right\Vert _{E^{\ast }}.$

If we equip D$\left( A\right) $ with the norm $\left\Vert \cdot \right\Vert
_{E}+\left\Vert \cdot \right\Vert _{E^{\ast }},$then D$\left( A\right) $ is
a Banach space.

By the condition, R$\left( A\right) $ is closed, therefore a Banach space
with the norm $\left\Vert \cdot \right\Vert _{E^{\ast }}.$

A is surjective from D$\left( A\right) $ to R$\left( A\right) ,$ then we can
apply open mapping Thm 

A$\left( B_{D\left( A\right) }\left( 0,1\right) \right) \supset B_{E^{\ast
}}\left( 0,c\right) \implies $A$\left( B_{D\left( A\right) }\left(
0,c^{\prime }\right) \right) \supset \overline{B_{E^{\ast }}\left(
0,1\right) },$then $\forall u\in E,Au\in E^{\ast },$

there exists $v\in E,\left\Vert v\right\Vert \leq c^{\prime },s.t.\frac{Au}{%
\left\Vert Au\right\Vert }=Av.$Let $v^{\prime }=\left\Vert Au\right\Vert
v\implies Au=Av^{\prime }$

$u-v^{\prime }\in N\left( A\right) ,$since $N\left( A\right) \subset N\left(
A^{\prime }\right) ,\implies u-v^{\prime }\in N\left( A^{\prime }\right) ,$%
which gives 

$\left\langle Av^{\prime },u-v^{\prime }\right\rangle =0\implies
\left\langle Av^{\prime },u\right\rangle =\left\langle Av^{\prime
},v^{\prime }\right\rangle \geq -\left\Vert Av^{\prime }\right\Vert
\left\Vert Au\right\Vert c^{\prime }$

Substituing $Av^{\prime }$ by $Au\implies \left\langle Au,u\right\rangle
\geq -c^{\prime }\left\Vert Au\right\Vert ^{2}.$

2.20 1.Suppose $u_{n}\in D\left( B\right) =D\left( A\right) $ and $%
u_{n}\rightarrow u$ in E, 

$Bu_{n}=\left( A+T\right) u_{n}\rightarrow v$ in $F.$ By the continuity of
T, $Tu_{n}\rightarrow Tu$ in F

$\implies Au_{n}\rightarrow v-Tu.$Since $A$ is closed, $u\in D\left(
A\right) $ and $v-Tu=Au$

$\implies v=\left( T+A\right) u=Bu,$and B is closed.

2. $D\left( A^{\ast }\right) =\left\{ v\in F^{\ast }|\exists c,\left\langle
v,Au\right\rangle \leq c\left\Vert u\right\Vert ,\forall u\in E\right\} $

$D\left( B^{\ast }\right) =\left\{ v\in F^{\ast }|\exists c,\left\langle
v,Au\right\rangle +\left\langle v,Tu\right\rangle \leq c\left\Vert
u\right\Vert ,\forall u\in E\right\} $

if $v\in D\left( A^{\ast }\right) ,\exists c,\left\langle v,Au\right\rangle
\leq c\left\Vert u\right\Vert ,$since $\left\langle v,Tu\right\rangle \leq
\left\Vert v\right\Vert \left\Vert T\right\Vert \left\Vert u\right\Vert ,$

$\left\langle v,Au\right\rangle +\left\langle v,Tu\right\rangle \leq \left(
c+\left\Vert v\right\Vert \left\Vert T\right\Vert \right) \left\Vert
u\right\Vert \implies v\in D\left( B^{\ast }\right) .$

On the other hand,

\bigskip if $v\in D\left( B^{\ast }\right) ,\exists c,\left\langle
v,Au\right\rangle +\left\langle v,Tu\right\rangle \leq c\left\Vert
u\right\Vert ,$since $\left\langle v,Tu\right\rangle \geq -\left\Vert
v\right\Vert \left\Vert T\right\Vert \left\Vert u\right\Vert ,$

$\left\langle v,Au\right\rangle \leq \left( c+\left\Vert v\right\Vert
\left\Vert T\right\Vert \right) \left\Vert u\right\Vert \implies v\in
D\left( A^{\ast }\right) .$

$\implies D\left( A^{\ast }\right) =D\left( B^{\ast }\right) .$

$\forall v\in F^{\ast },u\in D\left( A\right) ,\left\langle B^{\ast
}v,u\right\rangle =\left\langle v,Bu\right\rangle =\left\langle
v,Tu\right\rangle +\left\langle v,Au\right\rangle $

$=\left\langle T^{\ast }v,u\right\rangle +\left\langle A^{\ast
}v,u\right\rangle =\left\langle \left( A^{\ast }+T^{\ast }\right)
v,u\right\rangle \implies B^{\ast }=A^{\ast }+T^{\ast }.$

2.24 1.$D\left( B^{\ast }\right) =\left\{ v\in G^{\ast }|\exists
c,s.t.\left\langle v,Bu\right\rangle \leq c\left\Vert u\right\Vert ,\forall
u\in D\left( A\right) \right\} $

$\bigskip \left\langle v,Bu\right\rangle \leq c\left\Vert u\right\Vert \iff
\left\langle T^{\ast }v,Au\right\rangle \leq c\left\Vert u\right\Vert
\implies T^{\ast }\left( D\left( B^{\ast }\right) \right) \subset D\left(
A^{\ast }\right) $

$\forall v\in G^{\ast },u\in D\left( A\right) ,\left\langle B^{\ast
}v,u\right\rangle =\left\langle v,Bu\right\rangle =\left\langle v,T\circ
Au\right\rangle $

$=\left\langle T^{\ast }v,Tu\right\rangle =\left\langle A^{\ast }\circ
T^{\ast }v,u\right\rangle \implies B^{\ast }=A^{\ast }\circ T^{\ast }.$

\bigskip 2.Choose D$\left( A\right) \neq E$,T=0.Suppose G$\left( A\right) $
is closed, then $u_{n}\in D\left( A\right) $ and $u_{n}\rightarrow u$

and $Bu_{n}=0\rightarrow 0.$However, $u\notin D\left( A\right) $ if $D\left(
A\right) $ is not closed. 

\bigskip 2.25 1. Both $\left( S\circ T\right) ^{\ast }$ and $T^{\ast }\circ
S^{\ast }$ are defined on G$^{\ast }.$Further, 

$\forall v\in G^{\ast },u\in E,\left\langle \left( S\circ T\right) ^{\ast
},u\right\rangle =\left\langle v,\left( S\circ T\right) u\right\rangle
=\left\langle S^{\ast }v,Tu\right\rangle $

$=\left\langle T^{\ast }\circ S^{\ast }v,u\right\rangle \implies \left(
S\circ T\right) ^{\ast }=T^{\ast }\circ S^{\ast }.$

2 If $T^{\ast }v=0,$for some $v\in F^{\ast }\implies \left\langle
v,Tu\right\rangle =0\forall u\in E,$since T is surjective

$\implies \left\langle v,u^{\prime }\right\rangle =0\forall u^{\prime }\in
F\implies v=0,$by linearity of T$\implies T^{\ast }$ is injective.

$\forall v^{\prime }\in E^{\ast },v^{\prime }\circ T^{-1}\in F^{\ast }$ and $%
\left\langle T^{\ast }\left( v^{\prime }\circ T^{-1}\right) ,u\right\rangle
=\left\langle \left( v^{\prime }\circ T^{-1}\right) ,Tu\right\rangle
=\left\langle v^{\prime },u\right\rangle $

$\implies T^{\ast }\left( v^{\prime }\circ T^{-1}\right) =v^{\prime }.$

By 1. $T^{\ast }\circ \left( T^{-1}\right) ^{\ast }=\left( T\circ
T^{-1}\right) ^{\ast }=I_{F^{\ast }},$

$\left( T^{-1}\right) ^{\ast }\circ T^{\ast }=\left( T^{-1}\circ T\right)
^{\ast }=I_{E^{\ast }}.$

$\implies \left( T^{\ast }\right) ^{-1}=\left( T^{-1}\right) ^{\ast }.$

\end{document}
