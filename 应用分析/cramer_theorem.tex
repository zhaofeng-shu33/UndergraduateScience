\documentclass{article}
\usepackage{amssymb}
\usepackage{amsmath}

\title{A simple example for Cramér's Theorem}
\author{Feng Zhao}
\date

\begin{document}

\maketitle

\section{Introduction}
Cramér's Theorem is a general statement in large
derivation theory, both applicable to discrete and
continuous random variables. We use the notation
as Section 2.2 in \cite{dd}.
We use the example as Exercise 2.2.23 (d) in
\cite{dd},
which says that $\Lambda^*(x) = \frac{x^2}{2\sigma^2}$
if $X \sim \mathcal{N}(0, \sigma^2)$.
This can be got by consider the definition of
$\Lambda(x)$ in Definition 2.2.2
$$
\Lambda^*(x) \triangleq \sup_{\lambda \in R} \lambda x - \Lambda(\lambda)
$$
where $\Lambda(\lambda)$ is the logarithmic moment generating function of random variable $X$.
For Gaussian distribution $\mathcal{N}(0, \sigma^2)$ we have
$$
\Lambda(\lambda) = \frac{\lambda^2}{2}\sigma^2
$$
Therefore, we can get $\Lambda^*(x) = \frac{x^2}{2\sigma^2}$.

Now by Cramér's Theorem (Theorem 2.2.3 in \cite{dd}),
we have
\begin{equation}\label{eq:np}
\lim_{n\to \infty} \frac{1}{n}\log P_n = -\frac{x^2}{2\sigma^2}
\end{equation}
where
$$
P_n = P(\sum_{i=1}^n X_i > nx)
$$
and $X_i$ i.i.d. $\sim \mathcal{N}(0, \sigma^2)$.
Or equivalently, $P_n = P(\bar{X}>x)$ where
$\bar{X} = \frac{1}{n} \sum_{i=1}^n X_i$.

As an illustration, we use integral techniques to
show that the conclusion \eqref{eq:np} holds for
$\sigma = 1$. For general $\sigma$, a simple
scaling is enough.
\begin{align*}
P_n &= \int_{\sqrt{n}x}^{+\infty} \frac{1}{\sqrt{2\pi}}
\exp(-\frac{t^2}{2})dt \\
&= \frac{1}{\sqrt{2\pi n}x}\exp(-\frac{nx^2}{2})
- \int_{\sqrt{n}x}^{+\infty} \frac{1}{\sqrt{2\pi}t^2}
\exp(-\frac{t^2}{2})dt
\end{align*}
The second term has at least $\frac{1}{n^2}$ times
smaller than $P_n$ and we neglect it as high order term. Therefore, we have
$$
P_n \sim \frac{1}{\sqrt{2\pi n}x}\exp(-\frac{nx^2}{2})
$$
and \eqref{eq:np} holds.

This technique (integral by parts) in asymptotic analysis is called Laplace's method \cite{aa}.
\begin{thebibliography}{9}
\bibitem{dd} Zeitouni, A. Dembo O., and O. Dembo. "Large Deviations Techniques and Applications." (1998).
\bibitem{aa} J.D. Murray Asymptotic Analysis, Springer-Verlag, Applied Mathematical Sciences,
1974
\end{thebibliography}
\end{document}
