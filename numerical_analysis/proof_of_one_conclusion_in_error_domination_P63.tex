
\documentclass{article}
%%%%%%%%%%%%%%%%%%%%%%%%%%%%%%%%%%%%%%%%%%%%%%%%%%%%%%%%%%%%%%%%%%%%%%%%%%%%%%%%%%%%%%%%%%%%%%%%%%%%%%%%%%%%%%%%%%%%%%%%%%%%%%%%%%%%%%%%%%%%%%%%%%%%%%%%%%%%%%%%%%%%%%%%%%%%%%%%%%%%%%%%%%%%%%%%%%%%%%%%%%%%%%%%%%%%%%%%%%%%%%%%%%%%%%%%%%%%%%%%%%%%%%%%%%%%
%TCIDATA{OutputFilter=LATEX.DLL}
%TCIDATA{Version=5.00.0.2552}
%TCIDATA{<META NAME="SaveForMode" CONTENT="1">}
%TCIDATA{Created=Thursday, October 01, 2015 13:14:29}
%TCIDATA{LastRevised=Thursday, October 01, 2015 13:23:51}
%TCIDATA{<META NAME="GraphicsSave" CONTENT="32">}
%TCIDATA{<META NAME="DocumentShell" CONTENT="Standard LaTeX\Blank - Standard LaTeX Article">}
%TCIDATA{CSTFile=40 LaTeX article.cst}

\newtheorem{theorem}{Theorem}
\newtheorem{acknowledgement}[theorem]{Acknowledgement}
\newtheorem{algorithm}[theorem]{Algorithm}
\newtheorem{axiom}[theorem]{Axiom}
\newtheorem{case}[theorem]{Case}
\newtheorem{claim}[theorem]{Claim}
\newtheorem{conclusion}[theorem]{Conclusion}
\newtheorem{condition}[theorem]{Condition}
\newtheorem{conjecture}[theorem]{Conjecture}
\newtheorem{corollary}[theorem]{Corollary}
\newtheorem{criterion}[theorem]{Criterion}
\newtheorem{definition}[theorem]{Definition}
\newtheorem{example}[theorem]{Example}
\newtheorem{exercise}[theorem]{Exercise}
\newtheorem{lemma}[theorem]{Lemma}
\newtheorem{notation}[theorem]{Notation}
\newtheorem{problem}[theorem]{Problem}
\newtheorem{proposition}[theorem]{Proposition}
\newtheorem{remark}[theorem]{Remark}
\newtheorem{solution}[theorem]{Solution}
\newtheorem{summary}[theorem]{Summary}
\newenvironment{proof}[1][Proof]{\noindent\textbf{#1.} }{\ \rule{0.5em}{0.5em}}
\def\TEXTsymbol#1{\mbox{$#1$}}%
\def\NEG#1{\leavevmode\hbox{\rlap{\thinspace/}{$#1$}}}%
\def\QATOPD#1#2#3#4{{#3 \atopwithdelims#1#2 #4}}%
\def\QTP#1{}
\def\func#1{\mathop{\rm #1}}%

\begin{document}


Proof of Thm 4.3 lefthand in textbook P63

$Ax=b,A$ is a $n\times n$ real matrix, $b$ is a n-dimensional column vector,
x is the solution of this linear

equation. Suppose $\widetilde{x}$ is the approximate solution to the
equation, corresponding to the remaider 

$r=b-A\widetilde{x}.$Then 

$\frac{1}{cond(A)}\frac{\left\Vert r\right\Vert }{\left\Vert b\right\Vert }%
\leq \frac{\left\Vert \widetilde{x}-x\right\Vert }{\left\Vert x\right\Vert }.
$

Proof: $r=A\left( x-\widetilde{x}\right) .$

$\frac{1}{cond(A)}\frac{\left\Vert r\right\Vert }{\left\Vert b\right\Vert }=%
\frac{1}{\left\Vert A\right\Vert \left\Vert A^{-1}\right\Vert }\frac{%
\left\Vert A\left( x-\widetilde{x}\right) \right\Vert }{\left\Vert
b\right\Vert }\leq \frac{1}{\left\Vert A\right\Vert \left\Vert
A^{-1}\right\Vert }\frac{\left\Vert A\right\Vert \left\Vert \left( x-%
\widetilde{x}\right) \right\Vert }{\left\Vert b\right\Vert }=\frac{%
\left\Vert \left( x-\widetilde{x}\right) \right\Vert }{\left\Vert
A^{-1}\right\Vert \left\Vert b\right\Vert }\leq \frac{\left\Vert \left( x-%
\widetilde{x}\right) \right\Vert }{\left\Vert A^{-1}b\right\Vert }=\frac{%
\left\Vert \widetilde{x}-x\right\Vert }{\left\Vert x\right\Vert }$

\end{document}
