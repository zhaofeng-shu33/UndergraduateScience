
\documentclass{article}
%%%%%%%%%%%%%%%%%%%%%%%%%%%%%%%%%%%%%%%%%%%%%%%%%%%%%%%%%%%%%%%%%%%%%%%%%%%%%%%%%%%%%%%%%%%%%%%%%%%%%%%%%%%%%%%%%%%%%%%%%%%%%%%%%%%%%%%%%%%%%%%%%%%%%%%%%%%%%%%%%%%%%%%%%%%%%%%%%%%%%%%%%%%%%%%%%%%%%%%%%%%%%%%%%%%%%%%%%%%%%%%%%%%%%%%%%%%%%%%%%%%%%%%%%%%%
\usepackage{amssymb}
\usepackage{amsmath}

\setcounter{MaxMatrixCols}{10}
%TCIDATA{OutputFilter=LATEX.DLL}
%TCIDATA{Version=5.50.0.2953}
%TCIDATA{<META NAME="SaveForMode" CONTENT="1">}
%TCIDATA{BibliographyScheme=Manual}
%TCIDATA{Created=Monday, October 19, 2015 08:45:51}
%TCIDATA{LastRevised=Monday, January 04, 2016 00:41:46}
%TCIDATA{<META NAME="GraphicsSave" CONTENT="32">}
%TCIDATA{<META NAME="DocumentShell" CONTENT="Standard LaTeX\Blank - Standard LaTeX Article">}
%TCIDATA{CSTFile=40 LaTeX article.cst}

\newtheorem{theorem}{Theorem}
\newtheorem{acknowledgement}[theorem]{Acknowledgement}
\newtheorem{algorithm}[theorem]{Algorithm}
\newtheorem{axiom}[theorem]{Axiom}
\newtheorem{case}[theorem]{Case}
\newtheorem{claim}[theorem]{Claim}
\newtheorem{conclusion}[theorem]{Conclusion}
\newtheorem{condition}[theorem]{Condition}
\newtheorem{conjecture}[theorem]{Conjecture}
\newtheorem{corollary}[theorem]{Corollary}
\newtheorem{criterion}[theorem]{Criterion}
\newtheorem{definition}[theorem]{Definition}
\newtheorem{example}[theorem]{Example}
\newtheorem{exercise}[theorem]{Exercise}
\newtheorem{lemma}[theorem]{Lemma}
\newtheorem{notation}[theorem]{Notation}
\newtheorem{problem}[theorem]{Problem}
\newtheorem{proposition}[theorem]{Proposition}
\newtheorem{remark}[theorem]{Remark}
\newtheorem{solution}[theorem]{Solution}
\newtheorem{summary}[theorem]{Summary}
\newenvironment{proof}[1][Proof]{\noindent\textbf{#1.} }{\ \rule{0.5em}{0.5em}}
\def\TEXTsymbol#1{\mbox{$#1$}}%
\def\NEG#1{\leavevmode\hbox{\rlap{\thinspace/}{$#1$}}}%
\def\QATOPD#1#2#3#4{{#3 \atopwithdelims#1#2 #4}}%
\def\QTP#1{}
\def\func#1{\mathop{\rm #1}}%
\begin{document}


Numberical Analysis 第五周纸制作业%
\qquad \bigskip 赵丰2013012178

\bigskip

%FRAME

$A$为对称矩阵且对角元%
大于零,由书上定理2.2知J%
法收敛的充要条件为$A$%
和2D$-A$均正定,

A正定$\iff $A各阶主子式\TEXTsymbol{%
>}0$\iff 100-ac>0,\det A=500-15ac>0\iff ac<\allowbreak \frac{100}{3}$

2D$-A$正定$\iff 100-ac>0,\det \left( 2D-A\right)
=1500-25ac>0\iff ac<60.$

故J法收敛$\iff ac<\allowbreak \frac{100}{3}.$

$\bigskip $

According to the instruction of teacher assistant, $A$ is not a symmetric
matrix. As a result, we should proceed from $\rho \left( B\right) <1,$where $%
B$ is the recursive matrix.

\bigskip 

$A$为对称矩阵且对角元%
大于零,由书上定理2.2知%
GS法收敛的充要条件为$A$%
正定,

由上面结果:GS法收敛$\iff
ac<\allowbreak \frac{100}{3}$

2.$\left( a\right) $将$x^{\left( k+\frac{1}{2}\right) }$代%
入$x^{\left( k+1\right) }$得\qquad $\left( D-U\right) x^{\left(
k+1\right) }=L\left[ \left( D-L\right) ^{-1}Ux^{\left( k\right) }+\left(
D-L\right) ^{-1}b\right] +b\implies $

$x^{\left( k+1\right) }=\left( D-U\right) ^{-1}L\left( D-L\right)
^{-1}Ux^{\left( k\right) }+\left( D-U\right) ^{-1}\left( L\left( D-L\right)
^{-1}+I\right) b\implies $

$B=\left( D-U\right) ^{-1}L\left( D-L\right) ^{-1}U;g=\left( D-U\right)
^{-1}\left( L\left( D-L\right) ^{-1}+I\right) b$

因为$A=D-L-U=D(I-\widetilde{L}-\widetilde{U})$,所以$%
L=D\tilde{L},U=D\tilde{U}.$

将其分别代入$B$和$g$的%
表达式中得$\qquad B=\left( D-U\right)
^{-1}L\left( D-L\right) ^{-1}U$

=$\left( D-D\tilde{U}\right) ^{-1}D\tilde{L}\left( D-D\tilde{L}\right) ^{-1}D%
\tilde{U}=\left( I-\tilde{U}\right) ^{-1}\tilde{L}\left( I-\tilde{L}\right)
^{-1}\tilde{U},$考虑到$\tilde{L}$和$\left( I-\tilde{L%
}\right) ^{-1}$乘法可交换

$B=\left( I-\tilde{U}\right) ^{-1}\left( I-\tilde{L}\right) ^{-1}\tilde{L}%
\tilde{U}.$

$g=\left( D-D\tilde{U}\right) ^{-1}\left( D\tilde{L}\left( D-D\tilde{L}%
\right) ^{-1}+I\right) b=\left( I-\tilde{U}\right) ^{-1}D^{-1}\left( D\tilde{%
L}\left( I-\tilde{L}\right) ^{-1}D^{-1}+I\right) b$

=$\left( I-\tilde{U}\right) ^{-1}\left( \tilde{L}\left( I-\tilde{L}\right)
^{-1}+I\right) D^{-1}b=\left( I-\tilde{U}\right) ^{-1}\left( I-\tilde{L}%
\right) ^{-1}D^{-1}b.$

$\left( b\right) B=\left( \left( I-\tilde{L}\right) \left( I-\tilde{U}%
\right) \right) ^{-1}\tilde{L}\tilde{U},$

若设$\lambda $为B的一个特征%
值,对应的特征向量为$%
\vec{x}$,

则由$B\vec{x}=\lambda \vec{x}$,可推出$\left(
1-\lambda \right) \tilde{L}\tilde{U}\vec{x}=\left( I-\tilde{L}-\tilde{U}%
\right) \lambda \vec{x}$,利用$L=D\tilde{L},U=D\tilde{U}$和$%
U=L^{T}$可得到$\tilde{L}\tilde{U}=D^{-1}\tilde{U}^{T}D%
\tilde{U},$

代入上式有$\qquad \left( 1-\lambda \right)
\left( \tilde{U}^{T}D\tilde{U}\right) \vec{x}=\lambda A\vec{x},$注%
意到$D$中各对角元均正$%
, $因此$\tilde{U}^{T}D\tilde{U}=\left( D^{1/2}\tilde{U}\right)
^{T}D^{1/2}\tilde{U}$半正定

若$\left( \tilde{U}^{T}D\tilde{U}\right) \vec{x}=0$,则$\lambda A%
\vec{x}=0$,因A可逆,故必有$\lambda =0,$%
此种情况下特征值为%
实数

(取A为三阶Hilbert阵会出现B=%
%FRAME,此时B显然%
有一个特征值为0)

Remark: this case never happens since $A$ is positive definite$\implies $so
is $D\implies $so is $\tilde{U}^{T}D\tilde{U}.$

3若$\left( \tilde{U}^{T}D\tilde{U}\right) \vec{x}\neq 0,$则%
必有$\vec{x}^{\ast }\left( \tilde{U}^{T}D\tilde{U}\right) \vec{x}%
\neq 0,$indeed if $\vec{x}^{\ast }\left( \tilde{U}^{T}D\tilde{U}\right) \vec{%
x}=0,$then $\left( D^{1/2}\tilde{U}\vec{x}\right) ^{\ast }D^{1/2}\tilde{U}%
\vec{x}=0$

(* represents transpose conjugate)$\implies D^{1/2}\tilde{U}\vec{x}%
=0\implies \left( \tilde{U}^{T}D\tilde{U}\right) \vec{x}=0.$

因此可以在$\left( 1-\lambda \right) \left( 
\tilde{U}^{T}D\tilde{U}\right) \vec{x}=\lambda A\vec{x}$两边%
的左边同时乘以$\vec{x}^{\ast }$,%
利用矩阵的正定性可%
知

$\frac{1-\lambda }{\lambda }=\frac{\vec{x}^{\ast }A\vec{x}}{\vec{x}^{\ast }%
\tilde{U}^{T}D\tilde{U}\vec{x}}>0,\implies $in either case, $\lambda \in \NEG%
{R}$.同时我们也证明了$0\leq
\lambda <1$

进而证出$\rho \left( B\right) <1.$

Remark: this problem is the special case of SSOR when $\omega =1$, from P89
of textbook,

We can further show that if $B_{w}\vec{x}=\lambda \vec{x},$ then $\lambda $
is real and $0\leq \lambda <1$ as $0<w<2.$

Indeed, $\left[ \left( 1-w\right) I+w\tilde{L}\right] \left[ \left(
1-w\right) I+w\tilde{U}\right] x=\lambda \left( I-w\tilde{L}\right) \left(
I-w\tilde{U}\right) x,$

$\implies \left( 1-w\right) ^{2}Dx+w\left( 1-w\right) \left[ L+U\right]
x+w^{2}\tilde{U}^{T}D\tilde{U}x=\lambda \left[ D-w\left( L+U\right) +w^{2}%
\tilde{U}^{T}D\tilde{U}\right] x$

$L+D=D-A\implies \left( 1-w\right) Dx-w\left( 1-w\right) Ax+w^{2}\tilde{U}%
^{T}D\tilde{U}x=\lambda \left[ \left( 1-w\right) D+wA+w^{2}\tilde{U}^{T}D%
\tilde{U}\right] x\qquad \left( \ast \right) $

We can show that $\left[ \left( 1-w\right) D+wA+w^{2}\tilde{U}^{T}D\tilde{U}%
\right] $ is positive definite. 

\bigskip $\left( 1-w\right) D+wA+w^{2}\tilde{U}^{T}D\tilde{U}=D-\left(
U+U^{T}\right) w+w^{2}U^{T}D^{-1}U=\left( -wU^{T}D^{-\frac{1}{2}}+D^{\frac{1%
}{2}}\right) \left( -wD^{-\frac{1}{2}}U+D^{\frac{1}{2}}\right) $

$=\left( -wD^{-\frac{1}{2}}U+D^{\frac{1}{2}}\right) ^{T}\left( -wD^{-\frac{1%
}{2}}U+D^{\frac{1}{2}}\right) .-wU^{T}D^{-\frac{1}{2}}+D^{\frac{1}{2}}$ is
reversible$\implies $the positive definite property holds.

Similarly, we can show that $\left( 1-w\right) ^{2}D+w\left( 1-w\right) %
\left[ L+U\right] +w^{2}\tilde{U}^{T}D\tilde{U}$

$=\left( wD^{-\frac{1}{2}}U+\left( 1-w\right) D^{\frac{1}{2}}\right)
^{T}\left( wD^{-\frac{1}{2}}U+\left( 1-w\right) D^{\frac{1}{2}}\right) ,$%
which is also semi-positive definite

$\left( e.g.w=1,\text{it is semi-positive}\right) $

Therefore, multiplying $\left( \ast \right) $ with $x^{\ast }$ on the left
gives

$\lambda =\frac{\left( 1-w\right) x^{\ast }Dx+w\left( w-1\right) x^{\ast
}Ax+w^{2}x^{\ast }\tilde{U}D\tilde{U}x}{\left( 1-w\right) x^{\ast
}Dx+wx^{\ast }Ax+w^{2}x^{\ast }\tilde{U}D\tilde{U}x},$ the numerator is
nonnegative and the denominator is positive.

Notice $w-1<1\implies 0\leq \lambda <1,\lambda =0$ is reachable, as the
above example for 3D-Hilbert matrix with $w=1$ shows.

\bigskip 

$,$let $M=D^{-\frac{1}{2}}U^{T}D^{-\frac{1}{2}}\implies $

This property is obvious for $0<w\leq 1,$for $w>1,\left( 1-w\right)
D+wA=\left( w-1\right) \left( A-D\right) +A$

$\left\langle \left( \left( 1-w\right) D+wA\right) \left( x\right)
,x\right\rangle =\overset{n}{\underset{i=1}{\sum }}a_{ii}x_{i}^{2}+\overset{n%
}{\underset{\underset{i\neq j}{i,j=1}}{\sum }}wa_{ij}x_{i}x_{j}$

Thi

%FRAME

先证明迭代矩阵$B=\left( D-\omega
L\right) ^{-1}\left[ \left( 1-\omega \right) D+\omega U\right] $
inversible.Indeed it is easy to calculate the determinant of $B$ is $\left(
1-\omega \right) ^{n}$ which is non-negative guaranteed by given condition.

Assume $\lambda $ is an eigenvalue of $B,$ then $0=\det \left( B-\lambda
I\right) =\det \left[ \left( D-\omega L\right) ^{-1}\left[ \left( 1-\omega
\right) D+\omega U-\lambda \left( D-\omega L\right) \right] \right] $

$\det \left( D-\omega L\right) ^{-1}\det [\left( 1-\omega -\lambda \right)
D+\omega U+\lambda \omega L]$

If $\lambda \geq 1,$then $\left\vert \frac{1-\omega -\lambda }{\omega }%
\right\vert =\frac{\lambda +\omega -1}{\omega }\geq 1,$and $\left\vert \frac{%
1-\omega -\lambda }{\lambda \omega }\right\vert =\frac{\lambda +\omega -1}{%
\lambda \omega }\geq 1$(the latter is equivalent to

$\left( \lambda -1\right) \left( 1-\omega \right) \geq 0)$

If $\lambda \leq -1,$then $\left\vert \frac{1-\omega -\lambda }{\omega }%
\right\vert =\frac{-\lambda -\omega +1}{\omega }=\frac{-\lambda +1}{\omega }%
-1\geq \frac{2}{\omega }-1\geq 1,$and $\left\vert \frac{1-\omega -\lambda }{%
\lambda \omega }\right\vert =\frac{1-\omega -\lambda }{-\lambda \omega }>1$

(the latter is equivalent to $\left( 1-\lambda \right) \left( 1-\omega
\right) \geq 0)$

Hence for $\left\vert \lambda \right\vert \geq 1,$the coefficient of $D$ is
not smaller than that of $U$ or $L\implies \left( 1-\omega -\lambda \right)
D+\omega U+\lambda \omega L$

is still strict diagonal dominant and as a result is inversible. A
contradiction with $0=\det \left( B-\lambda I\right) $

$\implies \left\vert \lambda \right\vert <1$ for every eigenvalue of $%
B.\implies \rho \left( B\right) <1,$and SOR method has convergent solution
to the linear equation system.

$%FRAME$

由课本P92推导$\left( 4.11\right) $式,%
用此题的符号表示即%
有$F\left( x^{\left( k+1\right) }\right) =F\left( x^{\left( k\right)
}\right) -\alpha _{k+1}\left( b,p^{\left( k+1\right) }\right) +\alpha
_{k+1}^{2}\left( Ap^{\left( k+1\right) },p^{\left( k+1\right) }\right) .$%
此式成立是选择$p^{\left(
k+1\right) }$与$p^{\left( 0\right) },p^{\left( 1\right) },...p^{\left(
k\right) }$ A共轭的结果,而在%
取定$p^{\left( k+1\right) }$后,CG法再取$%
\alpha _{k+1}$极小化$F\left( x^{\left( k+1\right) }\right)
,$而$F\left( x^{\left( k\right) }\right) $对应$\alpha
_{k+1}=0$时的值,它当然比关%
于$\alpha _{k+1}$的二次函数最%
小值$F\left( x^{\left( k+1\right) }\right) $大,即%
我们有$F\left( x^{\left( k+1\right) }\right) \leq F\left(
x^{\left( k\right) }\right) .$

因$\alpha _{k+1}=\frac{\left( b,p^{\left( k+1\right) }\right) }{\left(
Ap^{\left( k+1\right) },p^{\left( k+1\right) }\right) }=\frac{\left(
r^{\left( k\right) },p^{\left( k+1\right) }\right) }{\left( Ap^{\left(
k+1\right) },p^{\left( k+1\right) }\right) }$因$r^{\left( k\right)
}=b-Ax^{\left( k\right) }$,而$x^{\left( k\right) }$可表%
为$p^{\left( 0\right) },p^{\left( 1\right) },...p^{\left( k\right) }$%
的线性组合$,$由$p^{\left( k+1\right) }$%
与$p^{\left( 0\right) },p^{\left( 1\right) },...p^{\left( k\right) }$ A%
共轭可知$\left( b,p^{\left( k+1\right) }\right)
=\left( r^{\left( k\right) },p^{\left( k+1\right) }\right) \implies \alpha
_{k+1}=0$ if $r^{\left( k\right) }=0,$因而此时%
二次函数最小值在$\alpha
_{k+1}=0$

处取到,同时也得到$\qquad
\alpha _{k+1}\neq 0$ if $r^{\left( k\right) }\neq 0,$即此时%
$F\left( x^{\left( k+1\right) }\right) \leq F\left( x^{\left( k\right)
}\right) $取严格不等号$\boxtimes $

\end{document}
