
\documentclass{article}
%%%%%%%%%%%%%%%%%%%%%%%%%%%%%%%%%%%%%%%%%%%%%%%%%%%%%%%%%%%%%%%%%%%%%%%%%%%%%%%%%%%%%%%%%%%%%%%%%%%%%%%%%%%%%%%%%%%%%%%%%%%%%%%%%%%%%%%%%%%%%%%%%%%%%%%%%%%%%%%%%%%%%%%%%%%%%%%%%%%%%%%%%%%%%%%%%%%%%%%%%%%%%%%%%%%%%%%%%%%%%%%%%%%%%%%%%%%%%%%%%%%%%%%%%%%%
\usepackage{amssymb}
\usepackage{amsmath}

\setcounter{MaxMatrixCols}{10}
%TCIDATA{OutputFilter=LATEX.DLL}
%TCIDATA{Version=5.50.0.2953}
%TCIDATA{<META NAME="SaveForMode" CONTENT="1">}
%TCIDATA{BibliographyScheme=Manual}
%TCIDATA{Created=Monday, October 19, 2015 08:45:51}
%TCIDATA{LastRevised=Monday, January 04, 2016 00:41:46}
%TCIDATA{<META NAME="GraphicsSave" CONTENT="32">}
%TCIDATA{<META NAME="DocumentShell" CONTENT="Standard LaTeX\Blank - Standard LaTeX Article">}
%TCIDATA{CSTFile=40 LaTeX article.cst}

\newtheorem{theorem}{Theorem}
\newtheorem{acknowledgement}[theorem]{Acknowledgement}
\newtheorem{algorithm}[theorem]{Algorithm}
\newtheorem{axiom}[theorem]{Axiom}
\newtheorem{case}[theorem]{Case}
\newtheorem{claim}[theorem]{Claim}
\newtheorem{conclusion}[theorem]{Conclusion}
\newtheorem{condition}[theorem]{Condition}
\newtheorem{conjecture}[theorem]{Conjecture}
\newtheorem{corollary}[theorem]{Corollary}
\newtheorem{criterion}[theorem]{Criterion}
\newtheorem{definition}[theorem]{Definition}
\newtheorem{example}[theorem]{Example}
\newtheorem{exercise}[theorem]{Exercise}
\newtheorem{lemma}[theorem]{Lemma}
\newtheorem{notation}[theorem]{Notation}
\newtheorem{problem}[theorem]{Problem}
\newtheorem{proposition}[theorem]{Proposition}
\newtheorem{remark}[theorem]{Remark}
\newtheorem{solution}[theorem]{Solution}
\newtheorem{summary}[theorem]{Summary}
\newenvironment{proof}[1][Proof]{\noindent\textbf{#1.} }{\ \rule{0.5em}{0.5em}}
\def\TEXTsymbol#1{\mbox{$#1$}}%
\def\NEG#1{\leavevmode\hbox{\rlap{\thinspace/}{$#1$}}}%
\def\QATOPD#1#2#3#4{{#3 \atopwithdelims#1#2 #4}}%
\def\QTP#1{}
\def\func#1{\mathop{\rm #1}}%
\begin{document}


Numberical Analysis \U{7b2c}\U{4e94}\U{5468}\U{7eb8}\U{5236}\U{4f5c}\U{4e1a}%
\qquad \bigskip \U{8d75}\U{4e30}2013012178

\bigskip

\FRAME{ftbpF}{4.5in}{0.9in}{0pt}{}{}{Figure}{\special{language "Scientific
Word";type "GRAPHIC";maintain-aspect-ratio TRUE;display "USEDEF";valid_file
"T";width 4.5in;height 0.9in;depth 0pt;original-width
12.5995in;original-height 2.8893in;cropleft "0";croptop "1";cropright
"1";cropbottom "0";tempfilename 'NWG7R702.wmf';tempfile-properties "XPR";}}

$A$\U{4e3a}\U{5bf9}\U{79f0}\U{77e9}\U{9635}\U{4e14}\U{5bf9}\U{89d2}\U{5143}%
\U{5927}\U{4e8e}\U{96f6},\U{7531}\U{4e66}\U{4e0a}\U{5b9a}\U{7406}2.2\U{77e5}J%
\U{6cd5}\U{6536}\U{655b}\U{7684}\U{5145}\U{8981}\U{6761}\U{4ef6}\U{4e3a}$A$%
\U{548c}2D$-A$\U{5747}\U{6b63}\U{5b9a},

A\U{6b63}\U{5b9a}$\iff $A\U{5404}\U{9636}\U{4e3b}\U{5b50}\U{5f0f}\TEXTsymbol{%
>}0$\iff 100-ac>0,\det A=500-15ac>0\iff ac<\allowbreak \frac{100}{3}$

2D$-A$\U{6b63}\U{5b9a}$\iff 100-ac>0,\det \left( 2D-A\right)
=1500-25ac>0\iff ac<60.$

\U{6545}J\U{6cd5}\U{6536}\U{655b}$\iff ac<\allowbreak \frac{100}{3}.$

$\bigskip $

According to the instruction of teacher assistant, $A$ is not a symmetric
matrix. As a result, we should proceed from $\rho \left( B\right) <1,$where $%
B$ is the recursive matrix.

\bigskip 

$A$\U{4e3a}\U{5bf9}\U{79f0}\U{77e9}\U{9635}\U{4e14}\U{5bf9}\U{89d2}\U{5143}%
\U{5927}\U{4e8e}\U{96f6},\U{7531}\U{4e66}\U{4e0a}\U{5b9a}\U{7406}2.2\U{77e5}%
GS\U{6cd5}\U{6536}\U{655b}\U{7684}\U{5145}\U{8981}\U{6761}\U{4ef6}\U{4e3a}$A$%
\U{6b63}\U{5b9a},

\U{7531}\U{4e0a}\U{9762}\U{7ed3}\U{679c}:GS\U{6cd5}\U{6536}\U{655b}$\iff
ac<\allowbreak \frac{100}{3}$

2.$\left( a\right) $\U{5c06}$x^{\left( k+\frac{1}{2}\right) }$\U{4ee3}%
\U{5165}$x^{\left( k+1\right) }$\U{5f97}\qquad $\left( D-U\right) x^{\left(
k+1\right) }=L\left[ \left( D-L\right) ^{-1}Ux^{\left( k\right) }+\left(
D-L\right) ^{-1}b\right] +b\implies $

$x^{\left( k+1\right) }=\left( D-U\right) ^{-1}L\left( D-L\right)
^{-1}Ux^{\left( k\right) }+\left( D-U\right) ^{-1}\left( L\left( D-L\right)
^{-1}+I\right) b\implies $

$B=\left( D-U\right) ^{-1}L\left( D-L\right) ^{-1}U;g=\left( D-U\right)
^{-1}\left( L\left( D-L\right) ^{-1}+I\right) b$

\U{56e0}\U{4e3a}$A=D-L-U=D(I-\widetilde{L}-\widetilde{U})$,\U{6240}\U{4ee5}$%
L=D\tilde{L},U=D\tilde{U}.$

\U{5c06}\U{5176}\U{5206}\U{522b}\U{4ee3}\U{5165}$B$\U{548c}$g$\U{7684}%
\U{8868}\U{8fbe}\U{5f0f}\U{4e2d}\U{5f97}$\qquad B=\left( D-U\right)
^{-1}L\left( D-L\right) ^{-1}U$

=$\left( D-D\tilde{U}\right) ^{-1}D\tilde{L}\left( D-D\tilde{L}\right) ^{-1}D%
\tilde{U}=\left( I-\tilde{U}\right) ^{-1}\tilde{L}\left( I-\tilde{L}\right)
^{-1}\tilde{U},$\U{8003}\U{8651}\U{5230}$\tilde{L}$\U{548c}$\left( I-\tilde{L%
}\right) ^{-1}$\U{4e58}\U{6cd5}\U{53ef}\U{4ea4}\U{6362}

$B=\left( I-\tilde{U}\right) ^{-1}\left( I-\tilde{L}\right) ^{-1}\tilde{L}%
\tilde{U}.$

$g=\left( D-D\tilde{U}\right) ^{-1}\left( D\tilde{L}\left( D-D\tilde{L}%
\right) ^{-1}+I\right) b=\left( I-\tilde{U}\right) ^{-1}D^{-1}\left( D\tilde{%
L}\left( I-\tilde{L}\right) ^{-1}D^{-1}+I\right) b$

=$\left( I-\tilde{U}\right) ^{-1}\left( \tilde{L}\left( I-\tilde{L}\right)
^{-1}+I\right) D^{-1}b=\left( I-\tilde{U}\right) ^{-1}\left( I-\tilde{L}%
\right) ^{-1}D^{-1}b.$

$\left( b\right) B=\left( \left( I-\tilde{L}\right) \left( I-\tilde{U}%
\right) \right) ^{-1}\tilde{L}\tilde{U},$

\U{82e5}\U{8bbe}$\lambda $\U{4e3a}B\U{7684}\U{4e00}\U{4e2a}\U{7279}\U{5f81}%
\U{503c},\U{5bf9}\U{5e94}\U{7684}\U{7279}\U{5f81}\U{5411}\U{91cf}\U{4e3a}$%
\vec{x}$,

\U{5219}\U{7531}$B\vec{x}=\lambda \vec{x}$,\U{53ef}\U{63a8}\U{51fa}$\left(
1-\lambda \right) \tilde{L}\tilde{U}\vec{x}=\left( I-\tilde{L}-\tilde{U}%
\right) \lambda \vec{x}$,\U{5229}\U{7528}$L=D\tilde{L},U=D\tilde{U}$\U{548c}$%
U=L^{T}$\U{53ef}\U{5f97}\U{5230}$\tilde{L}\tilde{U}=D^{-1}\tilde{U}^{T}D%
\tilde{U},$

\U{4ee3}\U{5165}\U{4e0a}\U{5f0f}\U{6709}$\qquad \left( 1-\lambda \right)
\left( \tilde{U}^{T}D\tilde{U}\right) \vec{x}=\lambda A\vec{x},$\U{6ce8}%
\U{610f}\U{5230}$D$\U{4e2d}\U{5404}\U{5bf9}\U{89d2}\U{5143}\U{5747}\U{6b63}$%
, $\U{56e0}\U{6b64}$\tilde{U}^{T}D\tilde{U}=\left( D^{1/2}\tilde{U}\right)
^{T}D^{1/2}\tilde{U}$\U{534a}\U{6b63}\U{5b9a}

\U{82e5}$\left( \tilde{U}^{T}D\tilde{U}\right) \vec{x}=0$,\U{5219}$\lambda A%
\vec{x}=0$,\U{56e0}A\U{53ef}\U{9006},\U{6545}\U{5fc5}\U{6709}$\lambda =0,$%
\U{6b64}\U{79cd}\U{60c5}\U{51b5}\U{4e0b}\U{7279}\U{5f81}\U{503c}\U{4e3a}%
\U{5b9e}\U{6570}

(\U{53d6}A\U{4e3a}\U{4e09}\U{9636}Hilbert\U{9635}\U{4f1a}\U{51fa}\U{73b0}B=%
\FRAME{itbpF}{1.5904in}{0.9003in}{0in}{}{}{Figure}{\special{language
"Scientific Word";type "GRAPHIC";display "USEDEF";valid_file "T";width
1.5904in;height 0.9003in;depth 0in;original-width 1.8749in;original-height
1.1044in;cropleft "0";croptop "1";cropright "1";cropbottom "0";tempfilename
'NWGKRO03.wmf';tempfile-properties "XPR";}},\U{6b64}\U{65f6}B\U{663e}\U{7136}%
\U{6709}\U{4e00}\U{4e2a}\U{7279}\U{5f81}\U{503c}\U{4e3a}0)

Remark: this case never happens since $A$ is positive definite$\implies $so
is $D\implies $so is $\tilde{U}^{T}D\tilde{U}.$

3\U{82e5}$\left( \tilde{U}^{T}D\tilde{U}\right) \vec{x}\neq 0,$\U{5219}%
\U{5fc5}\U{6709}$\vec{x}^{\ast }\left( \tilde{U}^{T}D\tilde{U}\right) \vec{x}%
\neq 0,$indeed if $\vec{x}^{\ast }\left( \tilde{U}^{T}D\tilde{U}\right) \vec{%
x}=0,$then $\left( D^{1/2}\tilde{U}\vec{x}\right) ^{\ast }D^{1/2}\tilde{U}%
\vec{x}=0$

(* represents transpose conjugate)$\implies D^{1/2}\tilde{U}\vec{x}%
=0\implies \left( \tilde{U}^{T}D\tilde{U}\right) \vec{x}=0.$

\U{56e0}\U{6b64}\U{53ef}\U{4ee5}\U{5728}$\left( 1-\lambda \right) \left( 
\tilde{U}^{T}D\tilde{U}\right) \vec{x}=\lambda A\vec{x}$\U{4e24}\U{8fb9}%
\U{7684}\U{5de6}\U{8fb9}\U{540c}\U{65f6}\U{4e58}\U{4ee5}$\vec{x}^{\ast }$,%
\U{5229}\U{7528}\U{77e9}\U{9635}\U{7684}\U{6b63}\U{5b9a}\U{6027}\U{53ef}%
\U{77e5}

$\frac{1-\lambda }{\lambda }=\frac{\vec{x}^{\ast }A\vec{x}}{\vec{x}^{\ast }%
\tilde{U}^{T}D\tilde{U}\vec{x}}>0,\implies $in either case, $\lambda \in \NEG%
{R}$.\U{540c}\U{65f6}\U{6211}\U{4eec}\U{4e5f}\U{8bc1}\U{660e}\U{4e86}$0\leq
\lambda <1$

\U{8fdb}\U{800c}\U{8bc1}\U{51fa}$\rho \left( B\right) <1.$

Remark: this problem is the special case of SSOR when $\omega =1$, from P89
of textbook,

We can further show that if $B_{w}\vec{x}=\lambda \vec{x},$ then $\lambda $
is real and $0\leq \lambda <1$ as $0<w<2.$

Indeed, $\left[ \left( 1-w\right) I+w\tilde{L}\right] \left[ \left(
1-w\right) I+w\tilde{U}\right] x=\lambda \left( I-w\tilde{L}\right) \left(
I-w\tilde{U}\right) x,$

$\implies \left( 1-w\right) ^{2}Dx+w\left( 1-w\right) \left[ L+U\right]
x+w^{2}\tilde{U}^{T}D\tilde{U}x=\lambda \left[ D-w\left( L+U\right) +w^{2}%
\tilde{U}^{T}D\tilde{U}\right] x$

$L+D=D-A\implies \left( 1-w\right) Dx-w\left( 1-w\right) Ax+w^{2}\tilde{U}%
^{T}D\tilde{U}x=\lambda \left[ \left( 1-w\right) D+wA+w^{2}\tilde{U}^{T}D%
\tilde{U}\right] x\qquad \left( \ast \right) $

We can show that $\left[ \left( 1-w\right) D+wA+w^{2}\tilde{U}^{T}D\tilde{U}%
\right] $ is positive definite. 

\bigskip $\left( 1-w\right) D+wA+w^{2}\tilde{U}^{T}D\tilde{U}=D-\left(
U+U^{T}\right) w+w^{2}U^{T}D^{-1}U=\left( -wU^{T}D^{-\frac{1}{2}}+D^{\frac{1%
}{2}}\right) \left( -wD^{-\frac{1}{2}}U+D^{\frac{1}{2}}\right) $

$=\left( -wD^{-\frac{1}{2}}U+D^{\frac{1}{2}}\right) ^{T}\left( -wD^{-\frac{1%
}{2}}U+D^{\frac{1}{2}}\right) .-wU^{T}D^{-\frac{1}{2}}+D^{\frac{1}{2}}$ is
reversible$\implies $the positive definite property holds.

Similarly, we can show that $\left( 1-w\right) ^{2}D+w\left( 1-w\right) %
\left[ L+U\right] +w^{2}\tilde{U}^{T}D\tilde{U}$

$=\left( wD^{-\frac{1}{2}}U+\left( 1-w\right) D^{\frac{1}{2}}\right)
^{T}\left( wD^{-\frac{1}{2}}U+\left( 1-w\right) D^{\frac{1}{2}}\right) ,$%
which is also semi-positive definite

$\left( e.g.w=1,\text{it is semi-positive}\right) $

Therefore, multiplying $\left( \ast \right) $ with $x^{\ast }$ on the left
gives

$\lambda =\frac{\left( 1-w\right) x^{\ast }Dx+w\left( w-1\right) x^{\ast
}Ax+w^{2}x^{\ast }\tilde{U}D\tilde{U}x}{\left( 1-w\right) x^{\ast
}Dx+wx^{\ast }Ax+w^{2}x^{\ast }\tilde{U}D\tilde{U}x},$ the numerator is
nonnegative and the denominator is positive.

Notice $w-1<1\implies 0\leq \lambda <1,\lambda =0$ is reachable, as the
above example for 3D-Hilbert matrix with $w=1$ shows.

\bigskip 

$,$let $M=D^{-\frac{1}{2}}U^{T}D^{-\frac{1}{2}}\implies $

This property is obvious for $0<w\leq 1,$for $w>1,\left( 1-w\right)
D+wA=\left( w-1\right) \left( A-D\right) +A$

$\left\langle \left( \left( 1-w\right) D+wA\right) \left( x\right)
,x\right\rangle =\overset{n}{\underset{i=1}{\sum }}a_{ii}x_{i}^{2}+\overset{n%
}{\underset{\underset{i\neq j}{i,j=1}}{\sum }}wa_{ij}x_{i}x_{j}$

Thi

\FRAME{ftbpF}{4.4996in}{0.3797in}{0pt}{}{}{Figure}{\special{language
"Scientific Word";type "GRAPHIC";display "USEDEF";valid_file "T";width
4.4996in;height 0.3797in;depth 0pt;original-width 12.5441in;original-height
0.736in;cropleft "0";croptop "1";cropright "1";cropbottom "0";tempfilename
'NWGLP104.wmf';tempfile-properties "XPR";}}

\U{5148}\U{8bc1}\U{660e}\U{8fed}\U{4ee3}\U{77e9}\U{9635}$B=\left( D-\omega
L\right) ^{-1}\left[ \left( 1-\omega \right) D+\omega U\right] $
inversible.Indeed it is easy to calculate the determinant of $B$ is $\left(
1-\omega \right) ^{n}$ which is non-negative guaranteed by given condition.

Assume $\lambda $ is an eigenvalue of $B,$ then $0=\det \left( B-\lambda
I\right) =\det \left[ \left( D-\omega L\right) ^{-1}\left[ \left( 1-\omega
\right) D+\omega U-\lambda \left( D-\omega L\right) \right] \right] $

$\det \left( D-\omega L\right) ^{-1}\det [\left( 1-\omega -\lambda \right)
D+\omega U+\lambda \omega L]$

If $\lambda \geq 1,$then $\left\vert \frac{1-\omega -\lambda }{\omega }%
\right\vert =\frac{\lambda +\omega -1}{\omega }\geq 1,$and $\left\vert \frac{%
1-\omega -\lambda }{\lambda \omega }\right\vert =\frac{\lambda +\omega -1}{%
\lambda \omega }\geq 1$(the latter is equivalent to

$\left( \lambda -1\right) \left( 1-\omega \right) \geq 0)$

If $\lambda \leq -1,$then $\left\vert \frac{1-\omega -\lambda }{\omega }%
\right\vert =\frac{-\lambda -\omega +1}{\omega }=\frac{-\lambda +1}{\omega }%
-1\geq \frac{2}{\omega }-1\geq 1,$and $\left\vert \frac{1-\omega -\lambda }{%
\lambda \omega }\right\vert =\frac{1-\omega -\lambda }{-\lambda \omega }>1$

(the latter is equivalent to $\left( 1-\lambda \right) \left( 1-\omega
\right) \geq 0)$

Hence for $\left\vert \lambda \right\vert \geq 1,$the coefficient of $D$ is
not smaller than that of $U$ or $L\implies \left( 1-\omega -\lambda \right)
D+\omega U+\lambda \omega L$

is still strict diagonal dominant and as a result is inversible. A
contradiction with $0=\det \left( B-\lambda I\right) $

$\implies \left\vert \lambda \right\vert <1$ for every eigenvalue of $%
B.\implies \rho \left( B\right) <1,$and SOR method has convergent solution
to the linear equation system.

$\FRAME{itbpF}{5.2434in}{0.7887in}{0in}{}{}{Figure}{\special{language
"Scientific Word";type "GRAPHIC";display "USEDEF";valid_file "T";width
5.2434in;height 0.7887in;depth 0in;original-width 12.8779in;original-height
1.9173in;cropleft "0";croptop "1";cropright "1";cropbottom "0";tempfilename
'NWGNMK05.wmf';tempfile-properties "XPR";}}$

\U{7531}\U{8bfe}\U{672c}P92\U{63a8}\U{5bfc}$\left( 4.11\right) $\U{5f0f},%
\U{7528}\U{6b64}\U{9898}\U{7684}\U{7b26}\U{53f7}\U{8868}\U{793a}\U{5373}%
\U{6709}$F\left( x^{\left( k+1\right) }\right) =F\left( x^{\left( k\right)
}\right) -\alpha _{k+1}\left( b,p^{\left( k+1\right) }\right) +\alpha
_{k+1}^{2}\left( Ap^{\left( k+1\right) },p^{\left( k+1\right) }\right) .$%
\U{6b64}\U{5f0f}\U{6210}\U{7acb}\U{662f}\U{9009}\U{62e9}$p^{\left(
k+1\right) }$\U{4e0e}$p^{\left( 0\right) },p^{\left( 1\right) },...p^{\left(
k\right) }$ A\U{5171}\U{8f6d}\U{7684}\U{7ed3}\U{679c},\U{800c}\U{5728}%
\U{53d6}\U{5b9a}$p^{\left( k+1\right) }$\U{540e},CG\U{6cd5}\U{518d}\U{53d6}$%
\alpha _{k+1}$\U{6781}\U{5c0f}\U{5316}$F\left( x^{\left( k+1\right) }\right)
,$\U{800c}$F\left( x^{\left( k\right) }\right) $\U{5bf9}\U{5e94}$\alpha
_{k+1}=0$\U{65f6}\U{7684}\U{503c},\U{5b83}\U{5f53}\U{7136}\U{6bd4}\U{5173}%
\U{4e8e}$\alpha _{k+1}$\U{7684}\U{4e8c}\U{6b21}\U{51fd}\U{6570}\U{6700}%
\U{5c0f}\U{503c}$F\left( x^{\left( k+1\right) }\right) $\U{5927},\U{5373}%
\U{6211}\U{4eec}\U{6709}$F\left( x^{\left( k+1\right) }\right) \leq F\left(
x^{\left( k\right) }\right) .$

\U{56e0}$\alpha _{k+1}=\frac{\left( b,p^{\left( k+1\right) }\right) }{\left(
Ap^{\left( k+1\right) },p^{\left( k+1\right) }\right) }=\frac{\left(
r^{\left( k\right) },p^{\left( k+1\right) }\right) }{\left( Ap^{\left(
k+1\right) },p^{\left( k+1\right) }\right) }$\U{56e0}$r^{\left( k\right)
}=b-Ax^{\left( k\right) }$,\U{800c}$x^{\left( k\right) }$\U{53ef}\U{8868}%
\U{4e3a}$p^{\left( 0\right) },p^{\left( 1\right) },...p^{\left( k\right) }$%
\U{7684}\U{7ebf}\U{6027}\U{7ec4}\U{5408}$,$\U{7531}$p^{\left( k+1\right) }$%
\U{4e0e}$p^{\left( 0\right) },p^{\left( 1\right) },...p^{\left( k\right) }$ A%
\U{5171}\U{8f6d}\U{53ef}\U{77e5}$\left( b,p^{\left( k+1\right) }\right)
=\left( r^{\left( k\right) },p^{\left( k+1\right) }\right) \implies \alpha
_{k+1}=0$ if $r^{\left( k\right) }=0,$\U{56e0}\U{800c}\U{6b64}\U{65f6}%
\U{4e8c}\U{6b21}\U{51fd}\U{6570}\U{6700}\U{5c0f}\U{503c}\U{5728}$\alpha
_{k+1}=0$

\U{5904}\U{53d6}\U{5230},\U{540c}\U{65f6}\U{4e5f}\U{5f97}\U{5230}$\qquad
\alpha _{k+1}\neq 0$ if $r^{\left( k\right) }\neq 0,$\U{5373}\U{6b64}\U{65f6}%
$F\left( x^{\left( k+1\right) }\right) \leq F\left( x^{\left( k\right)
}\right) $\U{53d6}\U{4e25}\U{683c}\U{4e0d}\U{7b49}\U{53f7}$\boxtimes $

\end{document}
