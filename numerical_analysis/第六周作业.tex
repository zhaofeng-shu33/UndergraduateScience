
\documentclass{article}
%%%%%%%%%%%%%%%%%%%%%%%%%%%%%%%%%%%%%%%%%%%%%%%%%%%%%%%%%%%%%%%%%%%%%%%%%%%%%%%%%%%%%%%%%%%%%%%%%%%%%%%%%%%%%%%%%%%%%%%%%%%%%%%%%%%%%%%%%%%%%%%%%%%%%%%%%%%%%%%%%%%%%%%%%%%%%%%%%%%%%%%%%%%%%%%%%%%%%%%%%%%%%%%%%%%%%%%%%%%%%%%%%%%%%%%%%%%%%%%%%%%%%%%%%%%%
\usepackage{amssymb}
\usepackage{amsmath}

\setcounter{MaxMatrixCols}{10}
%TCIDATA{OutputFilter=LATEX.DLL}
%TCIDATA{Version=5.00.0.2552}
%TCIDATA{<META NAME="SaveForMode" CONTENT="1">}
%TCIDATA{Created=Monday, October 19, 2015 23:38:54}
%TCIDATA{LastRevised=Thursday, October 29, 2015 00:54:06}
%TCIDATA{<META NAME="GraphicsSave" CONTENT="32">}
%TCIDATA{<META NAME="DocumentShell" CONTENT="Scientific Notebook\Blank Document">}
%TCIDATA{CSTFile=Math with theorems suppressed.cst}
%TCIDATA{PageSetup=72,72,72,72,0}
%TCIDATA{AllPages=
%F=36,\PARA{038<p type="texpara" tag="Body Text" >\hfill \thepage}
%}


\newtheorem{theorem}{Theorem}
\newtheorem{acknowledgement}[theorem]{Acknowledgement}
\newtheorem{algorithm}[theorem]{Algorithm}
\newtheorem{axiom}[theorem]{Axiom}
\newtheorem{case}[theorem]{Case}
\newtheorem{claim}[theorem]{Claim}
\newtheorem{conclusion}[theorem]{Conclusion}
\newtheorem{condition}[theorem]{Condition}
\newtheorem{conjecture}[theorem]{Conjecture}
\newtheorem{corollary}[theorem]{Corollary}
\newtheorem{criterion}[theorem]{Criterion}
\newtheorem{definition}[theorem]{Definition}
\newtheorem{example}[theorem]{Example}
\newtheorem{exercise}[theorem]{Exercise}
\newtheorem{lemma}[theorem]{Lemma}
\newtheorem{notation}[theorem]{Notation}
\newtheorem{problem}[theorem]{Problem}
\newtheorem{proposition}[theorem]{Proposition}
\newtheorem{remark}[theorem]{Remark}
\newtheorem{solution}[theorem]{Solution}
\newtheorem{summary}[theorem]{Summary}
\newenvironment{proof}[1][Proof]{\noindent\textbf{#1.} }{\ \rule{0.5em}{0.5em}}
\def\TEXTsymbol#1{\mbox{$#1$}}%
\def\NEG#1{\leavevmode\hbox{\rlap{\thinspace/}{$#1$}}}%
\def\QATOPD#1#2#3#4{{#3 \atopwithdelims#1#2 #4}}%
\def\QTP#1{}
\def\func#1{\mathop{\rm #1}}%

\begin{document}


\bigskip 赵丰 2013012178 Numerical Analysis 第四%
章纸制作业

%FRAME

\bigskip a为方程$f\left( x\right) =x^{2}-a$在某%
闭区间上的单根,

由迭代函数的构造可%
知:$\varphi _{3}\left( x\right) =x-\frac{f\left( x\right) }{f^{\prime
}\left( x\right) }+\frac{f\left( x\right) ^{2}}{2}\frac{d^{2}}{dt^{2}}%
f^{-1}\left( t\right) _{|t=f\left( x\right) }$

$=x-\frac{x^{2}-a}{2x}+\frac{\left( x^{2}-a\right) ^{2}}{2}\frac{d^{2}}{%
dt^{2}}\sqrt{t+a}_{|t=x^{2}-a}=\frac{x}{2}+\frac{a}{2x}-\frac{\left(
x^{2}-a\right) ^{2}}{2}\frac{1}{4x^{3}}=\allowbreak \allowbreak \frac{3}{8}x+%
\frac{3}{4}\frac{a}{x}-\frac{1}{8}\frac{a^{2}}{x^{3}}.$

%FRAME

由课本P111 Thm3.2知此时$\alpha $也%
是加速后的函数$\psi \left(
x\right) =x-\frac{\left[ \phi \left( x\right) -x\right] ^{2}}{\phi \left(
\phi \left( x\right) \right) -2\phi \left( x\right) +x}$ 的不%
动点,为证$\psi \left( x\right) $产生%
的迭代序列至少2阶收%
敛,只需证$\psi ^{\prime }\left( \alpha \right) =0,$%
为此,先对$\phi \left( x\right) $在$\alpha $%
处做Taylor expansion得: $\phi (x)=\alpha +\left( x-\alpha
\right) \phi ^{\prime }\left( \alpha \right) +o(x-\alpha )$

$\phi \left( \phi \left( x\right) \right) =\alpha +\left( \alpha +\left(
x-\alpha \right) \phi ^{\prime }\left( \alpha \right) +o(x-\alpha )-\alpha
\right) \phi ^{\prime }\left( \alpha \right) $

$+\left( \alpha +\left( x-\alpha \right) \phi ^{\prime }\left( \alpha
\right) +o(x-\alpha )-\alpha \right) o\left( 1\right) =\alpha +\left(
x-\alpha \right) \phi ^{\prime 2}\left( \alpha \right) +o(x-\alpha )$

$\left( \text{Noticing that }\frac{d}{dx}\phi \left( \phi \left( x\right)
\right) _{|x=\alpha }=\phi ^{\prime 2}\left( \alpha \right) ,\text{it is
also possible to directly expand }\phi \left( \phi \left( x\right) \right)
\right) $

$\phi \left( x\right) -x=\left( x-\alpha \right) \left( -1+\phi ^{\prime
}\left( \alpha \right) +o\left( 1\right) \right) $

$\implies \psi \left( x\right) =x-\frac{\left[ \phi \left( x\right) -x\right]
^{2}}{\phi \left( \phi \left( x\right) \right) -2\phi \left( x\right) +x}=x-%
\frac{\left( x-\alpha \right) ^{2}\left( -1+\phi ^{\prime }\left( \alpha
\right) +o\left( 1\right) \right) ^{2}}{\alpha +\left( x-\alpha \right) \phi
^{\prime 2}\left( \alpha \right) +o(x-\alpha )-2\left[ \alpha +\left(
x-\alpha \right) \phi ^{\prime }\left( \alpha \right) +o(x-\alpha )\right] +x%
}$

$=x-\frac{\left( x-\alpha \right) ^{2}\left( -1+\phi ^{\prime }\left( \alpha
\right) +o\left( 1\right) \right) ^{2}}{x-\alpha +\left( x-\alpha \right)
(\phi ^{\prime 2}\left( \alpha \right) -2\phi ^{\prime }\left( \alpha
\right) )+o(x-\alpha )}=x-\frac{\left( x-\alpha \right) \left( -1+\phi
^{\prime }\left( \alpha \right) +o\left( 1\right) \right) ^{2}}{\left( \phi
^{\prime }\left( \alpha \right) -1\right) ^{2}+o(1)}$

$\frac{\psi \left( x\right) -\alpha }{x-\alpha }=1-\frac{\left( -1+\phi
^{\prime }\left( \alpha \right) +o\left( 1\right) \right) ^{2}}{\left( \phi
^{\prime }\left( \alpha \right) -1\right) ^{2}+o(1)},$notice that $\phi
^{\prime }\left( \alpha \right) -1\neq 0\implies \frac{d}{dx}\psi \left(
x\right) _{|x=\alpha }=\underset{x->\alpha }{\lim }\frac{\psi \left(
x\right) -\alpha }{x-\alpha }=1-\frac{\left( -1+\phi ^{\prime }\left( \alpha
\right) \right) ^{2}}{\left( \phi ^{\prime }\left( \alpha \right) -1\right)
^{2}}=0\boxtimes $

%FRAME

$\left( 1\right) $ $\phi \left( x\right) =x-\frac{f\left( x\right) }{%
f^{\prime }\left( x\right) }=x-\frac{x^{3}-3x-1}{3x^{2}-3}=\allowbreak \frac{%
1}{3}\frac{2x^{3}+1}{x^{2}-1};$

$x_{0}=2,x_{1}=\phi \left( x_{0}\right) =\allowbreak \frac{1}{3}\frac{%
2\times 2^{3}+1}{2^{2}-1}=\allowbreak \frac{17}{9};$

$x_{2}=\phi \left( x_{1}\right) =\allowbreak \frac{1}{3}\frac{2\times \left( 
\frac{17}{9}\right) ^{3}+1}{\left( \frac{17}{9}\right) ^{2}-1}=\allowbreak 
\frac{10\,555}{5616}=\allowbreak 1.\,\allowbreak 879\,5.$

$x_{2}$ has four-digits significant figure.

$\left( 2\right) x_{0}=1,x_{1}=2,$by the formula of recursion: $%
x_{k+1}=x_{k}-\frac{f\left( x_{k}\right) \left( x_{k}-x_{k-1}\right) }{%
f\left( x_{k}\right) -f\left( x_{k-1}\right) },$

$x_{2}=2-\frac{f\left( 2\right) }{f\left( 2\right) -f\left( 1\right) }=\frac{%
7}{4};$

$x_{3}=x_{2}-\frac{f\left( x_{2}\right) \left( x_{2}-x_{1}\right) }{f\left(
x_{2}\right) -f\left( x_{1}\right) }=\frac{7}{4}-\frac{f\left( \frac{7}{4}%
\right) \left( -\frac{1}{4}\right) }{f\left( \frac{7}{4}\right) -1}=\frac{226%
}{121};$

Similarly, $x_{4}=1.8806,x_{5}=1.8793,$which has four-digits significant
figure.

$\left( 3\right) $Take $x_{0}=1,x_{1}=1.5,x_{2}=2,$programming gives the
following results:

$x_{3}=1.87394,$which already has four-digits significant figure.

%FRAME

Newton's Method:$\phi \left( x\right) =x-\frac{f\left( x\right) }{f^{\prime
}\left( x\right) }=x-\frac{\left( \frac{x}{2}-\sin x\right) }{1-2\cos x}%
,x_{0}=\frac{\pi }{2},$programming gives the following results:

\{x$_{1}=$1.7854, x$_{2}=$1.84456, 1.87083, 1.88335, 1.88946, 1.89249,
1.89399,

1.89475, 1.89512, 1.89531, 1.8954, 1.89545, 1.89547, 1.89548, x$_{15}=$%
1.89549\}

$\implies x_{12}=$1.89545 has 5-digits significant figure.

Let $\mu \left( x\right) =\frac{f\left( x\right) }{f^{\prime }\left(
x\right) }=\frac{\left( \frac{x}{2}-\sin x\right) }{1-2\cos x},$then $\mu
\left( x\right) $ has the same root with $f\left( x\right) $ but with
multiplicity one. Using Newton's Method for $\mu \left( x\right) $ gives: $%
\phi \left( x\right) =x-\frac{\mu \left( x\right) }{\mu ^{\prime }\left(
x\right) }=%FRAME.$

$x_{0}=\frac{\pi }{2},$programming gives the following results:

\{1.80175, 1.88963, x$_{3}=$1.89547, 1.89549, 1.89549\}, and x$_{3}$ has
5-digits significant figure.

%FRAME

Solution: $Jf=\left( 
\begin{array}{cc}
A-\lambda I & -x \\ 
2x^{T} & 0%
\end{array}%
\right) _{\left( n+1\right) ^{2}},$where $A-\lambda I$ is a n$\times n$
matrix and $x\in R.$ 

the iteration function is $\phi \binom{x}{\lambda }=\binom{x}{\lambda }%
-\left( Jf\right) ^{-1}\binom{Ax-\lambda x}{x^{T}x-1}.$Below we give
analytic simplification of $\phi \binom{x}{\lambda },$ such method is said
to be important because if we have $\left( x^{T},\lambda \right) $ near to
the exact eigenvector $\left( \text{but }\lambda \neq \lambda _{\func{real}%
}\right) $and eigenvalue of A and we also assume the normalization condition
x$^{T}x=1$,then we can proceed one-step Newton's method and get a more
accurate solution, the result is $\binom{\frac{\left( A-\lambda I\right)
^{-1}x}{x^{T}\left( A-\lambda I\right) ^{-1}x}}{\lambda +\frac{1}{%
x^{T}\left( A-\lambda I\right) ^{-1}x}}.$ This is the 

solution to the linear equation system $\binom{x_{s}}{\lambda _{s}}=\phi 
\binom{x}{\lambda }_{|x^{T}x=1},$and we can verify this result by
rearranging this equation $Jf\binom{x_{s}-x}{\lambda _{s}-\lambda }=-\binom{%
Ax-\lambda x}{0}.$Below is another approach to get the same equation system:

%FRAME

Notice that in the one dimensional equation, $2x^{T}\left( x_{s}-x\right) =0,
$2 is trivial and (4.4) is equivalent to

$\binom{x_{s}}{\lambda _{s}}=\phi \binom{x}{\lambda }_{|x^{T}x=1}.$

To get an explicit expression of $\phi \binom{x}{\lambda },$we need to
calculate $Jf^{-1}.$We assume that the initial value $\lambda _{0}$ is not 

the eigenvalue of A, otherwise we can solve $\left( A-\lambda I\right) x=0$
to get the exact solution of  $F\binom{x}{\lambda }=0.$

We can assume the inverse matrix of $Jf$ has the form $Jf^{-1}=\left( 
\begin{array}{cc}
B & \beta  \\ 
\alpha ^{T} & c%
\end{array}%
\right) ,$

\bigskip Using the relation  $Jf\cdot Jf^{-1}=I_{n+1}$ gives: $\left(
A-\lambda I\right) B-x\alpha ^{T}=I_{n}\left( 1\right) ,\left( A-\lambda
I\right) \beta =cx\left( 2\right) ,x^{T}B=0\left( 3\right) ,2x^{T}\beta
=1\left( 4\right) .$

From $\left( 2,4\right) $ follows that $c=\frac{1}{2x^{T}\left( A-\lambda
I\right) ^{-1}x},\beta =\frac{\left( A-\lambda I\right) ^{-1}x}{2x^{T}\left(
A-\lambda I\right) ^{-1}x}.$

Using the relation $Jf^{-1}\cdot Jf=I_{n+1}$ gives: 

$B\left( A-\lambda I\right) +2\beta x^{T}=I_{n}\left( 5\right) ,Bx=0\left(
6\right) ,\alpha ^{T}\left( A-\lambda I\right) +2cx^{T}=0\left( 7\right)
,-\alpha ^{T}x=1\left( 8\right) $

From $\left( 7\right) $ follows that $\alpha =\frac{-\left( A^{T}-\lambda
I\right) ^{-1}x}{x^{T}\left( A-\lambda I\right) ^{-1}x},$combining with $%
\left( 1\right) $ gives further $B=\left( A-\lambda I\right) ^{-1}-\frac{%
\left( A-\lambda I\right) ^{-1}xx^{T}\left( A-\lambda I\right) ^{-1}}{%
x^{T}\left( A-\lambda I\right) ^{-1}x}.$

We can verify the rightousness of $\left( 3,5,6,8\right) .$

Next we substitute $Jf^{-1}=$ $\left( 
\begin{array}{cc}
B & \beta  \\ 
\alpha ^{T} & c%
\end{array}%
\right) $ into $\binom{x}{\lambda }-\left( Jf\right) ^{-1}\binom{Ax-\lambda x%
}{x^{T}x-1},$ after simplification gives $\binom{\frac{\left(
1+x^{T}x\right) \left( A-\lambda \right) ^{-1}x}{2x^{T}\left( A-\lambda
I\right) ^{-1}x}}{\lambda +\frac{1+x^{T}x}{2x^{T}\left( A-\lambda I\right)
^{-1}x}}.$

If $x^{T}x=1,$we get the above results $\binom{\frac{\left( A-\lambda
I\right) ^{-1}x}{x^{T}\left( A-\lambda I\right) ^{-1}x}}{\lambda +\frac{1}{%
x^{T}\left( A-\lambda I\right) ^{-1}x}},$which is a more accurate solution
after one-iteration.

Below is a numerical example: consider the following matrix, which has
unique real eigenvalue 4 and corresponds the eigenvector $\left(
-1,-1,1\right) .$ %FRAME

Starting from \{-0.85/$\sqrt{3}$, -0.95/$\sqrt{3}$, 0.9/$\sqrt{3}$,
3.9\},programming produces the following results based on Newton's method
(we do not use the explicit form, since it contains the inverse matrix and
when $\lambda $ is very near to the real eigenvalue, the matrix is very
ill-conditioned and programming . Instead we solve linear equation system
about $x_{k}$ in the k$-th$ step.

\{-0.58166, -0.580349, 0.581341, 4.01181\}, \{-0.577363, -0.577361,
0.577364, 4.00007\},

\{-0.57735, -0.57735, 0.57735, 4.\}, \{-0.57735, -0.57735, 0.57735, 4.\}.

We can see that $x_{4}$ is very near to the exact solution, if we start from
points in R$^{4}$ in a small neighborhood of $\left( \frac{-1}{\sqrt{3}},%
\frac{-1}{\sqrt{3}},\frac{1}{\sqrt{3}},4\right) .$

\end{document}
