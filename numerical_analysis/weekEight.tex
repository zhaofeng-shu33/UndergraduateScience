
\documentclass{article}
\usepackage{amsmath}

%%%%%%%%%%%%%%%%%%%%%%%%%%%%%%%%%%%%%%%%%%%%%%%%%%%%%%%%%%%%%%%%%%%%%%%%%%%%%%%%%%%%%%%%%%%%%%%%%%%%%%%%%%%%%%%%%%%%%%%%%%%%%%%%%%%%%%%%%%%%%%%%%%%%%%%%%%%%%%%%%%%%%%%%%%%%%%%%%%%%%%%%%%%%%%%%%%%%%%%%%%%%%%%%%%%%%%%%%%%%%%%%%%%%
%TCIDATA{OutputFilter=LATEX.DLL}
%TCIDATA{Version=5.00.0.2552}
%TCIDATA{<META NAME="SaveForMode" CONTENT="1">}
%TCIDATA{Created=Saturday, November 07, 2015 19:44:57}
%TCIDATA{LastRevised=Sunday, November 08, 2015 15:41:25}
%TCIDATA{<META NAME="GraphicsSave" CONTENT="32">}
%TCIDATA{<META NAME="DocumentShell" CONTENT="Scientific Notebook\Blank Document">}
%TCIDATA{CSTFile=Math with theorems suppressed.cst}
%TCIDATA{PageSetup=72,72,72,72,0}
%TCIDATA{AllPages=
%F=36,\PARA{038<p type="texpara" tag="Body Text" >\hfill \thepage}
%}


\newtheorem{theorem}{Theorem}
\newtheorem{acknowledgement}[theorem]{Acknowledgement}
\newtheorem{algorithm}[theorem]{Algorithm}
\newtheorem{axiom}[theorem]{Axiom}
\newtheorem{case}[theorem]{Case}
\newtheorem{claim}[theorem]{Claim}
\newtheorem{conclusion}[theorem]{Conclusion}
\newtheorem{condition}[theorem]{Condition}
\newtheorem{conjecture}[theorem]{Conjecture}
\newtheorem{corollary}[theorem]{Corollary}
\newtheorem{criterion}[theorem]{Criterion}
\newtheorem{definition}[theorem]{Definition}
\newtheorem{example}[theorem]{Example}
\newtheorem{exercise}[theorem]{Exercise}
\newtheorem{lemma}[theorem]{Lemma}
\newtheorem{notation}[theorem]{Notation}
\newtheorem{problem}[theorem]{Problem}
\newtheorem{proposition}[theorem]{Proposition}
\newtheorem{remark}[theorem]{Remark}
\newtheorem{solution}[theorem]{Solution}
\newtheorem{summary}[theorem]{Summary}
\newenvironment{proof}[1][Proof]{\noindent\textbf{#1.} }{\ \rule{0.5em}{0.5em}}
\def\TEXTsymbol#1{\mbox{$#1$}}%
\def\NEG#1{\leavevmode\hbox{\rlap{\thinspace/}{$#1$}}}%
\def\QATOPD#1#2#3#4{{#3 \atopwithdelims#1#2 #4}}%
\def\QTP#1{}
\def\func#1{\mathop{\rm #1}}%

\begin{document}


\bigskip \bigskip 赵丰 2013012178 Numerical Analysis 第5 
章纸制作业

\bigskip 1. Since $\left\{ x^{\left( 1\right) },..x^{\left( n\right)
}\right\} $ is a basis of R$^{n},v^{\left( 0\right) }=v_{1}x^{\left(
1\right) }+\underset{i=2}{\overset{n}{\sum }}v_{i}x^{\left( i\right) },$

$\left\Vert v^{\left( k\right) }\right\Vert _{2}v^{\left( k\right)
}=A^{k}v^{\left( 0\right) }=v_{1}\lambda _{1}^{k}x^{\left( 1\right) }+%
\underset{i=2}{\overset{n}{\sum }}\lambda _{i}^{k}v_{i}x^{\left( i\right)
}=\lambda _{1}^{k}\left( v_{1}x^{\left( 1\right) }+\underset{i=2}{\overset{n}%
{\sum }}\left( \frac{\lambda _{i}}{\lambda _{1}}\right) ^{k}v_{i}x^{\left(
i\right) }\right) $

$R_{k}=\left( Av^{\left( k\right) },v^{\left( k\right) }\right) =\frac{%
\left( v_{1}\lambda _{1}x^{\left( 1\right) }+\underset{i=2}{\overset{n}{\sum 
}}\lambda _{2}\left( \frac{\lambda _{i}}{\lambda _{1}}\right)
^{k}v_{i}x^{\left( i\right) },v_{1}x^{\left( 1\right) }+\underset{i=2}{%
\overset{n}{\sum }}\left( \frac{\lambda _{i}}{\lambda _{1}}\right)
^{k}v_{i}x^{\left( i\right) }\right) }{\left\Vert v_{1}x^{\left( 1\right) }+%
\underset{i=2}{\overset{n}{\sum }}\left( \frac{\lambda _{i}}{\lambda _{1}}%
\right) ^{k}v_{i}x^{\left( i\right) }\right\Vert _{2}^{2}}$

$=\frac{\lambda _{1}v_{1}^{2}\left\Vert x^{\left( 1\right) }\right\Vert
_{2}^{2}+O\left( \left\vert \frac{\lambda _{2}}{\lambda _{1}}\right\vert
\right) ^{k}}{v_{1}^{2}\left\Vert x^{\left( 1\right) }\right\Vert
_{2}^{2}+O\left( \left\vert \frac{\lambda _{2}}{\lambda _{1}}\right\vert
\right) ^{k}}$

$\implies R_{k}-\lambda _{1}=\frac{O\left( \left\vert \frac{\lambda _{2}}{%
\lambda _{1}}\right\vert \right) ^{k}}{v_{1}^{2}\left\Vert x^{\left(
1\right) }\right\Vert _{2}^{2}+O\left( \left\vert \frac{\lambda _{2}}{%
\lambda _{1}}\right\vert \right) ^{k}}=O\left( \left\vert \frac{\lambda _{2}%
}{\lambda _{1}}\right\vert \right) ^{k},$that is 

$\left\vert R_{k}-\lambda _{1}\right\vert \leq Cr^{k},r=\left\vert \frac{%
\lambda _{2}}{\lambda _{1}}\right\vert ,k=1,2..$

2. Householder and Givens Transformations are orthogonal$\implies c^{2}=3.$

We calculate for the case $c=\sqrt{3},$ and $c=-\sqrt{3}$ is similar.

$w=\left( 1,1,1\right) ^{T}-\left( \sqrt{3},0,0\right) ^{T},H=I_{3}-2\frac{%
ww^{T}}{\left\Vert w\right\Vert _{2}^{2}}$

$=\frac{1}{6-2\sqrt{3}}%
\begin{pmatrix}
2\sqrt{3}-2 & 2\sqrt{3}-2 & 2\sqrt{3}-2 \\ 
2\sqrt{3}-2 & 4-2\sqrt{3} & -2 \\ 
2\sqrt{3}-2 & -2 & 4-2\sqrt{3}%
\end{pmatrix}%
,Hw=\left( \sqrt{3},0,0\right) ^{T}.$

Using the geometric interpretation of Givens Transformation in R$^{3},$ 

we first rotate $w$ around $z$ axis to make its y-coordinate vanish.

Such rotation matrix can be taken as G$_{1}=%
\begin{pmatrix}
\frac{\sqrt{2}}{2} & \frac{\sqrt{2}}{2} & 0 \\ 
-\frac{\sqrt{2}}{2} & \frac{\sqrt{2}}{2} & 0 \\ 
0 & 0 & 1%
\end{pmatrix}%
.$

G$_{1}w=\left( \sqrt{2},0,1\right) ^{T}.$Let $G_{2}=%
\begin{pmatrix}
\frac{\sqrt{6}}{3} & 0 & \frac{\sqrt{3}}{3} \\ 
0 & 1 & 0 \\ 
-\frac{\sqrt{3}}{3} & 0 & \frac{\sqrt{6}}{3}%
\end{pmatrix}%
.$

$G_{2}G_{1}w=\left( \sqrt{3},0,0\right) ^{T}.J=G_{1}G_{2}=%
\begin{pmatrix}
\frac{\sqrt{3}}{3} & \frac{\sqrt{2}}{2} & \frac{\sqrt{6}}{6} \\ 
-\frac{\sqrt{3}}{3} & \frac{\sqrt{2}}{2} & -\frac{\sqrt{6}}{6} \\ 
-\frac{\sqrt{3}}{3} & 0 & \frac{\sqrt{6}}{3}%
\end{pmatrix}%
.$

3.Using Givens Transformation can realize QR Decomposition of upper
Hessenberg matrix H as follows:

First choose $\theta _{1},s.t.H_{1}=J\left( 1,2,\theta _{1}\right) H$ that
makes the $\left( 2,1\right) $-element of H$_{1}$ vanish. For $i>j+1,$the  $%
\left( i,j\right) $-element of H$_{1}$ is $\underset{k=1}{\overset{n}{\sum }}
$ $J\left( 1,2,\theta _{1}\right) _{i,k}H_{k,j}.$Since $i>j+1\geq 2,$the
non-zero term in the summation satisfies $k=i,$then 

$\underset{k=1}{\overset{n}{\sum }}$ $J\left( 1,2,\theta _{1}\right)
_{i,k}H_{k,j}=H_{i,j}$ follows. Since $H$ is upper Hessenberg matrix, 

$H_{i,j}=0,$for $i>j+1.$ Then $H_{1}$ is an upper Hessenberg matrix with $%
H_{1,2,1}=0.$

Suppose we proceed to the s-th step, and 

H$_{s}=J\left( s,s+1,\theta s\right) ..J\left( 1,2,\theta _{1}\right) H$ is
an upper Hesenberg matrix with H$_{s,i,i-1}=0,$for $i=2,3..s$. Then we
choose $\theta _{s+1},s.t.H_{s+1}=J\left( s+1,s+2,\theta _{s+1}\right) H_{s}$
that makes the $\left( s+1,s+2\right) $-element of H$_{s+1}$ vanish. To show
the structure of $H_{s+1},$by the property of Givens tranformation we only
need to consider $i=s+1$ and $s+2,$since other rows are unchanged. For $%
i>j+1,$the $\left( i,j\right) $-element of H$_{2}$ is

$\underset{k=1}{\overset{n}{\sum }}$ $J\left( s+1,s+2,\theta _{s+1}\right)
_{i,k}H_{s,k,j}.$For non-vanishing term of $J\left( s+1,s+2,\theta
_{s+1}\right) _{i,k}$

$k\geq i>j+1,$but then $H_{s,k,j}$ vanish for such case since $H_{s,k,j}$ is
an upper Hessenberg matrix. $\implies H_{s+1}$ is an upper Hessenberg matrix
with H$_{s+1,i,i-1}=0,$for $i=2,3..s,s+1.$ By induction we show that 

$Q^{T}=J\left( n-1,n,\theta _{n-1}\right) ..J\left( 1,2,\theta _{1}\right)
,R:=Q^{T}H$ is an upper trianglar matrix.

Further, $Q=J\left( 1,2,-\theta _{1}\right) ..J\left( n-1,n,-\theta
_{n-1}\right) .$

\bigskip To show $Q$ is upper Hessenberg matrix, consider $Q_{s}=J\left(
1,2,-\theta _{1}\right) ..J\left( s,s+1,-\theta _{n-1}\right) .$ And suppose 
$Q_{s}$ is upper Hessenberg matrix with $Q_{s,i+1,i}=0,i=n,n-1,s+1$. $%
Q_{s+1}=Q_{s}J\left( s+1,s+2,-\theta _{n-1}\right) .$We need only consider
for j=s+1 and j=s+2.

$Q_{s+1,i,j,i>j+1}=$ $\underset{k=1}{\overset{n}{\sum }}$ $Q_{s,i,k}J\left(
s+1,s+2,-\theta _{n-1}\right) _{k,j}.j=s+1,$

$Q_{s+1,i,j,i>j+1}=Q_{s,i,j}J\left( s+1,s+2,-\theta _{n-1}\right)
_{s+1,s+1}+Q_{s,i,s+2}J\left( s+1,s+2,-\theta _{n-1}\right) _{s+2,s+1}$

Since $Q_{s,i,j,i>j+1}=0,Q_{s,s+3,s+2}=0\left( \text{induction assumption}%
\right) ,Q_{s+1,i,j,i>j+1}=0,$

Similarly, $j=s+2,Q_{s+1,i,j,i>j+1}=0.$By induction$\implies Q=Q_{n-1}$ is
upper Hessenberg matrix.

Then the original H has QR Decomposition $H=QR.$

Let $H^{\prime }=RQ,H_{i,j,i>j+1}^{\prime }=$ $\underset{k=1}{\overset{n}{%
\sum }}$ $R_{i,k}Q_{k,j},$ the non-vanishing term satisfies 

$i\leq k,k\leq j+1\implies i\leq j+1,$impossible$\implies
H_{i,j,i>j+1}^{\prime }=0\implies $

$H^{\prime }$ is upper Hessenberg matrix. Therefore, Hessenberg matrix is
unchanged under QR algorithm.

\end{document}
