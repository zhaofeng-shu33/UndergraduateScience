
\documentclass{article}
%%%%%%%%%%%%%%%%%%%%%%%%%%%%%%%%%%%%%%%%%%%%%%%%%%%%%%%%%%%%%%%%%%%%%%%%%%%%%%%%%%%%%%%%%%%%%%%%%%%%%%%%%%%%%%%%%%%%%%%%%%%%%%%%%%%%%%%%%%%%%%%%%%%%%%%%%%%%%%%%%%%%%%%%%%%%%%%%%%%%%%%%%%%%%%%%%%%%%%%%%%%%%%%%%%%%%%%%%%%%%%%%%%%%%%%%%%%%%%%%%%%%%%%%%%%%
%TCIDATA{OutputFilter=LATEX.DLL}
%TCIDATA{Version=5.00.0.2552}
%TCIDATA{<META NAME="SaveForMode" CONTENT="1">}
%TCIDATA{Created=Wednesday, December 02, 2015 21:44:04}
%TCIDATA{LastRevised=Wednesday, December 02, 2015 22:05:10}
%TCIDATA{<META NAME="GraphicsSave" CONTENT="32">}
%TCIDATA{<META NAME="DocumentShell" CONTENT="Standard LaTeX\Blank - Standard LaTeX Article">}
%TCIDATA{CSTFile=40 LaTeX article.cst}

\newtheorem{theorem}{Theorem}
\newtheorem{acknowledgement}[theorem]{Acknowledgement}
\newtheorem{algorithm}[theorem]{Algorithm}
\newtheorem{axiom}[theorem]{Axiom}
\newtheorem{case}[theorem]{Case}
\newtheorem{claim}[theorem]{Claim}
\newtheorem{conclusion}[theorem]{Conclusion}
\newtheorem{condition}[theorem]{Condition}
\newtheorem{conjecture}[theorem]{Conjecture}
\newtheorem{corollary}[theorem]{Corollary}
\newtheorem{criterion}[theorem]{Criterion}
\newtheorem{definition}[theorem]{Definition}
\newtheorem{example}[theorem]{Example}
\newtheorem{exercise}[theorem]{Exercise}
\newtheorem{lemma}[theorem]{Lemma}
\newtheorem{notation}[theorem]{Notation}
\newtheorem{problem}[theorem]{Problem}
\newtheorem{proposition}[theorem]{Proposition}
\newtheorem{remark}[theorem]{Remark}
\newtheorem{solution}[theorem]{Solution}
\newtheorem{summary}[theorem]{Summary}
\newenvironment{proof}[1][Proof]{\noindent\textbf{#1.} }{\ \rule{0.5em}{0.5em}}
\def\TEXTsymbol#1{\mbox{$#1$}}%
\def\NEG#1{\leavevmode\hbox{\rlap{\thinspace/}{$#1$}}}%
\def\QATOPD#1#2#3#4{{#3 \atopwithdelims#1#2 #4}}%
\def\QTP#1{}
\def\func#1{\mathop{\rm #1}}%

\begin{document}


使用Euclidean algorithm to compute the rational Pad\'{e}
approximant:

举例,课上以 $\log \left( 1+x\right) $ 的$%
R_{2,1}\left( x\right) $为例,分别介绍%
了用解矩阵方程组和%
连分式的方法求解此Pad%
\'{e} approximant,实际也可用高代%
学过的Euclidean algorithm求解之.$%
R_{2,1}\left( x\right) =\frac{P_{2}\left( x\right) }{Q_{1}\left( x\right) }$

$R_{2,1}\left( x\right) $与$\tilde{P}\left( x\right) =x-\frac{x^{2}}{2}%
+\frac{x^{3}}{3}$在$0$处相等 up to 三阶%
导,可理解为$R_{2,1}\left( x\right) $作
formal

power series expansion 前4项截断为$\tilde{P}%
\left( x\right) $,即$\frac{P_{2}\left( x\right) }{Q_{1}\left( x\right) 
}=\tilde{P}\left( x\right) \qquad \func{mod}\left( x^{4}\right) $

即存在多项式$K\left( x\right)
,s.t.P_{2}\left( x\right) (=\tilde{P}\left( x\right) Q_{1}\left( x\right)
+Q_{1}\left( x\right) K\left( x\right) x^{4})=\tilde{P}\left( x\right)
Q_{1}\left( x\right) +K_{1}\left( x\right) x^{4},$其中$x^{4}$%
与$\tilde{P}\left( x\right) $已知,于是可%
以用Euclidean algorithm对$x^{4}$与$\tilde{P}\left(
x\right) $做辗转相除,直到余%
项的次数为$2,$只要凑出$%
P_{2}\left( x\right) =\tilde{P}\left( x\right) Q_{1}\left( x\right)
+K_{1}\left( x\right) x^{4}$的表达式即可,%
对于此题,用计算软件%
算一步$x^{4}$关于$\tilde{P}\left( x\right) $%
的带余除法即给出商%
为$Q_{1}\left( x\right) $余数为$P_{2}\left( x\right)
.$注意$K_{1}\left( x\right) \left( =Q_{1}\left( x\right) K\left(
x\right) \right) $在这里取1,但它的%
次数是在模$x^{4}$的意义%
下讲的.

\end{document}
