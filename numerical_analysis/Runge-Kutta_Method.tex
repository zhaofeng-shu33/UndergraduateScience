
\documentclass{article}
\usepackage{amsmath}

%%%%%%%%%%%%%%%%%%%%%%%%%%%%%%%%%%%%%%%%%%%%%%%%%%%%%%%%%%%%%%%%%%%%%%%%%%%%%%%%%%%%%%%%%%%%%%%%%%%%%%%%%%%%%%%%%%%%%%%%%%%%%%%%%%%%%%%%%%%%%%%%%%%%%%%%%%%%%%%%%%%%%%%%%%%%%%%%%%%%%%%%%%%%%%%%%%%%%%%%%%%%%%%%%%%%%%%%%%%%%%%%%%%%
%TCIDATA{OutputFilter=LATEX.DLL}
%TCIDATA{Version=5.00.0.2552}
%TCIDATA{<META NAME="SaveForMode" CONTENT="1">}
%TCIDATA{Created=Wednesday, December 30, 2015 17:20:59}
%TCIDATA{LastRevised=Wednesday, December 30, 2015 23:29:55}
%TCIDATA{<META NAME="GraphicsSave" CONTENT="32">}
%TCIDATA{<META NAME="DocumentShell" CONTENT="Standard LaTeX\Blank - Standard LaTeX Article">}
%TCIDATA{CSTFile=40 LaTeX article.cst}

\newtheorem{theorem}{Theorem}
\newtheorem{acknowledgement}[theorem]{Acknowledgement}
\newtheorem{algorithm}[theorem]{Algorithm}
\newtheorem{axiom}[theorem]{Axiom}
\newtheorem{case}[theorem]{Case}
\newtheorem{claim}[theorem]{Claim}
\newtheorem{conclusion}[theorem]{Conclusion}
\newtheorem{condition}[theorem]{Condition}
\newtheorem{conjecture}[theorem]{Conjecture}
\newtheorem{corollary}[theorem]{Corollary}
\newtheorem{criterion}[theorem]{Criterion}
\newtheorem{definition}[theorem]{Definition}
\newtheorem{example}[theorem]{Example}
\newtheorem{exercise}[theorem]{Exercise}
\newtheorem{lemma}[theorem]{Lemma}
\newtheorem{notation}[theorem]{Notation}
\newtheorem{problem}[theorem]{Problem}
\newtheorem{proposition}[theorem]{Proposition}
\newtheorem{remark}[theorem]{Remark}
\newtheorem{solution}[theorem]{Solution}
\newtheorem{summary}[theorem]{Summary}
\newenvironment{proof}[1][Proof]{\noindent\textbf{#1.} }{\ \rule{0.5em}{0.5em}}
\def\TEXTsymbol#1{\mbox{$#1$}}%
\def\NEG#1{\leavevmode\hbox{\rlap{\thinspace/}{$#1$}}}%
\def\QATOPD#1#2#3#4{{#3 \atopwithdelims#1#2 #4}}%
\def\QTP#1{}
\def\func#1{\mathop{\rm #1}}%

\begin{document}


隐式一级二阶Runge-Kutta方法%
推导:

设线性迭代公式有形%
式$y_{n+1}=y_{n}+bhk,$

其中$k=f\left( t_{n}+ch,y_{n}+akh\right) ,$

局部截断误差为$L\left( y\left(
t_{n}\right) ,h\right) =y\left( t_{n+1}\right) -y\left( t_{n}\right)
-bhf\left( t_{n}+ch,y_{n}+akh\right) $

在$t=t_{n}$处做Taylor展开得$L\left(
y\left( t_{n}\right) ,h\right) =y^{\prime }\left( t_{n}\right) h+\frac{%
y^{\prime \prime }\left( t_{n}\right) }{2}h^{2}+O\left( h^{3}\right)
-bh\left( y^{\prime }\left( t_{n}\right) +\frac{\partial f}{\partial t}ch+%
\frac{\partial f}{\partial y}afh+O\left( h^{2}\right) \right) $

考虑到$y^{\prime \prime }\left( t_{n}\right) =\frac{%
\partial f}{\partial t}+\frac{\partial f}{\partial y}y^{\prime }\left(
t_{n}\right) $

$\implies L\left( y\left( t_{n}\right) ,h\right) =y^{\prime }\left(
t_{n}\right) h+\frac{y^{\prime \prime }\left( t_{n}\right) }{2}%
h^{2}-bh\left( y^{\prime }\left( t_{n}\right) +\frac{\partial f}{\partial t}%
ch+\left( y^{\prime \prime }\left( t_{n}\right) -\frac{\partial f}{\partial t%
}\right) ah\right) +O\left( h^{3}\right) $

$=\left( 1-b\right) y^{\prime }\left( t_{n}\right) h+\left( \left( \frac{1}{2%
}-ab\right) y^{\prime \prime }\left( t_{n}\right) -b\frac{\partial f}{%
\partial t}\left( c-a\right) \right) h^{2}+O\left( h^{3}\right) $

若要求精度达3阶则有$%
b=1,a=\frac{1}{2},c=\frac{1}{2}$

\bigskip 

隐式二级二阶Runge-Kutta方法%
推导:

设线性迭代公式有形%
式$y_{n+1}=y_{n}+h\left( b_{1}k_{1}+b_{2}k_{2}\right) ,$

其中$k_{1}=f\left( t_{n}+\left( a_{1}+a_{2}\right)
h,y_{n}+\left( a_{1}k_{1}+a_{2}k_{2}\right) h\right) ,$

$k_{2}=f\left( t_{n}+\left( a_{3}+a_{4}\right) h,y_{n}+\left(
a_{3}k_{1}+a_{4}k_{2}\right) h\right) ,$

局部截断误差为$L\left( y\left(
t_{n}\right) ,h\right) =$

$y\left( t_{n+1}\right) -y\left( t_{n}\right) -b_{1}hf\left( t_{n}+\left(
a_{1}+a_{2}\right) h,y_{n}+\left( a_{1}k_{1}+a_{2}k_{2}\right) h\right) $

$-b_{2}hf\left( t_{n}+\left( a_{3}+a_{4}\right) h,y_{n}+\left(
a_{3}k_{1}+a_{4}k_{2}\right) h\right) $

在$t=t_{n}$处做Taylor展开得$L\left(
y\left( t_{n}\right) ,h\right) =y^{\prime }\left( t_{n}\right) h+\frac{%
y^{\prime \prime }\left( t_{n}\right) }{2}h^{2}+O\left( h^{3}\right) $

$-b_{1}h\left( y^{\prime }\left( t_{n}\right) +\frac{\partial f}{\partial t}%
\left( a_{1}+a_{2}\right) h+\frac{\partial f}{\partial y}\left(
a_{1}k_{1}+a_{2}k_{2}\right) h+O\left( h^{2}\right) \right) $

\bigskip $-b_{2}h\left( y^{\prime }\left( t_{n}\right) +\frac{\partial f}{%
\partial t}\left( a_{3}+a_{4}\right) h+\frac{\partial f}{\partial y}\left(
a_{3}k_{1}+a_{4}k_{2}\right) h+O\left( h^{2}\right) \right) $

\bigskip 若只做2阶近似,$k_{1},k_{2}$%
均可用$y^{\prime }\left( t_{n}\right) +O\left( h\right) $%
来近代替

$L\left( y\left( t_{n}\right) ,h\right) =\left( 1-b_{1}-b_{2}\right)
y^{\prime }\left( t_{n}\right) h+\frac{y^{\prime \prime }\left( t_{n}\right) 
}{2}h^{2}+O\left( h^{3}\right) $

$-b_{1}h\left( \frac{\partial f}{\partial t}\left( a_{1}+a_{2}\right) h+%
\frac{\partial f}{\partial y}y^{\prime }\left( t_{n}\right) \left(
a_{1}+a_{2}\right) h\right) $

\bigskip $-b_{2}h\left( \frac{\partial f}{\partial t}\left(
a_{3}+a_{4}\right) h+\frac{\partial f}{\partial y}y^{\prime }\left(
t_{n}\right) \left( a_{3}+a_{4}\right) h\right) $

\bigskip 展开到$L\left( y\left( t_{n}\right) ,h\right)
=O\left( h^{5}\right) ,$可得到2级4阶的%
Runge Kutta 隐式格式

%FRAME

考虑到$y^{\prime \prime }\left( t_{n}\right) =\frac{%
\partial f}{\partial t}+\frac{\partial f}{\partial y}y^{\prime }\left(
t_{n}\right) $

$L\left( y\left( t_{n}\right) ,h\right) =\left( 1-b_{1}-b_{2}\right)
y^{\prime }\left( t_{n}\right) h+\left( \frac{1}{2}-b_{1}\left(
a_{1}+a_{2}\right) -b_{2}\left( a_{3}+a_{4}\right) \right) y^{\prime \prime
}\left( t_{n}\right) h^{2}+O\left( h^{3}\right) $

$\implies L\left( y\left( t_{n}\right) ,h\right) =y^{\prime }\left(
t_{n}\right) h+\frac{y^{\prime \prime }\left( t_{n}\right) }{2}%
h^{2}-bh\left( y^{\prime }\left( t_{n}\right) +\frac{\partial f}{\partial t}%
ch+\left( y^{\prime \prime }\left( t_{n}\right) -\frac{\partial f}{\partial t%
}\right) ah\right) +O\left( h^{3}\right) $

$=\left( 1-b\right) y^{\prime }\left( t_{n}\right) h+\left( \left( \frac{1}{2%
}-ab\right) y^{\prime \prime }\left( t_{n}\right) -b\frac{\partial f}{%
\partial t}\left( c-a\right) \right) h^{2}+O\left( h^{3}\right) $

若要求精度达3阶则有$%
b=1,a=\frac{1}{2},c=\frac{1}{2}$

\end{document}
