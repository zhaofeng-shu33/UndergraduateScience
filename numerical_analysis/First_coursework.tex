
\documentclass{article}
%%%%%%%%%%%%%%%%%%%%%%%%%%%%%%%%%%%%%%%%%%%%%%%%%%%%%%%%%%%%%%%%%%%%%%%%%%%%%%%%%%%%%%%%%%%%%%%%%%%%%%%%%%%%%%%%%%%%%%%%%%%%%%%%%%%%%%%%%%%%%%%%%%%%%%%%%%%%%%%%%%%%%%%%%%%%%%%%%%%%%%%%%%%%%%%%%%%%%%%%%%%%%%%%%%%%%%%%%%%%%%%%%%%%%%%%%%%%%%%%%%%%%%%%%%%%
%TCIDATA{OutputFilter=LATEX.DLL}
%TCIDATA{Version=5.00.0.2552}
%TCIDATA{<META NAME="SaveForMode" CONTENT="1">}
%TCIDATA{Created=Monday, September 14, 2015 18:11:48}
%TCIDATA{LastRevised=Thursday, September 17, 2015 09:11:38}
%TCIDATA{<META NAME="GraphicsSave" CONTENT="32">}
%TCIDATA{<META NAME="DocumentShell" CONTENT="Standard LaTeX\Blank - Standard LaTeX Article">}
%TCIDATA{Language=American English}
%TCIDATA{CSTFile=40 LaTeX article.cst}

\newtheorem{theorem}{Theorem}
\newtheorem{acknowledgement}[theorem]{Acknowledgement}
\newtheorem{algorithm}[theorem]{Algorithm}
\newtheorem{axiom}[theorem]{Axiom}
\newtheorem{case}[theorem]{Case}
\newtheorem{claim}[theorem]{Claim}
\newtheorem{conclusion}[theorem]{Conclusion}
\newtheorem{condition}[theorem]{Condition}
\newtheorem{conjecture}[theorem]{Conjecture}
\newtheorem{corollary}[theorem]{Corollary}
\newtheorem{criterion}[theorem]{Criterion}
\newtheorem{definition}[theorem]{Definition}
\newtheorem{example}[theorem]{Example}
\newtheorem{exercise}[theorem]{Exercise}
\newtheorem{lemma}[theorem]{Lemma}
\newtheorem{notation}[theorem]{Notation}
\newtheorem{problem}[theorem]{Problem}
\newtheorem{proposition}[theorem]{Proposition}
\newtheorem{remark}[theorem]{Remark}
\newtheorem{solution}[theorem]{Solution}
\newtheorem{summary}[theorem]{Summary}
\newenvironment{proof}[1][Proof]{\noindent\textbf{#1.} }{\ \rule{0.5em}{0.5em}}
\def\TEXTsymbol#1{\mbox{$#1$}}%
\def\NEG#1{\leavevmode\hbox{\rlap{\thinspace/}{$#1$}}}%
\def\QATOPD#1#2#3#4{{#3 \atopwithdelims#1#2 #4}}%
\def\QTP#1{}
\def\func#1{\mathop{\rm #1}}%

\begin{document}


\bigskip First Coursework of Numerical Analysis

(This is revised form, and should be treated as valid while the original one
invalid.)

\qquad \qquad \qquad \qquad \qquad \qquad \qquad \qquad \qquad \qquad \qquad
\qquad \qquad \qquad \qquad \qquad \qquad \qquad \qquad \qquad \qquad \qquad
\qquad \qquad \qquad \qquad \qquad \qquad \qquad \qquad \qquad \qquad \qquad
\qquad \qquad \qquad \qquad \qquad \qquad \qquad \qquad \qquad 赵%
丰 2013012178

%FRAME

Solution: Design an Algorithm to calculate the signifant figure of any give
pairs (x,x*), where x represents the 

true figure and x* represents the approximating figure.

Algorithm brief description: For the argument x*, returns its power index k
such that 

$x^{\ast }=\pm 10^{k}\times 0.d_{1}d_{2}...d_{i}...,d_{i}\in \{$integer 0%
\symbol{126}9\} and d$_{1}\neq 0.$Then f(x,x*) returns the relative error
and the significant figure.  the relative error can be easily calculated
from formular $|\frac{x-x\text{*}}{x}|$, while the significant figure can be
calculated using simple cycle program. From the definition of significant
figure, we assume at first this figure i = 0; While \TEXTsymbol{\vert}x - x1%
\TEXTsymbol{\vert} $\leq $ 1/2$\ast $10\symbol{94}(k - i), we let i=i+1;
This cycle continues until the cyclic condition is not satisified, then we
get the largest nonnegative integer such that \TEXTsymbol{\vert}x - x1%
\TEXTsymbol{\vert} $\leq $ 1/2$\ast $10\symbol{94}(t - i) holds. The
following results came from Mathematica (the code is omitted), where the
relative error is set precision to 3 digits:

%FRAME

%FRAMESolution (1) \ Let A=$\arctan
(N+1)-\arctan (N),$then $\tan A=\frac{1}{1+N(N+1)}\rightarrow A=\arctan (%
\frac{1}{1+N(N+1)}),$

\qquad \qquad \qquad then use the power series expansion of arctanx at 0,

\qquad \qquad \qquad $\arctan (N+1)-\arctan (N)\approx \frac{1}{1+N(N+1)}-%
\frac{1}{3}\frac{1}{(1+N(N+1))^{3}}+...$

\qquad \qquad\ (2) take the reciprocal,i.e. calculate $\frac{2}{x(\sqrt{x%
\text{-}\frac{1}{x}}+\sqrt{x+\frac{1}{x}})}$

\ \ \ \ (3) $\ln (x+1)-\ln (x)=\ln (1+\frac{1}{x})\approx \frac{1}{x}-\frac{1%
}{2}\frac{1}{x^{2}}+...$

\ \ \ \ (4) $\cos ^{2}x-\sin ^{2}x=\cos 2x$

\end{document}
