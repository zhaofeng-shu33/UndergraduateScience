\documentclass{ctexart}
\begin{document}
    In the third edition of 数值分析基础:
正文
\section{浮点数标准}
    P6页左右介绍的浮点数表示不符合IEEE754标准,
    如果采用 $\pm 1.d_1\dots d_t \times 2^J$的话
    第11行中应改为 $m=2^L, M=2^U(2-2^{-t})$,
    否则极易引起读者的误解。
\section{选主元的高斯消去法的数值稳定性}
更常用的一个名称是 LUP分解,其中 $P$ 是 pivoting 的首字母,
英文全称为 LU decomposition with pivoting。
定理为可逆矩阵存在 LUP分解。
P66页定理4.4的Comment中应该是(2)式从(1)式和定理4.1中推出来的。
误写成了定理4.3。
\section{SSOR方法}
P88 第9行,应为$(D-\omega L)D^{-1}(D-\omega U)$
\end{document}