
\documentclass{article}
%%%%%%%%%%%%%%%%%%%%%%%%%%%%%%%%%%%%%%%%%%%%%%%%%%%%%%%%%%%%%%%%%%%%%%%%%%%%%%%%%%%%%%%%%%%%%%%%%%%%%%%%%%%%%%%%%%%%%%%%%%%%%%%%%%%%%%%%%%%%%%%%%%%%%%%%%%%%%%%%%%%%%%%%%%%%%%%%%%%%%%%%%%%%%%%%%%%%%%%%%%%%%%%%%%%%%%%%%%%%%%%%%%%%%%%%%%%%%%%%%%%%%%%%%%%%
%TCIDATA{OutputFilter=LATEX.DLL}
%TCIDATA{Version=5.00.0.2552}
%TCIDATA{<META NAME="SaveForMode" CONTENT="1">}
%TCIDATA{Created=Thursday, October 01, 2015 13:14:29}
%TCIDATA{LastRevised=Thursday, October 01, 2015 13:23:51}
%TCIDATA{<META NAME="GraphicsSave" CONTENT="32">}
%TCIDATA{<META NAME="DocumentShell" CONTENT="Standard LaTeX\Blank - Standard LaTeX Article">}
%TCIDATA{CSTFile=40 LaTeX article.cst}
\usepackage{ctex}
\usepackage{amsmath}
\newtheorem{theorem}{Theorem}
\newtheorem{acknowledgement}[theorem]{Acknowledgement}
\newtheorem{algorithm}[theorem]{Algorithm}
\newtheorem{axiom}[theorem]{Axiom}
\newtheorem{case}[theorem]{Case}
\newtheorem{claim}[theorem]{Claim}
\newtheorem{conclusion}[theorem]{Conclusion}
\newtheorem{condition}[theorem]{Condition}
\newtheorem{conjecture}[theorem]{Conjecture}
\newtheorem{corollary}[theorem]{Corollary}
\newtheorem{criterion}[theorem]{Criterion}
\newtheorem{definition}[theorem]{Definition}
\newtheorem{example}[theorem]{Example}
\newtheorem{exercise}[theorem]{Exercise}
\newtheorem{lemma}[theorem]{Lemma}
\newtheorem{notation}[theorem]{Notation}
\newtheorem{problem}[theorem]{Problem}
\newtheorem{proposition}[theorem]{Proposition}
\newtheorem{remark}[theorem]{Remark}
\newtheorem{solution}[theorem]{Solution}
\newtheorem{summary}[theorem]{Summary}
\newenvironment{proof}[1][Proof]{\noindent\textbf{#1.} }{\ \rule{0.5em}{0.5em}}


\begin{document}



\section{Inequality about the condition number}
Proof of one conclusion in error domination
Proof of Thm 4.3 left hand side in textbook P63

$Ax=b,A$ is a $n\times n$ real matrix, $b$ is a n-dimensional column vector,
x is the solution of this linear

equation. Suppose $\widetilde{x}$ is the approximate solution to the
equation, corresponding to the remainder 

$r=b-A\widetilde{x}.$Then 

$\frac{1}{cond(A)}\frac{\left\Vert r\right\Vert }{\left\Vert b\right\Vert }%
\leq \frac{\left\Vert \widetilde{x}-x\right\Vert }{\left\Vert x\right\Vert }.
$

Proof: $r=A\left( x-\widetilde{x}\right) .$

$\frac{1}{cond(A)}\frac{\left\Vert r\right\Vert }{\left\Vert b\right\Vert }=%
\frac{1}{\left\Vert A\right\Vert \left\Vert A^{-1}\right\Vert }\frac{%
\left\Vert A\left( x-\widetilde{x}\right) \right\Vert }{\left\Vert
b\right\Vert }\leq \frac{1}{\left\Vert A\right\Vert \left\Vert
A^{-1}\right\Vert }\frac{\left\Vert A\right\Vert \left\Vert \left( x-%
\widetilde{x}\right) \right\Vert }{\left\Vert b\right\Vert }=\frac{%
\left\Vert \left( x-\widetilde{x}\right) \right\Vert }{\left\Vert
A^{-1}\right\Vert \left\Vert b\right\Vert }\leq \frac{\left\Vert \left( x-%
\widetilde{x}\right) \right\Vert }{\left\Vert A^{-1}b\right\Vert }=\frac{%
\left\Vert \widetilde{x}-x\right\Vert }{\left\Vert x\right\Vert }$

\section{主对角占优矩阵的Jacobi迭代收敛性证明}
课本81页,定理2.1
\begin{proof}
因为 $A$ 是对角占优矩阵,
$B_J=I-D^{-1}A$。设$B_J=\{b_{ij}\}$。则
对角元$b_{ii}=0$,非对角元满足
$\sum_{j=1}^{n} |b_{ij}| < 1$。
设 $\lambda, x$ 分别是$B_J$的特征值和特征向量,且
$||x||=1$。
由 $B_J x = \lambda x$ 可得
\begin{equation*}
    |\lambda x_i| = |\sum_{j=1}^n b_j x_j| 
    \leq \sum_{j=1}^{n} |b_{ij}| < 1, \textrm{ for } i=1,\dots, n
\end{equation*}
因此, $|\lambda|<1 \Rightarrow \rho(B_J)<1$。
由课本定理1.4可知Jacobi迭代法收敛到真值。
\end{proof}
\section{Steffensen 加速收敛}
此节作为课本112页定理3.3的一个注解。
$\varphi(x) = (x^2-3)/2$ 的不动点迭代用于求解
$x^2-2x-3=0$。但这个格式不收敛(初始值$x_0=4$),
使用
Steffensen 格式构造
\begin{equation}
    \psi(x) =     \frac{x\varphi(\varphi(x))
    - [\varphi(x)]^2}
    {\varphi(\varphi(x)) - 2\varphi(x) + x}
\end{equation}
后再使用不动点迭代即收敛到二次方程一个解$x=3$(迭代3次的结果保留4位有效数字
是 3.006)。
\end{document}
