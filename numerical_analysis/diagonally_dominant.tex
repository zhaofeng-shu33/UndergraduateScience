
\documentclass{article}
%%%%%%%%%%%%%%%%%%%%%%%%%%%%%%%%%%%%%%%%%%%%%%%%%%%%%%%%%%%%%%%%%%%%%%%%%%%%%%%%%%%%%%%%%%%%%%%%%%%%%%%%%%%%%%%%%%%%%%%%%%%%%%%%%%%%%%%%%%%%%%%%%%%%%%%%%%%%%%%%%%%%%%%%%%%%%%%%%%%%%%%%%%%%%%%%%%%%%%%%%%%%%%%%%%%%%%%%%%%%%%%%%%%%%%%%%%%%%%%%%%%%%%%%%%%%
\usepackage{amssymb}
\usepackage{amsmath}

\setcounter{MaxMatrixCols}{10}
%TCIDATA{OutputFilter=LATEX.DLL}
%TCIDATA{Version=5.00.0.2552}
%TCIDATA{<META NAME="SaveForMode" CONTENT="1">}
%TCIDATA{Created=Wednesday, September 16, 2015 15:35:16}
%TCIDATA{LastRevised=Wednesday, September 16, 2015 23:56:57}
%TCIDATA{<META NAME="GraphicsSave" CONTENT="32">}
%TCIDATA{<META NAME="DocumentShell" CONTENT="Standard LaTeX\Blank - Standard LaTeX Article">}
%TCIDATA{CSTFile=40 LaTeX article.cst}

\newtheorem{theorem}{Theorem}
\newtheorem{acknowledgement}[theorem]{Acknowledgement}
\newtheorem{algorithm}[theorem]{Algorithm}
\newtheorem{axiom}[theorem]{Axiom}
\newtheorem{case}[theorem]{Case}
\newtheorem{claim}[theorem]{Claim}
\newtheorem{conclusion}[theorem]{Conclusion}
\newtheorem{condition}[theorem]{Condition}
\newtheorem{conjecture}[theorem]{Conjecture}
\newtheorem{corollary}[theorem]{Corollary}
\newtheorem{criterion}[theorem]{Criterion}
\newtheorem{definition}[theorem]{Definition}
\newtheorem{example}[theorem]{Example}
\newtheorem{exercise}[theorem]{Exercise}
\newtheorem{lemma}[theorem]{Lemma}
\newtheorem{notation}[theorem]{Notation}
\newtheorem{problem}[theorem]{Problem}
\newtheorem{proposition}[theorem]{Proposition}
\newtheorem{remark}[theorem]{Remark}
\newtheorem{solution}[theorem]{Solution}
\newtheorem{summary}[theorem]{Summary}
\newenvironment{proof}[1][Proof]{\noindent\textbf{#1.} }{\ \rule{0.5em}{0.5em}}
\def\TEXTsymbol#1{\mbox{$#1$}}%
\def\NEG#1{\leavevmode\hbox{\rlap{\thinspace/}{$#1$}}}%
\def\QATOPD#1#2#3#4{{#3 \atopwithdelims#1#2 #4}}%
\def\QTP#1{}
\def\func#1{\mathop{\rm #1}}%

\begin{document}


$%FRAME$

Proof:

(b) Consider $A-\lambda I,$ if $\func{Re}(\lambda )\leq 0,$and $A-\lambda I$
is still a strictly diagonally

dominant matrix, which is nonsingular from (a), but contradicts with the fact

that there exists an eigenvector $v$ such that $\left( A-\lambda I\right)
v=0.$Therefore,

$\func{Re}(\lambda )\geq 0.$

Problem: (1)p-norm in $R^{n}$ means $||$x$||_{p}=\left( \overset{n}{\underset%
{i=1}{\sum }}\left\vert x_{i}\right\vert ^{p}\right) ^{\frac{1}{p}}, $show
that $||x||_{p}\leq ||x||_{q}$ if $p>q.$

Brief proof: Let $x=(x_{1},x_{2}...x_{n})^{T},$then $\frac{\left\vert
x_{i}\right\vert }{\left\vert \left\vert x\right\vert \right\vert _{p}}\leq
1,$therefore $\left( \frac{\left\vert x_{i}\right\vert }{\left\vert
\left\vert x\right\vert \right\vert _{p}}\right) ^{p}\leq \left( \frac{%
\left\vert x_{i}\right\vert }{\left\vert \left\vert x\right\vert \right\vert
_{p}}\right) ^{q}.$

Add the equations from $i=1$ to $i=n,$ we can get $1<\frac{\left\vert
x_{i}\right\vert ^{q}}{\left( \overset{n}{\underset{i=1}{\sum }}\left\vert
x_{i}\right\vert ^{p}\right) ^{\frac{q}{p}}},$which simplied to

$||x||_{p}\leq ||x||_{q}.$

(2)Show that $\underset{p\rightarrow \infty }{\lim }$ $||$x$||_{p}=\underset{%
1\leq i\leq n}{\max }\left\vert x_{i}\right\vert ,$which is the definition
of $\infty -$norm on $R_{n}:\left\vert \left\vert \cdot \right\vert
\right\vert _{\infty }$.

Brief proof: from the inequality $\underset{1\leq i\leq n}{\max }\left\vert
x_{i}\right\vert \leq \left( \overset{n}{\underset{i=1}{\sum }}\left\vert
x_{i}\right\vert ^{p}\right) ^{\frac{1}{p}}\leq \underset{1\leq i\leq n}{n^{%
\frac{1}{p}}\max }\left\vert x_{i}\right\vert ,$let $p\rightarrow \infty $
and then the

conclusion follows.

(3)Show that all the norms in $R^{n}$ are equivalent, which means there
exists $C_{1},C_{2}>0$ such that

$C_{1}\left\vert \left\vert x\right\vert \right\vert _{\alpha }\leq
||x||_{\beta }\leq C_{2}\left\vert \left\vert x\right\vert \right\vert
_{\alpha }$ holds for any two norms $\left\vert \left\vert \cdot \right\vert
\right\vert _{\alpha },\left\vert \left\vert \cdot \right\vert \right\vert
_{\beta }.$

Complete Proof: By the transmitting property of the norm, we need only show
the case for $\alpha =\infty .$

Let $\{e_{1,}e_{2}...e_{n}\}$ be a base of $R^{n}:,$and $\left\vert
\left\vert e_{i}\right\vert \right\vert _{a}$ represents the value of $e_{i}$
under the arbitrary

norm $\left\vert \left\vert \cdot \right\vert \right\vert _{a}.$For any $%
x\in R^{n},x=\overset{n}{\underset{i=1}{\sum }}x_{i}e_{i},\left\vert
\left\vert x\right\vert \right\vert _{a}=\left\vert \left\vert \overset{n}{%
\underset{i=1}{\sum }}x_{i}e_{i}\right\vert \right\vert _{a}\leq \overset{n}{%
\underset{i=1}{\sum }}\left\vert \left\vert e_{i}\right\vert \right\vert
_{a}\left\vert x_{i}\right\vert \leq \left\vert \left\vert x\right\vert
\right\vert _{\infty }\overset{n}{\underset{i=1}{\sum }}\left\vert
\left\vert e_{i}\right\vert \right\vert _{a},$ which shows

that we can let $C_{2}=\left( \overset{n}{\underset{i=1}{\sum }}\left\vert
\left\vert e_{i}\right\vert \right\vert _{a}\right) ^{-1}$and we have shown
that $||x||_{a}\leq C_{2}\left\vert \left\vert x\right\vert \right\vert
_{\infty }.$

To show there exists $C_{1}$ such that $C_{1}\left\vert \left\vert
x\right\vert \right\vert _{\infty }\leq ||x||_{a}$ holds for any $x\in
R^{n}. $ Firstly we notice that

the continuous function$\left\Vert \cdot \right\Vert _{a}$ has positive
minimum on the closed subset $U$ of $R^{n}:U=\{\left\Vert y\right\Vert
_{\infty }=1\},$

since every element of $U$ is nonzero, the minimum can not be zero by the
definity property of any norm.

then we notice that $\frac{x}{\left\Vert x\right\Vert _{\infty }}\in U,$ it
follows that the minimum $C_{1}\leq \frac{\left\Vert x\right\Vert _{a}}{%
\left\Vert x\right\Vert _{\infty }}$ by the property of absolutely \qquad
\qquad\ \ 

homogeneity of the norm. Hence the proof of the equivalence of any norm on $%
R^{n}$ is complete. $\boxtimes $

(3) Show that$\left\langle R^{n},\left\Vert \cdot \right\Vert
_{p}\right\rangle $ is complete space, where 1$\leq p\leq \infty $.

From the equivalence of norms on $R^{n}$(2) we only need to show the results
for one p, for example p=2,i.e. 

Euclid norm. SInce $\left\vert x_{i}\right\vert \leq ||x||_{2}$ holds for  1$%
\leq i\leq n,$from the Cauthy sequence$\left\{ x^{k}=\left(
x_{1}^{k},x_{2}^{k},...x_{n}^{k}\right) \right\} _{k=1}^{\infty }$ in $R^{n}$
it follows that $\{x_{i}^{k}\}_{k=1}^{\infty }$ is a Cauthy sequence in $%
R^{1}$ for $1\leq i\leq n,$ and by the completeness of $R^{1}$ it follows
that there exists $x_{i}$ such that $x_{i}^{k}\rightarrow x^{k}$ as $%
i\rightarrow \infty ,$and from $||x^{k}-x||_{2}\leq \overset{n}{\underset{i=1%
}{\sum }}\left\vert \left\vert x_{i}^{k}-x_{i}\right\vert \right\vert ,$%
where $x=\left( x_{1},x_{2},...x_{n}\right) $

it follows that $x^{k}\rightarrow x$ according to the norm $\left\Vert \cdot
\right\Vert _{2}.$

(4) Show that $\left\langle C_{[a,b]},\left\Vert \cdot \right\Vert _{\max
}\right\rangle $ is complete (i.e. Banach space), where $C_{[a,b]}$
represents all the continuous functions on the interval $[a,b].$

Refer to the proof of "dual space of normed space is complete" in file
"Functional Analysis", or 

Vladimir A. Zorich's "Mathematical Analysis":Volume Two,Chapter 9,section5
"the complete metric space".

\end{document}
