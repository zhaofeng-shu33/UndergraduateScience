
\documentclass{article}
%%%%%%%%%%%%%%%%%%%%%%%%%%%%%%%%%%%%%%%%%%%%%%%%%%%%%%%%%%%%%%%%%%%%%%%%%%%%%%%%%%%%%%%%%%%%%%%%%%%%%%%%%%%%%%%%%%%%%%%%%%%%%%%%%%%%%%%%%%%%%%%%%%%%%%%%%%%%%%%%%%%%%%%%%%%%%%%%%%%%%%%%%%%%%%%%%%%%%%%%%%%%%%%%%%%%%%%%%%%%%%%%%%%%%%%%%%%%%%%%%%%%%%%%%%%%
\usepackage{amssymb}
\usepackage{amsmath}

\setcounter{MaxMatrixCols}{10}
%TCIDATA{OutputFilter=LATEX.DLL}
%TCIDATA{Version=5.00.0.2552}
%TCIDATA{<META NAME="SaveForMode" CONTENT="1">}
%TCIDATA{Created=Sunday, September 27, 2015 19:40:53}
%TCIDATA{LastRevised=Saturday, October 10, 2015 14:18:31}
%TCIDATA{<META NAME="GraphicsSave" CONTENT="32">}
%TCIDATA{<META NAME="DocumentShell" CONTENT="Standard LaTeX\Blank - Standard LaTeX Article">}
%TCIDATA{CSTFile=40 LaTeX article.cst}
%TCIDATA{ComputeDefs=
%$[A$
%}


\newtheorem{theorem}{Theorem}
\newtheorem{acknowledgement}[theorem]{Acknowledgement}
\newtheorem{algorithm}[theorem]{Algorithm}
\newtheorem{axiom}[theorem]{Axiom}
\newtheorem{case}[theorem]{Case}
\newtheorem{claim}[theorem]{Claim}
\newtheorem{conclusion}[theorem]{Conclusion}
\newtheorem{condition}[theorem]{Condition}
\newtheorem{conjecture}[theorem]{Conjecture}
\newtheorem{corollary}[theorem]{Corollary}
\newtheorem{criterion}[theorem]{Criterion}
\newtheorem{definition}[theorem]{Definition}
\newtheorem{example}[theorem]{Example}
\newtheorem{exercise}[theorem]{Exercise}
\newtheorem{lemma}[theorem]{Lemma}
\newtheorem{notation}[theorem]{Notation}
\newtheorem{problem}[theorem]{Problem}
\newtheorem{proposition}[theorem]{Proposition}
\newtheorem{remark}[theorem]{Remark}
\newtheorem{solution}[theorem]{Solution}
\newtheorem{summary}[theorem]{Summary}
\newenvironment{proof}[1][Proof]{\noindent\textbf{#1.} }{\ \rule{0.5em}{0.5em}}
\def\TEXTsymbol#1{\mbox{$#1$}}%
\def\NEG#1{\leavevmode\hbox{\rlap{\thinspace/}{$#1$}}}%
\def\QATOPD#1#2#3#4{{#3 \atopwithdelims#1#2 #4}}%
\def\QTP#1{}
\def\func#1{\mathop{\rm #1}}%

\begin{document}


\bigskip Numurical Analysis 赵丰 2013012178

Problem 1:

The initial adjoint matrix takes the form:

2\qquad 1\qquad -3\qquad -1\qquad \TEXTsymbol{\vert}\qquad 1\qquad 0\qquad
0\qquad 0\qquad

3\qquad 1\qquad 0\qquad 7\qquad \TEXTsymbol{\vert}\qquad 0\qquad 1\qquad
0\qquad 0\qquad

-1\qquad 2\qquad 4\qquad -2\qquad \TEXTsymbol{\vert}\qquad 0\qquad 0\qquad
1\qquad 0\qquad

1\qquad 0\qquad -1\qquad 5\qquad \TEXTsymbol{\vert}\qquad 0\qquad 0\qquad
0\qquad 1\qquad

Exchanging the 2 row with the 1 row gives:

3\qquad 1\qquad 0\qquad 7\qquad \TEXTsymbol{\vert}\qquad 0\qquad 1\qquad
0\qquad 0\qquad

2\qquad 1\qquad -3\qquad -1\qquad \TEXTsymbol{\vert}\qquad 1\qquad 0\qquad
0\qquad 0\qquad

-1\qquad 2\qquad 4\qquad -2\qquad \TEXTsymbol{\vert}\qquad 0\qquad 0\qquad
1\qquad 0\qquad

1\qquad 0\qquad -1\qquad 5\qquad \TEXTsymbol{\vert}\qquad 0\qquad 0\qquad
0\qquad 1\qquad

Normalization of the 1 row gives:

1\qquad 0.33\qquad 0\qquad 2.3\qquad \TEXTsymbol{\vert}\qquad 0\qquad
0.33\qquad 0\qquad 0\qquad

2\qquad 1\qquad -3\qquad -1\qquad \TEXTsymbol{\vert}\qquad 1\qquad 0\qquad
0\qquad 0\qquad

-1\qquad 2\qquad 4\qquad -2\qquad \TEXTsymbol{\vert}\qquad 0\qquad 0\qquad
1\qquad 0\qquad

1\qquad 0\qquad -1\qquad 5\qquad \TEXTsymbol{\vert}\qquad 0\qquad 0\qquad
0\qquad 1\qquad

Elimination of the elements below the (1,1) position gives:

1\qquad 0.33\qquad 0\qquad 2.3\qquad \TEXTsymbol{\vert}\qquad 0\qquad
0.33\qquad 0\qquad 0\qquad

0\qquad 0.33\qquad -3\qquad -5.7\qquad \TEXTsymbol{\vert}\qquad 1\qquad
-0.67\qquad 0\qquad 0\qquad

0\qquad 2.3\qquad 4\qquad 0.33\qquad \TEXTsymbol{\vert}\qquad 0\qquad
0.33\qquad 1\qquad 0\qquad

0\qquad -0.33\qquad -1\qquad 2.7\qquad \TEXTsymbol{\vert}\qquad 0\qquad
-0.33\qquad 0\qquad 1\qquad

Exchanging the 3 row with the 2 row gives:

1\qquad 0.33\qquad 0\qquad 2.3\qquad \TEXTsymbol{\vert}\qquad 0\qquad
0.33\qquad 0\qquad 0\qquad

0\qquad 2.3\qquad 4\qquad 0.33\qquad \TEXTsymbol{\vert}\qquad 0\qquad
0.33\qquad 1\qquad 0\qquad

0\qquad 0.33\qquad -3\qquad -5.7\qquad \TEXTsymbol{\vert}\qquad 1\qquad
-0.67\qquad 0\qquad 0\qquad

0\qquad -0.33\qquad -1\qquad 2.7\qquad \TEXTsymbol{\vert}\qquad 0\qquad
-0.33\qquad 0\qquad 1\qquad

Normalization of the 2 row gives:

1\qquad 0.33\qquad 0\qquad 2.3\qquad \TEXTsymbol{\vert}\qquad 0\qquad
0.33\qquad 0\qquad 0\qquad

0\qquad 1\qquad 1.7\qquad 0.14\qquad \TEXTsymbol{\vert}\qquad 0\qquad
0.14\qquad 0.43\qquad 0\qquad

0\qquad 0.33\qquad -3\qquad -5.7\qquad \TEXTsymbol{\vert}\qquad 1\qquad
-0.67\qquad 0\qquad 0\qquad

0\qquad -0.33\qquad -1\qquad 2.7\qquad \TEXTsymbol{\vert}\qquad 0\qquad
-0.33\qquad 0\qquad 1\qquad

Elimination of the elements below the (2,2) position gives:

1\qquad 0.33\qquad 0\qquad 2.3\qquad \TEXTsymbol{\vert}\qquad 0\qquad
0.33\qquad 0\qquad 0\qquad

0\qquad 1\qquad 1.7\qquad 0.14\qquad \TEXTsymbol{\vert}\qquad 0\qquad
0.14\qquad 0.43\qquad 0\qquad

0\qquad 0\qquad -3.6\qquad -5.7\qquad \TEXTsymbol{\vert}\qquad 1\qquad
-0.71\qquad -0.14\qquad 0\qquad

0\qquad 0\qquad -0.43\qquad 2.7\qquad \TEXTsymbol{\vert}\qquad 0\qquad
-0.29\qquad 0.14\qquad 1\qquad

Normalization of the 3 row gives:

1\qquad 0.33\qquad 0\qquad 2.3\qquad \TEXTsymbol{\vert}\qquad 0\qquad
0.33\qquad 0\qquad 0\qquad

0\qquad 1\qquad 1.7\qquad 0.14\qquad \TEXTsymbol{\vert}\qquad 0\qquad
0.14\qquad 0.43\qquad 0\qquad

-0\qquad -0\qquad 1\qquad 1.6\qquad \TEXTsymbol{\vert}\qquad -0.28\qquad
0.2\qquad 0.04\qquad -0\qquad

0\qquad 0\qquad -0.43\qquad 2.7\qquad \TEXTsymbol{\vert}\qquad 0\qquad
-0.29\qquad 0.14\qquad 1\qquad

Elimination of the elements below the (3,3) position gives:

1\qquad 0.33\qquad 0\qquad 2.3\qquad \TEXTsymbol{\vert}\qquad 0\qquad
0.33\qquad 0\qquad 0\qquad

0\qquad 1\qquad 1.7\qquad 0.14\qquad \TEXTsymbol{\vert}\qquad 0\qquad
0.14\qquad 0.43\qquad 0\qquad

-0\qquad -0\qquad 1\qquad 1.6\qquad \TEXTsymbol{\vert}\qquad -0.28\qquad
0.2\qquad 0.04\qquad -0\qquad

0\qquad 0\qquad 0\qquad 3.4\qquad \TEXTsymbol{\vert}\qquad -0.12\qquad
-0.2\qquad 0.16\qquad 1\qquad

Normalization of the final row gives:

1\qquad 0.33\qquad 0\qquad 2.3\qquad \TEXTsymbol{\vert}\qquad 0\qquad
0.33\qquad 0\qquad 0\qquad

0\qquad 1\qquad 1.7\qquad 0.14\qquad \TEXTsymbol{\vert}\qquad 0\qquad
0.14\qquad 0.43\qquad 0\qquad

-0\qquad -0\qquad 1\qquad 1.6\qquad \TEXTsymbol{\vert}\qquad -0.28\qquad
0.2\qquad 0.04\qquad -0\qquad

0\qquad 0\qquad 0\qquad 1\qquad \TEXTsymbol{\vert}\qquad -0.035\qquad
-0.059\qquad 0.047\qquad 0.29\qquad

Elimination of the elements above the (4,4) position gives:

1\qquad 0.33\qquad 0\qquad 0\qquad \TEXTsymbol{\vert}\qquad 0.082\qquad
0.47\qquad -0.11\qquad -0.69\qquad

0\qquad 1\qquad 1.7\qquad 0\qquad \TEXTsymbol{\vert}\qquad 0.005\qquad
0.15\qquad 0.42\qquad -0.042\qquad

-0\qquad -0\qquad 1\qquad 0\qquad \TEXTsymbol{\vert}\qquad -0.22\qquad
0.29\qquad -0.035\qquad -0.47\qquad

0\qquad 0\qquad 0\qquad 1\qquad \TEXTsymbol{\vert}\qquad -0.035\qquad
-0.059\qquad 0.047\qquad 0.29\qquad

Elimination of the elements above the (3,3) position gives:

1\qquad 0.33\qquad 0\qquad 0\qquad \TEXTsymbol{\vert}\qquad 0.082\qquad
0.47\qquad -0.11\qquad -0.69\qquad

0\qquad 1\qquad 0\qquad 0\qquad \TEXTsymbol{\vert}\qquad 0.39\qquad
-0.35\qquad 0.48\qquad -0.042\qquad

-0\qquad -0\qquad 1\qquad 0\qquad \TEXTsymbol{\vert}\qquad -0.22\qquad
0.29\qquad -0.035\qquad -0.47\qquad

0\qquad 0\qquad 0\qquad 1\qquad \TEXTsymbol{\vert}\qquad -0.035\qquad
-0.059\qquad 0.047\qquad 0.29\qquad

Elimination of the elements above the (2,2) position gives:

1\qquad 0\qquad 0\qquad 0\qquad \TEXTsymbol{\vert}\qquad -0.047\qquad
0.59\qquad -0.11\qquad -0.69\qquad

0\qquad 1\qquad 0\qquad 0\qquad \TEXTsymbol{\vert}\qquad 0.39\qquad
-0.35\qquad 0.48\qquad -0.042\qquad

-0\qquad -0\qquad 1\qquad 0\qquad \TEXTsymbol{\vert}\qquad -0.22\qquad
0.29\qquad -0.035\qquad -0.47\qquad

0\qquad 0\qquad 0\qquad 1\qquad \TEXTsymbol{\vert}\qquad -0.035\qquad
-0.059\qquad 0.047\qquad 0.29\qquad

The matrix on the right hand of the vertical line is the required inverse
matrix:

-0.0471\qquad 0.588\qquad -0.11\qquad -0.686\qquad

0.388\qquad -0.353\qquad 0.482\qquad -0.042\qquad

-0.224\qquad 0.294\qquad -0.0353\qquad -0.471\qquad

-0.0353\qquad -0.0588\qquad 0.0471\qquad 0.294\qquad

Problem 2:

$\left( 1\right) $Initial condition: $l_{1}=\sqrt{b_{1}}$and $m_{2}=\frac{%
a_{2}}{l_{1}};$

Recursive formula: $l_{i}=\sqrt{b_{i}-m_{i}^{2}}$ and $m_{i+1}=\frac{a_{i+1}%
}{l_{i}}.$

$\left( 2\right) A=L^{T}L$ and euqation system $Ax=d\iff L\left(
L^{T}x\right) =d.$

We first solve $Ly=d,$then solve $L^{T}x=y.$

For $Ly=d.\qquad \qquad \qquad \qquad \qquad $For $L^{T}x=y$

$\bigskip \QATOPD\{ . {y_{1}=\frac{d_{1}}{l_{1}}}{y_{i}=\frac{%
d_{i}-m_{i}y_{i-1}}{l_{i}},i=2,3...n}\qquad \qquad \qquad \QATOPD\{ . {x_{n}=%
\frac{y_{n}}{l_{n}}}{x_{i}=\frac{y_{i}-m_{i+1}x_{i+1}}{l_{i}},i=n-1,n-2...1}$

%FRAME

Problem3 Proof: Since $\det A=\det A^{T},$we only need to show the result
for $A^{T},$that is

we show that $\left\vert \det A\right\vert ^{2}\leq \underset{i=1}{\overset{n%
}{\Pi }}\underset{j=1}{\overset{n}{\sum }}\left\vert a_{ji}\right\vert ^{2}$

$A=QR,$where Q is an orthogonal matrix and R is an upper right triangular
matrix.

$\left\vert \det A\right\vert ^{2}=\left\vert \det Q\right\vert
^{2}\left\vert \det R\right\vert ^{2}=\left\vert \det R\right\vert ^{2}=%
\underset{i=1}{\overset{n}{\Pi }}\left\vert r_{ii}\right\vert ^{2}.$

Let $A=\left( \vec{a}_{1},\vec{a}_{2}...\vec{a}_{n}\right) ,$where $\vec{a}%
_{i}$ represents the column vector of matrix A.

Since orthogonal transformation does not change the Euclid norm of the
column vector,

$\left\Vert Q^{-1}\vec{a}_{i}\right\Vert =\left\Vert \vec{a}_{i}\right\Vert
\implies \underset{j=1}{\overset{i}{\sum }}\left\vert r_{ji}\right\vert ^{2}=%
\underset{j=1}{\overset{n}{\sum }}\left\vert a_{ji}\right\vert ^{2}$
.Multiplying n equalities for $i=1,2...n$ gives

$\underset{i=1}{\overset{n}{\Pi }}\underset{j=1}{\overset{i}{\sum }}%
\left\vert r_{ji}\right\vert ^{2}$=$\underset{i=1}{\overset{n}{\Pi }}%
\underset{j=1}{\overset{n}{\sum }}\left\vert a_{ji}\right\vert ^{2}$ $%
\implies \underset{i=1}{\overset{n}{\Pi }}\left\vert r_{ii}\right\vert
^{2}\leq \underset{i=1}{\overset{n}{\Pi }}\underset{j=1}{\overset{n}{\sum }}%
\left\vert a_{ji}\right\vert ^{2}.$

That is $\left\vert \det A\right\vert ^{2}\leq \underset{i=1}{\overset{n}{%
\Pi }}\underset{j=1}{\overset{n}{\sum }}\left\vert a_{ji}\right\vert ^{2}$ .

%FRAME

Problem 4

Proof: $\left( 1\right) \left\Vert x\right\Vert _{A}\geq 0$ and $\left\Vert
x\right\Vert _{A}=0\implies x^{T}Ax=0\implies x=0,$ since $A$ is positive
definite.

$\left( 2\right) \left\Vert \lambda x\right\Vert _{A}=<A\lambda x,\lambda
x>^{1/2}=\left\vert \lambda \right\vert <Ax,x>=\left\vert \lambda
\right\vert \left\Vert x\right\Vert _{A}.$

$\left( 3\right) $By Cholesky Decomposition's method, we can find $L\in \NEG%
{R}^{n\times n},s.t.A=L^{T}L.$

Then $\left\Vert x\right\Vert _{A}=<L^{T}Lx,x>=\left\Vert Lx\right\Vert
_{2}, $where $\left\Vert \cdot \right\Vert _{2}$ represents Euclid norm on $%
\NEG{R}^{n}.$

$\left\Vert x+y\right\Vert _{A}=\left\Vert L\left( x+y\right) \right\Vert
_{2}=\left\Vert Lx+Ly\right\Vert _{2}\leq \left\Vert Lx\right\Vert
_{2}+\left\Vert Ly\right\Vert _{2}=\left\Vert x\right\Vert _{A}+\left\Vert
y\right\Vert _{A}.$

By $\left( 1,2,3\right) $ $\left\Vert \cdot \right\Vert _{A}$ is a norm on $%
\NEG{R}^{n}.$

%FRAME

\bigskip Problem 5

Solution: Since A is symmetric, $cond(A)_{2}=\frac{\left\vert \lambda _{\max
}\right\vert }{\left\vert \lambda _{\min }\right\vert },$where $\lambda $
represents the eigenvalue of A.

Solve the equation $\det (\lambda I-A)=0,$ we get $\lambda _{1,2}=99\pm 
\sqrt{9802}=99\pm \allowbreak 13\sqrt{58}.$

$\implies cond(A)_{2}=\left\vert \frac{99+13\sqrt{58}}{99-13\sqrt{58}}%
\right\vert =\left( 13\sqrt{58}+99\right) ^{2}=\allowbreak 2574\sqrt{58}%
+19\,603.$

$cond(A)_{\infty }=\left\Vert A\right\Vert _{\infty }\left\Vert
A^{-1}\right\Vert _{\infty },$where $\left\Vert \cdot \right\Vert _{\infty }$
represents the infinity norm on $\NEG{R}^{2}.$

$\left\Vert A\right\Vert _{\infty }=\underset{i=1or2}{\max }\left(
\left\vert a_{i1}\right\vert +\left\vert a_{i2}\right\vert \right)
=199.A^{-1}=\left( 
\begin{array}{cc}
-98 & 99 \\ 
99 & 100%
\end{array}%
\right) .$

and $\left\Vert A^{-1}\right\Vert _{\infty }=199.\implies cond\left(
A\right) _{\infty }=199^{2}=\allowbreak 39\,601.$

%FRAME%
Problem 6 \bigskip Proof:

$\frac{1}{cond\left( A\right) _{2}}=\frac{\lambda }{\left\Vert A\right\Vert
_{2}}\iff \left\Vert A^{-1}\right\Vert _{2}=\frac{1}{\lambda }.$Below the
notation of subscript is omitted for convenience.

We know if $\left\Vert A^{-1}\right\Vert \left\Vert B\right\Vert <1,$then $%
A+B=A(I+A^{-1}B)$ is non-singular.

In other word, if $A+B$ is singular, we must have $\left\Vert
A^{-1}\right\Vert \left\Vert B\right\Vert \geq 1.$

$\implies \left\Vert A^{-1}\right\Vert \geq \frac{1}{\left\Vert B\right\Vert 
}\implies \left\Vert A^{-1}\right\Vert \geq \frac{1}{\lambda }.\left(
1\right) $

\bigskip By the theory of singular decomposition, $A=VDU^{T},$where $%
D_{n\times n}=diag\left( \mu _{1},\mu _{2},...,\mu _{n}\right) ,$

$\mu _{1}\geq \mu _{2}\geq ...\geq \mu _{n}$ and by the definition $%
\left\Vert A\right\Vert _{2}=\mu _{1}\left( \text{the maximum of singular
value}\right) .$

$A^{-1}=UD^{-1}V^{-1},\implies \left\Vert A^{-1}\right\Vert =\frac{1}{\mu
_{n}}.$Without loss of generality, we can assume $A=D,$

otherwise we can always revise $B$ as $VBU^{T}$ without changing the norm of
B$.$ We choose

$A+B=diag\left( \mu _{1},\mu _{2},...,\mu _{n-1},0\right) .$Then $B=\left(
0,0,...,0,-\mu _{n}\right) .$

$\left\Vert A^{-1}\right\Vert =\frac{1}{\mu _{n}}=\frac{1}{\left\Vert
B\right\Vert }\leq \frac{1}{\lambda }.\left( 2\right) $

By $\left( 1,2\right) $ the proof is complete.$\boxtimes $

\end{document}
