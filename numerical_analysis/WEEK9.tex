
\documentclass{article}
\usepackage{amsmath}

%%%%%%%%%%%%%%%%%%%%%%%%%%%%%%%%%%%%%%%%%%%%%%%%%%%%%%%%%%%%%%%%%%%%%%%%%%%%%%%%%%%%%%%%%%%%%%%%%%%%%%%%%%%%%%%%%%%%%%%%%%%%%%%%%%%%%%%%%%%%%%%%%%%%%%%%%%%%%%%%%%%%%%%%%%%%%%%%%%%%%%%%%%%%%%%%%%%%%%%%%%%%%%%%%%%%%%%%%%%%%%%%%%%%
%TCIDATA{OutputFilter=LATEX.DLL}
%TCIDATA{Version=5.00.0.2552}
%TCIDATA{<META NAME="SaveForMode" CONTENT="1">}
%TCIDATA{Created=Wednesday, November 11, 2015 17:31:06}
%TCIDATA{LastRevised=Wednesday, November 11, 2015 22:44:17}
%TCIDATA{<META NAME="GraphicsSave" CONTENT="32">}
%TCIDATA{<META NAME="DocumentShell" CONTENT="Standard LaTeX\Blank - Standard LaTeX Article">}
%TCIDATA{CSTFile=40 LaTeX article.cst}

\newtheorem{theorem}{Theorem}
\newtheorem{acknowledgement}[theorem]{Acknowledgement}
\newtheorem{algorithm}[theorem]{Algorithm}
\newtheorem{axiom}[theorem]{Axiom}
\newtheorem{case}[theorem]{Case}
\newtheorem{claim}[theorem]{Claim}
\newtheorem{conclusion}[theorem]{Conclusion}
\newtheorem{condition}[theorem]{Condition}
\newtheorem{conjecture}[theorem]{Conjecture}
\newtheorem{corollary}[theorem]{Corollary}
\newtheorem{criterion}[theorem]{Criterion}
\newtheorem{definition}[theorem]{Definition}
\newtheorem{example}[theorem]{Example}
\newtheorem{exercise}[theorem]{Exercise}
\newtheorem{lemma}[theorem]{Lemma}
\newtheorem{notation}[theorem]{Notation}
\newtheorem{problem}[theorem]{Problem}
\newtheorem{proposition}[theorem]{Proposition}
\newtheorem{remark}[theorem]{Remark}
\newtheorem{solution}[theorem]{Solution}
\newtheorem{summary}[theorem]{Summary}
\newenvironment{proof}[1][Proof]{\noindent\textbf{#1.} }{\ \rule{0.5em}{0.5em}}
\def\TEXTsymbol#1{\mbox{$#1$}}%
\def\NEG#1{\leavevmode\hbox{\rlap{\thinspace/}{$#1$}}}%
\def\QATOPD#1#2#3#4{{#3 \atopwithdelims#1#2 #4}}%
\def\QTP#1{}
\def\func#1{\mathop{\rm #1}}%

\begin{document}


1.$\left( 1\right) $\U{6052}\U{7b49}\U{5f0f}\U{4e3a}\U{5173}\U{4e8e}$x^{k}$%
\U{7684}Langrage\U{63d2}\U{503c}\U{591a}\U{9879}\U{5f0f},\U{56e0}\U{800c}%
\U{6210}\U{7acb}.

$\left( 2\right) \underset{j=0}{\overset{n}{\sum }}\left( x_{j}-x\right)
^{k}l_{j}\left( x\right) =\underset{j=0}{\overset{n}{\sum }}\underset{t=0}{%
\overset{k}{\sum }}\binom{k}{t}x_{j}^{t}\left( -x\right) ^{k-t}l_{j}\left(
x\right) =\underset{t=0}{\overset{k}{\sum }}\binom{k}{t}\left( -x\right)
^{k-t}\underset{j=0}{\overset{n}{\sum }}x_{j}^{t}l_{j}\left( x\right) $

$\bigskip $\U{7531}$\left( 1\right) $\U{7684}\U{7ed3}\U{8bba}$\underset{j=0}{%
\overset{n}{\sum }}x_{j}^{t}l_{j}\left( x\right) =x^{t},$\U{56e0}\U{6b64}%
\U{4e0a}\U{5f0f}\U{53ef}\U{5316}\U{4e3a}

$=\underset{k=0}{\overset{m}{\sum }}\binom{k}{t}\left( -x\right)
^{k-t}x^{t}=\left( x-x\right) ^{k}=0,$for $k\neq 0.$

\bigskip $E\left( \left( \frac{\partial }{\partial \theta }\log f\left(
X|\theta \right) \right) ^{2}\right) =\int \left( \frac{\frac{\partial }{%
\partial \theta }f\left( \vec{x}|\theta \right) }{f\left( \vec{x}|\theta
\right) }\right) ^{2}f\left( \vec{x}|\theta \right) d\vec{x}$=$\int \frac{%
\left( \frac{\partial }{\partial \theta }f\left( \vec{x}|\theta \right)
\right) ^{2}}{f\left( \vec{x}|\theta \right) }d\vec{x}$

$-E\left( \frac{\partial ^{2}}{\partial \theta ^{2}}\log f\left( X|\theta
\right) \right) =-\int \left( \frac{\partial }{\partial \theta }\frac{\frac{%
\partial }{\partial \theta }f\left( \vec{x}|\theta \right) }{f\left( \vec{x}%
|\theta \right) }\right) f\left( \vec{x}|\theta \right) d\vec{x}$

$=-\int \left( \frac{f\left( \vec{x}|\theta \right) \frac{\partial ^{2}}{%
\partial \theta ^{2}}f\left( \vec{x}|\theta \right) -\frac{\partial }{%
\partial \theta }f\left( \vec{x}|\theta \right) ^{2}}{f\left( \vec{x}|\theta
\right) }\right) d\vec{x}=-\int \frac{\partial ^{2}}{\partial \theta ^{2}}%
f\left( \vec{x}|\theta \right) d\vec{x}+\int \frac{\frac{\partial }{\partial
\theta }f\left( \vec{x}|\theta \right) ^{2}}{f\left( \vec{x}|\theta \right) }%
d\vec{x}$

The condition says that $\frac{d}{d\theta }\int \frac{\partial }{\partial
\theta }f\left( \vec{x}|\theta \right) d\vec{x}=\int \frac{\partial ^{2}}{%
\partial \theta ^{2}}f\left( \vec{x}|\theta \right) d\vec{x}$

Then $E\left( \left( \frac{\partial }{\partial \theta }\log f\left( X|\theta
\right) \right) ^{2}\right) =-\frac{d}{d\theta }\int \frac{\partial }{%
\partial \theta }f\left( \vec{x}|\theta \right) d\vec{x}+\int \frac{\frac{%
\partial }{\partial \theta }f\left( \vec{x}|\theta \right) ^{2}}{f\left( 
\vec{x}|\theta \right) }d\vec{x}$

To prove the equality, we must further assume that 

$\int \frac{\partial }{\partial \theta }f\left( \vec{x}|\theta \right) d\vec{%
x}=\frac{\partial }{\partial \theta }$ $\int f\left( \vec{x}|\theta \right) d%
\vec{x}=1.$ The condition expression on the textbook is 

not clear, which does not give this assumption, though it is contained in
the proof of C-R Inequality. See, for example, "Theory of Point Estimation",
by Lehmann

and Casella, 1998 by Springer on page 116 Lemma 5.3, which clearly gives out
necessry condtions for this convinient expression of Fisher Information.

\FRAME{ftbpF}{4.5in}{0.9in}{0in}{}{}{Figure}{\special{language "Scientific
Word";type "GRAPHIC";maintain-aspect-ratio TRUE;display "USEDEF";valid_file
"T";width 4.5in;height 0.9in;depth 0in;original-width
10.8223in;original-height 3.237in;cropleft "0";croptop "1";cropright
"1";cropbottom "0";tempfilename 'NXNIVY00.wmf';tempfile-properties "XPR";}}

\end{document}
