
\documentclass{article}
%%%%%%%%%%%%%%%%%%%%%%%%%%%%%%%%%%%%%%%%%%%%%%%%%%%%%%%%%%%%%%%%%%%%%%%%%%%%%%%%%%%%%%%%%%%%%%%%%%%%%%%%%%%%%%%%%%%%%%%%%%%%%%%%%%%%%%%%%%%%%%%%%%%%%%%%%%%%%%%%%%%%%%%%%%%%%%%%%%%%%%%%%%%%%%%%%%%%%%%%%%%%%%%%%%%%%%%%%%%%%%%%%%%%%%%%%%%%%%%%%%%%%%%%%%%%
\usepackage{amsmath}

\setcounter{MaxMatrixCols}{10}
%TCIDATA{OutputFilter=LATEX.DLL}
%TCIDATA{Version=5.00.0.2552}
%TCIDATA{<META NAME="SaveForMode" CONTENT="1">}
%TCIDATA{Created=Sunday, December 27, 2015 15:48:17}
%TCIDATA{LastRevised=Tuesday, December 29, 2015 19:04:46}
%TCIDATA{<META NAME="GraphicsSave" CONTENT="32">}
%TCIDATA{<META NAME="DocumentShell" CONTENT="Standard LaTeX\Blank - Standard LaTeX Article">}
%TCIDATA{CSTFile=40 LaTeX article.cst}

\newtheorem{theorem}{Theorem}
\newtheorem{acknowledgement}[theorem]{Acknowledgement}
\newtheorem{algorithm}[theorem]{Algorithm}
\newtheorem{axiom}[theorem]{Axiom}
\newtheorem{case}[theorem]{Case}
\newtheorem{claim}[theorem]{Claim}
\newtheorem{conclusion}[theorem]{Conclusion}
\newtheorem{condition}[theorem]{Condition}
\newtheorem{conjecture}[theorem]{Conjecture}
\newtheorem{corollary}[theorem]{Corollary}
\newtheorem{criterion}[theorem]{Criterion}
\newtheorem{definition}[theorem]{Definition}
\newtheorem{example}[theorem]{Example}
\newtheorem{exercise}[theorem]{Exercise}
\newtheorem{lemma}[theorem]{Lemma}
\newtheorem{notation}[theorem]{Notation}
\newtheorem{problem}[theorem]{Problem}
\newtheorem{proposition}[theorem]{Proposition}
\newtheorem{remark}[theorem]{Remark}
\newtheorem{solution}[theorem]{Solution}
\newtheorem{summary}[theorem]{Summary}
\newenvironment{proof}[1][Proof]{\noindent\textbf{#1.} }{\ \rule{0.5em}{0.5em}}
\input{tcilatex}

\begin{document}


\bigskip Consider such a sphere moves on a turntable, 

$\frac{d\vec{L}_{c}}{dt}=a\vec{k}\times \vec{R},\vec{L}_{c},a,\vec{R}$

Then for every point of the sphere we use $\vec{r}_{i}$ to represent the
position vector from P, the angular momentum for this mass element is $%
\left( \vec{r}+\vec{r}_{i}\right) \times \left( \vec{v}_{P}+\overset{\cdot }{%
\vec{r}}_{i}\right) dm$

And the total angular momentum of the sphere with respect to $P$ is

$\vec{L}=\int \left( \vec{r}+\vec{r}_{i}\right) \times \left( \overset{\cdot 
}{\vec{r}}+\overset{\cdot }{\vec{r}}_{i}\right) dm$

$=\int \left( \vec{r}\times \overset{\cdot }{\vec{r}}\right) dm+\left( \int 
\vec{r}_{i}dm\right) \times \overset{\cdot }{\vec{r}}+\vec{r}\times \left(
\int \overset{\cdot }{\vec{r}}_{i}dm\right) +\int \vec{r}_{i}\times \overset{%
\cdot }{\vec{r}}_{i}dm$

By the symmetry of the sphere $\int \vec{r}_{i}dm=ma\vec{k},$where $a$
represents the radius of the sphere. We use $\vec{r}_{i}^{\prime }$ to
represent the position vector of a mass element from $G.$

Then $\vec{r}_{i}=\vec{r}_{i}^{\prime }+a\vec{k},\int \overset{\cdot }{\vec{r%
}}_{i}dm=\int \overset{\cdot }{\vec{r}^{\prime }}_{i}dm=m\vec{v}_{c}$

$\int \vec{r}_{i}\times \overset{\cdot }{\vec{r}}_{i}dm=$ $\int \left( \vec{r%
}_{i}^{\prime }+a\vec{k}\right) \times \overset{\cdot }{\vec{r}_{i}^{\prime }%
}dm$

$=$ $\int \vec{r}_{i}^{\prime }\times \overset{\cdot }{\vec{r}_{i}^{\prime }}%
dm+am\vec{k}\times \vec{v}_{G},$by the definition of velocity of $CM.$

By the definition of angular velocity in $3D,\overset{\cdot }{\vec{r}%
_{i}^{\prime }}=\vec{\omega}\times \vec{r}_{i}^{\prime }$

$\implies \vec{r}_{i}^{\prime }\times \overset{\cdot }{\vec{r}_{i}^{\prime }}%
=\vec{r}_{i}^{\prime }\times \left( \vec{\omega}\times \vec{r}_{i}^{\prime
}\right) =\left( \vec{r}_{i}^{\prime }\cdot \vec{r}_{i}^{\prime }\right) 
\vec{\omega}-\left( \vec{r}_{i}^{\prime }\cdot \vec{\omega}\right) \vec{r}%
_{i}^{\prime }$

In $3D-Cartesian$ coordiate: $\vec{r}_{i}^{\prime }\times \overset{\cdot }{%
\vec{r}_{i}^{\prime }}=\left( \left( y^{2}+z^{2}\right) \omega _{x}-xy\omega
_{y}-xz\omega _{z}\right) \vec{i}$

$+\left( -xy\omega _{x}+\left( x^{2}+z^{2}\right) \omega _{y}-yz\omega
_{z}\right) \vec{j}+\left( -xz\omega _{x}-yz\omega _{y}+\left(
x^{2}+y^{2}\right) \omega _{z}\right) \vec{k}.$

(To simplify the notation, $x,y,z$ are equivalent to $x_{i}^{\prime
},y_{i}^{\prime },z_{i}^{\prime }$ in the above equation)

\bigskip If we define $I_{i}=%
\begin{pmatrix}
y_{i}^{\prime 2}+z_{i}^{\prime 2} & -x_{i}^{\prime }y_{i}^{\prime } & 
-x_{i}^{\prime }z_{i}^{\prime } \\ 
-x_{i}^{\prime }y_{i}^{\prime } & x_{i}^{\prime 2}+z_{i}^{\prime 2} & 
-y_{i}^{\prime }z_{i}^{\prime } \\ 
-x_{i}^{\prime }z_{i}^{\prime } & -y_{i}^{\prime }z_{i}^{\prime } & 
x_{i}^{\prime 2}+y_{i}^{\prime 2}%
\end{pmatrix}%
,$ which is a $3\times 3$ matrix

We can write this in matrix form $\vec{r}_{i}^{\prime }\times \overset{\cdot 
}{\vec{r}_{i}^{\prime }}=I_{i}\vec{\omega}$

$\implies \int \vec{r}_{i}^{\prime }\times \overset{\cdot }{\vec{r}%
_{i}^{\prime }}dm=\left( \int I_{i}dm\right) \vec{\omega}.$

$\int I_{i}dm$ is the moment of inertia tensor, for the symmetric sphere, it
is equal to

$\frac{2}{5}ma^{2}I_{3},$where $I_{3}$ is the unit matrix.$\vec{L}_{c}=J\vec{%
\omega}$

$\implies \vec{L}=$ $\int \left( \vec{r}+\vec{r}_{i}\right) \times \left( 
\overset{\cdot }{\vec{r}}+\overset{\cdot }{\vec{r}}_{i}\right) dm$

$=m\vec{r}\times \overset{\cdot }{\vec{r}}+ma\vec{k}\times \overset{\cdot }{%
\vec{r}}+\vec{r}\times m\vec{v}_{c}+\frac{2}{5}ma^{2}\vec{\omega},$

\bigskip The total momentum of the sphere is $\vec{p}=\int \left( \overset{%
\cdot }{\vec{r}}+\overset{\cdot }{\vec{r}}_{i}\right) dm$

$\bigskip J\vec{\omega}=a\vec{k}\times m\overset{\cdot \cdot }{\vec{r}}$

$m\overset{\cdot \cdot }{\vec{r}}=\vec{R}-mg\vec{k}$

$=m\overset{\cdot }{\vec{r}}+\int \overset{\cdot }{\vec{r}^{\prime }}_{i}dm=m%
\overset{\cdot }{\vec{r}}+m\overset{\cdot }{\vec{r}}.$

$\vec{F}=\frac{d\vec{p}}{dt}\implies \vec{F}=m\overset{\cdot \cdot }{\vec{r}}%
+m\overset{\cdot }{\vec{v}}_{c}$

By Angular Momentum Thm$,$

$\frac{d\vec{L}}{dt}=\vec{r}\times \vec{F}\implies $

$ma\vec{k}\times \overset{\cdot \cdot }{\vec{r}}+\frac{2}{5}ma^{2}\overset{%
\cdot }{\vec{\omega}}=0\left( \ast \right) .$

Remark, this equation can be deduced directly from the angular momentum in $%
CM-reference$ system, in which $\frac{dL_{c}}{dt}=M_{c}\left( 1\right) .$

$\bigskip M_{c}=\left( -a\vec{k}\right) \times \vec{R}\left( 2\right) ,\vec{R%
}$ represents the restraint force.

Also by $CM-motion$ theorem, $\vec{R}-mg\vec{k}=m\overset{\cdot \cdot }{\vec{%
r}}\left( 3\right) $

Combining $\left( 1,2,3\right) $ we get $\left( \ast \right) $

The velocity of the touching point $\vec{v}_{P}=\Omega \vec{k}\times \vec{r},
$

where $\Omega $ is the constant angular speed of the turntable. 

This is the definition of the linear velocity on a rotating 

turntable and the velocity of P with respect to the sphere 

is equal to this because there is no slip. 

This is the restraint condition, which gives

$\Omega \vec{k}\times $

$\vec{r}$

$=$

$\overset{\cdot }{\vec{r}}+\vec{\omega}\times \left( -a\right) \vec{k}$

$\bigskip J\vec{\omega}=a\vec{k}\times m\overset{\cdot \cdot }{\vec{r}}$

$J=\frac{2}{5}ma^{2}$

$\vec{k}\times \overset{\cdot \cdot }{\vec{r}}+\frac{2}{5}a\overset{\cdot }{%
\vec{\omega}}=0\left( \ast \right) .$

Taking the derivative of $\left( \ast \ast \right) $ and substituting $\vec{%
\omega}$ of $\left( \ast \right) $ into $\left( \ast \ast \right) $ gives:

$\frac{7}{2}\overset{\cdot \cdot }{\vec{r}}=\Omega \vec{k}\times \overset{%
\cdot }{\vec{r}}.$

Integrating this equation gives:

$\frac{7}{2}\overset{\cdot }{\vec{r}}=\Omega \vec{k}\times \left( \vec{r}-%
\vec{r}_{0}\right) ,$ where $\vec{r}_{0}$ is the integration constant.

In $2D-Cartesian$ coordinate, $\frac{7}{2}\left( \dot{x}\vec{i}+\dot{y}\vec{j%
}\right) =\Omega \left( \left( x-x_{0}\right) \vec{j}-\left( y-y_{0}\right) 
\vec{i}\right) $

In Matrix form, $\overset{\cdot }{\vec{r}}=\frac{2}{7}\Omega 
\begin{pmatrix}
0 & -1 \\ 
1 & 0%
\end{pmatrix}%
\left( \vec{r}-\vec{r}_{0}\right) ,$ which allows solution like the form

$\qquad \vec{r}-\vec{r}_{0}=A\binom{\cos \frac{2}{7}\Omega t}{\sin \frac{2}{7%
}\Omega t}+B\binom{-\sin \frac{2}{7}\Omega t}{\cos \frac{2}{7}\Omega t},$%
where $\vec{r}_{0},$

$A$ and $B$ are constant determined by initial condition, 

$\implies \left\vert \vec{r}-\vec{r}_{0}\right\vert =\sqrt{A^{2}+B^{2}},$the
trajectory of the sphere is a circle.

\end{document}
