
\documentclass{article}
%%%%%%%%%%%%%%%%%%%%%%%%%%%%%%%%%%%%%%%%%%%%%%%%%%%%%%%%%%%%%%%%%%%%%%%%%%%%%%%%%%%%%%%%%%%%%%%%%%%%%%%%%%%%%%%%%%%%%%%%%%%%%%%%%%%%%%%%%%%%%%%%%%%%%%%%%%%%%%%%%%%%%%%%%%%%%%%%%%%%%%%%%%%%%%%%%%%%%%%%%%%%%%%%%%%%%%%%%%%%%%%%%%%%%%%%%%%%%%%%%%%%%%%%%%%%
\usepackage{amsmath}

\setcounter{MaxMatrixCols}{10}
%TCIDATA{OutputFilter=LATEX.DLL}
%TCIDATA{Version=5.00.0.2552}
%TCIDATA{<META NAME="SaveForMode" CONTENT="1">}
%TCIDATA{Created=Sunday, December 20, 2015 01:20:28}
%TCIDATA{LastRevised=Wednesday, December 23, 2015 07:53:31}
%TCIDATA{<META NAME="GraphicsSave" CONTENT="32">}
%TCIDATA{<META NAME="DocumentShell" CONTENT="Standard LaTeX\Blank - Standard LaTeX Article">}
%TCIDATA{CSTFile=40 LaTeX article.cst}

\newtheorem{theorem}{Theorem}
\newtheorem{acknowledgement}[theorem]{Acknowledgement}
\newtheorem{algorithm}[theorem]{Algorithm}
\newtheorem{axiom}[theorem]{Axiom}
\newtheorem{case}[theorem]{Case}
\newtheorem{claim}[theorem]{Claim}
\newtheorem{conclusion}[theorem]{Conclusion}
\newtheorem{condition}[theorem]{Condition}
\newtheorem{conjecture}[theorem]{Conjecture}
\newtheorem{corollary}[theorem]{Corollary}
\newtheorem{criterion}[theorem]{Criterion}
\newtheorem{definition}[theorem]{Definition}
\newtheorem{example}[theorem]{Example}
\newtheorem{exercise}[theorem]{Exercise}
\newtheorem{lemma}[theorem]{Lemma}
\newtheorem{notation}[theorem]{Notation}
\newtheorem{problem}[theorem]{Problem}
\newtheorem{proposition}[theorem]{Proposition}
\newtheorem{remark}[theorem]{Remark}
\newtheorem{solution}[theorem]{Solution}
\newtheorem{summary}[theorem]{Summary}
\newenvironment{proof}[1][Proof]{\noindent\textbf{#1.} }{\ \rule{0.5em}{0.5em}}
\input{tcilatex}

\begin{document}


$R_{AB,CD}:=\frac{U_{CD}}{I_{AB}},U_{SR}=\frac{I}{\pi d\sigma }\ln \frac{%
\left( a+b\right) \left( b+c\right) }{b\left( a+b+c\right) },\int \vec{j}%
\cdot d\vec{S}=I$

$e^{-\pi R_{AB,CD}d\sigma }+e^{-\pi R_{BC,DA}d\sigma }=1$

where $\vec{j}$ represents the current density emitting from $P.$

From Ohm law: $\vec{E}=\frac{\vec{j}}{\sigma }$ and the symmetric property

follows $E_{P}=\frac{I}{\sigma S}=\frac{I}{\pi \sigma d}\frac{1}{r},$where $%
r $ represents the

distance of the position from $P.$

Integrating $Edr$ from $S$ to $R$ alongs the boundary of

the sample gives $U_{SR}=\int \left( E_{Q}-E_{P}\right) dr.$ Choosing $P$ as

coordinate origin further gives: $U_{SR}=\int_{a+b}^{a+b+c}\left(
E_{Q}-E_{P}\right) dr$

Similarly, $E_{Q}=\frac{I}{\pi \sigma d}\frac{1}{r-a}$

$\implies U_{SR}=\frac{I}{\pi \sigma d}\int_{a+b}^{a+b+c}\left( \frac{1}{r-a}%
-\frac{1}{r}\right) dr=\frac{I}{\pi d\sigma }\ln \frac{\left( a+b\right)
\left( b+c\right) }{b\left( a+b+c\right) }$

In the same way, if the current enters from $P$ and leaves

at $S,$ we can show that under the (Cauchy PV) integral

$U_{QR}=$ $\frac{I}{\pi d\sigma }\ln \frac{\left( a+b\right) \left(
b+c\right) }{ac}.$

In general, if we choose from the remaining$\binom{4}{2}-1=5$

kinds of combination, the potential difference is the same.

In the last step, substituting $R_{SR,PQ}$ and $R_{QR,SP}$ into the left

hand of the formula gives $1.$

\bigskip

First we can treat the semi-infinite plane as $\left\{ z|\func{Re}z\geq
0\right\} $

in a complex plane.Let $u\left( x,y\right) $ represents the potential field

in the sample, By the theory of electrostatics we can show

$u\left( x,y\right) $ is harmonic real function and allows a conjugate

harmonic function $v\left( x,y\right) ,s.t.f\left( z\right) =u+iv$ is an
analytic

function.

\bigskip

$\nabla u\left( x,y\right) =E_{Q}-E_{P}$

$E_{P}=c\left( \frac{x}{\sqrt{x^{2}+y^{2}}},\frac{y}{\sqrt{x^{2}+y^{2}}}%
\right) $

$\bigskip \frac{\partial }{\partial y}\frac{x}{\sqrt{x^{2}+y^{2}}}=\frac{%
\partial }{\partial x}\frac{y}{\sqrt{x^{2}+y^{2}}}$

$\bigskip $By Green's formula:

$\doint E_{P}dr=0$

With translation we can assume $P$ at origin

of the complex plane, with $Q=a.$

Considering the direction of current at $P$ and $Q,$

Omiting the constant $u\left( z\right) =\log \left\vert z\right\vert -\log
\left\vert z-a\right\vert ,$

$\implies f\left( z\right) =\log z-\log \left( z-a\right) $

$\implies v\left( z\right) =\arg z-\arg \left( z-a\right) $

Travelling along the boundary of the upper half

plane from left to right. $v\left( z\right) $ remains constant

except for $z=0$ and $a.$

At $z=0,f\left( z\right) $ is not definied here, 

therefore the path should be bent 

as a half circle.

Passing along the half circle at $z=0$ decrease $v\left( z\right) $ by $\pi $

by the definition of $\arg z.$Similarly, $v\left( z\right) $ increases $\pi $
at $z=a.$

Now for the simple-connected region 

as the surface of an arbitrary sample, 

by Riemman-Conformal Mapping Thm,

there exists a conformal mapping $t\left( z\right) $ which maps the upper

half plane to $t-plane.$Let $A,B,C,D$ be the image of the 

points $P,Q,R,S.$The potential field $f\left( z\right) $ is mapped to $%
f\left( z\left( t\right) \right) =l+im.$

Similarly, $m$ remains constant when travelling in counter-clockwise

direction along the boundary of the sample, expect for $A$ and $B.$

At $A,m$ decreases $\pi $ and at $B$ $m$ increases $\pi .$

The physical meaning of the jump of $v$ is that there is external current

passing through $A$ and $B.$This follows from the continuous equation of 

steady current.

Because the upper-half plane is imaginary, we can set all its intensive 

quantities equal to those of the finite sample(like thickiness, input current

and resistivity). Then $m$ and $v$ shares the same change as they circle the 

boundary of each region counter-clockwisely. By harmonic function theory,

such information implies $m$ is the potential field of the sample.(At this
point

it needs further varification from \U{552f}\U{4e00}\U{6027}\U{5b9a}\U{7406})

$V_{CD}=\int \frac{\partial m}{\partial t_{x}}dt_{x}+\frac{\partial m}{%
\partial t_{y}}dt_{y},$ by Cauchy-Riemann Equation

$\implies V_{CD}=\int \frac{\partial m}{\partial t_{x}}dt_{x}-\frac{\partial
l}{\partial t_{x}}dt_{y}=\func{Re}\int \left( \frac{\partial m}{\partial
t_{x}}+\frac{\partial l}{\partial t_{x}}i\right) \left(
dt_{x}+idt_{y}\right) $

$=\func{Re}\int \frac{df\left( z\left( t\right) \right) }{dt}dt,$ Making
Transformation gives

$=\func{Re}\int f^{\prime }\left( z\right) dz=..=\int \frac{\partial u}{%
\partial x}dx+\frac{\partial u}{\partial y}dy=V_{RS}.$

Other potential equality can be proved similarly.

Therefore, the formula form is unchanged if finite sample is considered.

\end{document}
