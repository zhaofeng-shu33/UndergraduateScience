
\documentclass{ctexart}
\usepackage{amsmath}

%%%%%%%%%%%%%%%%%%%%%%%%%%%%%%%%%%%%%%%%%%%%%%%%%%%%%%%%%%%%%%%%%%%%%%%%%%%%%%%%%%%%%%%%%%%%%%%%%%%%%%%%%%%%%%%%%%%%%%%%%%%%%%%%%%%%%%%%%%%%%%%%%%%%%%%%%%%%%%%%%%%%%%%%%%%%%%%%%%%%%%%%%%%%%%%%%%%%%%%%%%%%%%%%%%%%%%%%%%%%%%%%%%%%
%TCIDATA{OutputFilter=LATEX.DLL}
%TCIDATA{Version=5.00.0.2552}
%TCIDATA{<META NAME="SaveForMode" CONTENT="1">}
%TCIDATA{Created=Tuesday, October 20, 2015 18:47:15}
%TCIDATA{LastRevised=Tuesday, October 20, 2015 20:41:58}
%TCIDATA{<META NAME="GraphicsSave" CONTENT="32">}
%TCIDATA{<META NAME="DocumentShell" CONTENT="Scientific Notebook\Blank Document">}
%TCIDATA{CSTFile=Math with theorems suppressed.cst}
%TCIDATA{PageSetup=72,72,72,72,0}
%TCIDATA{AllPages=
%F=36,\PARA{038<p type="texpara" tag="Body Text" >\hfill \thepage}
%}


\newtheorem{theorem}{Theorem}
\newtheorem{acknowledgement}[theorem]{Acknowledgement}
\newtheorem{algorithm}[theorem]{Algorithm}
\newtheorem{axiom}[theorem]{Axiom}
\newtheorem{case}[theorem]{Case}
\newtheorem{claim}[theorem]{Claim}
\newtheorem{conclusion}[theorem]{Conclusion}
\newtheorem{condition}[theorem]{Condition}
\newtheorem{conjecture}[theorem]{Conjecture}
\newtheorem{corollary}[theorem]{Corollary}
\newtheorem{criterion}[theorem]{Criterion}
\newtheorem{definition}[theorem]{Definition}
\newtheorem{example}[theorem]{Example}
\newtheorem{exercise}[theorem]{Exercise}
\newtheorem{lemma}[theorem]{Lemma}
\newtheorem{notation}[theorem]{Notation}
\newtheorem{problem}[theorem]{Problem}
\newtheorem{proposition}[theorem]{Proposition}
\newtheorem{remark}[theorem]{Remark}
\newtheorem{solution}[theorem]{Solution}
\newtheorem{summary}[theorem]{Summary}
\newenvironment{proof}[1][Proof]{\noindent\textbf{#1.} }{\ \rule{0.5em}{0.5em}}


\begin{document}


折射定律用Fermat Principle+Lagrange Multiplier 
证明如下

设光经过两种折射率%
分别为$n_{1},n_{2}$的介质分界%
面,

设介质$m_{i}$折射率为$n_{i}$,%
内有点$P_{i}\left( i=1,2\right) $

光经过$P_{1}\in m_{1},P_{2}\in m_{2}$的路%
径要极小化光程$s=n_{1}\int
ds_{1}+n_{2}\int ds_{2}$

$\int ds_{i}$分别表示光在介质%
$m_{i}$中的几何路径$,$若此%
路径与介面交点为$P$,则%
极小化$s\iff $极小化$P_{i}P$子%
路径$\left( i=1,2\right) $

由平面几何线段极小%
化两点间的路径知,路%
径$P_{i}P$应取线段,此时极%
小化光程的必要条件%
是$ds=0\iff n_{1}ds_{1}+n_{2}ds_{2}=0\qquad \left( \ast \right) $

设介面法线方向为单%
位向量$\vec{n}$,$\left\vert \overrightarrow{PP}_{i}\cdot 
\vec{n}\right\vert =l_{i},$由$S_{i}^{2}-l_{i}^{2}=const\implies $

$S_{i}dS_{i}-l_{i}dl_{i}=0$分别代入$\left( \ast
\right) $式$\implies $

$n_{1}\frac{l_{1}}{S_{1}}dl_{1}+n_{2}\frac{l_{2}}{S_{2}}dl_{2}=0$与$%
dl_{1}+dl_{2}=0\Longleftarrow l_{1}+l_{2}=const\qquad $联立%
以解除约束即推得

$n_{1}\frac{l_{1}}{S_{1}}=n_{2}\frac{l_{2}}{S_{2}}$即为$%
n_{1}\sin i=n_{2}\sin r.$

\end{document}
