
\documentclass{ctexart}
\usepackage{units}
%%%%%%%%%%%%%%%%%%%%%%%%%%%%%%%%%%%%%%%%%%%%%%%%%%%%%%%%%%%%%%%%%%%%%%%%%%%%%%%%%%%%%%%%%%%%%%%%%%%%%%%%%%%%%%%%%%%%%%%%%%%%%%%%%%%%%%%%%%%%%%%%%%%%%%%%%%%%%%%%%%%%%%%%%%%%%%%%%%%%%%%%%%%%%%%%%%%%%%%%%%%%%%%%%%%%%%%%%%%%%%%%%%%%%%%%%%%%%%%%%%%%%%%%%%%%
%TCIDATA{OutputFilter=LATEX.DLL}
%TCIDATA{Version=5.00.0.2552}
%TCIDATA{<META NAME="SaveForMode" CONTENT="1">}
%TCIDATA{Created=Sunday, September 20, 2015 15:43:13}
%TCIDATA{LastRevised=Sunday, September 20, 2015 21:03:45}
%TCIDATA{<META NAME="GraphicsSave" CONTENT="32">}
%TCIDATA{<META NAME="DocumentShell" CONTENT="Scientific Notebook\Blank Document">}
%TCIDATA{CSTFile=Math with theorems suppressed.cst}
%TCIDATA{PageSetup=72,72,72,72,0}
%TCIDATA{AllPages=
%F=36,\PARA{038<p type="texpara" tag="Body Text" >\hfill \thepage}
%}


\newtheorem{theorem}{Theorem}
\newtheorem{acknowledgement}[theorem]{Acknowledgement}
\newtheorem{algorithm}[theorem]{Algorithm}
\newtheorem{axiom}[theorem]{Axiom}
\newtheorem{case}[theorem]{Case}
\newtheorem{claim}[theorem]{Claim}
\newtheorem{conclusion}[theorem]{Conclusion}
\newtheorem{condition}[theorem]{Condition}
\newtheorem{conjecture}[theorem]{Conjecture}
\newtheorem{corollary}[theorem]{Corollary}
\newtheorem{criterion}[theorem]{Criterion}
\newtheorem{definition}[theorem]{Definition}
\newtheorem{example}[theorem]{Example}
\newtheorem{exercise}[theorem]{Exercise}
\newtheorem{lemma}[theorem]{Lemma}
\newtheorem{notation}[theorem]{Notation}
\newtheorem{problem}[theorem]{Problem}
\newtheorem{proposition}[theorem]{Proposition}
\newtheorem{remark}[theorem]{Remark}
\newtheorem{solution}[theorem]{Solution}
\newtheorem{summary}[theorem]{Summary}
\newenvironment{proof}[1][Proof]{\noindent\textbf{#1.} }{\ \rule{0.5em}{0.5em}}


\begin{document}


\bigskip 物理实验绪论课作%
业

赵丰2013012178

2. (1) 为五位\qquad (2)应用科学%
记数法表示较大的数%
\qquad $3.1690\times 10^{4}\pm 200\unit{kg}$\qquad $\left( 3\right) $未%
取齐$,$应为10.43$\pm 0.32\unit{cm}\qquad \left(
4\right) 18.55\pm 0.31\unit{cm}$ \qquad $\left( 5\right) 18.7\pm 1.4\unit{cm}%
\qquad \left( 6\right) 2.73\times 10^{5}\pm 2000\unit{km}\qquad \left(
7\right) R=6.371\times 10^{6}\unit{m}=6.371\times 10^{8}\unit{cm}$

4. $\rho =\frac{4M}{\pi D^{2}H}=\frac{4\times 236.124\unit{g}}{3.14\times
\left( 2.345\unit{cm}\right) ^{2}\times 8.21\unit{cm}}=\allowbreak
6.\,\allowbreak 662\,6\unit{g}/\unit{cm}^{3}.$

$\Delta \rho =\rho \sqrt{\left( \frac{\Delta M}{M}\right) ^{2}+\left( \frac{%
2\Delta D}{D}\right) ^{2}+\left( \frac{\Delta H}{H}\right) ^{2}}%
=\,\allowbreak 6.662\,6\unit{g}/\unit{cm}^{3}\times \sqrt{\left( \frac{0.004%
\unit{g}}{236.124\unit{g}}\right) ^{2}+\left( \frac{2\times 0.005\unit{cm}}{%
2.345\unit{cm}}\right) ^{2}+\left( \frac{0.03\unit{cm}}{8.21\unit{cm}}%
\right) ^{2}}=3.\,\allowbreak 741\,6\times 10^{-2}\unit{g}/\unit{cm}^{3}.$

由计算可知\qquad $\frac{2\Delta D}{D}>\frac{%
\Delta H}{H}>\frac{\Delta M}{M},$即直径$D$的不%
确定度一项对计算结%
果影响最大。

7.用直线$y=ax+b$对以下实验%
数据进行处理,求出$\sum
x_{i}^{2},\left( \sum x_{i}\right) ^{2},\sum x_{i}y_{i},a,b,r,$并%
写出拟合公式。

使用文件操作的方法%
设计一个C++小程序求出%
上述六个值分别为:

21226.16 153507.24 \qquad 8938.56 \qquad 18.37 \qquad 0.08 \qquad 0.99998

用图形编程的方法作%
图如下:

%\FRAME{ftbpF}{2.1456in}{1.5255in}{0pt}{}{}{Figure}{\special{language

6.\qquad $\left( 1\right) BD\qquad \left( 2\right) BD\qquad \left( 3\right)
ABCEF\qquad H$

\end{document}
