
\documentclass{article}
\usepackage{amsmath}
\def\QTP#1{}
%%%%%%%%%%%%%%%%%%%%%%%%%%%%%%%%%%%%%%%%%%%%%%%%%%%%%%%%%%%%%%%%%%%%%%%%%%%%%%%%%%%%%%%%%%%%%%%%%%%%%%%%%%%%%%%%%%%%%%%%%%%%%%%%%%%%%%%%%%%%%%%%%%%%%%%%%%%%%%%%%%%%%%%%%%%%%%%%%%%%%%%%%%%%%%%%%%%%%%%%%%%%%%%%%%%%%%%%%%%%%%%%%%%%
%TCIDATA{OutputFilter=LATEX.DLL}
%TCIDATA{Version=5.00.0.2552}
%TCIDATA{<META NAME="SaveForMode" CONTENT="1">}
%TCIDATA{Created=Monday, December 28, 2015 19:40:24}
%TCIDATA{LastRevised=Monday, December 28, 2015 19:40:37}
%TCIDATA{<META NAME="GraphicsSave" CONTENT="32">}
%TCIDATA{<META NAME="DocumentShell" CONTENT="Scientific Notebook\Blank Document">}
%TCIDATA{CSTFile=Math with theorems suppressed.cst}
%TCIDATA{PageSetup=72,72,72,72,0}
%TCIDATA{AllPages=
%F=36,\PARA{038<p type="texpara" tag="Body Text" >\hfill \thepage}
%}


\newtheorem{theorem}{Theorem}
\newtheorem{acknowledgement}[theorem]{Acknowledgement}
\newtheorem{algorithm}[theorem]{Algorithm}
\newtheorem{axiom}[theorem]{Axiom}
\newtheorem{case}[theorem]{Case}
\newtheorem{claim}[theorem]{Claim}
\newtheorem{conclusion}[theorem]{Conclusion}
\newtheorem{condition}[theorem]{Condition}
\newtheorem{conjecture}[theorem]{Conjecture}
\newtheorem{corollary}[theorem]{Corollary}
\newtheorem{criterion}[theorem]{Criterion}
\newtheorem{definition}[theorem]{Definition}
\newtheorem{example}[theorem]{Example}
\newtheorem{exercise}[theorem]{Exercise}
\newtheorem{lemma}[theorem]{Lemma}
\newtheorem{notation}[theorem]{Notation}
\newtheorem{problem}[theorem]{Problem}
\newtheorem{proposition}[theorem]{Proposition}
\newtheorem{remark}[theorem]{Remark}
\newtheorem{solution}[theorem]{Solution}
\newtheorem{summary}[theorem]{Summary}
\newenvironment{proof}[1][Proof]{\noindent\textbf{#1.} }{\ \rule{0.5em}{0.5em}}


\begin{document}


Consider such a sphere moves on a turntable, 

Then for every point of the sphere we use $\vec{r}_{i}$ 

to represent the position vector from the center 

of the turntable, by K\"{o}nig's Theorem, 

The total kinetic energy of the sphere

$T=\frac{1}{2}\int \overset{\cdot }{\vec{r}_{i}^{^{\prime }2}}dm+\frac{1}{2}m%
\overset{\cdot }{\vec{r}^{2}},$

the first term is the kinetic energy in CM-reference

system and the second term is the translational energy

of the CM. 

By the definition of angular velocity in $3D,\overset{\cdot }{\vec{r}%
_{i}^{\prime }}=\vec{\omega}\times \vec{r}_{i}^{\prime }$

$\implies \overset{\cdot }{\vec{r}_{i}^{^{\prime }2}}=\left( \vec{\omega}%
\times \vec{r}_{i}^{\prime }\right) \cdot \left( \vec{\omega}\times \vec{r}%
_{i}^{\prime }\right) =\left( \left( \vec{\omega}\times \vec{r}_{i}^{\prime
}\right) \times \vec{\omega}\right) \cdot \vec{r}_{i}^{\prime }$

By the formula:$A\times \left( B\times C\right) =\left( A\cdot C\right)
B-\left( A\cdot B\right) C$

$\implies \overset{\cdot }{\vec{r}_{i}^{^{\prime }2}}=\left( \vec{\omega}^{2}%
\vec{r}_{i}^{\prime }-\left( \vec{\omega}\cdot \vec{r}_{i}^{\prime }\right) 
\vec{\omega}\right) \cdot \vec{r}_{i}^{\prime }=\vec{\omega}^{2}\vec{r}%
_{i}^{\prime 2}-\left( \vec{\omega}\cdot \vec{r}_{i}^{\prime }\right) ^{2}$

$\int \overset{\cdot }{\vec{r}_{i}^{^{\prime }2}}=\vec{\omega}^{2}\int
\left( x_{i}^{\prime 2}+y_{i}^{\prime 2}+z_{i}^{\prime 2}\right) dm-\int
\left( \omega _{x}x_{i}^{\prime }+\omega _{y}y_{i}^{\prime }+\omega
_{z}z_{i}^{\prime }\right) ^{2}dm,$

By the symmetric property of the sphere, 

$=\vec{\omega}^{2}\int \left( x_{i}^{\prime 2}+y_{i}^{\prime
2}+z_{i}^{\prime 2}\right) dm-\int \left( \omega _{x}^{2}x_{i}^{\prime
2}+\omega _{y}^{2}y_{i}^{\prime 2}+\omega _{z}^{2}z_{i}^{\prime 2}\right) dm$

$=\int \left( y_{i}^{\prime 2}+z_{i}^{\prime 2}\right) \omega
_{x}^{2}dm+\int \left( x_{i}^{\prime 2}+z_{i}^{\prime 2}\right) \omega
_{y}^{2}dm+\int \left( x_{i}^{\prime 2}+y_{i}^{\prime 2}\right) \omega
_{z}^{2}dm$

By the definition of moment of inertia $J$ for the sphere 

$\implies \int \overset{\cdot }{\vec{r}_{i}^{^{\prime }2}}=J\vec{\omega}%
^{2}\implies T=\frac{1}{2}J\vec{\omega}^{2}+\frac{1}{2}m\overset{\cdot }{%
\vec{r}^{2}}$

We use $\vec{u}$ to represent the velocity on the bottom point of the ball 

The restraint condition is $\vec{u}=\overset{\cdot }{\vec{r}}+\vec{\omega}%
\times \left( -a\right) \vec{k}.$ where $\vec{\omega}\times \left( -a\right) 
\vec{k}$ is the velocity 

of $\vec{u}$ with respect to CM. $\vec{u}$ can be determined by other
conditions, now we 

solve from this equation $\vec{\omega}^{\prime }=\vec{\omega}-\omega _{z}%
\vec{k}:$

$\vec{u}=\overset{\cdot }{\vec{r}}+\vec{\omega}^{\prime }\times \left(
-a\right) \vec{k},$Multiplying both sides by $\vec{k}$ gives $\frac{1}{a^{2}}%
\left( \vec{u}-\overset{\cdot }{\vec{r}}\right) ^{2}=\vec{\omega}^{\prime 2},
$

since $\vec{\omega}^{\prime }\cdot \vec{k}=0$

We then substitute $\vec{\omega}^{2}=\vec{\omega}^{\prime 2}$+$\omega
_{z}^{2}$ into

$T=\frac{1}{2}J\vec{\omega}^{2}+\frac{1}{2}m\overset{\cdot }{\vec{r}^{2}}%
\implies $

$\frac{2T}{m}=\overset{\cdot }{\vec{r}}\overset{2}{}+\frac{J}{ma^{2}}\left( 
\vec{u}-\overset{\cdot }{\vec{r}}\right) ^{2}+C\omega _{z}^{2},$

We choose the sphere as our system,$M=%
\begin{pmatrix}
\cos \Omega t & -\sin \Omega t \\ 
\sin \Omega t & \cos \Omega t%
\end{pmatrix}%
$

$\vec{r}^{\prime }=M\vec{r},\overset{\cdot }{\vec{r}^{\prime }}=\Omega \vec{r%
}^{\prime }+M\overset{\cdot }{\vec{r}}.$

We can replace $\overset{\cdot }{\vec{r}}$ by $\vec{r}^{\prime },\overset{%
\cdot }{\vec{r}^{2}}=\left( \overset{\cdot }{\vec{r}^{\prime }}-\Omega \vec{r%
}^{\prime }\right) ^{2},$

$\frac{2T}{m}=\left( 1+\frac{J}{ma^{2}}\right) \overset{\cdot }{\vec{r}}%
\overset{2}{}+\frac{J}{ma^{2}}\left( \vec{u}^{2}-2\vec{u}\cdot \overset{%
\cdot }{\vec{r}}\right) +C\omega _{z}^{2}$

$=\left( 1+\frac{J}{ma^{2}}\right) \left( \overset{\cdot }{\vec{r}^{\prime }}%
-\Omega \vec{r}^{\prime }\right) ^{2}+\frac{J}{ma^{2}}\left( \vec{u}^{2}-2%
\vec{u}\cdot M^{-1}\left( \overset{\cdot }{\vec{r}^{\prime }}-\Omega \vec{r}%
^{\prime }\right) \right) +C\omega _{z}^{2}$

$\left( 1+\frac{J}{ma^{2}}\right) \overset{\cdot \cdot }{\vec{r}^{\prime }}%
-2\left( 1+\frac{J}{ma^{2}}\right) \Omega \vec{r}^{\prime }+\frac{J}{ma^{2}}%
-2\vec{u}\cdot M^{-1}$

$\left( 1+\frac{J}{ma^{2}}\right) \overset{\cdot \cdot }{\vec{r}}-\frac{J}{%
ma^{2}}\overset{\cdot }{\vec{u}}=\frac{J}{ma^{2}}\frac{\partial \vec{u}}{%
\partial \vec{r}}\left( \vec{u}-\overset{\cdot }{\vec{r}}\right) =\frac{J}{%
ma^{2}}\frac{\partial \vec{u}}{\partial \vec{r}}\left( \vec{\omega}\times
\left( -a\right) \vec{k}\right) $

$=\frac{J}{ma^{2}}\frac{\partial \vec{\omega}}{\partial \vec{r}}\times
\left( -a\right) \vec{k}\left( \vec{\omega}\times \left( -a\right) \vec{k}%
\right) =\frac{1}{m}\frac{\partial }{\partial \vec{r}}\frac{1}{2}J\left( 
\vec{\omega}\times \vec{k}\right) ^{2}$

$,\frac{\partial \vec{u}}{\partial \vec{r}}$ is a matrix 

$\vec{u}=\Omega \vec{k}\times \vec{r},\Omega \left( \overset{\cdot }{\vec{r}}%
-\vec{u}\right) $

$\implies \left( 1+\frac{J}{ma^{2}}\right) \overset{\cdot \cdot }{\vec{r}}-%
\frac{J}{ma^{2}}\overset{\cdot }{\vec{u}}=0$

$\implies \overset{\cdot \cdot }{\vec{r}}=\frac{1}{1+\frac{ma^{2}}{J}}%
\overset{\cdot }{\vec{u}}$

$\frac{J}{ma^{2}}\frac{\partial \vec{u}}{\partial \vec{r}}\vec{u}-\frac{J}{%
ma^{2}}\frac{\partial \vec{u}}{\partial \vec{r}}\cdot \overset{\cdot }{\vec{r%
}}$

$\vec{u}=\Omega \vec{k}\times \vec{r}$

\QTP{Body Math}
$V=mga-\frac{1}{2}m\Omega ^{2}\left( x^{2}+y^{2}\right) $

\QTP{Body Math}
$7\ddot{x}+4\Omega \dot{y}=7\Omega ^{2}x$

\QTP{Body Math}
$7\ddot{y}+4\Omega \dot{x}=7\Omega ^{2}y$

\QTP{Body Math}
$\dot{\omega}_{z}=0$

\QTP{Body Math}
$,\frac{d}{dt}\left( \frac{\partial L}{\partial \dot{q}}\right) -\frac{%
\partial L}{\partial q}=$

$T=m\left[ \frac{7}{10}\left( \dot{x}^{2}+\dot{y}^{2}\right) +\frac{1}{5}%
\Omega ^{2}\left( x^{2}+y^{2}\right) +\frac{2}{5}\Omega \left( -x\dot{y}+%
\dot{x}y\right) \right] $

\end{document}
